\documentclass[a4paper,12pt]{article}

\usepackage{ProfModels}


\begin{document}
\begin{Maquette}[Fiche]{Theme=Le parallélogramme,Niveau=1}

\begin{exercice}
$ABC$ un triangle et $O$ le milieu de $\lrc{AC}$.
\begin{enumerate}
\item Construire $D$ le symétrique de $B$ par rapport à $O$.
\item Montrer que $ABCD$ est un parallélogramme.
\end{enumerate}
\end{exercice}

\begin{exercice}
$ABCD$ est un parallélogramme de centre $O$ et $E$ un point de $\lrc{DO}$.
\begin{enumerate}
\item Construire $F$ le symétrique de $E$ par rapport à $O$.
\item Montrer que le quadrilatère $AFCE$ est un parallélogramme.
\end{enumerate}
\end{exercice}

\begin{exercice}
$ABC$ est un triangle tel que $\widehat{BAC}=70^{\circ}$;la droite parallèle à $\lrp{AC}$ passant par $B$ coupe la parallèle à $\lrp{AB}$passant par $C$ en $D$.
\begin{enumerate}
\item Construire une figure convenable.
\item Montrer que $ABCD$ est un parallélogramme.
\item Calculer $\widehat{BDC}$ et $\widehat{ABD}$.
\end{enumerate}
\end{exercice}

\begin{exercice}
$(\mathcal{C}1)$ et $(\mathcal{C}2)$ sont deux cercles de même centre et pas de même rayon.$\lrc{AC}$ est un diamètre de $(\mathcal{C}1)$ et $\lrp{BD}$ est diamètre de $(\mathcal{C}2)$
\begin{enumerate}
\item Trace une figure.
\item Prouver que $ABCD$ est un parallélogramme.
\end{enumerate}
\end{exercice}

\begin{exercice}
\begin{minipage}{0.5\linewidth}
On considère la figure ci-contre tel que $ABCD$ et $EFCD$ sont des parallélogrammes.
\begin{enumerate}
\item Montrer que $AE=BF$
\end{enumerate}
\end{minipage}%
\begin{minipage}{0.5\linewidth}
\begin{tikzpicture}
\tkzDefPoints{0/0/A,-1/-2/B,3/-1/C,5/-2/F}
\tkzDefParallelogram(A,B,C)\tkzGetPoint{D}
\tkzDefParallelogram(D,C,F)\tkzGetPoint{E}
\tkzDrawPolygon(A,B,C,D)
\tkzDrawPolygon(D,C,F,E)
\tkzDrawSegments[dashed](A,E B,F)
\tkzLabelPoints(B,C,F)
\tkzLabelPoints[above](A,D,E)
\end{tikzpicture}
\end{minipage}
\end{exercice}

\begin{exercice}
\begin{minipage}{0.6\linewidth}
On considère la figure ci-contre, tel que $\lrp{D}//\lrp{D'}$ et $\lrp{K}//\lrp{K'}$.On suppose que $\widehat{DAB}=130^{\circ}$ et $AB=5$ et $AD=3$ et que $E$ le milieu de $\lrc{BD}$.
\begin{enumerate}
\item Montrer que $ABCD$ est un parallélogramme.
\item Calculer $BC$ et $DC$.
\item Calculer $\widehat{BCD}$ et $\widehat{ABC}$.
\item Montrer que $E$ est le milieu de $\lrc{AC}$.
\end{enumerate}
\end{minipage}%
\begin{minipage}{0.4\linewidth}
\begin{tikzpicture}
\tkzDefPoints{0/0/A,4/0/B,3/-2/C}
\tkzDefParallelogram(A,B,C)
\tkzGetPoint{D}
\tkzDefMidPoint(B,D)\tkzGetPoint{E}
\tkzDrawLines(A,B D,C B,C A,D)
\tkzDrawSegment(B,D)
\tkzDrawPoint(E)
\tkzLabelLine[pos=1.3](A,B){(D)}
\tkzLabelLine[pos=1.3](D,C){(D')}
\tkzLabelLine[pos=1.4](A,D){(K)}
\tkzLabelLine[pos=1.4](B,C){(K')}
\tkzLabelPoints[above left](A,B,C,D,E)
\end{tikzpicture}
\end{minipage}

\end{exercice}




















\end{Maquette}
\end{document}