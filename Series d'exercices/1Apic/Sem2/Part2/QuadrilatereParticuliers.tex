\documentclass[a4paper,12pt]{article}


\usepackage{ProfModels}

 
\begin{document}

\begin{Maquette}[Fiche]{Theme=Les quadrilatère particuliers,Niveau=1}

\begin{exercice}
$EFG$ est un triange et $I$ le milieu de $\lrc{GE}$, $H$ est le symétrique de $F$ par rapport à $I$.
\begin{enumerate}
\item Construire une figure convenable.
\item Montrer que $EFGH$ est un parallélogramme.
\item Déduire que $\lrp{EF}//\lrp{HG}$.
\end{enumerate}
\end{exercice}

\begin{exercice}
$ABC$ est un triangle tel que : $AB=3$ et $AC=5$, le point $I$ est le milieu de $\lrc{BC}$ et $D$ le symétrique de $A$ par rapport à $I$.
\begin{enumerate}
\item Faire une figure.
\item Prouver que $ABCD$ est parallélogramme.
\item Calculer $BD$ et $CD$.
\end{enumerate}
\end{exercice}

\begin{exercice}
$EFG$ est triangle et $M$ le milieu de $\lrc{EG}$.
\begin{enumerate}
\item Construire $H$ le symétrique de $F$ par rapport à $M$.
\item Montrer que $EFGH$ est un parallélogramme.
\item Prouver que $\widehat{EMH}=\widehat{FMG}$ et $\widehat{EHF}=\widehat{HFG}$.
\item Comment choisir le triangle $EFG$ pour que $EFGH$ être un losange.Justifier.
\end{enumerate}
\end{exercice}

\begin{exercice}%4
$ABC$ est un triangle tel que $BC=5$ et $\widehat{ABC}=60^{\circ}$ et $I$ le milieu de $\lrc{AB}$,et $\lrp{AE}$ est la hauteur relative au côté $\lrc{BC}$.
\begin{enumerate}
\item Construire $D$ le symétrique de $E$ par rapport à $I$.
\item Montrer que $AEBD$ est un rectangle.
\item Montrer que $\widehat{EAB}=30^{\circ}$.
\item Construire $F$ le symétrique de $C$ par rapport à $I$.
\item Montrer que les points $A$, $D$ et $F$ sont alignés.
\item MOntrer que $\widehat{ADE}=\widehat{DEB} $.
\end{enumerate}
\end{exercice}

\begin{exercice}%5
$ABC$ est un triangle équilatéral tel que $AB=4$. Soit $I$ le milieu de $\lrc{AC}$ et $D$ le symétrique de $B$ par rapport à $I$.
\begin{enumerate}
\item Montrer que $\lrp{BC}//\lrp{AD}$.
\item Montrer que $\widehat{DAC}=60^{\circ}$.
\item Quelle est la nature du quadrilatère $ABCD$.
\end{enumerate}
\end{exercice}

\begin{exercice}
$ABD$ est un triangle isocèle en $A$ et $O$ le milieu de $\lrc{BD}$.
\begin{enumerate}
\item Construire $C$ le symétrique de $A$ par rapport à $O$.
\item Montrer que $ABCD$ est un parallélogramme.
\item Déduire que $\lrp{AC}\perp\lrp{BD}$.
\item Construire $E$ le symétrique de $B$ par rapport à $C$.
\item Prouver que $ADEC$ est un parallélogramme.
\item Montrer que $BDE$ est un triangle rectangle.
\end{enumerate}
\end{exercice}

\begin{exercice}
$ABCD$ est un parallélogramme.
\begin{enumerate}
\item Construire $E$ le projeté orthogonal de $A$ sur $CD$.
\item Construire $F$ le projeté orthogonal de $B$ sur $CD$.
\item Montrer que $ABFE$ est un parallélogramme.
\item Déduire que $FA=EB$.
\end{enumerate}
\end{exercice}

\begin{exercice}
$ABCD$ est un losange. La droite perpendiculaire à $\lrp{AC}$ passant par $A$ coupe $\lrp{BC}$ en $E$.
\begin{enumerate}
\item Construire une figure.
\item Montrer que $\lrp{AE}//\lrp{BD}$
\item Montrer que $AEBD$ est parallélogramme.
\end{enumerate}
\end{exercice}

\begin{exercice}
$ABCD$ est un parallélogramme de centre $O$.
\begin{enumerate}
\item Construire M le milieu de $\lrc{AB}$ et N le milieu de $\lrc{BC}$.
\item Construire $E$ et $F$ les symétriques de $O$ par rapport à M et à N respectivement.
\item Montrer que $AOBE$ et $BOCF$ sont des rectangles.
\item Déduire que $B$ est le milieu de $\lrc{EF}$.
\item Montrer que $OEF$ est un triangle isocèle.
\end{enumerate}
\end{exercice}

\begin{exercice}
$ABCD$ est un carré de centre O.
\begin{enumerate}
\item Construire E le milieu de $\lrc{AB}$ et $F$ le milieu de $\lrc{AD}$.
\item Montrer que $\lrp{OE}$ est la médiatrice de $\lrc{AB}$.
\item Montrer que $\lrp{OF}$ est la médiatrice de $\lrc{AD}$.
\item Montrer que $EOFA$ est un carré.
\item Déduire que $OA=EF$
\item Soit M le milieu de $\lrc{BC}$ et N le milieu de $\lrc{CD}$.
\begin{enumerate}
\item Montrer que $EM=MN=FN=FE$
\item Montrer que $EMNF$ est un carré.
\end{enumerate}
\item Construire les points R,S,P et Q les milieux de $\lrc{EM}$ et $\lrc{NM}$ et $\lrc{NF}$ et $\lrc{EF}$ respectivement.
\begin{enumerate}
\item Montrer que $RSPQ$ est carré.
\end{enumerate}
\end{enumerate}
\end{exercice}

\begin{exercice}
$ABC$ est un triangle isocèle en $A$ tel que $\widehat{BAC}=50^{\circ}$.
\begin{enumerate}
\item Construire $B'$ et $C'$ les symétriques respective de $B$ et $C$ par rapport à $A$.
\item Montrer que $BCB'C'$ est un rectangle.
\item Construire $A'$ le symétrique de $A$ par rapport à $\lrp{BC}$.
\begin{enumerate}
\item Montrer que $ABA'C$ est un losange.
\item Calculer les mesures des angles de losange $ABA'C$.
\end{enumerate}
\end{enumerate}
\end{exercice}













\end{Maquette}
\end{document}