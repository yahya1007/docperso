\documentclass[a4paper,12pt]{article}


\usepackage{ProfModels}

 
\begin{document}

\begin{Maquette}[Fiche]{Theme=Le cercle,Niveau=1}

\begin{exercice}
 On donne deux points $A$ et $B$, tels que : $AB= 2,5$.
 \begin{enumerate}
\item Trace le cercle $\lrp{\mathcal{C}1}$ de centre $A$ et de rayon 1,5
\item Trace le cercle $\lrp{\mathcal{C}2}$ de centre $B$ et de rayon 1
\item Les deux cercles se coupent au point $E$
\begin{enumerate}
\item Trace la droite $\lrp{AB}$.
La droite $\lrp{AB}$ coupe le cercle $\lrp{\mathcal{C}1}$ en $G$ et $E$ ; elle coupe le cercle $\lrp{\mathcal{C}2}$ en $E$ et $H$
\item Construis le cercle $\lrp{\mathcal{C}}$ de diamètre $\lrc{GH}$
\end{enumerate}
\item Hachure les disques $D\lrp{A ; 1,5}$ et $D\lrp{B ; 1}$
 \end{enumerate}
\end{exercice}

\begin{exercice}
\begin{minipage}{0.5\linewidth}
$\lrp{\mathcal{C}}$ est un cercle de centre $O$. Les droites $\lrp{AC}$ et $\lrp{BC}$ sont tangentes au cercle $\lrp{\mathcal{C}}$ tel que $\widehat{AOB}=140^{\circ}$.
\begin{enumerate}
\item Calculer la mesure de $\widehat{ACB}$.
\item Déduire $\widehat{ACO}$.
\end{enumerate}
\end{minipage}%
\begin{minipage}{0.5\linewidth}
\begin{tikzpicture}[scale=0.8]
\tkzDefPoint(0,0){O}
\tkzDefPoint(-1,2){A}
\tkzDefPointBy[rotation=center O angle 140](A)
\tkzGetPoint{B}
\tkzDefLine[perpendicular=through A](O,A)
\tkzGetPoint{x}
\tkzDefLine[perpendicular=through B](O,B)
\tkzGetPoint{y}
\tkzInterLL(A,x)(B,y)\tkzGetPoint{C}
\tkzDrawCircle(O,A)
\tkzDrawLines(A,C B,C)
\tkzDrawSegments(O,A O,B)
\tkzLabelPoints[above](C,A)
\tkzLabelPoint[below](B){B}
\tkzLabelPoint[right](O){O}
\tkzLabelCircle[above](O,A)(-45){($\mathcal{C}$)}
\tkzLabelAngle[rotate=90](A,O,B){$140^{\circ}$}
\tkzMarkAngle[arc=l , size=0.4 , mark=|](A,O,B)
\end{tikzpicture}
\end{minipage}
\end{exercice}

\begin{exercice}
$\lrp{\mathcal{C}}$ est un cercle de centre $O$. $H$ est un point de $\lrp{\mathcal{C}}$ et $A$ le symétrique de $O$ par rapport à $H$. Soient $\lrp{D}$ médiatrice du segment $\lrc{OA}$. $\lrp{\Delta}$ la droite parallèle à $\lrp{D}$ et passant par $O$. $\lrp{\Delta}$ coupe $\lrp{\mathcal{C}}$ en $E$ et $F$.
\begin{enumerate}
\item Construire une figure.
\item Montrer que $\lrp{D}$ est la tangente au cercle $\lrp{\mathcal{C}}$ en $H$.
\item Calculer la mesure de $\widehat{HFO}$.
\end{enumerate}
\end{exercice}

\begin{exercice}
$\lrp{\mathcal{C}}$ est un cercle de centre $O$. Soient $A$ et $B$ deux points tels que : $A\in \lrp{\mathcal{C}}$ , $\widehat{AOB}=48^{\circ}$ et $\widehat{OBA}=42^{\circ}$.
\begin{enumerate}
\item Faire un figure convenable.
\item Montrer que la droite $\lrp{AB}$ est tangente au cercle $\lrp{\mathcal{C}}$.
\end{enumerate}
\end{exercice}

\begin{exercice}
Soient $\lrp{\mathcal{C}}$ un cercle de centre $O$ et $A$ un point extérieur à $\lrp{\mathcal{C}}$.
Le cercle $\lrp{\mathcal{C'}}$ de diamètre $\lrc{OA}$ coupe $\lrp{\mathcal{C}}$ en deux points $C$ et $D$.
On note $O'$ le centre de $\lrp{\mathcal{C'}}$.
\begin{enumerate}
\item Montrer que les triangles $AO'C$, $OO'C$, $AO'D$ et $OO'D$ sont isocèles.
\item Montrer que $\lrp{AC}$ et $\lrp{AD}$ sont tangentes à $\lrp{\mathcal{C}}$.
\end{enumerate}
\end{exercice}

\begin{exercice}
\begin{minipage}{0.5\linewidth}
$\lrp{\mathcal{C}}$ et $\lrp{\mathcal{C'}}$ sont deux cercles de même rayon et de centres respectifs $O$ et $O'$. $\lrp{\mathcal{C}}$ et $\lrp{\mathcal{C'}}$ se coupent en $A$ et $B$.
\begin{enumerate}
\item Montrer que $\lrp{AB}$ est la médiatrice du $\lrc{OO'}$.
\item Quelle est la nature du triangle $AOO'$.
\item Quelle est la nature du quadrilatère $AOBO'$.
\end{enumerate}
\end{minipage}%
\begin{minipage}{0.5\linewidth}
\begin{tikzpicture}
\tkzDefPoint(0,0){O}
\tkzDefPoint(3,-1){O'}
\tkzDefCircle[R](O,2) \tkzGetPoint{a}
\tkzDefCircle[R](O',2) \tkzGetPoint{b}
\tkzInterCC(O,a)(O',b)
\tkzGetPoints{A}{B}
\tkzDrawCircles(O,a O',b)
\tkzDrawPoints(O,O')
\tkzLabelPoints(O,O')
\tkzLabelPoint[above=5pt](A){A}
\tkzLabelPoint[below=5pt](B){B}
\end{tikzpicture}
\end{minipage}
\end{exercice}

\begin{exercice}
$\lrp{\mathcal{C}}$ est un cercle de centre $O$. Soient $A$ et $B$ deux points non diamétralement opposés sur $\lrp{\mathcal{C}}$. $\lrp{D}$ est la tangente en $A$ au cercle $\lrp{\mathcal{C}}$, $\lrp{D'}$ est la tangente en $B$ au cercle $\lrp{\mathcal{C}}$; $\lrp{D}$ et $\lrp{D'}$ se coupent en $I$.
\begin{enumerate}
\item Faire une  figure.
\item Montrer que $O$ appartient à la bissectrice de l'angle $\widehat{AIB}$.
\end{enumerate}
\end{exercice}

\begin{exercice}
$ABC$ est un triangle rectangle en A, O est le milieu de $\lrc{BC}$
\begin{enumerate}
\item Construire le point $D$ le symétrique de $A$ par rapport à $O$.
\item Montrer que le quadrilatère $ABDC$ est un rectangle.
\item Déduire que A,B et C appartiennent tous à un cercle de centre O et de rayon OA.
\end{enumerate}
\end{exercice}

\begin{exercice}
\begin{minipage}{0.6\linewidth}
On considère la figure ci-contre.
\begin{enumerate}
\item Montrer que $OB=OC=OE$.
\item Déduire que les points $A$, $B$, $C$, $D$ et $E$ sont cocycliques.
\end{enumerate}
\end{minipage}%
\begin{minipage}{0.4\linewidth}
\begin{tikzpicture}
\tkzDefPoints{0/0/D,4/2/A}
\tkzDefPointOnLine[pos=0.5](A,D)
\tkzGetPoint{O}
\tkzDefTriangle[two angles=30 and 60](D,A)\tkzGetPoint{C}
\tkzDefTriangle[two angles=70 and 20](D,A)\tkzGetPoint{B}
\tkzDefTriangle[two angles=48 and 42](A,D)\tkzGetPoint{E}
\tkzDrawPoint(O)
\tkzDrawPolygon(A,D,B)
\tkzDrawPolygon(A,D,C)
\tkzDrawPolygon(A,D,E)
\tkzLabelPoints(E,D,O)
\tkzLabelPoints[above](B,C,A)
\tkzMarkRightAngles(A,E,D D,B,A D,C,A)
\tkzMarkSegments[mark=||](O,D O,A)
\end{tikzpicture}
\end{minipage}
\end{exercice}

\begin{exercice}
\begin{minipage}{0.6\linewidth}
Soit $\lrp{\mathcal{C}}$ un cercle de centre $O$, $\lrp{D}$ et $\lrp{D'}$ sont deux tangentes au cercle $\lrp{\mathcal{C}}$ respectivement en $A$ et $B$.Soit I le point d'interssection de $\lrp{D}$ et $\lrp{D'}$.
\begin{enumerate}
\item Montrer que $\lrp{OI}$ est la bissectrice de $\widehat{AOB}$.
\item Déduire que $\lrp{OI}$ est la médiatrice de $\lrc{AB}$
\end{enumerate}
\end{minipage}%
\begin{minipage}{0.4\linewidth}
\begin{tikzpicture}
\tkzDefPoint(0,0){O}
\tkzDefCircle[R](O,1.5)\tkzGetPoint{a}
\tkzDefPointOnCircle[through= center O angle 45 point a]
\tkzGetPoint{A}
\tkzDefPointOnCircle[through= center O angle -90 point a]
\tkzGetPoint{B}
\tkzDefLine[tangent at =A](O)\tkzGetPoint{x}
\tkzDefLine[tangent at =B](O)\tkzGetPoint{y}

\tkzInterLL(A,x)(B,y)\tkzGetPoint{I}
\tkzDrawLines(A,I B,I)
\tkzDrawSegments(O,A O,B)
\tkzLabelLine[pos=1.4](I,A){(D)}
\tkzLabelLine[pos=1.4](I,B){(D')}
\tkzDrawCircle(O,a)
\tkzDrawPoints(A,B)
\tkzLabelPoint[left](O){O}
\tkzLabelPoint[below](B){B}
\tkzLabelPoints[above](A,I)
\tkzMarkRightAngles(O,A,I I,B,O)
\end{tikzpicture}
\end{minipage}
\end{exercice}






\end{Maquette}
\end{document}