\documentclass[a4paper,12pt]{article}

\usepackage{ProfModels}

 
\begin{document}

\begin{Maquette}[Fiche]{Theme=Les équations ,présentation}

\begin{exercice}
Résoudre les équations :
\[
x+5=0\quad ;; \quad
x-9=0\quad ;; \quad
3-x=0\quad ;; \quad
2-x=3
\]
\[
4x=2\quad ;; \quad
6x=-4\quad ;; \quad
5x=-1\quad ;; \quad
-3x=-1
\]
\end{exercice}

\begin{exercice}
Résoudre les équations :
\[
3x-7=0\quad ;; \quad
6x-7=4\quad ;; \quad
7x-9=9\quad ;; \quad
6x-7=x
\]
\[
4x+6=7+x\quad ;; \quad
7x-8=6x-7\quad ;; \quad
3x-9=2x-7\quad ;; \quad
2x+1=2x-1
\]
\end{exercice}

\begin{exercice}
Résoudre les équations :
\[
3(x+6)=2x-7\quad ;; \quad
2x-(3x-7)=5(x+7)\quad ;; \quad
10x+6(x-2)=x-(5-x)
\]
\[
5-(x+9)=1+2x\quad ;; \quad
2x-(4-x)=-1\quad ;; \quad
36-(8-x)=x+2-(5-x)
\]
\end{exercice}

\begin{exercice}
Résoudre les équations :
\[
\dfrac{x+1}{2}+2=\dfrac{x+3}{3}\quad ;; \quad
\dfrac{x+1}{2}=\dfrac{x+2}{4}\quad ;; \quad
x+1+\dfrac{2x+1}{3}=0
\]
\[
\dfrac{2x-1}{2}+\dfrac{3x-4}{3}=1\quad ;; \quad
\dfrac{2}{3}(x+5)=\dfrac{5}{4}\quad ;; \quad
\dfrac{7}{4}x-x=\dfrac{5x-3}{6}
\]
\end{exercice}

\begin{exercice}
Résoudre les équations :
\[
\dfrac{4x-7}{3}+\dfrac{5x-11}{6}=\dfrac{x+2}{2}\quad ;; \quad
x+5+\dfrac{2x-8}{4}=\dfrac{3x-4}{7}
\]
\[
x+1+\dfrac{6x-1}{5}=\dfrac{5x-7}{2}\quad ;; \quad
\dfrac{3x-7}{6}-x=\dfrac{4x+4}{5}
\]
\end{exercice}
\begin{exercice}
Trois amis Ahmed, Jamal et Omar veulent partager la somme de 3600 DH.Si Jamal reçoit le double d'Ahmed et Omar le triple de Jamal, quel serais le montant pris par chacun des trois amis?
\end{exercice}
\begin{exercice}
On veut partager une somme d'argent à plusieurs personnes.Si chacun d'entre eux prends 400 DH, il reste 300 DH.Si on donne 550 DH à chacun,il manquera 950 DH.Déterminer le nombre de personnes.
\end{exercice}
\begin{exercice}
Un père et ses deux enfants ont respectivement pour âge 29, 7 et 4 ans.
Dans combien d'années l'âge du père sera la somme des âges de ses deux enfants?
\end{exercice}
\begin{exercice}
La moyenne d'un élève au premier trimestre était de 14.55, sachant que le nombre total des coefficients est 37 et le coefficient des mathématiques est 7, la moyenne des autres matière est : 13.71.
Quelle est la moyenne des mathématiques?
\end{exercice}
\begin{exercice}
Un théâtre reçoit 1700 personnes. 500 places sont situées devant la scène. Les autres sont un peu plus loin (les gradins).
Lors d'un concert, la salle est pleine. Les billets des places devant la scène sont vendus à 240 DH. Sachant que la recette du concert est de 300000 DH.
\begin{enumerate}
\item Déterminer le montant d'un billet des places en gradins.
\item Lors d'un deuxième concert, les places devant la scène sont de nouveau pleines et la recette est de 262500 Dh.
\begin{enumerate}
\item Déterminer le nombre de spectateurs dans les gradins. 
\end{enumerate}
\end{enumerate}
\end{exercice}
\end{Maquette}
\end{document}