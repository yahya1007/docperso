\documentclass[a4paper,12pt]{article}


\usepackage{ProfModels}
\newcolumntype{C}[1]{>{\centering\arraybackslash}p{#1}}
 
\begin{document}
\begin{Maquette}[Fiche]{Theme=Proportionnalité,Niveau=1}

\begin{exercice}
Ces tableaux sont-ils des tableaux de proportionnalité?

\begin{minipage}{0.24\linewidth}
\begin{tabular}{|c|c|c|c|}
\hline 
12 & 15 & 37 & 100 \\ 
\hline 
24 & 30 & 73 & 200 \\ 
\hline 
\end{tabular} 
\begin{center}
 tableau 1 
\end{center}
\end{minipage}\hfill%
\begin{minipage}{0.24\linewidth}
\begin{tabular}{|c|c|c|c|}
\hline 
3 & 5 & 1.5 & 10 \\ 
\hline 
4.5 & 7.5 & 2.25 & 15 \\ 
\hline 
\end{tabular}
\begin{center}
 tableau 2 
\end{center}
\end{minipage}\hfill%
\begin{minipage}{0.24\linewidth}
\begin{tabular}{|c|c|c|c|}
\hline 
4 & 5 & 1.5 & 11 \\ 
\hline 
12 & 15 & 7.5 & 15 \\ 
\hline 
\end{tabular} 
\begin{center}
 tableau 3 
\end{center}
\end{minipage}\hfill%
\begin{minipage}{0.24\linewidth}
\begin{tabular}{|c|c|c|c|}
\hline 
3 & 4 & 15 & 10 \\ 
\hline 
1.5 & 2.5 & 0.75 & 5 \\ 
\hline 
\end{tabular}
\begin{center}
 tableau 4 
\end{center}
\end{minipage}
\end{exercice}

\begin{exercice}
Compléter les tableaux de proportionnalité suivants:

\begin{minipage}{0.33\linewidth}
\begin{tabular}{|C{0.75cm}|C{0.75cm}|C{0.75cm}|C{0.75cm}|}
\hline 
4 & 5 &  & 14 \\ 
\hline 
5 &  & 9 &  \\ 
\hline 
\end{tabular} 
\end{minipage}\hfill%
\begin{minipage}{0.33\linewidth}
\begin{tabular}{|C{0.75cm}|C{0.75cm}|C{0.75cm}|C{0.75cm}|}
\hline 
 1 &  & 6 &  24\\ 
\hline 
  & 18 & 12 &  \\ 
\hline 
\end{tabular} 
\end{minipage}\hfill%
\begin{minipage}{0.33\linewidth}
\begin{tabular}{|C{0.75cm}|C{0.75cm}|C{0.75cm}|C{0.75cm}|}
\hline 
4 & 5 &  & 36 \\ 
\hline 
12 &  & 20 &  \\ 
\hline 
\end{tabular} 
\end{minipage}
\end{exercice}

\begin{exercice}
Compléter le tableau suivant 

\begin{tabular}{|c|C{1cm}|C{1cm}|C{1cm}|C{1cm}|C{1cm}|C{1cm}|}
\hline 
Longueur du côté d'un carré & 5 & 6.4 &  & 13.4 &  & 1.98 \\ 
\hline 
Périmètre du carré &  &  & 98 &  & 36.4 &  \\ 
\hline 
\end{tabular} 
\end{exercice}


\begin{exercice}
Une voiture consomme 10 litres de gasoil pour faire 150 km.
\begin{enumerate}
\item Quelle consommation peut-on prévoir pour 350 km ?
\item Avec 45 l dans le réservoir, peut-on espérer faire un trajet de 800 km ?
\end{enumerate}
\end{exercice}

\begin{exercice}
Dans une école de 135 élèves, on compte 72 filles et 63 garçons.
\begin{enumerate}
\item Quel est le pourcentage des garçons ?
\item Quel est le pourcentage des filles ?
\end{enumerate}
\end{exercice}

\begin{exercice}
Compléter le tableau : 

\begin{tabular}{|c|C{1cm}|C{1cm}|C{1cm}|C{1cm}|C{1cm}|C{1cm}|C{1cm}|}
\hline 
Nombre & 60 &  &  &  &  &  &  \\ 
\hline 
Pourcentage & 100 & 50 & 25 & 10 & 20 & 5 & 1 \\ 
\hline 
\end{tabular} 

Un pantalon d'une valeur de 160 Dhs est vendu avec une réduction de 20\%. Quel est le prix de ce pantalon ?
\end{exercice}

\begin{exercice}
\begin{minipage}{0.6\linewidth}
Le graphique suivant permet de déterminer le prix à payer pour une chambre pour deux personnes dans un hôtel en fonction du  nombre de nuits.
\begin{enumerate}
\item A l'aide du graphique, complète le tableau

\begin{tabular}{|c|c|c|c|c|c|c|}
\hline 
Nombre de nuits & 1 & 2 &  & 5 &  & 6 \\ 
\hline 
Prix à payer &  &  & 4000 &  & 3000 &  \\ 
\hline 
\end{tabular} 
\item Le prix à payer est-il proportionnel au nombre de nuits passées à l'hôtel ?
\end{enumerate}
\end{minipage}\hfill%
\begin{minipage}{0.38\linewidth}
\begin{tikzpicture}[scale=0.8]
\tkzInit[xmin=0, xmax=6, ymin=0, ymax=6000,ystep=1000]
\tkzGrid[sub,subxstep=0.5,subystep=500,color=black,line width=0.4pt]
\tkzDrawXY[label={}]
\tkzLabelXY
\tkzDefPoints{0/0/A,6/6000/B}
\tkzDrawLine[add=0 and 0,line width=1pt](A,B)
\end{tikzpicture}
\end{minipage}
\end{exercice}



\end{Maquette}
\end{document}