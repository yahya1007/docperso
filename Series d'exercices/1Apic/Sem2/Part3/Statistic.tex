\documentclass[a4paper,12pt]{article}


\usepackage{ProfModels}
\usepackage{diagrammes}
 
\begin{document}

\begin{Maquette}[Fiche]{Theme=Les statistiques,Niveau=1}

\begin{exercice}
Les notes obtenues lors d'un d'un contrôle de mathématiques par les élèves d'une classe sont :

\begin{tabular}{|c|c|c|c|c|c|c|c|c|c|}
7 & 9 & 10 & 15 & 10 & 17 & 16 & 9 & 9 & 9 \\ 
\hline 
10 & 10 & 13 & 16 & 16 & 10 & 10 & 15 & 13 & 9 \\ 
\hline 
7 & 7 & 9 & 10 & 16 & 17 & 17 & 10 & 13 & 13 \\ 
\end{tabular} 
\begin{enumerate}
\item Quel est le caractère étudié ?
\item Quelle est la population statistique ?
\item Classer les notes dans un tableau d'effectif.
\item Quel est l'effectif total de cette série statistique ?
\item Quel est le nombre des élèves ayant une note supérieur à 10 ?
\end{enumerate}
\end{exercice}

\begin{exercice}
On a demandé aux élèves d'une classe la somme d'argent de poche que leurs parents donnent chaque jour.

\begin{tabular}{|c|c|c|c|c|}
 \hline 
 Argent de poche & 0 & 5 & 10 & 20 \\ 
 \hline 
 Effectifs & 4 & 8 & 16 & 4 \\ 
 \hline 
 \end{tabular} 
 \begin{enumerate}
 \item Combien y a-t-il d'élèves dans cette classe ?
 \item Calculer la fréquence de chaque valeur.
 \item Calculer le pourcentage des élèves qui reçoivent 10 Dhs de leurs parents.
 \item Représenter cette série sur un diagramme circulaire .
 \end{enumerate}
\end{exercice}

\begin{exercice}
Le diagramme en bâtons ci-dessous représente les pointures des chaussures des élèves d'une classe de 1.AC.\newline
\begin{minipage}{.55\linewidth}
\begin{enumerate}
\item Recopier et compléter.

\begin{tabular}{|c|c|c|c|c|c|c|c|}
\hline 
Pointure & 36 & 37 & 38 & 39 & 40 & 41 & 42 \\ 
\hline 
Effectif &  &  &  &  &  &  &  \\ 
\hline 
Fréquence &  &  &  &  &  &  &  \\ 
\hline 
Pourcentage &  &  &  &  &  &  &  \\ 
\hline 
\end{tabular} 
\item Quel est l'effectif total des élèves ?
\end{enumerate}
\end{minipage}%
\begin{minipage}{.45\linewidth}
\Histogramme[Largeur=6,Hauteur=6,%
ListeCouleurs={orange,gray,blue,pink,red,black,gray},%
DebutOx=35,FinOx=42,GradX={35,36,...,42},GradY={0,2,...,10},%
AffEffectifs=false,LabelX={},LabelY={}]%
{35.8/36.2/4 36.8/37.2/8 37.8/38.2/4 38.8/39.2/2 39.8/40.2/10 40.8/41.2/8 41.8/42.2/6}
%\begin{tikzpicture}[scale=0.8]
%\tkzInit[xmin=35, xmax=43, ymin=0, ymax=13,xstep=1,ystep=2]
%\tkzGrid
%\tkzDrawX[label={}]
%\tkzLabelX
%\tkzDrawY[label={élèves}]
%\tkzLabelY
%\tkzDefSetOfPoints[prefix=A]%
%{36/0,37/0,38/0,39/0,40/0,41/0,42/0}
%\tkzDefSetOfPoints[prefix=P]%
%{36/6,37/8,38/12,39/4,40/4,41/2,42/2}
%\tkzDrawLines[add=-2pt and -2pt, line width=3pt](A1,P1 A2,P2 A3,P3 A4,P4 A5,P5 A6,P6 A7,P7)
%\end{tikzpicture}
\end{minipage}
\end{exercice}

\begin{exercice}
\begin{minipage}{0.55\linewidth}
le tableau suivant donne le nombre d'accident de la route durant 40 jours.
Construire un diagramme circulaire représente cette série statistique.
\end{minipage}%
\begin{minipage}{0.45\linewidth}
\begin{tabular}{|c|c|c|c|c|c|}
\hline 
Nombre d'accidents & 0 & 1 & 2 & 3 & 4 \\ 
\hline 
Nombre de jour & 14 & 5 & 16 & 3 & 2 \\ 
\hline 
\end{tabular} 
\end{minipage}
\end{exercice}



\begin{exercice}
\begin{minipage}{0.6\linewidth}
Le diagramme circulaire représente la répartition des élèves d'un collège suivant leur niveau en classe.Compléter le tableau suivant sachant qu'on a 450 élèves.

\begin{tabular}{|c|c|c|c|c|}
\hline 
Niveau & N1 & N2 & N3 & N4 \\ 
\hline 
les élèves &  &  &  &  \\ 
\hline 
Fréquence &  &  &  &  \\ 
\hline 
pourcentage &  &  &  &  \\ 
\hline 
Mesure d'angle & $90^{\circ}$ & $108^{\circ}$ & $63^{\circ}$ & $99^{\circ}$ \\ 
\hline 
\end{tabular} 
\end{minipage}%
\begin{minipage}{0.4\linewidth}
\begin{tikzpicture}[scale=0.8]
\pie[hide number]{25/N1, 30/N2, 17.5/N3, 27.5/N4}
\end{tikzpicture}
\end{minipage}
\end{exercice}


\begin{exercice}
\begin{minipage}{0.6\linewidth}
On considère une série statistique représentée par le diagramme en bâtons ci-contre.
\begin{enumerate}
\item Quelle est la valeur ayant le plus grand effectif ?
\item Quelle est la valeur ayant le plus petit effectif ?
\item Dresser le tableau des effectifs de la série.
\item Calculer le pourcentage de chaque valeur.
\item Représenter la série par un diagramme circulaire.
\end{enumerate}
\end{minipage}%
\begin{minipage}{0.4\linewidth}
\Histogramme[Largeur=6,Hauteur=6,%
ListeCouleurs={orange,gray,blue,pink,red,black,gray,green},%
DebutOx=0,FinOx=8,GradX={1,2,...,8},GradY={0,2,...,12},%
AffEffectifs=false,LabelX={},LabelY={}]%
{0.8/1.2/6 1.8/2.2/10 2.8/3.2/8 3.8/4.2/12 4.8/5.2/10 5.8/6.2/8 6.8/7.2/6 7.8/8.2/2}
\end{minipage}
\end{exercice}











\end{Maquette}
\end{document}