\documentclass[a4paper,12pt]{article}


\usepackage{TestProfModels}

 
\begin{document}

\begin{Maquette}[Fiche]{Theme=Les nombres en écriture fractionnaire : présentation,Niveau=1}

\begin{exercice}
\begin{enumerate}
\item \'Ecrire les nombres suivants sous forme d'écriture fractionnaire :
$$56.4 ;; 0 ;; 7 ;; 0.01 ;; 169.269 ;; 8.9 ;;  $$

\item Complète par qui convient :
$$\dfrac{\cdots}{4}=\dfrac{2}{15} ;; \dfrac{5}{8}=\dfrac{5}{\cdots} ;; \dfrac{8}{\cdots}=\dfrac{48}{\cdots} ;; \dfrac{0}{627}=\dfrac{\cdots}{5} ;; \dfrac{\cdots}{4}=\dfrac{432}{120}$$
\end{enumerate}
\end{exercice}

\begin{exercice}
\begin{enumerate}
\item Rendre les dénominateurs entiers :
$$\dfrac{8}{3.5} ;; \dfrac{1.44}{0.62} ;; \dfrac{34.5}{34.15} ;; \dfrac{10}{0.00085} $$

\item Rendre les numérateurs entiers :
$$\dfrac{6.1}{2.7} ;; \dfrac{64.24}{1.5} ;; \dfrac{59.325}{6.22} ;; \dfrac{0.063}{3.84} $$
\end{enumerate}
\end{exercice}

\begin{exercice}
Compléter par qui convient :
$$\dfrac{36}{42}=\dfrac{\cdots}{7} ;; \dfrac{125}{225}=\dfrac{\cdots}{9} ;; \dfrac{24}{18}=\dfrac{\cdots}{8}$$
$$\dfrac{1.4}{\cdots}=\dfrac{7}{5} ;; \dfrac{12}{18}=\dfrac{8}{\cdots} ;; \dfrac{2}{11}=\dfrac{16}{\cdots} ;; \dfrac{33}{15}=\dfrac{\cdots}{5} ;; \dfrac{12}{7}=\dfrac{\cdots}{28} $$
\end{exercice}

\begin{exercice}
Rendre les nombres suivants au même dénominateur :
$$\dfrac{7}{10}et \dfrac{3}{5};;;\dfrac{11}{6}et\dfrac{5}{24};;\dfrac{5}{8}et\dfrac{13}{12};;;\dfrac{1}{15}et\dfrac{4}{12};;\dfrac{6}{17}et\dfrac{7}{11};;;\dfrac{3.9}{3.6}et\dfrac{15}{60} $$
\end{exercice}

\begin{exercice}
Donner la forme irréductible des fractions suivantes :
$$\dfrac{121}{33};;\dfrac{981}{27};;\dfrac{235}{40};;;\dfrac{16}{48};;;\dfrac{169}{39} $$
$$\dfrac{2\times 5\times 24}{10\times 27}\;; \dfrac{12\times 25\times 14}{15\times 49\times 27} ;; \dfrac{200\times 124 \times 18}{39\times 45 \times 16} $$

\end{exercice}

\begin{exercice}
\begin{enumerate}
\item Comparer les nombres suivants :
$$ \dfrac{8}{10}\cdots \dfrac{4}{15};; \dfrac{17}{20}\cdots 0.24 ;;; \dfrac{78}{2008}\cdots \dfrac{78}{2009} ;;; \dfrac{2023}{2020}\cdots \dfrac{2023}{2020} $$
\item Ranger les nombres dans l'ordre croissant :
$$\dfrac{2}{5};\dfrac{11}{15};\dfrac{7}{10};\dfrac{13}{30};\dfrac{21}{6} $$
\item Ranger les nombres dans l'ordre décroissant :
$$\dfrac{15}{21};\dfrac{15}{35};\dfrac{12}{42};\dfrac{44}{77};\dfrac{9}{28} $$
\end{enumerate}

\end{exercice}

\end{Maquette}

\end{document}