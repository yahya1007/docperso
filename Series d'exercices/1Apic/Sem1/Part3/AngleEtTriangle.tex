\documentclass[a4paper,12pt]{article}


\usepackage{ProfModels}

 
\begin{document}

\begin{Maquette}[Fiche]{Theme=Les angles et le triangle,Niveau=1}

\begin{exercice}
\begin{enumerate}
\item Construire le triangle $ABC$ tel que $\widehat{ABC}=78^{\circ}$ et $\widehat{ACB}=54^{\circ}$.
\item Calculer la mesure de $\widehat{BAC}$.
\item $GHI$ est un triangle tel que $\widehat{GHI}=34^{\circ}$ et $\widehat{HGI}=47^{\circ}$. Calculer la mesure de $\widehat{GIH}$.
\end{enumerate}
\end{exercice}

\begin{exercice}
\begin{minipage}{.5\linewidth}
Calculer la mesure des angles $\widehat{ADC}$ et $\widehat{ABC}$ de la figure ci-dessous.
\end{minipage}%
\begin{minipage}{.5\linewidth}
\begin{tikzpicture}
\tkzDefPoints{0/0/B,4/1/C}
\tkzDefTriangle[two angles=86 and 40](B,C)
\tkzGetPoint{A}
\tkzDefLine[bisector](B,A,C) \tkzGetPoint{x}
\tkzInterLL(A,x)(B,C)
\tkzGetPoint{D}
\tkzDrawPolygon(A,B,C)
\tkzDrawLine(A,D)
\tkzLabelPoint[left](B){B}
\tkzLabelPoint[above](A){A}
\tkzLabelPoint[right](C){C}
\tkzLabelPoint[below left](D){D}
\tkzMarkAngle[arc=l , size=0.5 , mark=|](A,C,B)
\tkzMarkAngle[arc=ll , size=0.5 , mark=|](C,B,A)
\tkzMarkAngle[arc=l , size=0.5 , mark=|](C,D,A)
\tkzMarkAngles[size=0.5,mark=s|](D,A,C B,A,D)
\tkzLabelAngle(A,C,B){$40^{\circ}$}
\tkzLabelAngle(D,A,C){$28^{\circ}$}
\end{tikzpicture}
\end{minipage}
\end{exercice}

\begin{exercice}
\begin{minipage}{.6\linewidth}
On considère la figure ci-dessous
\begin{enumerate}
\item Calculer la mesure de $\widehat{TMR}$.
\item Construire la figure.
\item Calculer les mesures des angles $\widehat{MTH}$ et $\widehat{HTR}$.
\end{enumerate}
\end{minipage}%
\begin{minipage}{.4\linewidth}
\begin{tikzpicture}
\tkzDefPoints{0/0/M,5/1/R}
\tkzDefTriangle[two angles=36 and 54](M,R)
\tkzGetPoint{T}
\tkzDefPointBy[projection=onto M--R](T)\tkzGetPoint{H}
\tkzDrawPolygon(M,T,R)
\tkzDrawLine[add=0.1 and 0.1](T,H)
\tkzMarkAngle[arc=l , size=0.5 , mark=|](T,R,M)
\tkzMarkRightAngles(T,H,M M,T,R)
\tkzLabelSegment[above](M,T){5}
\tkzLabelPoints(M,H,R)
\tkzLabelPoint[above](T){T}
\tkzLabelAngle(T,R,M){$54^{\circ}$}
\end{tikzpicture}
\end{minipage}
\end{exercice}

\begin{exercice}
\begin{enumerate}
\item $ABC$ est un triangle isocèle de sommet principal $A$ tel que $\widehat{ABC}=55.8^{\circ}$.\newline
Calculer la mesure des angles $\widehat{BCA}$ et $\widehat{BAC}$.
\item $DEF$ est un triangle isocèle en $D$ tel que $\widehat{EDF}=42.6^{\circ}$.\newline
Calculer la mesure des angles $\widehat{DEF}$ et $\widehat{DFE}$.
\end{enumerate}
\end{exercice}

\begin{exercice}
\begin{minipage}{0.5\linewidth}
Le quadrilatère $ABCD$ est un rectangle; le point $E$ appartient au côté $\lrc{AB}$.
Le triangle $CDE$ est-il rectangle en $E$? Justifier la réponse.
\end{minipage}\hfill%
\begin{minipage}{0.4\linewidth}
\begin{tikzpicture}
\tkzDefPoints{0/0/A,5/-2/C}
\tkzDefRectangle(A,C)
\tkzGetPoints{B}{D}
\tkzDefPointOnLine[pos=0.3](A,B)
\tkzGetPoint{E}
\tkzDrawPolygon(A,B,C,D)
\tkzDrawSegments(D,E E,C)
\tkzLabelPoints[above](A,B,E)
\tkzLabelPoints(D,C)
\tkzLabelAngle(E,C,D){$36^{\circ}$}
\tkzLabelAngle(A,E,D){$54^{\circ}$}
\tkzMarkAngle[arc=l , size=0.5](E,C,D)
\tkzMarkAngle[arc=l , size=0.5 , mark=|](A,E,D)
\end{tikzpicture}
\end{minipage}
\end{exercice}

\begin{exercice}
\begin{minipage}{0.6\linewidth}
\begin{enumerate}
\item Calculer la mesure de $\widehat{MNP}$.
\item Calculer la mesure de $\widehat{ANM}$.
\end{enumerate}
\end{minipage}%
\begin{minipage}{0.4\linewidth}
\begin{tikzpicture}[scale=0.8]
\tkzDefPoints{0/0/N,4/1/P}
\tkzDefTriangle[two angles=45 and 45](N,P)
\tkzGetPoint{A}
\tkzDefTriangle[two angles=120 and 30](N,P)
\tkzGetPoint{M}
\tkzDrawPolygon(A,N,P)
\tkzDrawPolygon(M,N,P)
\tkzLabelPoints[left](N,M)
\tkzLabelPoints[right](A,P)
\tkzMarkAngle[arc=l , size=0.5 , mark=|](M,P,N)
\tkzMarkAngle[arc=l , size=0.5 , mark=||](N,M,P)
\tkzLabelAngle(M,P,N){$30^{\circ}$}
\tkzLabelAngle(N,M,P){$50^{\circ}$}
\tkzMarkSegments[mark=||](A,P A,N)
\tkzMarkRightAngle(N,A,P)
\end{tikzpicture}
\end{minipage}
\end{exercice}

\begin{exercice}
\begin{minipage}{0.6\linewidth}
On considère la figure ci-contre.
\begin{enumerate}
\item Calculer la mesure de $\widehat{BCD}$.
\item Déduire que $ACD$ est rectangle en $C$.
\end{enumerate}
\end{minipage}%
\begin{minipage}{0.4\linewidth}
\begin{tikzpicture}
\tkzDefPoints{0/0/A,3/0/B}
\tkzDefTriangle[equilateral](A,B)\tkzGetPoint{C}
\tkzDefTriangle[two angles=30 and 103](C,B)\tkzGetPoint{D}
\tkzDrawPolygon(A,B,D,C)
\tkzDrawSegment(C,B)
\tkzLabelPoints(A,B,D)
\tkzLabelPoint[left](C){C}
\tkzMarkAngle[arc=l , size=0.5 , mark=|](D,B,C)
\tkzMarkAngle[arc=ll , size=0.5 , mark=x](C,D,B)
\tkzLabelAngle[left](D,B,C){$125^{\circ}$}
\tkzLabelAngle(C,D,B){$25^{\circ}$}
\tkzMarkSegments[mark=||](A,C A,B B,C)
\end{tikzpicture}
\end{minipage}
\end{exercice}

\begin{exercice}
\begin{minipage}{0.6\linewidth}
Les points $H$, $A$ et $P$ sont alignés. Avec les informations codées sur cette figure :
\begin{enumerate}
\item Calculer la mesure de $\widehat{CAH}$.
\item Calculer la mesure de $\widehat{TAP}$.
\item Le triangle $CAT$ est-il rectangle en $A$.
\end{enumerate}
\end{minipage}%
\begin{minipage}{0.4\linewidth}
\begin{tikzpicture}
\tkzDefPoints{0/0/A,3/0/P,-2/0/H}
\tkzDefPoint(-2,2){C}
\tkzDefTriangle[two angles=47 and 90](A,P)\tkzGetPoint{T}
\tkzDrawPolygon(C,H,A)
\tkzDrawPolygon(T,A,P)
\tkzLabelPoints(H,A,P)
\tkzLabelPoints[above](C,T)
\tkzMarkRightAngles(C,H,A T,P,A)
\tkzLabelAngle(H,C,A){$45$}
\tkzLabelAngle(A,T,P){$43$}
\tkzMarkAngle[arc=l , size=0.5 , mark=|](H,C,A)
\tkzMarkAngle[arc=ll , size=0.5 , mark=|](A,T,P)
\end{tikzpicture}
\end{minipage}
\end{exercice}

\begin{exercice}
\begin{minipage}{0.6\linewidth}
On considère la figure ci-contre.
\begin{enumerate}
\item Calculer la mesure de $\widehat{AOC}$.
\item Déduire la mesure de $\widehat{BCO}$ et $COB$.
\end{enumerate}
\end{minipage}%
\begin{minipage}{0.4\linewidth}
\begin{tikzpicture}
\tkzDefPoints{0/0/A,5/1/B}
\tkzDefTriangle[two angles=64 and 26](A,B)
\tkzGetPoint{O}
%\tkzDefCircle[R](O,A)
\tkzInterLC(A,B)(O,A)
\tkzGetSecondPoint{C}
\tkzDrawPolygon(A,B,O)
\tkzDrawSegment(O,C)
\tkzLabelPoints(A,C,B)
\tkzLabelPoint[above](O){O}
\tkzMarkSegments[mark=||](O,A O,C)
\tkzMarkAngle[arc=l , size=0.5 , mark=|](O,B,A)
\tkzLabelAngle(O,B,A){$26^{\circ}$}
\tkzMarkRightAngle(A,O,B)
\tkzFillAngle[opacity=.5,size=0.5](O,C,A)
\tkzFillAngle[opacity=.5,size=0.5](C,A,O)
\end{tikzpicture}
\end{minipage}
\end{exercice}

\begin{exercice}
\begin{minipage}{0.6\linewidth}
On considère la figure ci-contre.
\begin{enumerate}
\item Calculer les mesures des angles de triangle $ABE$.
\item Calculer les mesures des angles de triangle $BED$.
\item Calculer les mesures des angles de triangle $DBC$.
\end{enumerate}
\end{minipage}%
\begin{minipage}{0.4\linewidth}
\begin{tikzpicture}[scale=0.8]
\tkzDefPoints{0/0/A,3/0/B}
\tkzDefTriangle[equilateral](A,B)
\tkzGetPoint{E}
\tkzDefTriangle[two angles=90 and 45](E,B)
\tkzGetPoint{D}
\tkzInterLC(A,B)(D,B)\tkzGetSecondPoint{C}
\tkzDrawPolygon(A,C,D,E)
\tkzDrawSegments(B,E B,D)
\tkzMarkSegments[mark=|](A,B B,E E,A E,D)
\tkzMarkSegments[mark=||](D,B D,C)
\tkzLabelPoints(A,B,C)
\tkzLabelPoints[left](E,D)
\tkzMarkRightAngle(B,E,D)
\end{tikzpicture}
\end{minipage}
\end{exercice}

\begin{exercice}
\begin{minipage}{0.6\linewidth}
On considère la figure ci-contre.
\begin{enumerate}
\item Calculer les mesures des angles du triangle $ACD$.
\item Calculer les mesures des angles du triangle $ABE$.
\item Calculer les mesures des angles du triangle $ACB$.
\end{enumerate}
\end{minipage}\hfill%
\begin{minipage}{0.4\linewidth}
\begin{tikzpicture}[scale=0.8]
\tkzDefPoints{0/0/D,3/0/C,4.5/0/B}
\tkzDefTriangle[two angles=65 and 50](D,C)
\tkzGetPoint{A}
\tkzInterLC(D,B)(B,A)
\tkzGetSecondPoint{E}
\tkzDrawPolygon(A,D,E)
\tkzDrawSegments(A,C A,B)
\tkzLabelPoints(D,C,B,E)
\tkzLabelPoint[above](A){A}
\tkzMarkSegments[mark=||](C,D C,A)
\tkzMarkSegments[mark=|||](B,A B,E)
\tkzLabelAngle(A,C,D){$50^{\circ}$}
\tkzMarkAngle[arc=l , size=0.5 , mark=|](A,C,D)
\tkzLabelAngle[left](A,E,B){$20^{\circ}$}
\tkzMarkAngle[arc=l , size=0.5 , mark=||](A,E,B)
\end{tikzpicture}
\end{minipage}
\end{exercice}




\end{Maquette}
\end{document}