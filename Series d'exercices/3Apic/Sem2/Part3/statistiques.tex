\documentclass[a4paper,12pt]{article}


\usepackage{ProfModels}
\usepackage{diagrammes}

 
\begin{document}

\begin{Maquette}[Fiche]{Theme=Les statistiques,Niveau=3}

\begin{exercice}
Un prélèvement de Institut National de la Météorologie des précipitations quotidiennes (en mm):

\begin{tabular}{c|c|c|c|c|c|c|c|c|c|c}
4 & 6 & 6 & 7 & 4 & 5 & 8 & 8 & 8 & 6 & 5 \\ 
\hline 
8 & 7 & 4 & 4 & 5 & 5 & 6 & 6 & 6 & 7 & 7 \\ 
\hline 
4 & 5 & 4 & 6 & 6 & 7 & 7 & 8 & 8 & 6 & 4 \\ 
\end{tabular} 
\begin{enumerate}
\item Donner le tableau des effectifs et des effectifs cumulés.
\item Déterminer le mode de cette série statistique.
\item Calculer la moyenne de précipitations.
\item Déterminer la médiane de cette série statistique.
\end{enumerate}
\end{exercice}

\begin{exercice}
Le tableau suivant donne une classification de 30 jeunes dans un club sportif selon leurs âges:

\begin{tabular}{|c|c|c|c|c|}
\hline 
Classe: Age & $8\leq <10$ & $10\leq <12$ & $12\leq <14$ & $14\leq <16$ \\ 
\hline 
Effectifs & 9 & 4 & 12 & 5 \\ 
\hline 
\end{tabular}
\begin{enumerate}
\item Déterminer la classe modale de cette série statistique.
\item calculer la moyenne de cette série statistique.
\item Donner le tableau des effectifs cumulés.
\item Déterminer la médiane
\item Calculer le pourcentage des jeunes ayant un âge entre 12 et 16 ans.
\end{enumerate}
\end{exercice}

\begin{exercice}
Le tableau suivant donne le nombres des opérations chirurgicales effectués par équipe médicale des médecins pendant 30 jours.

\begin{tabular}{|c|c|c|c|c|c|c|}
\hline 
Nombre des opérations & 0 & 1 & 2 & 3 & 4 & 5 \\ 
\hline 
Nombred de jours & 5 & 6 & 10 & 8 & 1 & 0 \\ 
\hline 
\end{tabular} 
\begin{enumerate}
\item Déterminer le mode de cette série statistique.
`\item Déterminer l'effectif cumulé 
\item Calculer la moyenne de cette série statistique.
\item Déterminer la médiane de cette série statistique.
\item Calculer le pourcentage des jours où aucune opération chirurgicale n'a été effectuée.
\end{enumerate}
\end{exercice}

\begin{exercice}
Le tableau suivant donne le nombre des enfants de 25 familles.Recopier puis compléter le tableau:
\begin{tabular}{|c|c|c|c|c|c|}
\hline 
Nombre des enfants & 1 & 2 & 3 & 4 & 5 \\ 
\hline 
Nombre des familles & 5 &  & 7 & 3 & 4 \\ 
\hline 
Effectif cumulé &  & 11 &  &  &  \\ 
\hline 
\end{tabular}
\begin{enumerate}
\item Déterminer le mode de cette série statistique.
\item Déterminer la médiane de cette série statistique.
\item Calculer la moyenne des enfants de ces familles.
\item Calculer le pourcentage des familles ayant plus de trois enfants.
\end{enumerate}
\end{exercice}

\begin{exercice}
Le graphique suivant présent les notes des élèves d'une classe dans un contrôle surveillé de mathématiques.

\Histogramme[Largeur=12,Hauteur=6,%
ListeCouleurs={orange,gray,blue,pink,red,black,gray,green,orange,gray,blue},%
DebutOx=0,FinOx=18,GradX={1,2,...,18},GradY={0,1,...,8},%
AffEffectifs=false,LabelX={},LabelY={}]%
{1.8/2.2/3 3.8/4.2/2 5.8/6.2/4 7.8/8.2/5 8.8/9.2/6 9.8/10.2/8 10.8/11.2/3 12.8/13.2/4 13.8/14.2/6 15.8/16.2/7 17.8/18.2/3}
\begin{enumerate}
\item Déterminer l'effectif total.
\item Déterminer le mode de cette série statistique.
\item Calculer le pourcentage des élèves ayant une note supérieur à 12.
\item Calculer la note moyenne du classe.
\item Déterminer la note médiane de la série statistique.
\end{enumerate}
\end{exercice}

\begin{exercice}%6
Le tableau suivant donne les distances en Km parcourus par 80 élèves de maisons vers l'école:
\begin{tabular}{|c|c|c|c|c|c|}
\hline 
Classe: Distance & $0\leq d<2$ & $2\leq d<4$ & $4\leq d<6$ & $6\leq d<8$ & $8\leq d<10$ \\ 
\hline 
Effectif & 15 & 5 & x & 20 & 10 \\ 
\hline 
\end{tabular} 
\begin{enumerate}
\item Montrer que $x=30$ puis déterminer les effectifs cumulés.
\item Déterminer la classe modale de cette série statistique.
\item Déterminer la médiane de cette série statistique.
\item Calcule la moyenne des distances parcourus par ces élèves.
\item Calcule le pourcentage des élèves qui parcourent des distances supérieurs ou égales à 4 Km.
\end{enumerate}
\end{exercice}

\begin{exercice}
Le tableau suivant donne la répartition des notes d'une classe à un contrôle.

\begin{tabular}{|c|c|c|c|c|c|c|c|c|c|}
\hline 
Note & 6 & 8 & 9 & 10 & 11 & 12 & 13 & 16 & 19 \\ 
\hline 
Effectif & 2 & $\cdots$ & 4 & 5 & $\cdots$ & 6 & 7 & 2 & 1 \\ 
\hline 
Effectif cumulé & $\cdots$ & 4 & $\cdots$ & $\cdots$ & $\cdots$ & $\cdots$ & $\cdots$ & $\cdots$ & 30 \\ 
\hline 
\end{tabular} 
\begin{enumerate}
\item Quel est l'effectif total de cette série?
\item Recopier et compléter le tableau.
\item Calculer la fréquence de 12.
\item Calculer le pourcentage des élèves qui ont obtenu une note supérieur à 12.
\item Calculer la note moyenne de la classe.
\item Déterminer la médiane de cette série
\item Déterminer le mode de cette série statistique
\end{enumerate}
\end{exercice}

\begin{exercice}
Dans le tableau ci-dessous est donnée la masse en Kg des élèves d'un collège.

\begin{tabular}{|c|c|c|c|c|}
\hline 
Masse en Kg & $20<m\leq 26$ & $26<m\leq 32$  & $32<m\leq 38$  & $38<m\leq 50$  \\ 
\hline 
Effectif & 20 & 45 & 51 & 28 \\ 
\hline 
Centre de classe &  &  &  &  \\ 
\hline 
Effectif cumulé &  &  &  &  \\ 
\hline 
Fréquence &  &  &  &  \\ 
\hline 
\end{tabular} 
\begin{enumerate}
\item Compléter le tableau.
\item Calculer l'effectif total.
\item Calculer le poids moyen des élèves.
\item Détermine le mode de cette série statistique.
\item A quelle classe appartient la médiane de cette série?
\end{enumerate}

\end{exercice}



\end{Maquette}
\end{document}