\documentclass[a4paper,12pt]{article}


\usepackage{ProfModels}
%\usepackage{diagrammes}

 
\begin{document}

\begin{Maquette}[Fiche]{Theme=Vecteurs et Translation,Niveau=3}

\begin{exercice}
\begin{enumerate}
\item Exprimer le plus simple possible les expressions suivantes :
\end{enumerate}
\[
\vv{AC}-\vv{BC} \quad ;; \quad
\vv{DA}+\vv{AB}-\vv{DB}\quad ;; \quad
\vv{MO}+\vv{AM}+\vv{OA}\quad ;; \quad
-\vv{AO}+\vv{CB}+\vv{BO}
\]
\[
\vv{OA}+\vv{DO}+\vv{AB}+\vv{CD}+\vv{BC}\quad ;; \quad
\vv{AD}-\vv{FD}+\vv{ED}-\vv{AF}+\vv{BE}+\vv{AB}\quad ;; \quad
3(\vv{AB}-2\vv{DA})-2(\vv{AB}-3\vv{DA})
\]
\end{exercice}

\begin{exercice}
\begin{enumerate}
\item Compléter les égalités suivantes selon la relation de Chasles.
\end{enumerate}
\[
\vv{AB}+\vv{BD}=\cdotsx{3}\quad ;; \quad
\vv{BO}+\vv{AB}=\cdotsx{3}\quad ;; \quad
\vv{ED}+\vv{DE}=\cdotsx{3}\quad ;; \quad
\vv{\cdotsx{2} E}-\vv{\cdotsx{2} E}=\vv{AC}\quad ;; \quad
\]
\[
\vv{CA}+\vv{BC}+\vv{AF}=\cdotsx{3}\quad ;; \quad
\vv{A\cdotsx{2}}+\vv{B\cdotsx{2}}=\vv{AD}\quad ;; \quad
\vv{O\cdotsx{2}}+\vv{M\cdotsx{2}}=\vv{OP}
\]
\[
\vv{A\cdotsx{2}}+\vv{D\cdotsx{2}}+\vv{M\cdotsx{2}}=\vv{AG}\quad ;; \quad
\vv{FH}+\cdotsx{3} +\vv{HE}=\vv{FL}\quad ;; \quad
\vv{MN}-\vv{PN}=\cdotsx{3}
\]
\end{exercice}

\begin{exercice}
On considère le triangle $ABC$.
\begin{enumerate}
\item Construire les points $E$ , $F$ , $G$  et $H$ tel que :
\[
\vv{BE}=2\vv{BC}\quad ;; \quad
\vv{AF}=-\dfrac{1}{2}\vv{AC}\quad ;; \quad
\vv{GC}=\dfrac{3}{2}\vv{CB}\quad ;; \quad
\vv{BH}=\vv{AC}
\]
\item Construire les points $K$ , $L$ , $M$  et $N$ tel que :
\[
\vv{AK}=\vv{AB}+2\vv{AC}\quad ;; \quad
\vv{AL}=\vv{AB}+\vv{AC}\quad ;; \quad
\vv{AM}=-\dfrac{3}{2}\vv{AB}\quad ;; \quad
\vv{AN}=\dfrac{1}{2}\vv{AB}+\dfrac{1}{2}\vv{AC}
\]
\end{enumerate}
\end{exercice}

\begin{exercice}
Soit $ABC$ un triangle.
\begin{enumerate}
\item Construire le point $M$ tel que $\vv{AM}=\vv{BC}$.
\item Construire le point $N$ tel que $\vv{BN}=-\vv{AC}$.
\item Construire le point $P$ tel que $\vv{CP}=\vv{AB}$.
\end{enumerate}
\end{exercice}

\begin{exercice}
Soit $ABCD$ un rectangle.
\begin{enumerate}
\item Construire le point $E$ image du point $C$ par la translation de vecteur $\vv{AD}$.
\item Construire le point $F$ image du point $A$ par la translation de vecteur $\vv{BC}$.
\end{enumerate}
\end{exercice}

\begin{exercice}
$ABCD$ un parallélogramme de centre $O$.
\begin{enumerate}
\item Construire la figure.
\item Construire le point $M$ image  du point $D$ par la translation qui transforme $A$ en $B$.
\item Construire le point $N$ image du point $O$ par la translation qui transforme $D$ en $C$.
\end{enumerate}
\end{exercice}

\begin{exercice}
Soit $ABC$ un triangle.
\begin{enumerate}
\item Construire les points $D$ et $E$ tel que : $\vv{BD}=\dfrac{1}{3}\vv{BC}$ , $\vv{CE}=2\vv{AB}$.
\item Montrer que : $\vv{AD}=\dfrac{2}{3}\vv{AB}+\dfrac{1}{3}\vv{AC}$ , $\vv{AE}=2\vv{AB}+\vv{AC}$
\item En déduire que les points $A$ , $E$ et $D$ sont alignés.
\end{enumerate}
\end{exercice}

\end{Maquette}
\end{document}