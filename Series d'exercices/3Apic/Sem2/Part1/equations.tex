\documentclass[a4paper,12pt]{article}


\usepackage{TestProfModels}


 
\begin{document}

\begin{Maquette}[Fiche]{Theme=Les équations et les inéquations,Niveau=3}

\begin{exercice}
Résoudre les équations suivantes.
\begin{tasks}(3)
\task $\sqrt{3}x+4=-12$
\task $-5x+\sqrt{2}=0$
\task $-4x-12=-20$
\task $7x+4(5-4x)=3x$
\task* $4x-7-2(4x+1)=-2(3+2x)-3$
\task $3x+\sqrt{2}(1+2x)=-1$
\task* $8x-(11-5x)=-3(x+5)+7$
\task $\dfrac{2x-1}{3}+\dfrac{6x}{5}=\dfrac{9x}{2}$
\task $\dfrac{x+7}{\sqrt{3}}-5x=\sqrt{2}$
\task $\dfrac{7x-8}{7}=\dfrac{2x+4}{4}$
\end{tasks}
\end{exercice}

\begin{exercice}
Résoudre les équations
\begin{tasks}(2)
\task $2x(-x+\sqrt{2})(x\sqrt{3}-1)=0$
\task $\dfrac{2x+3}{4}(-3x+5)=0$
\task $(\sqrt{2}x-1)(\sqrt{3}x-4)=0$
\task $(2x+3)(x-1)+2x(x-1)=0$
\task $(x-2)^{2}(2x+3)=0$
\task $25x^{2}-30x+9=0$
\task $4x^{2}-9+(2x+3)(2x-5)=0$
\task $x^{2}-25=0$
\task $(x+\sqrt{3})^{2}=(2-3x)(x+\sqrt{3})$
\task $25x^{2}+1=10$
\task $(7x-2)^{2}=16$
\end{tasks}
\end{exercice}

\begin{exercice}
Résoudre les inéquations suivants et représenter les solutions sur une droite graduée.
\begin{tasks}(3)
\task $3x+2\leq 0$
\task $-4x-7\geq 2$
\task $-7x+6\leq 9+\sqrt{3}x$
\task $\sqrt{5}x-7\geq 3+3x$
\task $-x+5\leq 3(-x+8)-7+x$
\task $\dfrac{3+2x}{6}-\dfrac{3+x}{4}>0$
\task*(2) $\dfrac{2x+7}{4}< \dfrac{-5x+4}{5}-\dfrac{2x-7}{2}$
\task $\dfrac{-9x+4}{-2}>\dfrac{x}{-7}$
\end{tasks}
\end{exercice}

\begin{exercice}
L'age de Luc est le double de l'age de sa
sœur Sylvie. L'an prochain, ils auront à
eux deux 23 ans. Calculer les ages
actuels de Luc et de Sylvie. 
\end{exercice}

\begin{exercice}
Un randonneur parcours 100 km en 3
jours.
Le deuxième jour il parcourt 10 km de
moins que le premier jour
Le troisième jour il parcourt le double de
ce qu'il a parcouru le deuxième jours.
Calculer les distances parcourues le
premier le deuxième et le troisième jours.
\end{exercice}

\begin{exercice}
Des amis veulent louer un voilier.
S'il participent avec 170 Dh chacun il y aura
330 Dh en trop. ( 1er cas)
S'il participent avec 130 Dh chacun il
manquera 150 Dh. ( 2ème cas)
On cherche le nombre d'amis et le prix
de location du voilier.
\end{exercice}


\begin{exercice}
Un livre m'a coûté 120 Dh ce qui représente
les trois cinquièmes de mes économies.
Quel était le montant de mes économies . 
\end{exercice}

\begin{exercice}
Luc a dépensé un tiers de ses économies
pour l'achat de livres. Il a en plus dépensé
deux cinquièmes de ses économies pour
l'achat d'un disque. Après ces achats il
lui reste encore 160 Dh.
Quel était le montant de ses économies?
\end{exercice}

\begin{exercice}
Sophie a dépensé les deux septièmes de
ses économies pour l'achat de livres. Elle
a en plus dépensé les deux tiers de ce
qu'il lui reste pour l'achat d'une robe.
Après ces achats il lui reste encore 350 Dh.
Quel était le montant de ses économies. 
\end{exercice}

\begin{exercice}
Un particulier a des marchandises à transporter.Un premier transporteur lui demande 650 Dh au départ et 3.20 Dh par Kilomètre. Un second transporteur lui demande 850 Dh au départ et 1.90 Dh par kilomètre. Pour quelles distances à parcourir est-il plus avantageux de s'adresser au second transporteur?
\end{exercice}





\end{Maquette}

\end{document}