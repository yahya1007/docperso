\documentclass[a4paper,12pt]{article}


\usepackage{ProfModels}
\usepackage{diagrammes}

 
\begin{document}

\begin{Maquette}[Fiche]{Theme=Repère dans le plan.,Niveau=3}

\begin{exercice}
(O;I;J) est un repère orthonormé dans le plan.
\begin{enumerate}
\item Construis les points : 
\[
A(2,-4)\quad ;;\quad B(3,4)\quad ;;\quad C(-1,3)\quad ;;\quad D(-2,-2)\quad ;;\quad E(0,-3)\quad ;;\quad F(2,0)
\]
\item Détermine les coordonnées de $M$ milieu de $[AC]$.
\item Détermine les coordonnées de $N$ tel que $F$ milieu de $[DN]$
\item Montre que : $E$ milieu de $[AD]$.
\item Montre que : $\vv{AB}(1,8)$
\item Calcule les distances : $AB$ et $DC$.
\item Détermine les coordonnées de $K$ image de $A$ par la translation qui transforme $B$ en $C$.
\item Détermine les coordonnées de : $\vv{BC}+\vv{AD}$ et $3\vv{EC}$
\item Détermine les coordonnées de $R$ tel que : $\vv{AR}=2\vv{AF}-\vv{BE}$
\end{enumerate}
\end{exercice}

\begin{exercice}
(O;I;J) est un repère orthonormé dans le plan.
tel que : $A(1,-3)$ ; $ B(3,7) $ et $  C(-3,1)$
\begin{enumerate}
\item Montrer que $ABC$ est un triangle rectangle.
\item Calculer $S$ la surface du triangle $ABC$.
\end{enumerate}
\end{exercice}

\begin{exercice}
(O;I;J) est un repère orthonormé dans le plan. On considère les points :
\[
A(2,5)\quad ;;\quad B(-4,1)\quad ;;\quad C(-2,-1)\quad ;;\quad D(4,3)\quad ;;\quad E(-1,3)
\]
\begin{enumerate}
\item Montrer que $ABCD$ est un parallélogramme.
\item Détermine les coordonnées de $M$ centre de $ABCD$
\item Montre que le point $N(3,-3)$ appartient au médiatrice du segment $[AB]$.
\item Montre que $F(6,1)$ est l'image de $C$ par la translation qui transforme $B$ en $D$.
\item Prouve que les points $A$ , $B$ et $C$  sont alignés.
\end{enumerate}
\end{exercice}

\begin{exercice}
le plan est muni du repère orthonormé $\oij$ ci-dessous.\newline
\begin{minipage}{.5\linewidth}
\begin{AffRepere}[-4][4][-4][3]
\coordpoints
{1}{2}{A}
{-1}{-3}{B}
{4}{-1}{C}
{0}{3}{D}
{-3}{0}{E}
{2}{-4}{K}
\end{AffRepere}
\end{minipage}
\begin{minipage}{.5\linewidth}
\begin{enumerate}
\item Donner les coordonnées des points A,B,C,D,et E.
\item Placer dans ce repère les points :
\[
F\lrp{3;1} ;; G\lrp{-2;4} ;; H\lrp{5;0} ;; L\lrp{0;2}
\]
\item Calculer les coordonnées des vecteurs :
\[
\vv{AB} ;; \vv{AC} ;; \vv{BC} ;; \vv{CD}
\]
\item Déterminer les coordonnées des points P,Q et S tels que :
\[
\vv{OP}\lrp{4;2} ;; \vv{PQ}\lrp{-4;4} ;; \vv{QS}\lrp{-2;-2}
\]
\end{enumerate}
\end{minipage}
\end{exercice}

\begin{exercice}
Le plan est muni d'un repère orthonormé $\oij$, tels que $\vv{AB}\lrp{a;5}$,$\vv{DC}\lrp{-3;b-7}$,$\vv{EF}\lrp{5+x;3}$ et $\vv{MN}\lrp{2;3-y}$.
\begin{enumerate}
\item Calculer a et b pour que $\vv{AB}=\vv{DC}$.
\item Calculer x et y pour que $\vv{EF}=\vv{MN}$.
\end{enumerate}
\end{exercice}

\begin{exercice}
On donne les points $E\lrp{4;3}$ ; $F\lrp{-2;2}$ ; $G\lrp{-5;1}$ et $H\lrp{10;4}$ 
\begin{enumerate}
\item Calculer les coordonnées du point $ A $ tel que  $ \vv{EA} = \vv{GF} $ 
\item Calculer les coordonnées du point $ B $ pour que $ EFGB $ soit un parallélogramme. 
\item Calculer les coordonnées du point $ C $ image de $ G $ par la translation de vecteur $ \vv{BF} $  
\item Montrer que $ AHBF $ est parallèlogramme .
\end{enumerate}
\end{exercice}

\begin{exercice}
Le plan est muni d'un repère orthonormé $ \oij$.
\begin{enumerate} 
\item Dans chaque cas,calculer les coordonnées de $M$ le milieu de $[AB]$:
\begin{multicols*}{2} 
\begin{enumerate} 
\item $A\lrp{2;3}$ et $B\lrp{6,-1}$
\item $A\lrp{2 ; 3}$ et $B\lrp{6 ; -1}$ ; 
\item $A\lrp{12 ; 1}$ et $B\lrp{-2 ; 5} $ ; 
\item $A\lrp{-3 ; 2} $ et $B \lrp{3 ; -2} $ ; 
\item $A\lrp{\dfrac{3}{2} ; 0}$ et $B\lrp{-1 ; 2} $ . 
\end{enumerate}
\end{multicols*} 
\item On donne les points $ M\lrp{-1;-2} $  et  $ N \lrp{3;-2} $ 
 \begin{enumerate} 
\item Montrer que le point $ K \lrp{1 ; -2 } $ le milieu de $ \left[ MN \right] $.
\item Calculer les coordonnées de $L$ symétrique $M$ par rapport à $N$ 
\item Calculer les coordonnées de $P$ symétrique $L$ par rapport à $I$. 
\end{enumerate}
\end{enumerate}
\end{exercice}

%\begin{exercice}
%\begin{enumerate}
%\item On considère les points  : $ M\lrp{ -5 ; 1 } $  ; $ N\lrp{  -3 ; 4 } $ ;  $ P\lrp{ 3 ; 2 } $  et $ Q\lrp{  1 ; -1 } $ .
%\begin{enumerate}
%\item  Montrer que $ MNPQ $   est un parallélogramme.
%\item Déterminer les coordonnées de son centre $ K $ .
%\end{enumerate}
%\end{enumerate}
%\end{exercice}

%\begin{exercice}
%Le plan est muni d’un repère orthonormé $ \oij $ d’unité $1cm$ . \\
%On considère trois points $ A(-5;2)$ , $ B(4;-1)$ et $ C(-2;5)$ .
%\begin{enumerate}
%\item Calculer les distances $AB$, $AC$ et $BC$.
%\item En déduire la nature du triangle $ABC$. Justifier.
%\end{enumerate}
%\end{exercice}

%\begin{exercice}
%Le plan est muni d’un repère orthonormé $ \oij$ .  On considère les points $ A\lrp{-3;0} $ , $ B\lrp{2;1} $ , $ C\lrp{4;3} $  et $ D\lrp{-1; 2} $ .
%\begin{enumerate}
%\item Placer les points $ A $ , $ B $  , $ C $ et $ D $.
%\item Démontrer que les segments $ [AC] $ et $ [BD] $  ont le même milieu  .
%\item Montrer que le triangle $ OBD $ est rectangle est isocèle.
% \item On considère le point $ E $ du plan tel que $ BODE $  soit un parallélogramme.
% \item Déterminer les coordonnées de $ E $.
%\item Calculer $ AE $ 
%\end{enumerate}
%\end{exercice}

%\begin{exercice}
%Le plan est muni d'un repère orthonormal $\oij$.
%Soient les points $ A \lrp{-2 ; 1} $ , $ B\lrp{4 ; 3} $ et $ C\lrp{-1 ;-2} $
%\begin{enumerate}
%\item Place les points $ A $ , $ B $ et $ C $ .
%\item Déterminer les coordonnées de $  \vv{AB} $  puis calculer la distance $ AB $
%\item Calculer les coordonnées de $ K $ milieu de $ [BC]$.
%\item Calcule les coordonnées de $ D $ symétrique de $ A $ par rapport à $ K $.
%\item Montrer que le quadrilatère $ ABDC $ est un rectangle.
%\item Montre que les points $ A $ , $ B $ , $ C $ et $ D $ appartiennent à un même cercle $(C)$ dont on donnera le centre et le rayon $ r $ .
%\item Calculer les coordonnées de $ E $ , image de $ A $  par la translation de vecteur $ \vv{BC} $ . 
%\end{enumerate}
%\end{exercice}

\end{Maquette}
\end{document}