\documentclass[a4paper,12pt]{article}


\usepackage{ProfModels}

 
\begin{document}

\begin{Maquette}[Fiche]{Theme=Les systèmes de deux équations,Niveau=3}

\begin{exercice}
\begin{enumerate}
\item Résoudre par la méthode de substitution les systèmes suivants:

$(S_1)\systeme{3x+y=9,-2x-y=-11}$
$(S_2)\systeme{x-y=9,5x+y=9}$
$(S_3)\systeme{3x+2y=1,2x-y=-1}$
$(S_4)\systeme{\dfrac{-3}{4}x+y=9,-2x-\dfrac{5}{2}y=-11}$

\item  Résoudre par la méthode de combinaison linéaire les systèmes suivants:

$(S_5)\systeme{3x-2y=1,4x-y=-1}$
$(S_6)\systeme{x+2y=1,x-4y=-4}$
$(S_7)\systeme{4x+3y=0,2x-3y=-1}$
$(S_8)\systeme{x-y=-7,2x+y=-2}$
\end{enumerate}
\end{exercice}

\begin{exercice}
Résoudre  les systèmes suivants:
\begin{tasks}(3)
\task $\systeme{3x-y=-7,2x+5y=-2}$
\task $\systeme{\sqrt{2}x-y=3,x+\sqrt{3}y=-2}$
\task $\systeme{\cfrac{3}{4}x-\cfrac{2}{3}y=\cfrac{-3}{8},\cfrac{5}{3}x+y=-\cfrac{8}{9}}$
\task $\systeme{\sqrt{3}x-\sqrt{2}y=\sqrt{8},2\sqrt{2}x+\sqrt{3}y=-2}$
\task $\systeme{\sqrt{13}x-\sqrt{2}y=\sqrt{5},\sqrt{3}x+\sqrt{5}y=-2\sqrt{3}}$
\task $\systeme{2\sqrt{3}x-\sqrt{3}y=12,3\sqrt{6}x-3\sqrt{2}y=\sqrt{12}}$
\end{tasks}
\end{exercice}

\begin{exercice}
Résoudre , en utilisant la méthode graphique, chacun des systèmes ci-dessous :
\begin{tasks}(3)
\task[$S_1$] $\systeme{2x+y=-1,-x-y=4}$
\task[$S_2$] $\systeme{4x-2y=6,2x-y=0}$
\task[$S_3$] $\systeme{3x+y=-2,x-2y=4}$
\end{tasks}
\end{exercice}

\begin{exercice}
\begin{enumerate}
\item Résoudre le système: $\systeme{x+y=110,2x+5y=340}$
\item Un théâtre propose deux types de billets les uns à 100dh et les autres à 250dh. On sait que 110 spectateurs ont assisté à cette représentation théâtrale et que la recette totale s'élève à 17000dh.\newline
Calcule le nombre de billets vendus pour chaque type.
\end{enumerate}
\end{exercice}

\begin{exercice}
Yassir a 10 pièces dans son porte-monnaie. Ce sont uniquement des pièces de 5dh et 10dh . Le montant contenu dans le porte monnaie est de 75dh . Combien a-t-il de pièces de chaque sorte ?
\end{exercice}

\begin{exercice}
Dans le panier de Fatima , il y a 5kg de pommes et 2kg de carottes.
Dans le panier de Youssef, il y a 3kg de pommes et 7kg de carottes.
Fatima a payé 33dh alors que Youssef a payé 43dh .
Quel est le prix d’un kg de pommes et d’un kg de carottes ?
\end{exercice}
\begin{exercice}
Dans une ferme, il y a des lapins et des poules . On compte 120 têtes et 298 pattes.
Combien y a-t-il de lapins et de poules dans la ferme ?
\end{exercice}

\begin{exercice}
Pour classer des photos, un magasin propose deux types de rangement : des albums ou des boîtes. Sara achète 6 boîtes et 5 albums et paie 57dh . Hicham achète 3 boîtes et 7 albums et paie 55,50dh.
Quel est le prix d'une boîte ? Quel est le prix d'un album ?
\end{exercice}

\begin{exercice}
Dans une classe... Au début, il y a deux fois plus de garçons que de filles. Six garçons quittent la salle et six filles arrivent ; il y a alors deux fois plus de filles que de garçons.
Combien de garçons et de filles y avait-t-il au début ?
\end{exercice}

\begin{exercice}
Une usine fabrique deux sortes d'objets en bois : A et B.
L'objet A nécessite 1,2 kg de bois et 2h de fabrication. L'objet B nécessite 2 kg de bois et 3h de fabrication.
Pour produire ces objets, on a utilisé 54 kg de bois et la fabrication a duré 84 heures.
Combien d'objets A a-t-on fabriqués ? d'objets B ?
\end{exercice}

\begin{exercice}
Quand le père avait l'âge du fils, le fils avait 10 ans.
Quand le fils aura l'âge du père, le père aura 70 ans.
Quels sont leurs âges respectifs ?
\end{exercice}

\begin{exercice}
J'ai deux fois l'âge que vous aviez quand j'avais l'âge que vous avez .
Et quand vous aurez mon âge , nous aurons à nous deux 63 ans.
Quel est mon âge ?
\end{exercice}


\end{Maquette}
\end{document}