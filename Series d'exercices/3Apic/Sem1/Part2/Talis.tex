\documentclass[a4paper,12pt]{article}

\usepackage{ProfModels}

 
\begin{document}

\begin{Maquette}[Fiche]{Theme=Théorème Thalès,Niveau=3}

\begin{exercice}
$ABC$ un triangle tel que : $AB=4$ , $AC=6$ et $BC=5$. Soit $M$ un point de $[AB]$ avec $AM=1$.

La droite passante de $M$ qui est parallèle de $(BC)$ coupe $(AC)$ en $N$.
\begin{enumerate}
\begin{minipage}{.5\linewidth}
\item Faire une figure convenable.
\item Calculer les distances $AN$ et $MN$.
\end{minipage}%
\begin{minipage}{.5\linewidth}
\item Soit $F$ un point de $[BC]$ tel que $BF=3.75$
\begin{enumerate}
\item Construire le point $F$.
\item Montrer que : $(MF)//(AC)$.
\end{enumerate}
\end{minipage}
\end{enumerate}
\end{exercice}

\begin{exercice}
$ABCD$ un parallélogramme tel que : $AB=8$ et $AD=4.5$ et $E$ un point de $[DA)$ qui n'appartient pas à $[DA]$ avec $AE=1.5$; la droite $(EC)$ coupe $(AB)$ en $M$.
\begin{enumerate}
\item Faire un schéma.
\item Calculer $MA$.
\item Soit $F$ un point de $[DC]$ tel que : $DF=\dfrac{3}{4}DC$
\begin{enumerate}
\item Compléter la figure.
\item Prouver que $(EC)//(AF)$.
\end{enumerate}
\end{enumerate}
\end{exercice}

\begin{exercice}
\begin{minipage}{.6\linewidth}
On considère la figure ci-contre tel que $AB=16$ , $AC=8$ , $BM=4$ ,$AE=4$ et $AF=2$
\begin{enumerate}
\item Montrer que $AN=6$.
\item Calculer $NC$.
\item Prouver que $(BC)//(EF)$.
\end{enumerate}
\end{minipage}%
\begin{minipage}{.4\linewidth}
\begin{tikzpicture}[scale=0.8]
\tkzDefPoint(0,0){A}
\tkzDefPoint(1,-3){C}
\tkzDefPoint(-3,-2){B}
\tkzDefPointOnLine[pos=0.75](A,B)\tkzGetPoint{M}
\tkzDefPointOnLine[pos=0.75](A,C)\tkzGetPoint{N}
\tkzDefPointOnLine[pos=-0.5](A,B)\tkzGetPoint{E}
\tkzDefPointOnLine[pos=-0.5](A,C)\tkzGetPoint{F}
\tkzLabelPoint[left=3pt](A){A}
\tkzLabelPoint[above left](B){B}
\tkzLabelPoint[above right](C){C}
\tkzLabelPoint[above left=5pt](M){M}
\tkzLabelPoint[above right](N){N}
\tkzLabelPoint[below right](E){E}
\tkzLabelPoint[below left](F){F}
\tkzDrawLines(E,B F,C M,N B,C E,F)
\end{tikzpicture}
\end{minipage}
\end{exercice}

\begin{exercice}
\begin{minipage}{.65\linewidth}
On considère la figure ci-contre tel que : $ABCD$ un trapèze de bases $[AB]$ et $[CD]$, $AB=3$ et $DC=8$ et $AE=a$ et $AD=6$.
\begin{enumerate}
\item Compare les rapports $\dfrac{OB}{OD}$ et $\dfrac{AB}{DC}$
\item Déduire $\dfrac{OB}{OD}$
\item Calculer la valeur de $a$.
\item Prouve que $OB\times EC=OD\times EB$
\end{enumerate}
\end{minipage}%
\begin{minipage}{.35\linewidth}
\begin{tikzpicture}
\tkzDefPoints{0/0/D,5/0/C,3/4/E}
\tkzDefPointOnLine[pos=0.4](E,D)\tkzGetPoint{A}
\tkzDefPointOnLine[pos=0.4](E,C)\tkzGetPoint{B}
\tkzDrawPolygon(E,D,C)
\tkzDrawSegments(A,B B,D A,C)
\tkzInterLL(A,C)(B,D)\tkzGetPoint{O}
\tkzLabelPoints[left](E,A,D)
\tkzLabelPoints[right](B,C)
\tkzLabelPoint[below](O){O}
\tkzLabelSegment[left](E,A){a}
\tkzLabelSegment[left](A,D){6}
\tkzLabelSegment(D,C){8}
\tkzLabelSegment[above](A,B){3}
\end{tikzpicture}
\end{minipage}
\end{exercice}

\begin{exercice}
\begin{minipage}{.6\linewidth}
On considère la figure ci-contre tel que $(AB)//(EF)$ et $AB=24$ et $OB=21$ et $OE=12$ et $OF=14$ et $BC=7$ et $BD=8$.
\begin{enumerate}
\item Calculer $OA$ et $EF$.
\item Montrer que $(OA)//(DC)$.
\item Calculer $DC$.
\end{enumerate}
\end{minipage}%
\begin{minipage}{.4\linewidth}
\begin{tikzpicture}
\tkzDefPoints{0/0/A,5/1/B,2/3/O}
\tkzDefPointOnLine[pos=1.2](B,O)\tkzGetPoint{F}
\tkzDefPointOnLine[pos=1.2](A,O)\tkzGetPoint{E}
\tkzDefPointOnLine[pos=0.4](B,O)\tkzGetPoint{C}
\tkzDefPointOnLine[pos=0.4](B,A)\tkzGetPoint{D}
\tkzDrawSegments(A,B B,F F,E E,A)
\tkzDrawPoints(C,D)
\tkzDrawLine[dashed,add=0.2 and 0.3](C,D)

\tkzLabelPoints(A,D,B)
\tkzLabelPoints[right](O,C)
\tkzLabelPoints[above](F,E)
\end{tikzpicture}
\end{minipage}
\end{exercice}

\begin{exercice}
\begin{minipage}{.6\linewidth}
On considère la figure ci-contre tel que $AB=6$ et $BC=8$ et $BM=3$
\begin{enumerate}
\item Calculer $AC$ et $AM$.
\item Calculer $PC$ et $MP$.
\item Montrer que $(AM)$ est la bissectrice d'angle $\widehat{BAC}$.
\end{enumerate}
\end{minipage}%
\begin{minipage}{.4\linewidth}
\begin{tikzpicture}
\tkzDefPoints{0/0/C,4/0/P,0/4/B,-2/4/A}
\tkzInterLL(A,P)(B,C)\tkzGetPoint{M}
\tkzDrawSegments(C,P P,A A,B B,C)
\tkzMarkRightAngles(A,B,C P,C,B)
\tkzLabelPoints(C,P)
\tkzLabelPoints[above](A,B)
\tkzLabelPoint[right](M){M}
\end{tikzpicture}
\end{minipage}
\end{exercice}

\begin{exercice}
$ABC$ est un triangle rectangle en $A$ tel que $AC=4$ et  $AB=8$.
\begin{enumerate}
\item Montrer que : $BC=4\sqrt{5}$
\end{enumerate}
Soit $E$ un point de $[AB]$ tel que $BE=6$, la droite passante par $E$ et parallèle à $(AC)$ coupe $[CB]$ en $F$.
\begin{enumerate}[start=2]
\item Calculer $EF$ et $BF$
\end{enumerate}
Soit  $G$ un point de $(EF)$ tel que $EG=1$ et $G\notin [EF]$
\begin{enumerate}[start=3]
\item Montrer que $(AG)//(BF)$.
\end{enumerate}
\end{exercice}

\begin{exercice}
Soit $(C)$ un cercle de centre $O$ et de rayon $R=6$, $[AB]$ un diamètre de $(C)$, $N$ un point de de $[OB]$ tel que $BN=4$ et $M$ un point tel que $BM=3.2$ et $MN=2.4$.
\begin{enumerate}
\item Construire une figure convenable.
\item Montrer que $BMN$ est un triangle rectangle.
\end{enumerate}
La droite $(BM)$ coupe le cercle $(C)$ en $P$.
\begin{enumerate}[start=3]
\item Montrer que $(MN)//(AP)$.
\item Calculer $AP$ puis $BP$
\item Soit $E$ le milieu de $[BN]$.Montrer que $(ME)//(PO)$
\end{enumerate}
La droite $(PO)$ coupe le cercle $(C)$ en $K$ et $(PN)$ coupe $(BK)$ en $I$.
\begin{enumerate}
\item Montrer que $I$ est le milieu de $[BK]$
\end{enumerate}
\end{exercice}






















\end{Maquette}





\end{document}