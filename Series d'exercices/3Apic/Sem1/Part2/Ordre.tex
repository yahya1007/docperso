\documentclass[a4paper,12pt]{article}

\usepackage{ProfModels}

 
\begin{document}

\begin{Maquette}[Fiche]{Theme=L'ordre et les opérations ,Niveau=3}

\begin{exercice}
\begin{enumerate}
\item Comparer ce qui suit :
$$\dfrac{-7}{18} et \dfrac{-5}{9};;;-\sqrt{2} et -\sqrt{2}+\dfrac{1}{2};;;; \dfrac{3}{7}+3^{2023} et \dfrac{12}{5}+3^{2023}  $$
$$-\sqrt{5}\times\dfrac{11}{2}et -\sqrt{5}\times \dfrac{13}{7};;;;2\sqrt{7}\times \dfrac{18}{5}et 2\sqrt{7}\times \dfrac{11}{25} $$
\item Soit $x$ et $y$ deux nombres réels tels que $x>0$ et $y<0$, comparer :
$$y-x et x+y ;;;;;; 4y+x et 3y+x $$
\item Comparer les nombres $a$ et $b$ tel que : $a=\sqrt{12}+\sqrt{27}$ et $\sqrt{48}$
\end{enumerate}
\end{exercice}

\begin{exercice}
\begin{enumerate}
\item Comparer ce qui suit :
$$2\sqrt{17}et3\sqrt{7};;;-3\sqrt{11}et-5\sqrt{5};;3\sqrt{5}et\sqrt{3}-\sqrt{17};;;\sqrt{7+2\sqrt{11}}et\sqrt{3}+2 $$
\item Soit $a$ et $b$ deux nombres réels positifs tel que $a\leq b$
\begin{enumerate}
\item Montrer que $a+1\leq b+\dfrac{5}{4}$ et que $b+\sqrt{7}\geq a-3\sqrt{7}$
\item Comparer $b^{2}et\dfrac{a^{2}+3b^{2}}{4}$
\end{enumerate}
\item Soient $a$,$b$ et $c$ des nombres réels positifs.
\begin{enumerate}
\item Montrer que : $a^{2}+b^{2}\geq 2ab$
\item Déduire que : $a^{2}+b^{2}+c^{2}\geq ab+bc+ac$
\item Si $a^{2}+b^{2}+c^{2}=1$,montrer que $(a+b+c)^{2}=1+2(ab+bc+ac)$
\item Déduire que : $a+b+c\leq \sqrt{3}$
\end{enumerate}
\end{enumerate}
\end{exercice}

\begin{exercice}
On pose $x=\dfrac{2}{\sqrt{3}+1}$ et $y=\dfrac{5+\sqrt{3}}{2}$
\begin{enumerate}
\item Montrer que : $x-y=\dfrac{\sqrt{3}-7}{2}$
\item Comparer : $\sqrt{3}$ et $7$
\item Déduire  la comparaison de $x$ et $y$.
\end{enumerate}
\end{exercice}

\begin{exercice}
\begin{enumerate}
\item
\begin{enumerate}
\item Comparer les deux nombres: $\sqrt{7}$ et $2$ puis $\sqrt{3}$ et $5$.
\item Déduire une simplification pour : $\sqrt{(\sqrt{7}-2)^{2}}$ et $\sqrt{(\sqrt{3}-5)^{2}}$
\end{enumerate}
\item 
\begin{enumerate}
\item Développer les expressions :$(\sqrt{5}-4)^{2})^{2}$ et $(6-\sqrt{2})^{2}$
\item Déduire une simplification pour : 
$\sqrt{21-8\sqrt{5}}$ et $\sqrt{38-12\sqrt{2}}$
\end{enumerate}
\end{enumerate}
\end{exercice}

\begin{exercice}
\begin{enumerate}
\item Montrer que : $\dfrac{1}{\sqrt{5}+\sqrt{3}}=\dfrac{\sqrt{5}-\sqrt{3}}{2}$
\item Comparer les nombres : $\sqrt{5}+\sqrt{3}$ et $\sqrt{3}+1$
\item Déduire une comparaison $\dfrac{\sqrt{5}-\sqrt{3}}{2}$ et $\dfrac{\sqrt{3}-1}{2}$
\end{enumerate}
\end{exercice}

\begin{exercice}
Soit $x$ et $y$ deux nombres réels tels que : $-4\leq x\leq -1$ et $5\leq y\leq 9$
\begin{enumerate}
\item Encadrer ce qui suit :$x+y$ et $x-y$ et $xy$ et $\dfrac{x}{y}$
\item Encadrer : $2x+3y$ et $-5x-y$ et $4y-8$et $\dfrac{-4x}{3y}$
\item Encadrer : $x^{2}+y^{2}$ et $\dfrac{x^{2}}{y^{2}-2xy}$
\end{enumerate}
\end{exercice}

\begin{exercice}
Soit $a$ et $b$ deux nombres réels tels que : $4\leq a\leq 12$ et $5\leq b\leq 9$
\begin{enumerate}
\item Encadrer ce qui suit :$a+b$ et $a-b$ et $ab$ et $\dfrac{a}{b}$
\item Encadrer : $2a+3b$ et $-5a-2b$ et $4a+5$et $\dfrac{2b}{5a}$
\item Encadrer : $a^{2}+b^{2}$ et $\dfrac{a^{2}}{b^{2}+2ab}$ et $\dfrac{2a+3b}{5a+9b}$
\end{enumerate}
\end{exercice}

\begin{exercice}
Soient $a$,$b$,$c$ et $d$ des nombres réels tels que: 
$$9\leq a\leq 16 et -7\leq b\leq -6 et \dfrac{1}{2}\leq \dfrac{3c-1}{2}\leq 1 et -2\leq d \leq -1 $$
\begin{enumerate}
\item Encadrer ce qui suit :
$$a+b ;; ab ;; a-b ;; \dfrac{a}{b} ;; -3a+2b-1 ;; 2\sqrt{a}+d $$
$$a^{2}+bd-b^{2};;;; \dfrac{2b-d}{a+b};;\sqrt{a^{2}-ab+b^{2}} $$
\item Montrer que $\dfrac{2}{3}\leq c \leq 1$
\end{enumerate}
\end{exercice}





\end{Maquette}




\end{document}