\documentclass[a4paper,12pt]{article}

\usepackage{ProfModels}

 
\begin{document}

\begin{Maquette}[Fiche]{Theme=Trigonométrie,Niveau=3}

\begin{exercice}
\begin{minipage}{.5\linewidth}
\begin{enumerate}
\item $ABC$ est un triangle rectangle en $A$ tel que $AC=7$ et $BC=9$
\begin{enumerate}
\item Calculer $\sin\widehat{B}$.
\item Montrer que $AB=4\sqrt{2}$.
\item Calculer $cos\widehat{B}$ et $tan\widehat{B}$
\end{enumerate}
\end{enumerate}
\begin{enumerate}[start=2]
\item $EFG$ est un triangle rectangle en $E$ tel que $EF=\sqrt{3}$ et $EG=1$.
\begin{enumerate}
\item Calculer $FG$.
\item Calculer les rapports trigonométriques d'angle $\widehat{G}$.
\item Déduire la mesure de l'angle $\widehat{G}$.
\end{enumerate}
\end{enumerate}
\end{minipage}\hspace{0.1\linewidth}
\begin{minipage}{.4\linewidth}
\begin{enumerate}[start=3]
\item Sachant que $\sin x=\dfrac{\sqrt{5}}{3}$
\begin{enumerate}
\item Calculer $\cos x$ et $\tan x$
\end{enumerate} 
\end{enumerate}
\begin{enumerate}[start=4]
\item Sachant que $\sin\alpha=\dfrac{2\sqrt{6}}{7}$
\begin{enumerate}
\item Calculer $\cos\alpha$ et $\tan\alpha$
\end{enumerate}
\end{enumerate}
\begin{enumerate}[start=5]
\item Sachant que $\cos x=\dfrac{10}{11}$
\begin{enumerate}
\item Calculer $\sin x$ et $\tan x$.
\end{enumerate}
\end{enumerate}
\begin{enumerate}[start=6]
\item Sachant que $\tan x=2$
\begin{enumerate}
\item Calculer $\sin x$ et $\cos x$.
\end{enumerate}
\end{enumerate}
\end{minipage}
\end{exercice}

\begin{exercice}
Soit $ABC$ un triangle rectangle en $A$ tel que $BC=6$ et $\tan\widehat{C}=\sqrt{3}$
\begin{enumerate}
\item Trouve la mesure d'angle $\widehat{C}$ à l'aide d'une calculatrice.
\item Construire une figure convenable.
\item Donne les rapports trigonométriques d'angle $\widehat{B}$.
\item Calculer $AB$ et $AC$.
\end{enumerate}
\end{exercice}

\begin{minipage}{.5\linewidth}
\begin{exercice}
Soit $ABC$  un triangle rectangle en $A$ tel que $BA=\sqrt{3}$ et $\tan\widehat{B}=\sqrt{2}$.
\begin{enumerate}
\item Montrer que $AC=\sqrt{6}$.
\item Calculer $BC$.
\item Calculer $\sin\widehat{B}$ et $\cos\widehat{B}$.
\end{enumerate}
\end{exercice}
\end{minipage}%
\begin{minipage}{.5\linewidth}
\begin{exercice}
Soit $ABC$  un triangle rectangle en $A$ tel que $AB=6$ et $\cos\widehat{B}=\dfrac{12}{13}$.
\begin{enumerate}
\item Calculer $\sin\widehat{B}$ et $\tan\widehat{B}$.
\item Calculer $AC$ et $BC$.
\item Calculer les rapports trigonométriques de $\widehat{C}$.
\end{enumerate}
\end{exercice}
\end{minipage}

\begin{exercice}
\begin{minipage}{.5\linewidth}
Simplifier les expressions suivantes:
$$A=\cos^{2}35^{\circ}+\sin^{2}33^{\circ}+\sin^{2}35^{\circ}+\cos^{2}33^{\circ}$$
$$B=\cos^{2}15^{\circ}+\cos^{2}75^{\circ}-2tan35^{\circ}\times tan55^{\circ} $$
$$C=\sin25^{\circ}-\sin65^{\circ}+\cos25^{\circ}-\cos65^{\circ} $$
\end{minipage}%
\begin{minipage}{.5\linewidth}
$$D=(\cos x + \sin x)^{2}+(\cos x - \sin x)^{2} $$
$$E=2\cos^{2}x+3\sin^{2}x-2 $$
$$F=\dfrac{1}{1+\cos x}+\dfrac{1}{1-\cos x}-\dfrac{2}{\sin^{2}x} $$
\end{minipage}
$$G=\sin x\sqrt{1-\cos x}\sqrt{1+\cos x}+\cos x\sqrt{1+\sin x }\sqrt{1-\sin x} $$
\end{exercice}


\begin{exercice}
$ABC$ un triangle rectangle en $A$ tel que $AB=3$ et $BC=5$.
\begin{enumerate}
\item Montrer que $AC$.
\item Calculer les rapports trigonométriques de $\widehat{C}$.
\end{enumerate}
Soit $E$ un point de $[BC]$ tel que $CE=3$ et $H$ le projeté orthogonal de $E$ sur $(AC)$.
\begin{enumerate}[start=3]
\item Calculer $EH$ et $HC$.
\end{enumerate}
\end{exercice}

\begin{exercice}
$ABC$ un triangle rectangle en $A$ tel que $AB=6$ et $BC=12$.
\begin{enumerate}
\item Montrer que $AC=6\sqrt{3}$.
\item Calculer les rapports trigonométriques de $\widehat{B}$.
\end{enumerate}
Soit $H$ le projeté orthogonal de $A$ sur $(BC)$
\begin{enumerate}[start=3]
\item Calculer $AH$ et $CH$.
\end{enumerate}
\end{exercice}



\begin{exercice}
\setlength{\columnseprule}{0.5pt}
\begin{multicols}{2}
\begin{enumerate}
\item Simplifier
$$A=cos a(sin a + cos a )-sin a(cos a -sin a) $$
$$B=\dfrac{1}{1+sin a}+\dfrac{1}{1-sin a}-\dfrac{2}{cos^{2}a} $$
$$C=(cosa + sina)^{2}+(cosa-sina)^{2} $$
$$D=cos^{4}a-sin^{4}a-cos^{2}a+3sin^{2}a$$
$$E=sina\times \sqrt{1-cosa}\times \sqrt{1+cosa}+cos^{2}a$$
$$F=\sqrt{2}sin^{2}a+\sqrt{2}cos^{2}a$$
\item Montrer que :
$$\dfrac{cos^{4}a-sin^{4}a}{cos^{2}a-sin^{2}a}=1$$
$$\dfrac{1-cosx}{sinx}=\dfrac{sinx}{1+cosx}$$
$$\sqrt{1-sinx}\times\sqrt{1+sinx}=cosx$$
$$sin^{2}a=\dfrac{tan^{2}a}{1+tan^{2}a}$$
$$1+tan^{2}x=\dfrac{1}{cos^{2}x}$$
\item Sachant que $sina=\dfrac{\sqrt{3}}{2}$
\begin{enumerate}
\item Calculer : $cosa$ et $tana$
\item Déduire : $sin(90-a)$ et $cos(90-a)$ et $tan(90-a)$.
\end{enumerate}
\item Calculer 
$$ X= 2cos15^{\circ}+cos^{2}36^{\circ}-2sin75^{\circ}+cos^{2}54^{\circ}$$
$$ Y=cos^{2}28^{\circ}-sin^{2}51^{\circ}+cos^{2}62^{\circ}+cos^{2}39^{\circ}$$
$$ Z=tan73^{\circ}\times tan17^{\circ}-sin^{2}40^{\circ}-sin^{2}50^{\circ}$$
\end{enumerate}
\end{multicols}

\end{exercice}

\begin{exercice}
$ABC$ est un triangle rectangle en $A$ tel que $BC=15$ et $sin\widehat{B}=\dfrac{3}{5}$.
\begin{enumerate}
\item Calculer $cos\widehat{B}$ et $tan\widehat{B}$
\item Calculer $AB$ et $AC$
\end{enumerate}


\end{exercice}














\end{Maquette}
\end{document}