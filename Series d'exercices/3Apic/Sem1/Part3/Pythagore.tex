\documentclass[a4paper,12pt]{article}

\usepackage{ProfModels}

 
\begin{document}

\begin{Maquette}[Fiche]{Theme=Théorème Pythagore,Niveau=3}

\begin{exercice}
\begin{minipage}{.6\linewidth}
On considère la figure ci-contre, tel que $H$ est le projeté orthogonal de $A$ sur $(BC)$.
\begin{enumerate}
\item Calculer $AH$ et $BH$.
\item Le triangle $ABC$ est-il rectangle.
\end{enumerate}

\end{minipage}%
\begin{minipage}{.4\linewidth}
\begin{tikzpicture}[scale=0.8]
\tkzDefPoints{0/0/A,4/0/B}
\tkzDefTriangle[pythagore](B,A)
\tkzGetPoint{C}
\tkzDefPointBy[projection=onto B--C ](A)
\tkzGetPoint{H}
\tkzDrawSegments(A,B B,C C,A A,H)
\tkzLabelPoint[](A){A}
\tkzLabelPoint[](B){B}
\tkzLabelPoint[left](C){C}
\tkzLabelPoint[above](H){H}
\tkzMarkRightAngle(A,H,B)
\tkzLabelSegment(A,B){$3\sqrt{2}$}
\tkzLabelSegment[left](A,C){$\sqrt{7}$}
\tkzLabelSegment(H,C){2}
\end{tikzpicture}
\end{minipage}
\end{exercice}

\begin{exercice}
\begin{minipage}{.5\linewidth}
\begin{tikzpicture}[scale=0.8]
\tkzDefPoints{0/0/A,5/1/C}
\tkzDefTriangle[school](C,A)
\tkzGetPoint{B}
\tkzDefPointBy[projection=onto B--C](A)
\tkzGetPoint{H}
\tkzDrawPolygon
(A,B,C)
\tkzDrawSegment(A,H)
\tkzLabelPoint[above](A){A}
\tkzLabelPoint[below](B){B}
\tkzLabelPoint[above](C){C}
\tkzLabelPoint[below](H){H}
\tkzMarkRightAngle(A,H,C)
\tkzLabelSegment(B,H){2}
\tkzLabelSegment(H,C){8}
\tkzLabelSegment(A,H){4}
\end{tikzpicture}
\end{minipage}%
\begin{minipage}{.5\linewidth}
On considère la figure ci-contre.
\begin{enumerate}
\item Calculer $AC$.
\item Calculer $AB$.
\item Prouver que $ABC$ est un triangle rectangle.
\end{enumerate}
\end{minipage}
\end{exercice}

\begin{exercice}
\begin{minipage}{0.5\linewidth}
On considère la figure ci-contre. Prouve que $AB^{2}+AC^{2}=BD^{2}+DC^{2}$
\end{minipage}%
\begin{minipage}{.5\linewidth}
\begin{tikzpicture}[scale=0.8]
\tkzDefPoints{0/0/B,5/-3/C}
\tkzDefTriangle[two angles=30 and 60](B,C)
\tkzGetPoint{D}
\tkzDefTriangle[two angles=-65 and -25](B,C)
\tkzGetPoint{A}
\tkzDrawPolygon(A,B,D,C)
\tkzDrawSegment(B,C)
\tkzLabelPoint[](A){A}
\tkzLabelPoint[left](B){B}
\tkzLabelPoint[](C){C}
\tkzLabelPoint[right](D){D}
\tkzMarkRightAngle(B,D,C)
\tkzMarkRightAngle(C,A,B)
\end{tikzpicture}
\end{minipage}
\end{exercice}

\begin{exercice}
\begin{minipage}{.6\linewidth}
$ABCD$ est un rectangle, $M$ et $N$ deux points tels que : $AB=6$, $AD=4$ et $CM=CN=2$.
\begin{enumerate}
\item Calculer les distances $AM$ , $MN$ et $AN$.
\item En déduire la nature du triangle $AMN$.
\end{enumerate}
\end{minipage}%
\begin{minipage}{.4\linewidth}
\begin{tikzpicture}[scale=1]
\tkzDefPoints{0/0/A,5/3/C}
\tkzDefRectangle(A,C)
\tkzGetPoints{B}{D}
\tkzDrawPolygon(A,B,C,D)
\tkzLabelPoints(A,B)
\tkzLabelPoints[above](C,D)
\tkzDefCircle[R](C,2)\tkzGetPoint{r}
\tkzInterLC(D,C)(C,r)
\tkzGetPoints{M}{M'}
\tkzInterLC(C,B)(C,r)
\tkzGetPoints{N'}{N}
\tkzDrawPoint(M)
\tkzLabelPoint[above](M){M}
\tkzDrawPoint(N)
\tkzLabelPoint[right](N){N}
\tkzDrawSegments(A,M M,N N,A)
\end{tikzpicture}
\end{minipage}
\end{exercice}

\begin{exercice}
\begin{minipage}{.5\linewidth}
On considère la figure ci-contre,
\begin{enumerate}
\item Calculer $HC$
\item Calculer $CG$
\end{enumerate}
\end{minipage}%
\begin{minipage}{.5\linewidth}
\begin{tikzpicture}[scale=0.8]
\tkzDefPoints{0/0/E,4/0/C}
\tkzDefTriangle[pythagore](C,E)\tkzGetPoint{H}
\tkzDefTriangle[school](H,C)\tkzGetPoint{G}
\tkzDrawPolygon(E,C,H)
\tkzDrawSegments(H,G C,G)
\tkzLabelPoints(E,C,G)
\tkzLabelPoint[left](H){H}
\tkzLabelSegment[left](E,H){4}
\tkzLabelSegment(E,C){10}
\tkzLabelSegment[above](H,G){16}
\tkzMarkRightAngles(C,E,H G,C,H)
\end{tikzpicture}
\end{minipage}
\end{exercice}

\begin{exercice}
\begin{minipage}{.7\linewidth}
$ABC$ est un triangle et $H$ le projeté orthogonal de $A$ sur $(BC)$ tel que $BH=4$, $CH=9$ et $AH=6$.
\begin{enumerate}
\item Calculer $AB$ et $AC$.
\item Montrer que $ABC$ est un triangle rectangle en $A$.
\end{enumerate}
\end{minipage}%
\begin{minipage}{.3\linewidth}
\begin{tikzpicture}[scale=0.8]
\tkzDefPoint(0,0){A}
\tkzDefPoint(4,0){B}
\tkzDefTriangle[pythagore](B,A)\tkzGetPoint{C}
\tkzDefPointBy[projection=onto B--C](A)\tkzGetPoint{H}
\tkzDrawPolygon(A,B,C)
\tkzDrawSegment(A,H)
\tkzLabelPoints[above](B,C,H)
\tkzLabelPoint[](A){A}
\tkzLabelSegment[above](B,H){9}
\tkzLabelSegment[above](C,H){4}
\tkzLabelSegment(A,H){6}
\tkzMarkRightAngle(A,H,B)
\end{tikzpicture}
\end{minipage}
\end{exercice}

\begin{exercice}
\begin{minipage}{.6\linewidth}
$EFGH$ est un trapèze rectangle en $E$ et $H$ tel que : $EF=8$ , $EH=4$ et $HG=10$.
\begin{enumerate}
\item Montrer que $FH=4\sqrt{5}$ et $FG=2\sqrt{5}$
\item Montrer que $FGH$ est un triangle rectangle. 
\end{enumerate}
\end{minipage}%
\begin{minipage}{.4\linewidth}
\begin{tikzpicture}[scale=0.8]
\tkzDefPoint(0,0){F}
\tkzDefPoint(4,-3){H}
\tkzDefTriangle[two angles=30 and 60](F,H)\tkzGetPoint{E}
\tkzDefTriangle[two angles=30 and 70](H,F)\tkzGetPoint{G}
\tkzDrawPolygon(F,E,H,G)
\tkzDrawSegment(F,H)
\tkzLabelPoints[above](F,E)
\tkzLabelPoints(G,H)
\tkzMarkRightAngles(F,E,H E,H,G)
\end{tikzpicture}
\end{minipage}
\end{exercice}

\begin{exercice}
\begin{minipage}{.6\linewidth}
$ABCD$ est un rectangle tel que $AB=10$ et $AD=4$
\begin{enumerate}
\item Calculer $BD$
\item Calculer $AE$
\item Calculer $EF$
\end{enumerate}
\end{minipage}%
\begin{minipage}{.4\linewidth}
\begin{tikzpicture}[scale=0.8]
\tkzDefPoints{0/0/A,5/3/C}
\tkzDefRectangle(A,C)\tkzGetPoints{B}{D}
\tkzDrawSegment(B,D)
\tkzDrawPolygon(A,B,C,D)
\tkzDefPointBy[projection=onto D--B](A)\tkzGetPoint{E}
\tkzDefPointBy[projection=onto D--B](C)\tkzGetPoint{F}
\tkzDrawSegments(A,E C,F)
\tkzLabelPoints[above](D,C)
\tkzLabelPoints(A,B)
\tkzLabelPoint[above](E){E}
\tkzLabelPoint[](F){F}
\tkzMarkRightAngles(B,F,C D,E,A)
\end{tikzpicture}
\end{minipage}
\end{exercice}

\begin{exercice}
\begin{minipage}{.75\linewidth}
$ABCD$ un trapèze rectangle en $B$ et $C$ tel que : $AB=4$ et $BC=2$ et $CD=5$
\begin{enumerate}
\item Montrer que $AC=2\sqrt{5}$.
\item Soit $H$ le projeté orthogonal de $A$ sur $(DC)$.
\begin{enumerate}
\item Calculer $AD$.
\item Montrer que $ADC$ est un triangle rectangle en $A$.
\end{enumerate}
\end{enumerate}
\end{minipage}%
\begin{minipage}{.25\linewidth}
\begin{tikzpicture}[scale=0.8]
\tkzDefPoints{0/0/A,0/4/B,2/4/C,2/-1/D,2/0/H}
\tkzDrawSegments(A,B B,C C,D D,A A,H A,C)
\tkzLabelPoints[left](A,B)
\tkzLabelPoints[right](C,H,D)
\tkzMarkRightAngles(A,H,D A,B,C B,C,H)
\end{tikzpicture}
\end{minipage}
\end{exercice}

\begin{exercice}
$ABC$ est un triangle rectangle en $A$, soit $H$ le projeté orthogonal de $A$ sur $(BC)$.

Montrer que :
\begin{enumerate}
\item $BH^{2}+CH^{2}+2AH^{2}=BC^{2}$
\item $AH^{2}=BH\times CH$
\item $AB^{2}=BH\times BC$ et $AC^{2}=CH\times BC$
\item $AB\times AC=AH\times BC$
\end{enumerate}
\end{exercice}



\end{Maquette}
\end{document}