\documentclass[a4paper,12pt]{article}


\usepackage{ProfModels}

 
\begin{document}

\begin{Maquette}[Fiche]{Theme=Puissance de 10,Niveau=2}

\begin{exercice}
Calculer 
\begin{itemize*}
\item $10^{5}$
\item $10^{-3}$
\item $10^{8}$
\item $10^{0}$
\item $10^{-8}$
\end{itemize*}
\end{exercice}
\begin{exercice}
Mettre sous la forme d'une puissance de 10 :
\begin{itemize*}
\item 10000
\item 0.001
\item 0.1
\item 0.000000001
\item 1000000000000 
\end{itemize*}
\end{exercice}
\begin{exercice}
Mettre sous la forme d'une puissance de 10 :
\begin{tasks}(4)
\task $10^{-5}\times 10^{8}$
\task  $\lrp{10^{-2}}^{-4}$
\task $\dfrac{ 10^{26}}{ 10^{-12}}$
\task $ 10^{9}\times 10^{-7}$
\task $\dfrac{ 10^{-4}}{ 10^{-8}}$
\task $\lrp{ 10^{-5}}^{-4}$
\task $\dfrac{ 10^{24}\times  10^{-4}}{ 10^{-7}}$
\task $ 10^{0}\times  10^{-27}$
\end{tasks}
\end{exercice}

\begin{exercice}
Mettre sous la forme d'une puissance de 10 :
\begin{tasks}(4)
\task $\dfrac{10^{52}}{10^{26}\times 10^{14}}$
\task $\dfrac{10^{-2}\times 10^{-47}}{10^{-26}\times 10^{23}}$
\task $\dfrac{10^{-8}}{10^{-6}\times 10^{-14}}$
\task $\dfrac{10^{21}\times 10^{-4}}{10^{6}\times 10^{-4}}$
\task $\dfrac{10^{20}}{10^{-5}\times \lrp{10^{4}}^{5}}$
\task $\dfrac{\lrp{10^{-2}}^{-7}}{10^{6}\times 10^{-14}}$
\task $\dfrac{10^{5}}{10^{2}\times 10^{10}}$
\task $\dfrac{\lrp{10^{-4}}^{-7}}{10^{6}\times \lrp{10^{-4}}^{-12}}$
\end{tasks}
\end{exercice}

\begin{exercice}
Donner l'écriture décimale :
\begin{tasks}(3)
\task $6.08\times 10^{5}$
\task $672.34\times 10^{-2}$
\task $879.254\times 10^{4}$
\task $9124.23\times 10^{3}$
\task $0.00458\times 10^{5}$
\task $6.253687\times 10^{5}$
\end{tasks}
\end{exercice}

\begin{exercice}
Mettre en notation scientifique :
\begin{tasks}(3)
\task $540 000 000 000$
\task $650 000 000$
\task $0,000 000 006$
\task $1 048 000 000 000$
\task $0,000 002$
\task $64 20 300 000$
\task $673,185 8 070 000 000$
\task $4000,007$
\task $0,700 600 000$
\end{tasks}
\end{exercice}
\begin{exercice}
Mettre en notation scientifique :
\begin{tasks}(3)
\task $63000\times 10^{-15} $
\task $450\times 10^{6} $
\task $0.00002564\times 10^{-9}$
\task $0.012\times 500\times 10^{-12} $
\task $815000000000\times 10^{-15} $
\task $\dfrac{8\times 10^{32}\times 15\times 10^{-6}}{20\times 10^{-9}} $
\task $\dfrac{15\times 10^{14}\times 4\times 10^{-5}}{0.25\times 10^{-4}} $
\task $\dfrac{6\times \lrp{10^{-5}}^{-4}}{4\times 10^{6}} $
\task $0.002\times 10^{25}\times 0.00005\times 10^{-15} $
\end{tasks}
\end{exercice}

\begin{exercice}
Le corps humain contient $25\times 10^{12}$ globules rouges. Suite à une maladie,
un individu perd 12\% de ses globules rouges. Combien de globules rouges lui reste-t-il ?
(Donner le résultat en notation scientifique)
\end{exercice}

\begin{exercice}
Environ $78\times 10^{10} $ sacs plastiques ont été utilisés en 2006 par les $ 65\times 10^{8}$
 habitants de la planète. Cette même année les $61\times 10^{6} $ Français ont consommé en moyenne 360 sacs par habitant.
 \begin{enumerate}
 \item Calculer le nombre de sacs plastiques utilisés en moyenne par un habitant de la planète en 2006.
 \item Comparer ce résultat avec le nombre de sacs utilisés par un Français.
 \item Calculer le nombre de sacs plastiques utilisés en France en 2006
 \item Donner des idées permettant de limiter le nombre de sacs plastiques.
 \end{enumerate}
\end{exercice}

\begin{exercice}
$A =2105395$ et $B =0.0594$.
\begin{enumerate}
\item Encadrer A, puis B par deux puissances de 10 d'exposants consécutifs.
\item Donner un ordre de grandeur de $A\times B$.
\end{enumerate}
\end{exercice}



\end{Maquette}
\end{document}