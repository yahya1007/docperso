\documentclass[a4paper,12pt]{article}

\usepackage{ProfModels}

 
\begin{document}


\begin{Maquette}[Fiche]{Theme=La symétrie axiale ,Niveau=2}

\begin{exercice}
\begin{minipage}{.7\linewidth}
On considère la figure ci-contre.
\begin{enumerate}
\item Construis $E$ le symétrique du point $M$ par rapport à $(BC)$.
\item Construis $F$ le symétrique du point $M$ par rapport à $(AC)$.
\item Construis $G$ le symétrique du point $M$ par rapport à $(AB)$.
\item Que pouvez-vous dire des points $E$, $F$ et $G$.
\end{enumerate}
\end{minipage}%
\begin{minipage}{.3\linewidth}
\begin{tikzpicture}
\tkzDefPoint(0,0){O}
\tkzDefPoint(2,0){A}
\tkzDrawCircle(O,A)

\tkzDefPointOnCircle[through= center O angle 120 point A]
\tkzGetPoint{B}
\tkzDefPointOnCircle[through= center O angle -120 point A]
\tkzGetPoint{C}

\tkzDefPointOnCircle[through= center O angle 90 point A]
\tkzGetPoint{M}

\tkzDrawPoints(A,B,C,M)
\tkzDrawSegments(A,B B,C C,A)
\tkzLabelPoint[right](A){A}
\tkzLabelPoint[above left](B){B}
\tkzLabelPoint[above](M){M}
\tkzLabelPoint[](C){C}
\end{tikzpicture}
\end{minipage}
\end{exercice}

\begin{exercice}
On considère la figure ci-contre , tel que $(D_{1})//(D_{2})$ et $E$ le milieu de $[AB]$.
\begin{enumerate}
\item Construis $M$ le symétrique du point $E$ par rapport à $(D_{1})$.
\item Construis $N$ le symétrique du point $E$ par rapport à $(D_{2})$.
\item Montrer que $M$,$N$ et $E$ sont alignés.
\item La droite $(MN)$ coupe $(D_{1}$ en $I$ et $(D_{2})$ en $J$. Montrer que $E$ est le milieu de $[IJ]$
\end{enumerate}

\begin{tikzpicture}
\tkzDefPoint(3,3){A}
\tkzDefPoint(-3,3){D}
\tkzDefPoint(0,1){B}
\tkzDefMidPoint(A,B)
\tkzGetPoint{E}
\tkzDefLine[parallel=through B](A,D)
\tkzGetPoint{C}
\tkzDrawLines(A,D B,C A,B)
\tkzDrawPoints(A,B,E)
\tkzLabelPoints(A,B,E)
\tkzMarkSegments[mark=||](A,E E,B)
\tkzLabelLine[pos=1.25,below left](A,D){$(D_1)$}
\tkzLabelLine[pos=1.25,below left](B,C){$(D_2)$}
\end{tikzpicture}
\end{exercice}

\begin{exercice}
$ABC$ est un triangle isocèle en $A$.
Soient $B'$ le symétrique de $B$ par rapport à $(AC)$ et $C'$ le symétrique de $C$ par rapport à $(AB)$.
\begin{enumerate}
\item Construis la figure.
\item Montrer que $AC'=AB'$.
\item En déduire que les points $B'$ , $C'$ , $C$ et $B$ appartiennent au même cercle.
\item Montrer que $C'B=BC=CB'$.
\end{enumerate}
\end{exercice}

\begin{exercice}
$EFG$ est un triangle rectangle en $E$, tel que $EF=4$.
\begin{enumerate}
\item Construis le point $M$ le symétrique du point $E$ par rapport à la droite $(FG)$.
\item Calculer la distance $FM$.
\end{enumerate}
\end{exercice}

\begin{exercice}
$ABC$ un triangle et $O$ le milieu du segment $[BC]$.
\begin{enumerate}
\item Construis les points $E$ et $F$ les symétriques de $B$ et $C$ respectivement par rapport à la droite $(AO)$.
\item Déterminer le symétrique du segment $[BC]$ par rapport à la droite $(AO)$.
\item Montrer que $O$ est le milieu du segment $[EF]$.
\end{enumerate}
\end{exercice}

\begin{exercice}
$ABC$ un triangle tel que $AB=3$ et $AC=5$ et $\widehat{ABC}=60^{\circ}$. Soit $M$ le milieu du segment $[AC]$.
\begin{enumerate}
\item Construire les points $E$ et $F$ les symétriques respectifs des points $A$ et $C$ par rapport à la droite $(BM)$.
\item Calculer $EF$ et $BE$.
\item Quelle est la mesure de l'angle $\widehat{BEF}$.
\end{enumerate}
\end{exercice}

\begin{exercice}
$ABCD$ un trapèze tel que $\widehat{ADC}=40^{\circ}$ et $I$ le milieu du segment $[CD]$.
\begin{enumerate}
\item Construire une figure convenable.
\item Construire les points $D'$ , $C'$ et $I'$ les symétriques respectifs des points $D$ , $C$ et $I$ par rapport à $(AB)$.
\item Montrer que les points $D'$ , $C'$ et $I'$ sont alignés.
\item Montrer que $I'$ est le milieu du segment $[D'C']$.
\item Quelle est la mesure de $\widehat{A'D'C'}$.
\end{enumerate}
\end{exercice}

\begin{exercice}
$C(O;r)$ et $C'(O';r)$ deux cercles de même rayon et ne sont pas sécantes.Soit $(\Delta)$ la médiatrice du segment $[OO']$.
Soit $M$ un point de $(C)$ tel que la demi-droite $[OM)$ coupe $(\Delta)$ en $I$. Soit $M'$ le point d'intersection de la droite $(O'I)$ et le cercle $(C')$
\begin{enumerate}
\item Construire une figure convenable.
\item Quel est le métrique du cercle $(C)$ par rapport à la droite  $(\Delta)$ .
\item Montrer que $M'$ est le symétrique de $M$ par rapport à la droite  $(\Delta)$.
\end{enumerate}
\end{exercice}





\end{Maquette}


\end{document}