\documentclass[a4paper,12pt]{article}


\usepackage{ProfModels}

 
\begin{document}

\begin{Maquette}[Fiche]{Theme=Triangle et parallèle,Niveau=2}

\begin{exercice}
$ABCD$ est un trapèze de bases $\lrc{AB}$ et $\lrc{CD}$, soit $I$, $J$ et $K$ les milieux respectifs de $\lrc{BC}$, $\lrc{BD}$ et  $\lrc{AD}$.
\begin{enumerate}
\item Construire une figure.
\item Montrer que $I$, $J$ et $K$ sont des points alignés.
\end{enumerate}
\end{exercice}

\begin{exercice}
$ABCD$ est un parallélogramme, $E$ le symétrique de $D$ par rapport à $A$; la droite $\lrp{EC}$ coupe $\lrc{AB}$ en $M$.
\begin{enumerate}
\item Construire la figure.
\item Montrer que $M$ est le milieu de $\lrc{AB}$.
\item Prouver que $AEBC$ est un parallélogramme.
\end{enumerate}
\end{exercice}

\begin{exercice}
\begin{minipage}{0.5\linewidth}
Dans la figure ci-contre : $AE=2$; $EB=7$; $AF=3$ et $\lrp{EF}//\lrp{BC}$.
\begin{enumerate}
\item Calculer $AC$.
\item Déduire $FC$.
\item Calculer $BC$.
\end{enumerate}
\end{minipage}
\begin{minipage}{0.5\linewidth}
\begin{tikzpicture}
\tkzDefPoints{0/0/C,4/1/B,1/4/A}
\tkzDefPointOnLine[pos=0.3](A,C)\tkzGetPoint{F}
\tkzDefPointOnLine[pos=0.3](A,B)\tkzGetPoint{E}
\tkzDrawPolygon(A,B,C)
\tkzDrawSegment(E,F)
\tkzLabelPoints[left](A,C,F)
\tkzLabelPoints[right](E,B)
\tkzLabelSegment[left](A,F){3}
\tkzLabelSegment[right](A,E){2}
\tkzLabelSegment[right](E,B){7}
\tkzLabelSegment(E,F){2.4}
\end{tikzpicture}
\end{minipage}
\end{exercice}

\begin{exercice}
$ABC$ est un triangle rectangle en $A$. Soit $M$, $N$ et $P$ les milieux respectifs de $\lrc{AB}$,  $\lrc{BC}$ et  $\lrc{AC}$.
\begin{enumerate}
\item Montrer que $\lrp{AB}//\lrp{MN}$.
\item Montrer que $AN=BN=CN$.
\end{enumerate}
\end{exercice}

\begin{exercice}
$ABC$ est un triangle et $M$ le milieu de $\lrc{BC}$ et $O$ le milieu de $\lrc{AM}$. La droite $\lrp{OB}$ coupe $\lrc{AC}$ en $D$ et la droite parallèle à $\lrp{OB}$ passant par $M$ coupe $\lrc{AC}$ en $E$.
\begin{enumerate}
\item Construire une figure.
\item Montrer que $D$ est le milieu de $\lrc{AE}$.
\item Montrer que $E$ est le milieu de $\lrc{DC}$.
\item Déduire que $EC=DE=AD$.
\item Prouver que $DC=2AD$.
\end{enumerate}
\end{exercice}

\begin{exercice}
$ABCD$ est un quadrilatère tel que $M$, $N$, $P$ et $Q$ sont les milieux respectifs de $\lrc{AB}$, $\lrc{CB}$, $\lrc{CD}$ et $\lrc{AD}$.
\begin{enumerate}
\item Trace une figure convenable.
\item Montrer que $\lrp{MN}//\lrp{PQ}$.
\end{enumerate}
\end{exercice}

\begin{exercice}
$ABCD$ est un parallélogramme et $O$ le milieu de $\lrc{AD}$.
La droite $\lrp{OC}$ coupe $\lrp{AB}$ en $M$.
\begin{enumerate}
\item Construire une figure.
\item Montrer que $A$ est le milieu de $\lrc{MD}$.
\item Construire la droite parallèle à $\lrp{MO}$ passant par $D$ et qui coupe $\lrp{AB}$ en $N$.
\item Montrer que $M$ est milieu de $\lrc{AN}$.
\end{enumerate}
\end{exercice}

\begin{exercice}
$ABCD$ est un parallélogramme et $E\in [DA)$ tel que : $AD=AE$; la droite $\lrp{EC}$ coupe $\lrp{AB}$ en $F$.
\begin{enumerate}
\item Construire une figure.
\item Montrer que $E$ est le symétrique de $C$ par rapport à $F$.
\item Construire $G$ le symétrique de $D$ par rapport à $F$.
\item Montrer que $\lrp{EG}//\lrp{AB}$.
\item Montrer que $AGBD$ est un parallélogramme.
\item Déduire que $G$, $B$ et $C$ sont alignés.
\end{enumerate} 
\end{exercice}

































\end{Maquette}
\end{document}