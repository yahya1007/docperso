\documentclass[a4paper,12pt]{article}


\usepackage{ProfModels}

\begin{document}


\begin{Maquette}[Fiche]{Theme=Les droites remarquables dans un triangle,Niveau=2}

\begin{exercice}
\begin{minipage}{0.25\linewidth}
\begin{tikzpicture}
\tkzDefPoints{0/0/A,4/0/B,1/3/C}
\tkzDefMidPoint(A,B)\tkzGetPoint{I}
\tkzDrawPolygon(A,B,C)
\tkzDrawLine[add=0.5 and 0.5](C,I)
\tkzLabelLine[pos=1.5,right](C,I){$(D1)$}
\tkzMarkSegments[mark=||](A,I I,B)
\end{tikzpicture}%
\end{minipage}%
\begin{minipage}{0.25\linewidth}
\begin{tikzpicture}
\tkzDefPoints{0/0/A,2.5/0/B,-1/3/C}
\tkzDefPointBy[projection = onto A--B](C)
\tkzGetPoint{I}
\tkzDrawPolygon(A,B,C)
\tkzDrawLine[add=0.5 and 0.5](C,I)
\tkzDrawLine[add=0 and 0.25,dashed](A,I)
\tkzLabelLine[pos=1.25,right](I,C){$(D2)$}
\tkzMarkRightAngle(C,I,A)
\end{tikzpicture}%
\end{minipage}%
\begin{minipage}{0.25\linewidth}
\begin{tikzpicture}[scale=0.8]
\tkzDefPoints{0/0/A,3/0/B,-1/3/C}
\tkzDefMidPoint(B,C)\tkzGetPoint{I}
\tkzDefLine[mediator](C,B)
\tkzGetPoints{i}{j}
\tkzInterLL(i,j)(A,B)\tkzGetPoint{K}
\tkzDrawPolygon(A,B,C)
\tkzDrawLine[add=1 and 1 ](I,K)
\tkzMarkRightAngle(C,I,K)
\tkzMarkSegments[mark=||](C,I I,B)
\tkzLabelLine[pos=1.5,right](I,K){$(D3)$}
\end{tikzpicture}%
\end{minipage}%
\begin{minipage}{0.25\linewidth}
\begin{tikzpicture}[scale=0.8]
\tkzDefPoints{0/0/A,4/0/B,3/3/C}
\tkzDefLine[bisector](B,A,C)
\tkzGetPoint{a}
\tkzInterLL(A,a)(B,C)\tkzGetPoint{I}	
\tkzDrawPolygon(A,B,C)
\tkzDrawLine[add=0.2 and 0.2](I,A)
\tkzLabelLine[pos=1.2,above](A,I){$(D4)$}
\tkzMarkAngle(I,A,C)
\tkzMarkAngle[size=0.8](B,A,I)
\end{tikzpicture}
\end{minipage}%

Que représente les droites ci-dessus pour chaque figure?
\end{exercice}

\begin{exercice}
$EFGH$ est un carré de centre $I$ et $O$ le milieu de $\lrc{HG}$; la droite $\lrp{FO}$ coupe $\lrp{EG}$ en $K$.
\begin{minipage}{0.55\linewidth}
\begin{enumerate}
\item Que représente le point $K$ pour le triangle $FGH$?
\item Calculer $GK$ sachant que $GI=6$.
\item Montrer que $\lrp{OI}$ est la médiatrice de $\lrc{HG}$.
\item Que représente le point $I$ pour le triangle $FGH$?
\end{enumerate}
\end{minipage}%
\begin{minipage}{0.45\linewidth}
\begin{center}
\begin{tikzpicture}[scale=0.8]
\tkzDefPoints{0/0/E,4/0/F}
\tkzDefSquare(E,F)\tkzGetPoints{G}{H}
\tkzDefPointOnLine[pos=0.5](H,G)\tkzGetPoint{O}
\tkzInterLL(E,G)(F,H)\tkzGetPoint{I}
\tkzInterLL(F,O)(E,G)\tkzGetPoint{K}
\tkzDrawPolygon(E,F,G,H)
\tkzDrawSegments(E,G F,H F,O)
\tkzLabelPoints(E,F,I)
\tkzLabelPoints[above](H,G,O)
\tkzLabelPoint[right](K){K}
\tkzMarkSegments[mark=||](H,O O,G)
\end{tikzpicture}
\end{center}
\end{minipage}
\end{exercice}

\begin{exercice}
$ABC$ est un triangle isocèle en $A$ tel que $AB=AC=6$ et $BC=5$.
\begin{enumerate}
\item Construire $G$ le centre de gravité du triangle $ABC$.
\item Construire $I$ le centre du cercle inscrit dans le triangle $ABC$.
\item Construire $H$ l'orthocentre du triangle $ABC$.
\item Construire $O$ le centre du cercle conscrit au triangle $ABC$.
\item Montrer que $G$, $H$, $I$ et $O$ sont alignés.
\end{enumerate}
\end{exercice}

\begin{exercice}
$EAF$ est un triangle rectangle en $A$ tel que $AE=6$ et $\widehat{FEA}=30^{\circ}$.
\begin{enumerate}
\item Détermine l'orthocentre du triangle $EAF$.
\item Construire le point $G$ symétrique de $F$ par rapport à $A$.
\item Montrer que $EF=EG$.
\item Soit $B$ le milieu de $\lrc{EG}$ et la droite $\lrp{FB}$ coupe $\lrp{EA}$ en $M$.
\begin{enumerate}
\item Montrer que $M$ est le centre du gravité de triangle $EFG$.
\item Calculer $EM$.
\end{enumerate}
\item La droite $\lrp{GM}$ coupe $\lrc{EF}$ en $C$.
\begin{enumerate}
\item Montrer que $C$ est le milieu de $\lrc{EF}$.
\end{enumerate}
\item Montrer que $EFG$ est équilatéral.
\item Construire le cercle inscrit et circonscrit au triangle $EFG$.
\end{enumerate}
\end{exercice}

\begin{exercice}
$ABCD$ est un parallélogramme de centre $O$.$G$ est le centre de gravité de $ABD$ et $G'$ le centre de gravité de $BCD$.
\begin{enumerate}
\item Construire une figure.
\item Montrer que $AG=\dfrac{1}{3}AC$.
\item Montrer que $O$ est le milieu de $\lrc{GG'}$.
\end{enumerate}
\end{exercice}

\begin{exercice}
$ABCD$ est parallélogramme de centre $I$ tel que : $AB=6$  et $BC=3$,le point $E$ est le symétrique de $B$ par rapport à $C$ et la droite $\lrp{IE}$ coupe $\lrc{DC}$ en $F$.
\begin{enumerate}
\item Construire une figure convenable.
\item Montrer que $F$ est le centre de gravité de $DBE$.
\item Calculer $DF$.
\end{enumerate}
\end{exercice}

\begin{exercice}
$SPR$ est un triangle tel que $PS=PR=6$ et $\widehat{SPR}=120^{\circ}$; Soit $H$ l'orthocentre de $SPR$ et $M$ le point d'intersection de la hauteur correspond au côté $\lrc{RS}$.
\begin{enumerate}
\item Faire une figure.
\item Construire le cercle de centre $E$ et qui  inscrit au triangle $SPR$.
\item Montrer que $M$, $P$, $E$ et $H$ sont alignés.
\item Quel est la nature du triangle $HRS$? Calculer $PM$.
\end{enumerate}
\end{exercice}

\begin{exercice}
Soit $ABCD$ un parallélogramme de centre $O$. Le point $E$ est le milieu du segment $[AB]$ et les segments $[AC]$
et $[DE]$ se coupent en $G$.
\begin{enumerate}
\item Que représente le segment $[AO]$ pour le triangle $ABD$ ? Justifie.
\item Que représente le point $G$ pour le triangle $ABD$ ? Justifie.
\item Démontre que la droite $(BG)$ coupe le segment $[AD]$ en son milieu.
\end{enumerate}
\end{exercice}



\end{Maquette}


\end{document}