\documentclass[a4paper,12pt]{article}


\usepackage{ProfModels}

 
\begin{document}

\begin{Maquette}[Fiche]{Theme=Les équations,Niveau=2}


\begin{exercice}
\begin{enumerate}
\item Résoudre les équations suivantes 
\end{enumerate}
\[
3x-\dfrac{1}{2}\quad ;; \quad
4x+\dfrac{2}{3}=\dfrac{1}{2}\quad ;; \quad
\dfrac{-x}{2}+\dfrac{8}{7}=\dfrac{-1}{4}\quad ;; \quad
5x-\dfrac{2}{3}=\dfrac{5x}{4}+1
\]
\[
5x+2=\dfrac{1}{2}-3x\quad ;; \quad
2x-\dfrac{1}{2}=\dfrac{5}{3}-4x\quad ;; \quad
\dfrac{2x+6}{7}-1=\dfrac{3x-5}{4}
\]
\end{exercice}

\begin{exercice}
\begin{enumerate}
\item Résoudre les équations suivantes 
\end{enumerate}
\[
\dfrac{x-1}{2}-\dfrac{x}{3}=0\quad ;; \quad
6\left(2x+5\right)+\dfrac{1}{2}\left(4x-1\right)=0\quad ;; \quad
-2\left(3x+\dfrac{4}{3}\right)-4\left(\dfrac{1}{2}x-5\right)=0
\]
\[
3\left(5-\dfrac{x}{3}\right)=-\dfrac{3}{2}-12\left(\dfrac{x}{5}+\dfrac{1}{3}\right)\quad ;; \quad
\dfrac{5}{2}\left(x+\dfrac{2}{5}\right)=\dfrac{5}{4}-3\left(\dfrac{2}{3}-\dfrac{x}{2}\right)
\]
\end{exercice}
\begin{exercice}
\begin{enumerate}
\item Résoudre les équations suivantes 
\end{enumerate}
\[
\dfrac{x}{4}-\dfrac{2x-1}{3}=\dfrac{x}{2}\quad ;; \quad
2x+\dfrac{2\left(3x-2\right)}{5}=\dfrac{x+1}{2}\quad ;; \quad
\dfrac{6x-1}{6}-\dfrac{3-4x}{4}=1-\dfrac{2-x}{3}
\]
\[
\dfrac{x+1}{14}-\dfrac{1-x}{21}=\dfrac{3-5x}{42}\quad ;; \quad
\dfrac{1-4x}{4}+\dfrac{2x+3}{10}=\dfrac{x}{2}-\dfrac{5-x}{5}\quad ;; \quad
\dfrac{x+1}{2}+x-3=\dfrac{3x-1}{2}+4
\]
\end{exercice}

\begin{exercice}
\begin{enumerate}
\item Résoudre les équations suivantes 
\end{enumerate}
\[
\dfrac{x-3}{4}=\dfrac{2x+5}{3}\quad ;; \quad
\dfrac{7x-1}{12}=\dfrac{x+3}{9}\quad ;; \quad
\dfrac{7x}{3}-\dfrac{x+1}{2}=\dfrac{12x}{5}-\dfrac{1}{2}
\]
\[
\dfrac{1-x}{5}+\dfrac{1-2x}{2}=1-\dfrac{x-2}{2}\quad ;; \quad
-\dfrac{4x-1}{3}=3x-\dfrac{2x+5}{8}+\dfrac{23-50x}{24}
\]
\end{exercice}

\begin{exercice}
Résoudre les équations suivantes 
\[
4x^{2}-9 \quad ;; \quad
25x^{2}-1+(2x+3)(5x+1)=0\quad ;; \quad
49x^{2}+28x+4=0
\]
\[
16x^{2}-25=0\quad ;; \quad
x^{2}+2x+1=0\quad ;; \quad
x^{2}=1
\]
\[
49x^{2}=16\quad ;; \quad
81x^{2}-18x+1=0\quad ;; \quad
4x^{2}+1=0
\]
\end{exercice}

\begin{exercice}
Karim a obtenu 11 et 16 aux deux premiers contrôles de Maths.

Quelle note doit-il avoir au troisième contrôle pour obtenir 16 de moyenne ?
\end{exercice}

\begin{exercice}
Aya achète 24 assiettes plates, 12 assiettes creuses et 12 assiettes à dessert. Une assiette creuse coûte 2 DH de moins qu'une assiette plate. Une assiette à dessert coûte 5 DH de moins qu'une assiette plate. Elle dépense en tout 540 DH. Quel est le prix de chaque sorte d'assiette ?
\end{exercice}

\begin{exercice}
la somme des âges de Marie, de sa mère et de sa grand-mère est 90 ans.La grand-mère a le double de l'âge de la mère et l'âge de Marie est le tiers de celui de sa mère.
Quel est l'âge de chacune ?
\end{exercice}

\begin{exercice}
Pierre dit :<< il y a 10 ans , j'avais la moitié de l'âge que j'aurai dans 10 ans>>. Quel est l'âge de Pierre ?
\end{exercice}

\begin{exercice}
Christian dépense $\dfrac{3}{5}$ d'une somme puis les deux tiers du reste. Finalement, il lui reste 39 DH. Quelle était la somme initiale ?
\end{exercice}

\begin{exercice}
On retranche un même nombre au numérateur et au dénominateur de la fraction $\dfrac{23}{38}$ . Quel est ce nombre sachant que l'on obtient l'inverse de la fraction initiale ?
\end{exercice}

\begin{exercice}
Deux enfants ont ensemble 200 DH. L'un des deux  enfants a 20 DH de plus que l'autre. Combien a chaque enfant ?
\end{exercice}

\begin{exercice}
Cindy, Eric et Kevin se sont partagés 89 pin's(épingles).Cindy a pris trois fois plus de pin's que Eric et Kevin a pris 5 pin's de plus que Cindy. Combien ont-ils de pin's chacun ?
\end{exercice}

\begin{exercice}
Pour offrir un cadeau à leur prof de Math, les élèves d'une classe ont collecté 74 DH en pièce de 1 DH et de 2 DH, soit 43 pièces en tout. Calculer le nombre de pièces de  chaque sorte.
\end{exercice}

\begin{exercice}
Si on augmente de 5 m un côté d'un carré et si on diminue de 3 m l'autre côté, on obtient un rectangle de même aire que celle du carré. Combien mesure le côté de ce carré ?
\end{exercice}

\begin{exercice}
Si tous les inscrits étaient venus, la sortie en autocar aurait coûté 250 DH par personne. Mais il y a eu 3 absents et chaque participant a du payer un supplément de 15 DH . Combien y avait-il d'inscrits ?
\end{exercice}



\end{Maquette}
\end{document}