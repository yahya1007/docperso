\documentclass[a4paper,12pt]{article}


\usepackage{ProfModels}
\usepackage{diagrammes}
 
\begin{document}

\begin{Maquette}[Fiche]{Theme=Proportionnalité ,Niveau=2}
\begin{exercice}
Dans un immeuble, les charges payées sont proportionnelles à la surface au sol de la propriété pour
chacun des propriétaires. Trouver la valeur de x, y et de z du tableau des charges de quelques propriétaires. 

\begin{tabular}{|Oc|Oc|Oc|Oc|Oc|}
\hline 
Surface au sol en $m^{2}$ & x & 61.2 & y & 72.9 \\ 
\hline 
Montant des charges  & 82.32 & 171.36 & 189 & z \\ 
\hline 
\end{tabular} 
\end{exercice}

\begin{exercice}
Sur une carte à l’échelle $\dfrac{1}{100000}$
, deux villes sont séparées par 4,5 cm. Quelle est la distance réelle entre elles ? 
\end{exercice}

\begin{exercice}
Un globule blanc monocyte est un disque de 0,002 mm de diamètre. On souhaite en faire un dessin à l’échelle
$\dfrac{25000}{1}$ . Calculer le diamètre du disque à représenter à cette échelle. On donnera une réponse en cm. 
\end{exercice}

\begin{exercice}
Une spore de fougère est représentée par un disque de 1 cm de diamètre. Son diamètre réel est de 0,5 mm. Quelle
est l’échelle du schéma ? 
\end{exercice}

\begin{exercice}
Une voiture roulant à vitesse constante, a parcouru 105 km en 1 h 15min. Combien de temps lui faudra-t-il pour
parcourir 189 km ? 
\end{exercice}

\begin{exercice}
Lorsqu’il a battu le record du monde de l’heure le 6 septembre 1956, le champion Chris
Boardman a parcouru 27,06 m chaque fois qu’il a fait 3 tours de pédalier. Combien de tours de pédaliers a-t-il fait
pour parcourir les 56,3759 km de son record ? 
\end{exercice}

\begin{exercice}
Un train qui roule d’un mouvement uniforme à la vitesse de 80 km par heure défile en 12 s devant un passage à
niveau. Calculer la longueur du train. 
\end{exercice}

\begin{exercice}
Pour peindre un mur, un peintre mélange de la peinture blanche et de la peinture rouge. 
Pour 2,5 L de peinture blanche, il met 1,7 L de peinture rouge. Les volumes de peinture blanche et de peinture rouge sont proportionnels. 
Quel volume de peinture rouge ajoute-t-il à 3,5 L de peinture blanche ?
\end{exercice}

\begin{exercice}
Un club de sports compte 260 membres dont 120 garçons.
15\% des garçons et 25\% des filles participent à des compétitions.
\begin{enumerate}
\item Combien de garçons participent à des compétitions ?
\item Combien de filles participent à des compétitions ?
\item Quel pourcentage des membres de ce club participent à des compétitions ?
\end{enumerate}
\end{exercice}

\begin{exercice}
\begin{enumerate}
\item Un cycliste parcourt 48 km en une heure et demie. Quelle est alors sa vitesse moyenne ?
\item Plus tard, il fait le même trajet à la vitesse moyenne de 38,4 km/h. Combien de temps roule-t-il ?
\item Quelle distance parcourt-il s'il roule pendant 1 h 40 min à la vitesse moyenne de 35 km/h ?
\end{enumerate}
\end{exercice}

\begin{exercice}
\begin{minipage}{0.6\linewidth}
Le graphique donne, pour un opérateur téléphonique, le prix payé selon la durée de communication.
\begin{enumerate}
\item Ce graphique illustre-t-il une situation de
proportionnalité ? Justifie.
\item Quel est le prix à payer pour 25 minutes de
communication ?
\item Combien de temps peut-on téléphoner pour
200 Dh ? Donne une valeur approchée en minutes.
\end{enumerate}
\end{minipage}%
\begin{minipage}{0.4\linewidth}
\begin{flushright}
\begin{tikzpicture}
\tkzInit[xmax=50,xstep=10,ymax=40,ystep=10]
\tkzGrid[sub,subxstep=5,subystep=5]
\tkzDrawX[label=min,right]
\tkzDrawY[label=Dh,above]
\tkzLabelX[orig=false]
\tkzLabelY
\tkzDefPoints{0/0/O,50/30/A}
\tkzDrawLine[add=0.1 and 0.1](O,A)
\end{tikzpicture}
\end{flushright}
\end{minipage}
\end{exercice}

\begin{exercice}
Une entreprise propose deux tarifs pour la location d'un ordinateur. Le premier tarif est donné dans
le tableau ci-dessous.

\begin{tabular}{|Oc|Oc|Oc|Oc|}
\hline 
Nombre de jour de locaton & 1 & 2 & 5 \\ 
\hline 
Prix payé avec le tarif A en (Dh) & 150 & 300 & 750 \\ 
\hline 
\end{tabular} 
\begin{enumerate}
\item Le prix payé est-il proportionnel à la durée de location avec le tarif A ?
\item Dans un repère (en prenant 1 cm pour 1 jour en abscisse et 1 cm pour 100 Dh en ordonnée), place
les points correspondants au tarif A. Pourquoi la représentation est-elle une droite ?
\item Avec le tarif B, le client paye un forfait de 200 Dh puis 100 Dh par jour de location. Calcule le prix payé
avec le tarif B pour 1 jour de location puis pour 5 jours de location.
\item Le prix payé est-il proportionnel à la durée de location avec le tarif B ?
\item Détermine le tarif le plus avantageux pour 3 jours de location.
\end{enumerate}
\end{exercice}



\end{Maquette}
\end{document}