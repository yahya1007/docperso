\documentclass[a4paper,12pt]{article}

\usepackage{ProfModels}


\begin{document}
\begin{Maquette}[Fiche]{Theme=Triangle rectangle et cercle,Niveau=2}

\begin{exercice}%1
$ABC$ est un triangle rectangle en $A$ tel que $AB<AC$ et $O$ le milieu de $\lrc{AC}$,le cercle de centre $O$ et son diamètre $\lrc{AC}$ coupe $\lrc{BC}$ en $C$ et $M$.
\begin{enumerate}
\item Construire une figure convenable.
\item Montrer que $M$ est le projeté orthogonal de $A$ sur $\lrc{BC}$.
\item Soit $E$ le milieu de $\lrc{AB}$.
\begin{enumerate}
\item Montrer que $BEM$ est un triangle isocèle.
\end{enumerate}
\end{enumerate}
\end{exercice}

\begin{exercice}
On considère la figure ci-contre tel que $\lrp{C}$ est un cercle de centre $O$ et $\lrc{MN}$ sa corde.La droite perpendiculaire à $\lrp{MN}$ en $N$ coupe $\lrp{C}$ en $N$ et $L$.
\begin{minipage}{0.5\linewidth}
\begin{enumerate}
\item faire une schéma.
\item Montrer que $O$ est le milieu de $\lrc{ML}$
\item La droite $\lrp{ON}$ coupe le cercle en $N$ et $P$.
\begin{enumerate}
\item Prouver que $\lrp{PL}//\lrp{MN}$.
\end{enumerate}
\end{enumerate}
\end{minipage}%
\begin{minipage}{0.5\linewidth}
\begin{tikzpicture}
\tkzDefPoints{0/0/O,2/0/B}

\tkzDrawCircle(O,B)
\tkzDefPointBy[rotation=center O angle 50](B)
\tkzGetPoint{M}
\tkzDefPointBy[rotation=center O angle -40](B)
\tkzGetPoint{N}
\tkzDrawPoints(O,M,N)
\tkzDrawSegment(M,N)
\tkzLabelPoints[right](M,N)
\tkzLabelPoint[below](O){O}
\tkzLabelCircle[above=4pt](O,B)(120){$\lrp{\mathcal{C}}$}
\end{tikzpicture}
\end{minipage}
\end{exercice}

\begin{exercice}
\begin{minipage}{0.5\linewidth}
On considère la figure ci-contre tel que : \newline
$A\in \lrp{MN}$ et $A\in \lrp{EF}$
\begin{enumerate}
\item Montrer que $\lrp{NF}//\lrp{EM}$
\end{enumerate}
\end{minipage}%
\begin{minipage}{0.5\linewidth}
\begin{tikzpicture}
\tkzDefPoints{0/0/A,2/0/O,-1.5/0/O',4/0/M,-3/0/N}
\tkzDefPointBy[rotation=center O angle 40](M)\tkzGetPoint{E}
\tkzDefPointBy[rotation=center O' angle 40](N)\tkzGetPoint{F}
\tkzDrawCircle(O,M)
\tkzDrawCircle(O',N)
\tkzDrawSegments(M,N E,F)
\tkzDrawLines(E,M N,F)
\tkzDrawPoints(O,O')
\tkzLabelPoints[right](M,E)
\tkzLabelPoints[left](N,F)
\tkzLabelPoints[below right](O,A)
\tkzLabelPoint[above](O'){O'}
\end{tikzpicture}
\end{minipage}
\end{exercice}

\begin{exercice}
$ABC$ est un triangle isocèle en $A$.Et soit $H$ le projeté orthogonal de $A$ sur $\lrp{BC}$.
\begin{enumerate}
\item Construire le point $M$ milieu de $\lrc{AB}$ et $N$ milieu de $\lrc{AC}$.
\item Montrer que $HM=HN$.
\end{enumerate}
\end{exercice}

\begin{exercice}
\begin{minipage}{0.65\linewidth}
On considère la figure ci-contre tel que : $FGH$ et $EFG$ sont deux triangles rectangles respectives en $H$ et $E$.
\begin{enumerate}
\item Montrer que $OE=OH$.
\end{enumerate}
\end{minipage}%
\begin{minipage}{0.35\linewidth}
\begin{tikzpicture}[rotate=-40]
\tkzDefPoints{0/0/F,4/0/H}
\tkzDefTriangle[two angles= 40 and 90](F,H)\tkzGetPoint{G}
\tkzDefPointOnLine[pos=0.5](F,G)\tkzGetPoint{O}
\tkzDefPointOnCircle[through= center O angle -100 point G]
\tkzGetPoint{E}
\tkzDrawPoint(O)
\tkzDrawSegments(F,H H,G G,F F,E E,G)
\tkzLabelPoint[above](G){G}
\tkzLabelPoint[above](O){O}
\tkzLabelPoint[left](F){F}
\tkzLabelPoint[right](H){H}
\tkzLabelPoint[below](E){E}
\tkzMarkSegments[mark=||](O,F O,G)
\tkzMarkRightAngles(G,H,F G,E,F)
\end{tikzpicture}
\end{minipage}
\end{exercice}

\begin{exercice}
ABC est un triangle rectangle en A tel que $\widehat{ABC}=50^{\circ}$ et E le milieu de $\lrc{BC}$
\begin{enumerate}
\item Construire une figure convenable
\item Calculer $\widehat{EAB}$ et $\widehat{AEB}$.
\end{enumerate}
\end{exercice}

\begin{exercice}
\begin{minipage}{0.65\linewidth}
On considère la figure ci-contre tel que $\lrp{\mathcal{C}}$ est un cercle de centre $O$ et de diamètre $\lrc{BC}$ et $\lrp{\mathcal{C'}}$ est un cercle de centre $O'$ et de diamètre $\lrc{CF}$.
\begin{enumerate}
\item Montrer que $\lrp{AB}//\lrp{EF}$
\end{enumerate}
\end{minipage}%
\begin{minipage}{0.35\linewidth}
\begin{tikzpicture}
\tkzDefPoints{0/0/O,2.5/0/B,-2.5/0/C,-1/0/O',0.5/0/F}
\tkzDefPointOnCircle[through= center O angle 60 point B]\tkzGetPoint{A}
\tkzDefPointOnCircle[through= center O' angle 60 point F]\tkzGetPoint{E}
\tkzDrawPoints(C,O',O,F,B,A,E)
\tkzDrawCircles(O',C O,C)
\tkzDrawSegments(C,B C,A E,F A,B)
\tkzLabelPoints(O,O')
\tkzLabelPoints[left](C)
\tkzLabelPoints[right](A,B)
\tkzLabelPoint[above](E){E}
\tkzLabelPoint[below right](F){F}
\tkzLabelCircle[below=4pt](O,B)(-40){$(\mathcal{C})$}
\tkzLabelCircle[below=4pt](O',F)(-40){$(\mathcal{C'})$}
\end{tikzpicture}
\end{minipage}
\end{exercice}

\begin{exercice}
$\lrp{\mathcal{C}}$ est un cercle de centre $O$ et de rayon $r$ et $\lrc{AB}$ son diamètre. Soient $E$ et $F$ deux points différentes de $\lrp{\mathcal{C}}$.
\begin{enumerate}
\item Construire une figure convenable.
\item Détermine la nature des triangles $AEB$ et $AFB$.
\item Déduire que : $OE=OF$.
\end{enumerate}
\end{exercice}

\begin{exercice}
Soit $\lrc{AB}$ un segment et $E$ son milieu. $C$ un point tel que $EAC$ est un triangle équilatéral.
\begin{enumerate}
\item faire une schéma
\item Montrer que $ABC$ est un triangle rectangle.
\item Déduire la mesure d'angle $\widehat{ECB}$.
\end{enumerate}
\end{exercice}

\begin{exercice}
$AOB$ est un triangle isocèle en $O$ et $C$ le symétrique de $B$ par rapport à $O$.
\begin{enumerate}
\item Construire une figure.
\item Montrer que $ABC$ est un triangle rectangle.
\end{enumerate}
\end{exercice}

\begin{exercice}
$ABC$ est un triangle rectangle en $A$ tel que $AB=7$ et $AC=5$ et $H$ est le projeté orthogonal de $A$ sur $\lrp{BC}$.Soient $M$ et $N$ les milieux de $\lrc{AB}$ et $\lrc{AC}$ respectivement.
\begin{enumerate}
\item Construire la figure.
\item Calculer $MH$ et $NH$.
\end{enumerate}
\end{exercice}




\end{Maquette}
\end{document}