\documentclass[a4paper,12pt]{article}

\usepackage{ProfModels}


\begin{document}
\begin{Maquette}[Fiche]{Theme=Théorème de Pythagore,Niveau=2}

\begin{exercice}%1
\begin{minipage}{0.65\linewidth}
On considère la figure ci-contre, tel que $AD=9$ et $BD=12$ et $BC=13$.
\begin{enumerate}
\item Calculer les longueurs $AB$ et $DC$.
\item Donner le périmètre et l'aire du triangle $ABC$.
\end{enumerate}
\end{minipage}%
\begin{minipage}{0.35\linewidth}
\begin{tikzpicture}
\tkzDefPoints{0/0/A,4/0/C}
\tkzDefTriangle[two angles=50 and 70](A,C)
\tkzGetPoint{B}
\tkzDefPointBy[projection=onto A--C](B)
\tkzGetPoint{D}
\tkzDrawPolygon(A,C,B)
\tkzDrawSegment(B,D)
\tkzLabelPoints(A,D,C)
\tkzLabelPoint[above](B){B}
\tkzLabelSegment(A,D){9}
\tkzLabelSegment[left=7pt,rotate=90](B,D){12}
\tkzLabelSegment[right](B,C){13}
\tkzMarkRightAngle(B,D,A)
\end{tikzpicture}
\end{minipage}
\end{exercice}

\begin{exercice}
\begin{minipage}{0.62\linewidth}
On considère la figure ci-contre, tel que $AD=15$ et $CD=6$ et $BC=10$.
\begin{enumerate}
\item Calculer les longueurs $AB$ et $BD$.
\item Donner le périmètre et l'aire du triangle $ABC$.
\end{enumerate}
\end{minipage}%
\begin{minipage}{0.38\linewidth}
\begin{tikzpicture}
\tkzDefPoints{0/0/A,6/0/C}
\tkzDefTriangle[two angles=40 and 50](A,C)
\tkzGetPoint{B}
\tkzDefPointBy[projection=onto A--C](B)
\tkzGetPoint{D}
\tkzDrawPolygon(A,C,B)
\tkzDrawSegment(B,D)
\tkzLabelPoints(A,D,C)
\tkzLabelPoint[above](B){B}
\tkzLabelSegment(A,D){15}
\tkzLabelSegment(C,D){6}
\tkzLabelSegment[right](B,C){10}
\tkzMarkRightAngle(B,D,A)
\end{tikzpicture}
\end{minipage}
\end{exercice}

\begin{exercice}
$ABC$ est triangle équilatéral tel que $AB=3$ et $M$ le symétrique de $B$ par rapport à $A$.
\begin{enumerate}
\item Construire une figure.
\item Montrer que $BMC$ est un triangle rectangle.
\item Déduire $MC^{2}$.
\end{enumerate}
\end{exercice}

\begin{exercice}
\begin{minipage}{0.5\linewidth}
On considère la figure ci-contre.
\begin{enumerate}
\item Calculer $AH$.
\item Calculer $BH$ et $HC$.
\end{enumerate}
\end{minipage}%
\begin{minipage}{0.5\linewidth}
\begin{tikzpicture}
\tkzDefPoints{0/0/A,4/0/B}
\tkzDefTriangle[pythagore](B,A)\tkzGetPoint{C}
\tkzDefPointBy[projection=onto B--C](A)\tkzGetPoint{H}
\tkzDrawPolygon(A,B,C)
\tkzDrawSegment(A,H)
\tkzLabelPoints(A,B)
\tkzLabelPoints[above](C,H)
\tkzLabelSegment(A,B){3}
\tkzLabelSegment[left](A,C){4}
\tkzMarkRightAngles(B,A,C C,H,A)
\end{tikzpicture}
\end{minipage}
\end{exercice}

\begin{exercice}
\begin{minipage}{0.5\linewidth}
On considère la figure ci-contre ,tel que $AB=9$ et $BC=3$ et $CD=6$.
\begin{enumerate}
\item Calculer $BD$ et $AD$.
\item Quelle est la nature de triangle $ADC$.
\end{enumerate}
\end{minipage}%
\begin{minipage}{0.5\linewidth}
\begin{tikzpicture}
\tkzDefPoints{0/0/C,4/0/D}
\tkzDefTriangle[pythagore](D,C)\tkzGetPoint{B}
\tkzDefTriangle[two angles=30 and 90](B,D)\tkzGetPoint{A}
\tkzDrawPolygon(A,B,C,D)
\tkzDrawSegment(B,D)
\tkzMarkRightAngles(A,D,B D,C,B)
\tkzLabelSegment(C,D){6}
\tkzLabelSegment[left](B,C){3}
\tkzLabelSegment[above](A,B){9}
\tkzLabelPoints(C,D)
\tkzLabelPoints[above](A,B)
\end{tikzpicture}
\end{minipage}
\end{exercice}

\begin{exercice}
EFG est un triangle rectangle en E tel que : $EF=4$ et $\cos\widehat{EFG}=\dfrac{2}{3}$.
\begin{enumerate}
\item Calculer $FG$.
\end{enumerate}
\end{exercice}

\begin{exercice}
ABC est un triangle rectangle en A tel que : $AC=21$ et $\cos\widehat{ACB}=\dfrac{21}{29}$.
\begin{enumerate}
\item Calculer $BC$.
\item Déduire la longueur $AB$.
\end{enumerate}
\end{exercice}

\begin{exercice}
\begin{minipage}{0.5\linewidth}
On considère la figure ci-contre tel que $AC=6.5$ et $BH=1.2$ et $CH=1.6$
\begin{enumerate}
\item Calculer $\cos\widehat{HCB}$ et $\cos\widehat{HCA}$.
\item Calculer $\cos\widehat{HAC}$.
\end{enumerate}
\end{minipage}%
\begin{minipage}{0.5\linewidth}
\begin{tikzpicture}
\tkzDefPoints{0/0/H,5/0/A,3/0/B,0/4/C}
\tkzDrawPolygon(C,H,A)
\tkzDrawSegment(B,C)
\tkzMarkRightAngle(A,H,C)
\tkzLabelSegment[left](H,C){1.6}
\tkzLabelSegment[right](A,C){6.5}
\tkzLabelSegment(H,B){1.2}
\tkzLabelPoints(A,B,H)
\tkzLabelPoint[above](C){C}
\end{tikzpicture}
\end{minipage}
\end{exercice}

\begin{exercice}
$ABC$ est un triangle isocèle en $A$. M est le projeté orthogonal de $C$ sur $\lrp{AB}$ et N est le projeté orthogonal de $B$ sur $\lrp{AC}$
\begin{enumerate}
\item Construire une figure.
\item Montrer que $\dfrac{AM}{AC}=\dfrac{AN}{AB}$.
\end{enumerate}
\end{exercice}

\begin{exercice}
\begin{minipage}{0.5\linewidth}
On considère la figure suivante tel que $M$ est le milieu de $\lrc{AE}$.
\begin{enumerate}
\item Prouver que :
\begin{enumerate}
\item $MB^{2}-AB^{2}=MF^{2}-EF^{2}$
\item $MG^{2}+AG^{2}=MF^{2}-EF^{2}$
\item $MB^{2}-AB^{2}=MH^{2}+EH^{2}$
\item $MG^{2}+AG^{2}=MH^{2}+EH^{2}$
\end{enumerate}
\end{enumerate}
\end{minipage}\hfill
\begin{minipage}{0.4\linewidth}
\begin{tikzpicture}
\tkzDefPoints{0/0/A,2.5/0/M,5/0/E,0/2/B,5/-2/F}
\tkzDefPointBy[projection=onto B--M](A)
\tkzGetPoint{G}
\tkzDefPointBy[projection=onto M--F](E)
\tkzGetPoint{H}
\tkzDrawPolygon(A,E,F,B)
\tkzDrawSegments(A,G E,H)
\tkzLabelPoints(A,M,H,F)
\tkzLabelPoints[above](B,G,E)
\tkzMarkRightAngles(M,A,B M,E,F A,G,M E,H,M)
\end{tikzpicture}
\end{minipage}
\end{exercice}

\begin{exercice}
$ABC$ est un triangle rectangle en $A$ tel que $\widehat{ABC}=60^{\circ}$; la bissectrice de $\widehat{ABC}$ coupe $\lrc{AC}$ en $E$.
\begin{enumerate}
\item Construire une figure.
\item Montrer que $EC^{2}=AB^{2}+AE^{2}$
\item Montrer que $\cos^{2}\widehat{ABC}+\cos^{2}\widehat{ACB}=1$
\end{enumerate}
\end{exercice}

\begin{exercice}
\begin{minipage}{0.5\linewidth}
On considère la figure ci-contre tel que $ABCD$ est un parallélogramme.
\begin{enumerate}
\item Montrer que :
\begin{enumerate}
\item $AF^{2}+FB^{2}=DE^{2}+EC^{2}$.
\item $DE^{2}+EA^{2}=BF^{2}+FC^{2}$
\end{enumerate}
\end{enumerate}
\end{minipage}\hfill
\begin{minipage}{0.4\linewidth}
\begin{tikzpicture}
\tkzDefPoints{0/0/A,4/1/B,2/-3/C}
\tkzDefParallelogram(A,B,C)
\tkzGetPoint{D}
\tkzDefPointBy[projection=onto A--C](D)
\tkzGetPoint{E}
\tkzDefPointBy[projection=onto A--C](B)
\tkzGetPoint{F}
\tkzDrawPolygon(A,B,C,D)
\tkzDrawSegments(A,C D,E B,F)
\tkzLabelPoints[above](A,B)
\tkzLabelPoints(D,C)
\tkzLabelPoint[right](E){E}
\tkzLabelPoint[left](F){F}
\tkzMarkRightAngles(D,E,C B,F,A)
\end{tikzpicture}
\end{minipage}
\end{exercice}

\begin{exercice}
$(\mathcal{C})$ est un cercle de centre $O$ et de rayon $r=5$; son diamètre  est $\lrc{BC}$, et $A$ un point de $(\mathcal{C})$ tel que $AB=6$.
\begin{enumerate}
\item Construire une figure convenable.
\item Montrer que $ABC$ est triangle rectangle.
\item Soit $H$ le projeté orthogonal de $A$ sur $\lrp{BC}$.
\begin{enumerate}
\item Calculer $\cos\widehat{ABC}$ et $\cos\widehat{ACB}$.
\item Déduire les longueurs $BH$ et $CH$.
\end{enumerate}
\end{enumerate}
\end{exercice}






\end{Maquette}
\end{document}