\documentclass{fiche}
\begin{document}
\entete{3\ieme\ 2}{14 d´ecembre 2006}{Devoir \no 5}
\exe Le plan est muni d’un rep`ere orthonormal (O, I, J). On prendra le
centim`etre comme unit´e de longueur.
\que Placer dans ce rep`ere les points : A($-4$ ; 3), B(3 ; 2) et C(1 ; $-2$).
\que Calculer les coordonn´ees des points E et F, milieux respectifs des
segments [BC] et [BA]. Placer les points E et F dans le rep`ere.
\que Placer le point G, sym´etrique du point E par rapport au point F.
\que Quelle est la nature du quadrilat`ere AEBG ? Justifier votre r´eponse.
\qsq Calculer la valeur exacte de la distance AB.
\qsq Calculer la valeur exacte de la distance AC.
\squ On admet que $AC=\sqrt{50}$ et $BC=\sqrt{20}$. Quelle est la nature
du triangle ABC ?
\squ En d´eduire que le quadrilat`ere AEBG est un rectangle.
\exe On donne $E=(5x-4)^2+(5x-4)(x+3)$.
\que D´evelopper et r´eduire E.
\que Factoriser E.
\que Calculer E pour $x=-1$.
\que R´esoudre l’´equation $(5x-4)(6x-1)=0$.
\end{document}