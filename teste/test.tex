\documentclass{article}

\usepackage{ProfModels}
\usepackage{pstyle}
\begin{document}


\begin{exoo}[][style7]
On considère les expressions $A=2x(3x+5)$ et $B=x(7x-1)$.
\begin{enumerate}
\item  Développer l'expression $A$.\newline
\notes[15pt]{3}{\linewidth}
\fullwidth{yahya $\cdotsx{15}$}
\item  Développer l'expression $B$.
\end{enumerate}
\end{exoo}
















%\begin{Maquette}[DevS]{Prive=false,Date=10/24}
%\begin{exercice}[BaremeDetaille=true] % ici le barème est total, pas de détail : comportement par défaut
%On considère les expressions $A=2x(3x+5)$ et $B=x(7x-1)$.
%\begin{enumerate}
%\item \brm{2} Développer l'expression $A$.\newline
%\notes[15pt]{3}{\linewidth}
%\fullwidth{yahya $\cdotsx{15}$}
%\item \brm{1.5} Développer l'expression $B$.
%\end{enumerate}
%\end{exercice}
%
%\begin{exercice}[BaremeDetaille=true]%ici le barème est total ET détaillé
%On considère les expressions $A=2x(3x+5)$ et $B=x(7x-1)$.
%\begin{enumerate}
%\item\brm{2.5} Développer l'expression $A$.\newline
%\anserline[4]
%\item Développer l'expression $B$.
%\anserline[5]
%\begin{enumerate}
%\item\brm{6} yahkdi qie dk\newline\begin{EnvUplevel}\anserline[3]\end{EnvUplevel}
%\end{enumerate}
%\end{enumerate}
%\end{exercice}
%
%\end{Maquette}


%\begin{Maquette}[Fiche]{Theme=Algorithmique}
%\colorlet{PfMColCpt}{red}
%\colorlet{PfMColSrc}{blue}
%\begin{exercice}[Source=Olympiades 2019,Titre=Modifier des mots,Competence=Raisonner]
%Dans ce problème, on appellera {\em mot} toute suite de lettres formée des
%lettres A, D et G. Par exemple : ADD, A, AAADG sont des {\em mots}.
%\\Astrid possède un logiciel qui fonctionne de la manière suivante : un
%utilisateur entre un {\em mot} et, après un clic sur EXÉCUTER, chaque
%lettre A du {\em mot} (s'il y en a) est remplacée par le {\em mot}
%AGADADAGA. Ceci donne un nouveau {\em mot}.\\Par exemple, si l'
%utilisateur rentre le {\em mot} AGA, on obtient le {\em mot}
%AGADADAGAGAGADADAGA. Un deuxième clic sur EXÉCUTER réitère la
%transformation décrite ci-dessus au nouveau {\em mot}, et ainsi de suite
%.
%\begin{enumerate}
%\item Quels sont les {\em mots} qui restent inchangés quand on clique sur
%EXÉCUTER ?
%\end{enumerate}
%\begin{enumerate}
%\item  Développer l'expression $A$.
%
%\item  Développer l'expression $B$.
%
%\end{enumerate}
%\end{exercice}
%\end{Maquette}
%------------------------------------------------------
\end{document}