\documentclass[12pt,a4paper]{book}
\usepackage[utf8]{inputenc}
%\usepackage[french]{babel}
\usepackage[T1]{fontenc}
\usepackage{amsmath,amssymb,amsfonts,amsthm}
\usepackage{mathtools,thmtools}
\usepackage[explicit]{titlesec}
\usepackage{titletoc}
\usepackage[many]{tcolorbox}
\tcbuselibrary{skins}
\usepackage{array}
\usepackage{pifont}
\usepackage{xcolor,colortbl}
\usepackage{tikz}
\usepackage[framemethod=tikz]{mdframed}
\usetikzlibrary{matrix,fit,calc,shadows,shadows.blur}
\usepackage[left=2cm,right=2.5cm,top=2.5cm,bottom=2.5cm]{geometry}
\usepackage{eso-pic}
\usepackage{tikzpagenodes}
\usepackage{nopageno}
%============================================
\definecolor{myblue3}{RGB}{36, 57, 126}
\definecolor{YankeesBlue}{RGB}{28,40,65}
\definecolor{mygray}{RGB}{112,121,139}
%=================================================
%===========================================
\definecolor{myblue}{RGB}{0,82,155}
\definecolor{myblueii}{RGB}{0, 0, 100}
\definecolor{MainRed}{rgb}{.6, .1, .1}
\definecolor{berry}{RGB}{235, 37, 129}
\definecolor{Kgolden}{RGB}{255, 188, 0}
\definecolor{golden}{RGB}{255,223,0}
%===========================================
\definecolor{coralred}{rgb}{1.0, 0.25, 0.25}
%===========================================
\definecolor{definitioncolor}{gray}{0.65}
\definecolor{clight2}{RGB}{212, 237, 244}
\definecolor{col}{RGB}{31, 170, 31}
\definecolor{colf}{RGB}{208, 61, 66}
\definecolor{ocre}{RGB}{52,177,201}
\definecolor{ultramarine}{RGB}{0,45,97}
\definecolor{mybluei}{RGB}{0,173,239}
\definecolor{myblueii}{RGB}{63,200,244}
\definecolor{myblueiii}{RGB}{199,234,253}
\definecolor{MainRed}{rgb}{.6, .1, .1}
\definecolor{berry}{RGB}{235, 37, 129}
\definecolor{Kgolden}{RGB}{255, 188, 0}
\definecolor{coralred}{rgb}{1.0, 0.25, 0.25}
%===========================================
\definecolor{shadow@color}{cmyk}{.07,0,0,0.49}
\definecolor{col1}{RGB}{12, 102, 98}
\definecolor{col2}{RGB}{248, 193, 12}
\definecolor{col3}{RGB}{236, 67, 1} 
%=========================================
\definecolor{shamrockgreen}{rgb}{0.0, 0.62, 0.38}
\definecolor{rose}{rgb}{1.0, 0.0, 0.5}
\definecolor{richlavender}{rgb}{0.67, 0.38, 0.8}
\definecolor{tangelo}{rgb}{0.98, 0.3, 0.0}
% % % % % % % % % % % % % % % % % % % % % % %
%===========================================
\titlespacing*{\chapter}{0pt}{2cm}{3.5cm}%left,top,bottom
\titleformat{\chapter}[display]
{\startcontents\bfseries\sffamily\color{MainRed}}
{
\begin{tikzpicture}[remember picture,overlay]
%\path (current page.north west) coordinate (A); 
\coordinate (D) at ([yshift=-1cm]current page.north);
\path[fill=yellow](D)--++(0,-6)--++(2.5cm,-3mm)--++(2.5,3mm)--++(0,6) --cycle;
\node[scale=9.5,font=\color{red}] (num) at ([shift={(2.5,-2)}]D) {\thechapter};
\path[fill=cyan]([yshift=-4cm]D)--++(0,-1cm+2mm) -- ++(2mm,-2mm) --++(5cm-4mm,0) --++ (2mm,2mm)--++(0,1cm-2mm)--cycle;
\path[fill=cyan]([yshift=-.75cm]current page.north west) rectangle ([yshift=-1cm-3pt]current page.north east);
\node[scale=2.2,color=coralred] at ([yshift=-2.5cm]num) {\chaptertitlename};
\node[minimum height=1.5cm, fill=white,font=\color{coralred},text width=5cm]  at ([shift={(-3,-6.25)}]num) [scale=2.5,align=left,]{#1};
%==============================================
\path 
([shift={(-11,-6.25)}]num) coordinate (A);
%triangle
\def\r{1}
\foreach \c in {0,1,2}{
 \path 
 ($(A)+(.1,-.1)+(120*\c:\r)$)coordinate(T\c) ;
}
\begin{scope}[transform canvas={xshift=0.04cm,yshift=-0.05cm}]
\fill [black, opacity=0.3] 
(T0) -- (T1) -- (T2) -- cycle; 
\end{scope}
\shade [top color=coralred!60!white, bottom color=coralred] (T0) -- (T1) -- (T2) -- cycle; 
\end{tikzpicture}}
{10ex}
{}
[]
%------------------------------------------------------
\titleformat{name=\chapter,numberless}[display]
{\Huge\bfseries\sffamily\color{MainRed}}
{#1}
{1ex}
{\titlerule[2pt]\vspace*{5ex}\huge\normalfont}
%------------------------------------------------------
\titleformat{\section}
{\sffamily\Large\bfseries\color{cyan}}
{}
{1em}
{%
 \begin{tikzpicture}
 \node[name=r1, rectangle, fill=cyan, anchor=north west, font=\bfseries\color{white},
 inner sep=3pt, minimum width=40mm, minimum height=10mm,
 xshift=3mm, align=center, text width=10cm,
 ] {#1};
 \node[name=c, rectangle, fill=blue, font=\color{white}, anchor=north west,
 minimum width=10mm,
 ] {\thesection};
 \draw[cyan, line width=5pt] ($(c.north east)+(0,-.5\pgflinewidth)$)
 -- ({(\textwidth)},{-.5\pgflinewidth});
 \end{tikzpicture}
}
%==============================================
%=============================================
% % % % % % % % % % % % % % % % % % % % % % %
% Définitions environment
\declaretheoremstyle[
headfont=\large\normalfont\bfseries\color{MainRed},
notefont=\mdseries, notebraces={(}{)},
bodyfont=\normalfont,
postheadspace=0.5em,
mdframed={
 skipabove=\topsep,
 skipbelow=\topsep,
 hidealllines=true,
 backgroundcolor={definitioncolor!10},
 innerleftmargin=0pt,
 innerrightmargin=0pt,
}
]{mystyle}
\declaretheorem[style=mystyle,name=D\'{e}finition]{definition}
\newenvironment{Definition}[1]
{\renewcommand\thedefinition{#1}\begin{definition}}
 {\end{definition}}
\numberwithin{definition}{chapter}
% % % % % % % % % % % % % % % % % % % % % % %
% % % % % %% Example environment
\theoremstyle{plain}
\newmdtheoremenv[%
innertopmargin=0pt,% make the frame "tight"
innerbottommargin=0pt,%
rightline=false,% Kill all the lines except the left one
topline=false,%
bottomline=false,%
linecolor=gray,%
linewidth=4pt,%
leftmargin=10pt%
ntheorem = true% since we are using ntheorem to configure the style
]{example}{\color{blue!30!black}Exemple}[chapter]
% % % % % % % % % % % % % % % % % % %
\AddToShipoutPictureBG{
 \begin{tikzpicture}[remember picture,overlay]
 \fill[cyan]([yshift=1cm]current page.south west)rectangle(current page.south east) ;
 \node[circle,draw=white,line width=2pt,minimum size=1cm,fill=cyan]at([yshift=0.9cm]current page.south){\bfseries\Large\textcolor{MainRed}{\thepage}};
 \end{tikzpicture} }
% % % %
\begin{document}
\thispagestyle{empty}
\chapter{Matrices}
\section{Notions générales}
\begin{definition}
 Une Matrice est un tableau rectangulaire de la forme
 \begin{equation*}
 A = 
 \left[
 \begin{array}{ccc>{\columncolor{clight2}}c cc} 
 a_{11} & a_{12} & \cdots &  a_{1j} & \cdots & a_{1n}\\ 
 a_{21} & a_{22} & \cdots &  a_{2j} & \cdots & a_{2n}\\ 
 \vdots & \vdots &\vdots & \vdots & \vdots&\vdots \\
 \rowcolor{clight2} a_{i1} & a_{i2} & \cdots & \cellcolor[gray]{.6}a_{ij} & \cdots & a_{in}\\
 \vdots & \vdots &\ddots & \vdots & \vdots &\vdots \\
 a_{m1} & a_{m2} & \cdots & a_{mj} & \cdots & a_{mn}
 \end{array}
 \right]
 \end{equation*}
 où les $a_{ij}$ sont des nombres réels appelés les éléments ou coefficients de la matrice $A$. La matrice précédente est aussi notée par $\left(a_{ij}\right),\;i=1,\cdots,m,\;j=1,\cdots,n$, ou simplement par $\left(a_{ij}\right)$. L'élément $a_{ij}$ est situé à l’intersection de la $i$-ème ligne et de la $j$-ème colonne. Une matrice ayant $m$ lignes et $n$ colonnes
 est appelée une matrice d'ordre $(m,n)$, ou de dimension $m\times n$. Les matrices seront notées habituellement par des lettres capitales $A,\,B,\cdots$, et les éléments par des lettres minuscules $a,\,b,\cdots$.
\end{definition}
\begin{example}:
 On considère les matrices suivantes:
 \begin{equation*}
 A=
 \begin{bmatrix}
 1 & -1 & 2\\
 3 &  \frac{1}{2} & \sqrt{2}
 \end{bmatrix},\;
 B=
 \begin{bmatrix}
 -\sqrt{3} \\
 \frac{1}{5}\\
 1
 \end{bmatrix},\;
 C=
 \begin{bmatrix}
 -1 & 0 & 1 & 5
 \end{bmatrix},\;
 D=
 \begin{bmatrix}
 -1 & 1 & 3 \\
 5 & 6 & 2 \\
 1 & -1 & 0
 \end{bmatrix}.
 \end{equation*}
 \begin{enumerate}
 \item La matrice $A$ est de dimension $2\times 3$ et on a $a_{23}=\sqrt{2},\quad a_{13}=2,\quad a_{22}=\frac{1}{2}$.
 \item La matrice $B$ est de dimension $3\times 1$ et on a $b_{11}=-\sqrt{3},\quad b_{21}=\frac{1}{5},\quad b_{31}=1$.
 \item La matrice $C$ est de dimension $1\times 4$ et on a $C_{11}=-1,\quad C_{12}=0,\quad C_{13}=1,\quad C_{14}=5$. 
 \item La matrice $D$ est de dimension $3\times 3$ et on a $D_{33}=0,\quad D_{23}=2,\quad D_{32}=-1$. 
 \end{enumerate} 
\end{example}
\section{Matrices particulières}
\begin{definition}
 Une matrice ligne est une matrice comportant une seule ligne. Une matrice ligne a donc
 pour dimension $1\times n$. Une matrice ligne a la forme suivante:
 \begin{equation*}
 A=
 \begin{bmatrix}
 a_{11} & a_{12} & \cdots & a_{1n}
 \end{bmatrix}_{1\times n}.
 \end{equation*}
\end{definition}
\begin{example}: La matrice
 $
 A=
 \begin{bmatrix}
 1 & 2 & -1 & 5 & 0
 \end{bmatrix}
 $ est une matrice ligne de dimension $1\times 5$.
\end{example}
\begin{definition}
 Une matrice colonne est une matrice comportant une seule colonne.  Une matrice colonne a donc
 pour dimension $m\times 1$. Une matrice colonne a la forme suivante:
 \begin{equation*}
 A=\begin{bmatrix}
 a_{11}  \\
 a_{21}  \\
 \vdots  \\
 a_{n1}
 \end{bmatrix}_{n\times 1}.
 \end{equation*}
\end{definition}
\begin{example}: La matrice
 $
 A=
 \begin{bmatrix}
 -1 \\
 0 \\
 2 \\
 4
 \end{bmatrix}$ est une matrice colonne de dimension $4\times 1$.
\end{example}
\section{Opérations sur les matrice}
\begin{definition}
 Soit $A$ et $B$ deux matrices de même dimension. La somme de $A$ et $B$, écrite $A+B$, est la matrice obtenue en ajoutant les éléments correspondants des deux matrices.
 \begin{equation*}
 \begin{aligned}
 &\text{Si}\quad A=\begin{bmatrix}
 a_{11} & a_{12} & a_{13}\\
 a_{21} & a_{22}  & a_{23}
 \end{bmatrix}_{2\times 3}\;\text{et}\quad
 B=\begin{bmatrix}
 b_{11} & b_{12} & b_{13}\\
 b_{21} & b_{22}  & b_{23}
 \end{bmatrix}_{2\times 3},
 \\ \ \\
 &\text{alors}\quad A+B=\begin{bmatrix}
 a_{11}+b_{11} & a_{12}+b_{12} & a_{13}+b_{13}\\
 a_{21}+b_{21} & a_{22}+b_{22}  & a_{23}+b_{23}
 \end{bmatrix}_{2\times 3}.
 \end{aligned}
 \end{equation*}
\end{definition}
\begin{example}: On considère les matrices: 
 \begin{equation*}
 A=
 \begin{bmatrix}
 -2 & 0 & 1\\
 1 & 3  & -1\\
 \end{bmatrix},\;
 B=
 \begin{bmatrix}
 1 & 2 \\
 1 & 0 \\
 0 & 4
 \end{bmatrix},\;
 C=
 \begin{bmatrix}
 -2 & 1 \\
 4 & 3 \\
 -1 & -1
 \end{bmatrix},\;
 D=
 \begin{bmatrix}
 1 & 1 & 0\\
 2 & 0 & 4
 \end{bmatrix}.
 \end{equation*}
 \begin{itemize}
 \item[\ding{43}] Calchler $B+C$.
 \begin{equation*}
 B+C=\begin{bmatrix}
 1 & 2 \\
 1 & 0 \\
 0 & 4
 \end{bmatrix}+
 \begin{bmatrix}
 -2 & 1 \\
 4 & 3 \\
 -1 & -1
 \end{bmatrix}=
 \begin{bmatrix}
 1+(-2) & 2+1 \\
 1+4 & 0+3 \\
 0+1 & 4+(-1)
 \end{bmatrix}=
 \begin{bmatrix}
 -1 & 3 \\
 5 & 3 \\
 1 & 3
 \end{bmatrix}.
 \end{equation*} 
 \item[\ding{43}] Calculer $A+D$.
 \begin{equation*}
 A+D=\begin{bmatrix}
 -2 & 0 & 1 \\
 1 & 3  & -1
 \end{bmatrix}+
 \begin{bmatrix}
 1 & 1 & 0\\
 2 & 0 & 4
 \end{bmatrix}=
 \begin{bmatrix}
 -2+1 & 0+1 & 1+0\\
 1+2 & 3+0 & -1+4
 \end{bmatrix}=
 \begin{bmatrix}
 -1 & 1 & 1\\
 3 & 3 & 3
 \end{bmatrix}.
 \end{equation*}
 \item[\ding{43}]  La somme de $A$ et $B$ n’est pas définie car $A$ et $B$ ne sont pas de même dimension.
 \end{itemize}
\end{example}
\end{document}
