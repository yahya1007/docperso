\documentclass{article}
   \usepackage{statistics}

   \begin{document}

   \StatsSortData \mydata = { 2, 11=8, 6=3, 2=2, 11=1 }
 \def \rawdata { 2=2, 11=9, 6, 2, 6, 6, 11}
 \StatsSortData \yourdata = \rawdata
 mydata contains [\mydata]\\
 yourdata contains [\yourdata]
   
 \StatsRangeData \facebook = { 0, 1, 1.5, 1.5, 2, 3, 2.4, 2, 2.4=5,
 3, 4=10, 5=6, 6=9, 6.5=5, 7, 7.1, 7.2,
 7.3, 7.4, 7.5, 7.6, 7.7, 7, 7, 8, 8, 8,
 9=5, 12=12}
 (\IN[0;1;[, \IN[1;2;[, \IN[2;4;[,
 \IN[4;7;[, \IN[7;10;[, \IN[10;14;[)
%\detokenize\expandafter{\facebook}



%\statisticssetup{table/values=Caractere}
% \statisticssetup[table]{counts=Effets}
% \StatsTable \facebook

\statisticssetup{table/showonly/hidden=\color{white}#1}
 
 
 \StatsTable \facebook[ values=Caractere,counts=Effet, frequencies, frame=full, showonly=2-4 ]
 
 
 
 \def \combdata { 36=3, 37=8, 38=2, 39=6, 40=6, 41=3, 42=2, 45=2, 46=2 }
 \StatsTable \combdata[ values=Caractere,counts=Effet, frame=full ]
 
 \StatsGraph \combdata[ values=Caractere,counts=Effet]















\end{document}