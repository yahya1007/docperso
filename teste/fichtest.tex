\documentclass[a4paper,12pt]{article}


\usepackage{TestProfModels}
\usepackage{diagrammes}
%\usepackage{ProfMaquette}
 
\begin{document}

\begin{Maquette}[DevS]{Theme=Fonctions linéaires et affines,Niveau=3,Date=21/05/2024,Calculatrice}

%\begin{exercice}
%Soit $f$ une fonction tel que $f(5)=4$.
%\begin{enumerate}
%\item Détermine le coefficient de $f$.
%\item Donne l'éxpression de la fonction $f$.
%\item Calcule les images de $10$ et $\dfrac{-7}{2}$ par $f$.
%\item Quel nombre a pour image $12$ par $f$.
%\item Sans calcul, détermine la valeur de $\dfrac{f(1436)}{1436}$
%\item Représenter graphiquement $f$ dans un repère orthonormé $\oij$.
%\end{enumerate}
%\end{exercice}
%
%\begin{exercice}
%Soit $f$ une fonction affine tel que $f(4)=7$ et $f(7)=13$.
%\begin{enumerate}
%\item Déterminer la fonction $f$.
%\item Calculer les images de $5$ et $0$.
%\item Quel nombre a pour image 12 par $f$.
%\item Représenter graphiquement $f$ dans un repère orthonormé $\oij$.
%\end{enumerate}
%\end{exercice}
%
%\begin{exercice}
%\begin{minipage}{.5\linewidth}
%\begin{AffRepere}[-4][3][-2][4]
% \draw[domain=-1.5:1.5] plot(\x, 2*\x+1) node[above]{$(D)$};
%  \draw[domain=-4:3] plot(\x, -0.5*\x) node[right]{$(\Delta)$};
%  \draw[domain=-2:4] plot(2,\x ) node[right]{$(\delta)$};
% \end{AffRepere}
%\end{minipage}
% \begin{minipage}{.5\linewidth}
%  $(D)$ et $(\Delta)$ sont les représentations graphiques respectives des fonctions $f$ et $g$.
% \begin{enumerate}
% \item Déterminer la nature de $f$ et $g$.
% \item Quel est l'image de $1$ et $-1$ par $f$.
%  \item Quel est l'image de $2$ par $g$.
% \item Calculer le coefficient de $f$ puis déterminer sa formule.
%  \item Calculer le coefficient de $g$ puis déterminer sa formule.
% \end{enumerate}
% \end{minipage}
%\end{exercice}
%
%
%\begin{exercice}
%Soit $h$ une fonction affine et $(D)$ sa représentation graphique passant par $A(-1,4)$ et $B(1,-2)$.
%\begin{enumerate}
%\item Représenter graphiquement la fonction $h$.
%\item Déterminer le coefficient de $h$.
%\item Donner l'expression de $h$.
%\end{enumerate}
%\end{exercice}
%\newpage
%
%\begin{exercice}
%Soit $g$ la fonction affine définit par $g(x)=2x+5$.
%\begin{enumerate}
%\item Calculer $g(0)$ , $g(-3)$ et $g(1-\sqrt{3})$.
%\item Représenter graphiquement $g$ dans un repère orthonormé $\oij$.
%\item Est-ce que la représentation graphique de $g$  passe par le point $A(101,205)$.
%\item Quel nombre a pour image 13 par $g$. 
%\item Sans calcul, détermine la valeur de $\dfrac{g(2023)-g(1444)}{2023-1444}$.
%\end{enumerate}
%\end{exercice}
%
%\begin{exercice}
%Soit $f$ une fonction linéaire telle que : $f(6)=4$, et $g$ une fonction affine telle que : $g(5)-g(2)=-3$ et $g(0)=5$.
%\begin{enumerate}
%\item Vérifier que l'expression de $f$ est : $f(x)=\dfrac{2}{3}x$.
%\item Déterminer le nombre dont l'image par $f$ est 2.
%\item Montrer que le coefficient de $g$ est -1.
%\item Vérifier que l'expression de $g$ est : $g(x)=-x+5$.
%\item Déterminer l'image de 3 par $g$.
%\item Soient $(D)$ la représentation graphique de $f$ et $(\Delta)$ la représentation graphique de $g$ dans un repère $\oij$.
%\begin{enumerate}
%\item Construire $(D)$ et $(\Delta)$.
%\item Résoudre $f(x)=g(x)$.
%\end{enumerate}
%\end{enumerate}
%\end{exercice}
%
%
\end{Maquette}
\end{document}