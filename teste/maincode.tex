%=============================================================================================================
%=============================================================================================================
%==============================This Template Was Created By Mr. O. Nabil======================================
%=============================================================================================================
%=============================================================================================================
\documentclass[12pt,a4paper]{report}
\usepackage[top=0.75cm,bottom=1cm,right=.85cm,left=.85cm]{geometry}

%======================= All Libraries You Need For This Template =============================================
\usepackage{lmodern}
%\usepackage[french]{babel}
\usepackage[utf8]{inputenc}
\usepackage[T1]{fontenc}
\usepackage{tcolorbox,tikz,everypage,amsmath,etoolbox,pifont,multicol,colortbl,array,xparse}
\usepackage{lipsum,tasks}
\tcbuselibrary{skins,breakable}
\usepackage{amsmath,amsfonts,amssymb,mathrsfs,tikz,fancyhdr,array, cases}
\usepackage{fourier}
\usetikzlibrary{patterns}
\usepackage{fancyhdr,lastpage}
\newcommand{\R}{\mathbb{R}}
\newcommand{\N}{\mathbb{N}}
\newcommand{\C}{\mathbb{C}}
\newcommand{\Z}{\mathbb{Z}}
\newcommand{\Q}{\mathbb{Q}}
\usepackage[]{anttor}
\everymath{\displaystyle}

%================================ For Enumerate ==============================================
\renewcommand{\labelenumi}{\tcbox[enhanced,colback=teal,size=fbox,frame hidden,arc=1mm,box align=base,colupper=white,fontupper=\bfseries]{\arabic{enumi}}}
\renewcommand{\labelenumii}{\tcbox[enhanced,colback=lime,size=fbox,arc=1mm,box align=base,colupper=white,fontupper=\bfseries]{\alph{enumii}}}

%===================================== For Pages Border =======================================
\parindent=0mm
\usepackage{tikz}
\usetikzlibrary{positioning,shadows.blur,shapes.multipart,shapes.geometric}
\usetikzlibrary{decorations.pathmorphing}
\tikzset{render blur shadow/.prefix code={\colorlet{black}{gray}}}
\AddEverypageHook{%
	\tikzpicture[remember picture,overlay]
	\coordinate (ne) at ([shift={(-.5,-.5)}]current page.north east);
	\coordinate (nw) at ([shift={(.5,-.5)}]current page.north west);
	\coordinate (sw) at ([shift={(.5,.65)}]current page.south west);
	\coordinate (se) at ([shift={(-.5,.65)}]current page.south east);
	\node[text opacity=.75]at(current page){};
	\draw[teal!85!black,line width=.68mm] (ne)rectangle(sw);
	\draw[fill=white, lime!3!white] ([shift={(-.15,-.15)}]ne)rectangle([shift={(.15,.15)}]sw);
	\foreach \ico/\nm in {ne/M,nw/N,se/E,sw/O}{
		\node[draw = lime!75!black, color=white, fill = teal!67!black, minimum width = .5cm,rotate=.5] at(\ico) {\nm};
	}
	\endtikzpicture
}

%================================== For Exercies Box ===============================================================
\newcounter{myexo}% preamble

\newtcolorbox[use counter=myexo,number format=\arabic]{mybox}[2][]{%
	enhanced jigsaw, breakable,
	arc=2mm,
	arc is angular,
	center,
	attach boxed title to top left={xshift=1cm,yshift=-5mm},
	top=8mm,
	left=3mm,
	right=3mm,
	bottom=4mm,
	coltitle=white,
	boxed title style={
		colback=teal!65!white,
		colframe=teal!75!white,
		leftrule=3pt, toprule=0pt, bottomrule=0pt, rightrule=0pt,
		bottom=2pt,
		drop large lifted shadow=teal!78!white,
		sharp corners
	},
	watermark color=teal!65!white,
	watermark opacity=.35,
	watermark tikz={\node[rotate=30]at(0,0){\textsf{Prof. O. Nabil}};},
	drop fuzzy shadow=teal!78!white,
	colback=teal!5!white,
	colframe=teal!65!white,
	fonttitle=\bfseries,title=\Huge Exercie \thetcbcounter: #2,#1}

%================================= For Foote Pages =================================================
\newtcbox{\fpage}[1][red]{enhanced, rotate=30,
	drop fuzzy shadow=lime!78!black,
	colupper=white,
	on line,arc=2pt,colback=#1!65!white,colframe=#1!50!black,before upper={\rule[-3pt]{0pt}{10pt}},boxrule=1pt,boxsep=0pt,left=6pt,right=6pt,top=2pt,bottom=2pt}
\pagestyle{fancy}
\cfoot{
	\fpage[teal]{\Large\thepage}	
}


%========================= The Body Of Code =========================================
\begin{document}
	
	%================ The Head Of Page =================================
	\tcbset{
		enhanced,
		fonttitle=\bfseries\large,
		fontupper=\large\sffamily,
		colback=teal!3!white,
		colframe=teal!70!black,
		colupper=black!77!white,
		drop fuzzy shadow=teal!67!white,
		top=.3mm,left=.3mm,right=.3mm,bottom=.3mm,
		arc is angular,
		arc=2mm,
		center title}
	\begin{tcolorbox}
		\renewcommand{\arraystretch}{2}
		\renewcommand{\arrayrulewidth}{.38mm}
		\arrayrulecolor{teal!65!white}
		\begin{tabular}{|m{.3\linewidth}|>{\centering\bfseries}m{.35\linewidth}|m{.275\linewidth}|}
			\hline
			Lycée : ............
			&
			\textbf{\Large Test diagnostique}
			&
			Niveau : ..............
			\\\cline{1-1}\cline{3-3}\vglue1mm
			Année scolaire : 2024/2025
			&\vglue1mm
			Durée :...... 
			&\vglue1mm
			\large Prof. O. Nabil
			\\\hline
		\end{tabular}
	\end{tcolorbox}
	
	%====================== Satart Exercise One ==========================
	\begin{mybox}{}
		Déterminer l'ensemble de définition de la fonction $f$ dans chacun des cas suivants:\\
		\begingroup
		\arraycolsep=.2cm%
		\begin{displaymath}	
			\begin{array}{llll}
				1)f(x)=\frac{6x+7}{x^3-3x^2+2x} & 2)f(x)=\frac{\sqrt{1-2x}}{x^2-2x-8} & 3)f(x)=\sqrt{\frac{2x+3}{5-x}} &  4)f(x)=\frac{\sqrt{2x+3}}{\sqrt{5-x}}\\ 
				5)f(x)=\sqrt{\frac{x^2-2x-15}{x^2-x-2}} & 6)f(x)=\frac{x}{\sqrt{x^2-x+3}} &
				7)f(x)=\frac{\sin x}{5-\left\lvert x - 2\right\rvert} & 8)f(x)=\frac{3x-4}{\sqrt{\left\lvert x\right\lvert-2}}\\
			\end{array}%
		\end{displaymath}
		\endgroup
	\end{mybox}
	%====================== End Exercise One =============================
	
	%====================== Satart Exercise Two ==========================
	\begin{mybox}{}
		\begin{enumerate}
			\item Calculer les limites suivantes:\\
			\begin{tabular}{lll}
				1) $\lim\limits_{x \to 2}(5x^2-3x-2)$ & 
				2) $\lim\limits_{\substack{x \to 4\\x>4}}\frac{x}{2-\sqrt{x}}$ &
				3) $\lim\limits_{x \to 1}\frac{2x-7}{(x-1)^2}$ \\
				4) $\lim\limits_{x \to +\infty}(x^3-4x^2+5x-1)$ &
				5) $\lim\limits_{x \to +\infty}\frac{3x^2-7x+4}{(2x+1)^3}$ &
				6) $\lim\limits_{x \to -\infty}\frac{5x^3+4x-1}{(x-2)(x^2+1)}$ \\
				7) $\lim\limits_{x \to -\infty}\sqrt{x^2+x}+2x+5$ &
				8) $\lim\limits_{x \to +\infty}\frac{\sqrt{x+1}}{x+3}$ &
				9) $\lim\limits_{x \to 1}\frac{x^2-1}{2x^2-5x+3}$ \\
				10) $\lim\limits_{x \to 1}\frac{x^3+3x^2-4}{x-1}$ &
				11) $\lim\limits_{x \to 3}\frac{x^3+2x^2-11x-12}{x^4-81}$ &
				12) $\lim\limits_{x \to 2}\frac{x^2+x-6}{\sqrt{x}-\sqrt{2}}$\\
				13) $\lim\limits_{\substack{x \to -2\\x<-2}}\frac{\sqrt{x^2+x-2}}{x^2-4}$ &
				14) $\lim\limits_{x \to +\infty}\frac{\sqrt{x^2+3x}-x}{\sqrt{2x^2+x-5}}$ &
				15) $\lim\limits_{x \to +\infty}\frac{x\sqrt{x}}{x-3\sqrt{x}}$ \\
				16) $\lim\limits_{x \to -\infty}(\sqrt{x^2+3x+2}+x+1)$ & &\\
			\end{tabular}
			\item Calculer les limites trigonométriques suivantes:\\
			\\		   
			\begin{tabular}{llll}
				1) $\lim\limits_{x \to 0}\frac{\sin(7x)}{4x}$ & 
				2) $\lim\limits_{x \to 0}\frac{\tan(4x)}{\sin(3x)}$ & 
				3) $\lim\limits_{x \to 0}\frac{2x-\tan(3x)}{x+\sin(4x)}$ &  
				4) $\lim\limits_{x \to 0}\frac{1-\cos(4x)}{\sin(2x)\tan(3x)}$\\
				5) $\lim\limits_{x \to 0}\frac{\sqrt{x+4}-2}{\sin x}$ & 
				6) $\lim\limits_{x \to +\infty}\frac{\sin x}{x^2+1}$ &
				7) $\lim\limits_{x \to -\infty}\frac{1+x+\cos x}{x-1}$ & 
				8) $\lim\limits_{x \to +\infty}\frac{x^2+1+\sin x}{x}$\\
			\end{tabular}%
		\end{enumerate}
	\end{mybox}
	%====================== End Exercise Two =============================
	
	%====================== Satart Exercise Three ==========================
	\begin{mybox}{}
		Soit $f$ la fonction numérique définie sur $\R-\{1\}$ par:\\
		\\
		$\left\{
		\begin{array}{l}
			f(x)=\frac{\sqrt{x^2-x}}{x-1}\; \quad si x>1\\
			\\
			f(x)=\frac{x^2-5}{\sqrt{1-x}}\; \quad si x<1
		\end{array}\right.$\\
		
		\begin{enumerate}
			\item Calculer les limites suivantes:\\
			$\lim\limits_{\substack{x \to 1\\ x<1}}f(x)$\; ;\qquad $\lim\limits_{\substack{x \to 1\\ x>1}}f(x)$\; ;\qquad
			$\lim\limits_{x \to +\infty}f(x)$\; ;\qquad $\lim\limits_{x \to -\infty}f(x)$\; et \; $\lim\limits_{x \to -\infty}\frac{f(x)}{x}$
		\end{enumerate}
	\end{mybox}
	%====================== End Exercise Three =============================
	
\end{document}