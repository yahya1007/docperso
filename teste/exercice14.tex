
%compiler with xelatex
\documentclass[12pt,a4paper,dvipsnames,svgnames,x11names,table]{article}
\usepackage[top=1cm,bottom=1cm,right=1.2cm,left=2.2cm]{geometry}
\usepackage{pgfplots}
\pgfplotsset{compat=1.15}
\usepackage{mathrsfs}
\usetikzlibrary{arrows}
\usepackage{amsmath,amsfonts,amssymb,fancyhdr,txfonts,pxfonts}
%\usepackage[table]{xcolor}
\usepackage{tikz,xcolor,tcolorbox}
%\usepackage{amsmath}
\usetikzlibrary{calc}
\usepackage[object=vectorian]{pgfornament}
\tcbuselibrary{skins,xparse,hooks,vignette}
\usepackage{tikzrput}
\usepackage[tikz]{bclogo}
\usepackage{varwidth}
\usetikzlibrary{shapes}
\pagestyle{empty}
\renewcommand{\headrulewidth}{0pt}
\pagecolor{white}
\usepackage{polyglossia}
\setotherlanguage{french}
\newfontfamily\frenchfont[Scale=1.2]{Times New Roman}
\newfontfamily\frenchfontt[Scale=1.2]{Algerian}
\newfontfamily\frenchfonts[Scale=1.2]{Andalus}
\usepackage{eso-pic}
\AddToShipoutPicture{%
\begin{tikzpicture}[remember picture,overlay]
\fill[black,rounded corners=4mm]([shift={(-0.2,-0.2)}]current page.north east)rectangle([shift={(.25,.2)}]current page.south west);
\fill[white]([shift={(-.4,-.4)}]current page.north east)rectangle([shift={(.45,.4)}]current page.south west);
\foreach \i in {1,3,5}{
\path (current page.north west)--
node[right=0mm,pos=\i/6]{%
\tikzpicture
\foreach \j in {0,1,2,3}{
\node[circle,inner color=black,outer color=black,inner sep=1.2mm,outer sep=0mm,anchor=center](\j) at (0,-\j*.4){};
\path[rounded corners=1mm,inner color=white,outer color=black] (-1,-.08-\j*.4)rectangle(0,.08-\j*.4);}
\endtikzpicture
}
(current page.south west);
\fill[black]($(current page.north west)+(1.6cm,-1cm)$)rectangle($(current page.south west)+(1.7cm,1cm)$);
}
\node[black,rotate=90]at([xshift=1cm,yshift=9.9cm]current page.south west){\textit{{ {\color{black}{\frenchfontt{2022/2023}}}}}};
\node[black,rotate=90]at([xshift=1cm,yshift=19.6cm]current page.south west){\textit{{ {\color{black}{\frenchfontt{1BACSEG}}}}}};
\node[black,rotate=90]at([xshift=1cm,yshift=2.2cm]current page.south west){\textit{{ {\color{black}{\frenchfonts{}}}}}};
\node[black,rotate=90]at([xshift=1cm,yshift=27.2cm]current page.south west){\textit{{ {\color{black}{\frenchfonts{}}}}}};
\end{tikzpicture}}
%%%%%%%%%%%%%%%%%%%%%%%%%    myboxe1    %%%%%%%%%%%%%%%%%%%%%%%%%%%%%%
\newtcolorbox{myboxe1}[2][]{skin=enhancedlast jigsaw,interior hidden,
boxsep=0pt,top=0pt,colframe=black,coltitle=black,
fonttitle=\bfseries\sffamily,
attach boxed title to bottom center,
boxed title style={empty,boxrule=0.5mm},
varwidth boxed title=0.5\linewidth,
underlay boxed title={
\draw[white,line width=0.5mm]
([xshift=0.3mm-\tcboxedtitleheight*2,yshift=0.3mm]title.north west)
--([xshift=-0.3mm+\tcboxedtitleheight*2,yshift=0.3mm]title.north east);
\path[draw=black,top color=white,bottom color=white,line width=0.5mm]
([xshift=0.25mm-\tcboxedtitleheight*2,yshift=0.25mm]title.north west)
cos +(\tcboxedtitleheight,-\tcboxedtitleheight/2)
sin +(\tcboxedtitleheight,-\tcboxedtitleheight/2)
-- ([xshift=0.25mm,yshift=0.25mm]title.south west)
-- ([yshift=0.25mm]title.south east)
cos +(\tcboxedtitleheight,\tcboxedtitleheight/2)
sin +(\tcboxedtitleheight,\tcboxedtitleheight/2); },
title={#2},#1}
%%%%%%%%%%%%%%%%%%%%%%%%%    myboxe11    %%%%%%%%%%%%%%%%%%%%%%%%%%%%%
\newtcolorbox{myboxe11}[2][]{skin=enhancedlast jigsaw,interior hidden,
boxsep=0pt,top=0pt,colframe=black,coltitle=black,
fonttitle=\bfseries\sffamily,
attach boxed title to top center,
boxed title style={empty,boxrule=0.5mm},
varwidth boxed title=0.5\linewidth,
underlay boxed title={
\draw[black,line width=0.5mm]
([xshift=0.3mm-\tcboxedtitleheight*2,yshift=0.3mm]title.north west)
--([xshift=-0.3mm+\tcboxedtitleheight*2,yshift=0.3mm]title.north east);
\path[draw=black,top color=white,bottom color=white,line width=0.5mm]
([xshift=0.25mm-\tcboxedtitleheight*2,yshift=0.25mm]title.north west)
cos +(\tcboxedtitleheight,-\tcboxedtitleheight/2)
sin +(\tcboxedtitleheight,-\tcboxedtitleheight/2)
-- ([xshift=0.25mm,yshift=0.25mm]title.south west)
-- ([yshift=0.25mm]title.south east)
cos +(\tcboxedtitleheight,\tcboxedtitleheight/2)
sin +(\tcboxedtitleheight,\tcboxedtitleheight/2); },
title={#2},#1}
%%%%%%%%%%%%%%%tableau%%%%%%%%%%%%%%%%%%%
\setlength{\arrayrulewidth}{1pt}
%\setlength{\tabcolsep}{8pt}
%\renewcommand{\arraystretch}{2.5}
\newcolumntype{s}{>{\columncolor{black!30}} p{2cm}}
\arrayrulecolor{black}
%%%%%%%%%%%%%%%%%%%%%%%%%%%%%%%%%%%%%%
 \begin{document}
 \begin{myboxe1}{Devoir à la maison 3 S1 }
\emph{\textbf{\frenchfontt{Pr: R. IBOURK}}} \hfill \emph{\textbf{\frenchfontt{Lycée Allal bn Abdallah}}}         
\end{myboxe1}
\begin{myboxe11}{\frenchfontt {\emph{\textbf{\bcplume Exercice 1}}}}
       \bccrayon Soit $\left( U_n \right)$ la suite numérique définie par: $u_0=5$ et $u_{n+1}=\dfrac{4u_n-9}{u_n-2}$ pour tout $n$ de $\mathbb{N}$.
\begin{enumerate}
\item Calculer $u_1$ et $u_2$. 
\item Montrer par récurrence que pour tout $n$ de $\mathbb{N}$: $u_n>3$
\item \begin{enumerate}
\item Montrer que pour tout $n$ de $\mathbb{N}$: $ u_{n+1}-u_n=\dfrac{\left(u_n-3 \right)^2}{u_n-2}$ 
\item En déduire que la suite $\left( U_n \right)$ est une suite décroissante. 
\end{enumerate}
\item On pose pour tout $n\in \mathbb{N}$: $V_n=\dfrac{1}{u_n-3}$.
\begin{enumerate}
\item Calculer $V_0$.
\item Montrer que la suite $\left( V_n \right)$ est arithmétique de raison $1$ 
\item Montrer que $V_n=\dfrac{1}{2}+n$ pour tout $n$ de $\mathbb{N}$.
\item Vérifier que pour tout $n$ de $\mathbb{N}$ $u_n=\dfrac{3V_n+1}{V_n}$
\item En déduire que pour tout $n$ de $\mathbb{N}$: $u_n=\dfrac{6n+5}{2n+1}$
\end{enumerate}
\end{enumerate}
\end{myboxe11}
\begin{myboxe11}{\frenchfontt {\emph{\textbf{\bcplume Exercice 2}}}}
\bccrayon Une urne contient $4$ boules rouges et une seule boule verte. On tire successivement et sans remise deux boules dans cette urne
\begin{enumerate}
\item Construire l'arbre des cas possibles.
\item Montrer que le nombre des tirages possibles est $20$
\item Déterminer le nombre des tirages possibles contenant exactement deux boules rouges
%\item Déterminer le nombre des tirages possibles contenant exactement une boule verte
\end{enumerate}
\end{myboxe11} 
 \begin{myboxe11}{\frenchfontt {\emph{\textbf{\bcplume Exercice 4}}}}
  \bccrayon Un sac $A$ contient $7$ boules rouges, $4$ boules noires et $3$ boules vertes. On suppose que toutes les boules sont identiques.
On tire simultanément $3$ boules de ce sac.
\begin{enumerate}
\item Quel est le nombre des tirages possibles ? 
\item Quel est le nombre des tirages possibles contenant trois boules de  même couleurs ? 
\item Quel est le nombre des tirages possibles contenant exactement deux boules rouges ? 
\item Quel est le nombre des tirages possibles ne contenant aucune boule noire ?
\end{enumerate}
\end{myboxe11}
 \end{document}
