%https://tex.stackexchange.com/questions/157317/some-questions-about-titlepage-and-tikz
\documentclass[12pt,a4paper]{book}
\usepackage[utf8]{inputenc}
\usepackage[french]{babel}
\usepackage[T1]{fontenc}
\usepackage{amsmath,amssymb,amsfonts}
\usepackage[left=3cm,right=2.5cm,top=2.5cm,bottom=2.5cm]{geometry}
\usepackage[svgnames,x11names]{xcolor}
\usepackage{colortbl}
\usepackage{fancyhdr}
\usepackage[explicit]{titlesec}
\usepackage{tikz}
\usetikzlibrary{matrix,fit,calc,shadows,shadows.blur,shapes.symbols,positioning}
\usepackage[most]{tcolorbox}
\usepackage{pifont}
\usepackage{multicol}
\usepackage{enumerate}
\usepackage[shortlabels,inline]{enumitem}
\usepackage{eso-pic}
\usepackage{tikzpagenodes}
\usepackage{nopageno}
%===========================================================
%définition des couleurs
\definecolor{MainRed}{rgb}{.6, .1, .1}
\definecolor{mybluei}{RGB}{0,173,239}
\definecolor{bubblegum}{rgb}{0.99, 0.76, 0.8}
\definecolor{clight2}{RGB}{212, 237, 244}
\definecolor{blueslategray}{cmyk}{0.189, 0.091, 0, 0.439}
\definecolor{greenbrown}{HTML}{555544}
\definecolor{cottonseed}{cmyk}{0, 0.026, 0.047, 0.255}
\definecolor{lightblue}{cmyk}{0.272, 0.068, 0, 0.192}
\definecolor{bluetrout}{cmyk}{0.18, 0.15, 0, 0.608}
\definecolor{brownred}{cmyk}{0, 0.531, 0.557, 0.239}
\definecolor{orangewhite}{cmyk}{0, 0.031, 0.075, 0.004}
\definecolor{avocado}{cmyk}{0.06, 0, 0.55, 0.37}
\definecolor{eggplant}{cmyk}{0.04, 0.24, 0, 0.47}
\definecolor{darkblue}{rgb}{0.12,0.47,0.87}
%===============================================================
%===========================================
\definecolor{shadow@color}{cmyk}{.07,0,0,0.49}
\definecolor{col1}{RGB}{12, 102, 98}
\definecolor{col2}{RGB}{248, 193, 12}
\definecolor{col3}{RGB}{236, 67, 1} 
%========================================
%=================================================
\definecolor{gray}{RGB}{236,236,236}
\definecolor{shadow@color}{cmyk}{.07,0,0,0.49}
\definecolor{section@title@color}{cmyk}{1,0.2,0.3,0.1}
\definecolor{subsection@title@color}{cmyk}{0,0.6,0.9,0}
%=================================================
%========================================
%Commands
%newcommand % % % % %
\newcommand\z{\cellcolor{bubblegum}}
\newcommand\y{\cellcolor{clight2}}
% % % % % % % % % % % % % % % % % % % % % % %
\newcommand\highlight[1]{\tikz[overlay, remember picture,baseline=-\the\dimexpr\fontdimen5\textfont2\relax]\node[rectangle,fill=blue!50,rounded corners,fill opacity = 0.2,draw,thick,text opacity =1] {$#1$};} 
% % % % % % % % % % % % % % % % % % % % % % %
\newcommand{\tikzmark}[2]{\tikz[overlay,remember picture,baseline=(#1.base)] \node (#1) {#2};}
% % % % % % % % % % % % % % % % % % % % % % %
\newcommand\tikznode[3][]%
{\tikz[remember picture,baseline=(#2.base)]
	\node[minimum size=0pt,inner sep=0pt,#1](#2){#3};%
}
%==============================================
%=============================================
\newlength{\mylen}
\newlength{\hauteur}
\newlength{\malongueur}
\settowidth{\mylen}{Méthode :}
\setlength{\hauteur}{\baselineskip}
\addtolength{\hauteur}{0.5mm}
\settowidth{\malongueur}{408.52pt}
%\addtolength{\malongueur}{-\mylen}
%===========================================================
\renewcommand{\sfdefault}{lmss}%Latin Modern fonts
%============================================
%==============================================
%fancy chapters
\definecolor{titlepagecolor}{cmyk}{0,0,0,80}
\definecolor{titlepagecolor2}{RGB}{196,175,153}
\DeclareFixedFont{\chapternumberfont}{T1}{phv}{bx}{sc}{1.5in}
\DeclareFixedFont{\bigsf}{T1}{phv}{b}{n}{1cm}
\titlespacing*{\chapter}{0pt}{2cm}{2.5cm}%left,top,bottom
\titleformat{\chapter}[display]
{\bfseries\sffamily\color{Red}}
{\begin{tikzpicture}[remember picture, overlay]%
\node[
fill=section@title@color,
anchor=north west,
text width=\paperwidth,
text height=2cm,
text depth=2cm,
inner xsep=1cm,
font=\color{white}\bigsf 
] 
at ([yshift=-2.5cm]current page.north west) (blackrect) {#1};
% the khaki stripe
\path[fill=eggplant] 
(blackrect.north west) rectangle ++(\paperwidth,2cm);
% the banner
\node[
signal,
signal from=north,
signal pointer angle=150,
fill=orangewhite,
drop shadow={shadow xshift=0pt},
rotate=-90,
text width=6cm,
text height=5cm,
anchor=north west
] (banner) at ([xshift=-3cm,yshift=1cm]blackrect.north east) {};
% the image in the banner
\node[anchor=north] at ([yshift=-1.5cm]banner.west)
{\color{darkblue}\chapternumberfont\thechapter};
% the text in the banner
\node[
anchor=south,
text width=3cm,
align=center,
font=\sffamily\LARGE\color{darkblue}] 
at ([yshift=1.5cm]banner.east) {\chaptertitlename};
\end{tikzpicture}}
{1ex}
{}
[]
%------------------------------------------------------
\titleformat{name=\chapter,numberless}[display]
{\Huge\bfseries\sffamily\color{MainRed}}
{#1}
{1ex}
{\titlerule[2pt]\vspace*{5ex}\huge\normalfont}
%===============================================================
%%%%%%%%%%%%% Titre de section %%%%%%%%%%%%%%%%%%%%%
\renewcommand{\thesection}{\Roman{section}}
\def\sectiontitle@font{\fontfamily{ugq}\selectfont}
%================================================
\newcommand\SecTitle[4]{%
	\hspace*{-1.5cm}\begin{tikzpicture}
	\node[inner xsep=0pt,minimum height=3cm,text width=0.8\textwidth,
	align=left,left color=gray,right color=white, signal to=#1,font=\LARGE,anchor=#2] (A)
	at (#3,0) {\hspace*{2.25cm}#4};
	%===============================================================
	%==================================================
	\node (numsec) at ($0.5*(A.north west)-0.5*(A.north west)+(1,0)$) {\Huge\textbf{\thesection}};
	\fill[rounded corners=2pt,fill=shadow@color] ($(numsec.north west)+(2pt,-2pt)$) -- ($(numsec.north east)+(1mm,0mm)+(2pt,-2pt)$) -- ($(numsec.south east)+(2pt,-2pt)$) -- ($(numsec.south west)+(-1mm,0)+(2pt,-2pt)$) -- cycle;
	\fill[rounded corners=2pt,fill=section@title@color] (numsec.north west) -- ($(numsec.north east)+(1mm,0mm)$) -- (numsec.south east) -- ($(numsec.south west)+(-1mm,0)$) -- cycle;
	\node[white] at (numsec) {\LARGE\sectiontitle@font\thesection};
	\end{tikzpicture}%
}
%\titleformat{command}[shape]{format}{label}{sep}{before-code}{after-code}
\titleformat{\section}
{\sectiontitle@font\color{section@title@color}}{}{0em}
{\SecTitle{east}{west}{0\paperwidth}{#1}}

%======================================================
%%%%%%%%%%%%%%%%Titre de subsection %%%%%%%%%%%%%%%%%%%%%
\renewcommand{\thesubsection}{\arabic{subsection}}
\def\subsectiontitle@font{\fontfamily{ugq}\selectfont}
\newcommand\SubsecTitle[4]{%
	\hspace*{-1.5cm}
	\tikz[ overlay]{
		\node (num) at (1.5,0) {\subsectiontitle@font\thesubsection};
		\fill[rounded corners=2pt,fill=shadow@color] ($(num.north west)+(2pt,-2pt)$) -- ($(num.north east)+(1mm,0mm)+(2pt,-2pt)$) -- ($(num.south east)+(2pt,-2pt)$) -- ($(num.south west)+(-1mm,0)+(2pt,-2pt)$) -- cycle;
		\fill[rounded corners=2pt,fill=subsection@title@color] (num.north west) -- ($(num.north east)+(1mm,0mm)$) -- (num.south east) -- ($(num.south west)+(-1mm,0)$) -- cycle;
		\node[white]  at (num) {\subsectiontitle@font\thesubsection};
		\node[color=subsection@title@color, minimum height=.5cm,minimum width=15cm,text width=10cm, align=left,font=\subsectiontitle@font](title) at ([xshift=6cm]num){#4};
	}
}
%\titleformat{command}[shape]{format}{label}{sep}{before-code}{after-code}
\titleformat{\subsection}
{\subsectiontitle@font\color{subsection@title@color}}{}{0em}
{\SubsecTitle{east}{west}{0\paperwidth}{#1}}

%===============================================================
%=============================================
% ===== Environment "definition" ======================

\newtcolorbox[auto counter, number within=chapter]{definition}[1][]{%
	enhanced, breakable, fonttitle=\large\sffamily\bfseries,
	title=D,
	left=\tcboxedtitlewidth,
	boxsep=0pt, top=2mm, right=2mm, bottom=2mm+0.7\baselineskip,
	sharp corners, colback=white, colbacktitle=orange, frame hidden,
	pad before break=-0.7\baselineskip,
	before upper = {\begingroup
		\large\bfseries\textcolor{green!50!black}{éfinition}
		\textcolor{orange}{\thetcbcounter}\quad\endgroup},
	attach boxed title to top left={
		xshift=2pt,
		yshift=-\tcboxedtitleheight+1.5pt-2mm},
	boxed title style={
		empty, left=0mm, right=0mm, top=0mm, bottom=0mm, boxsep=2pt},
	underlay boxed title={
		\path[fill=orange] (title) circle (\tcboxedtitleheight/2);},
	overlay unbroken={
		\coordinate[xshift=\tcboxedtitlewidth/2,
		yshift=-\tcboxedtitleheight-0.7\baselineskip] (nw) at (frame.north west);
		\node (sw) [circle, fill=orange, inner sep=2pt,
		xshift=\tcboxedtitlewidth/2, yshift=\tcboxedtitlewidth/2]
		at (frame.south west) {};
		\draw[orange, line cap=round, line width=2pt,
		dash pattern=on 0pt off 4\pgflinewidth] (nw) -- (sw);},
	overlay first={
		\coordinate[xshift=\tcboxedtitlewidth/2,
		yshift=-\tcboxedtitleheight-0.7\baselineskip] (nw) at (frame.north west);
		\coordinate[xshift=\tcboxedtitlewidth/2,
		yshift=0.7\baselineskip] (sw) at (frame.south west);
		\draw[gray, line cap=round, line width=2pt,
		dash pattern=on 0pt off 4\pgflinewidth] (nw) -- (sw);},
	#1}
%=====================================================
% % % % % %% Example environment
\newcounter{examp}
\setcounter{examp}{0}

\newcommand{\mytitle}[2]%
{%
	\stepcounter{examp}
	\begin{tikzpicture}[font=\sffamily]
	\def\cola{pink!50!purple}
	\def\colb{cyan!80!blue}
	\node[circle,fill=\cola,minimum size=8mm] at (0,0) {};
	\node[fill=\cola,anchor=west,minimum height=8mm,minimum width=2cm] at (0,0) {};
	\node[fill=white,line width=2pt,circle,draw=\cola,minimum size=1cm] at (2.1,0) {\theexamp};
	\node[text=white,anchor=west] at (-0.2,0) {Example};
	\node[fill=\colb,signal,signal to=east,text=white,minimum height=8mm] (arg1) at (4,0) {#1};
	\node[right=2mm of arg1,\colb] (arg2) {#2};
	\end{tikzpicture}
}
%==============================================================
%Exercice résolu
\makeatletter
\newtcolorbox{mybox}[2][]{
	enhanced,
	breakable,
	sharp corners,
	rounded corners=northwest,
	colback=white,
	colframe=orange!80!black,
	fonttitle=\bfseries\large\sffamily,
	frame hidden,
	title=#2,
	attach boxed title to top left,
	boxed title size=standard,
	boxed title style={%
		empty,
		rounded corners=north,
		boxrule=0pt,
		bottom=0pt,
	},
	underlay boxed title={%
		\filldraw[rounded corners=\kvtcb@arc, orange!80!black, line width=.5mm]
		(title.south east)--++(93:\tcboxedtitleheight)--++(183:\tcboxedtitlewidth)--++(-87:\tcboxedtitleheight)|-cycle;
		\draw[orange!80!black, line width=.5mm] (title.south)|-(frame.north east);
	},
	#1
}
\makeatother
%===============================================================
%Exercice
%commands
\newcommand{\sect}[1]{\begin{tcolorbox}[colframe=white,top=2pt,bottom=2pt,colback=red!14]
		\centering \textbf{#1}
	\end{tcolorbox}
}
\setlength{\columnsep}{30pt}
\setlength{\columnseprule}{0.2pt}
\setlength{\parindent}{0pt}%
\newtcolorbox[auto counter ]{Exercice}[1][]{
	enhanced,left=0pt,top=8pt,frame hidden,bottom=0pt,before upper={\hspace*{1cm}} ,colback=white,
	interior code app={\node[fill=col3, anchor=west,text width=0.5cm,text badly centered]at ([shift={(0.1,-0.5)}]frame.north west){{\large\color{white} \textbf{\centering\thetcbcounter}}};},nobeforeafter}
%============================================================
\AddToShipoutPictureBG{
	\begin{tikzpicture}[remember picture,overlay]
	\fill[cyan]([yshift=1cm]current page.south west)rectangle(current page.south east) ;
	\node[circle,draw=white,line width=2pt,minimum size=1cm,fill=cyan]at([yshift=0.9cm]current page.south){\bfseries\Large\textcolor{MainRed}{\thepage}};
	\node[ minimum height=2cm,minimum width=15cm,text width=10cm, align=left,font=\bfseries\color{white}] at ([shift={(6.5,.5)}]current page.south west) [align=left,scale=1]{rezazsouilah@gmail.com};
	\end{tikzpicture} }
%======================================================
%list
%========================================
\newtcbox{\lblbox}{enhanced, drop fuzzy shadow=Gray, tcbox raise base, boxrule=0.6pt, left=1pt, right=2pt, top=0pt, bottom=0pt, colback=cyan!30, colframe=SlateGrey}
%=======================================
\newlist{listexos}{enumerate}{2} % en espérant qu’il n’ait pas plus d’un niveau de sous-questions
\setlist[listexos,1]{label =\lblbox{\arabic*}, font=\itshape\bfseries\large, wide=0pt, leftmargin=*}

\begin{document}
\chapter{Matrices}

\section{Notions générales}

\begin{definition}
	Une Matrice est un tableau rectangulaire de la forme
	
	\begin{equation*}
	A = 
	\left[
	\begin{array}{ccc>{\columncolor{clight2}}c cc} 
	a_{11} & a_{12} & \cdots &  a_{1j} & \cdots & a_{1n}\\ 
	a_{21} & a_{22} & \cdots &  a_{2j} & \cdots & a_{2n}\\ 
	\vdots & \vdots &\vdots & \vdots & \vdots&\vdots \\
	\rowcolor{clight2} a_{i1} & a_{i2} & \cdots & \cellcolor[gray]{.6}a_{ij} & \cdots & a_{in}\\
	\vdots & \vdots &\ddots & \vdots & \vdots &\vdots \\
	a_{m1} & a_{m2} & \cdots & a_{mj} & \cdots & a_{mn}
	\end{array}
	\right]
	\end{equation*}
	
	où les $a_{ij}$ sont des nombres réels appelés les éléments ou coefficients de la matrice $A$. La matrice précédente est aussi notée par $\left(a_{ij}\right),\;i=1,\cdots,m,\;j=1,\cdots,n$, ou simplement par $\left(a_{ij}\right)$. L'élément $a_{ij}$ est situé à l’intersection de la $i$-ème ligne et de la $j$-ème colonne.	Une matrice ayant $m$ lignes et $n$ colonnes
	est appelée une matrice d'ordre $(m,n)$, ou de dimension $m\times n$. Les matrices seront notées habituellement par des lettres capitales $A,\,B,\cdots$, et les éléments par des lettres minuscules $a,\,b,\cdots$.
\end{definition}

\mytitle{SKILLS}{PROBLEM-SOLVING}

On considère les matrices suivantes:
\begin{equation*}
A=
\begin{bmatrix}
1 & -1 & 2\\
3 &  \frac{1}{2} & \sqrt{2}
\end{bmatrix},\;
B=
\begin{bmatrix}
-\sqrt{3} \\
\frac{1}{5}\\
1
\end{bmatrix},\;
C=
\begin{bmatrix}
-1 & 0 & 1 & 5
\end{bmatrix},\;
D=
\begin{bmatrix}
-1 & 1 & 3 \\
5 & 6 & 2 \\
1 & -1 & 0
\end{bmatrix}.
\end{equation*}
\begin{listexos}
	\item La matrice $A$ est de dimension $2\times 3$ et on a $a_{23}=\sqrt{2},\quad a_{13}=2,\quad a_{22}=\frac{1}{2}$.
	\item La matrice $B$ est de dimension $3\times 1$ et on a $b_{11}=-\sqrt{3},\quad b_{21}=\frac{1}{5},\quad b_{31}=1$.
	\item La matrice $C$ est de dimension $1\times 4$ et on a $C_{11}=-1,\quad C_{12}=0,\quad C_{13}=1,\quad C_{14}=5$. 
	\item La matrice $D$ est de dimension $3\times 3$ et on a $D_{33}=0,\quad D_{23}=2,\quad D_{32}=-1$.	
\end{listexos}	

\subsection{Égalité de deux matrices}

\begin{definition}
	Deux matrices $A$ et $B$ sont égales, et on écrit
	$A = B$, si elles ont même dimension et si leurs éléments correspondants sont égaux.
\end{definition}


\mytitle{SKILLS}{PROBLEM-SOLVING}

On considère les matrices $
A=
\begin{bmatrix}
0 & 1  \\
2 & 8 
\end{bmatrix},
\;
B=
\begin{bmatrix}
0 & x^{2}\\
\sqrt{y} & z 
\end{bmatrix}$.
\begin{itemize}
	\item Trouver $x,\,y\in\mathbb{R}$ tels que $A=B$.
\end{itemize}

\vspace{1cm}

\begin{mybox}{Exercice résolu}
	Trouver les valeurs possibles de $x,\,y\in\mathbb{R}$ telles que les matrices
	
	\[ 
	E=
	\begin{bmatrix}
	x^{2}-5 & 1 \\
	3 & -2y-4  
	\end{bmatrix}
	\quad
	F=
	\begin{bmatrix}
	1 & 1 \\
	x+2 & 0  
	\end{bmatrix}.
	\]
	soit égales.
	
\end{mybox}


\section{Matrices particulières}
Dans cette on va définir quelques matrices spéciales.
\subsection{Matrice ligne}

\begin{definition}
	Une matrice ligne est une matrice comportant une seule ligne. Une matrice ligne a donc
	pour dimension $1\times n$. Une matrice ligne a la forme suivante:
	\begin{equation*}
	A=
	\begin{bmatrix}
	a_{11} & a_{12} & \cdots & a_{1n}
	\end{bmatrix}_{1\times n}.
	\end{equation*}
\end{definition}

\mytitle{SKILLS}{PROBLEM-SOLVING}

La matrice
$
A=
\begin{bmatrix}
1 & 2 & -1 & 5 & 0
\end{bmatrix}
$ est une matrice ligne de dimension $1\times 5$.


\subsection{Matrice colone}

\begin{definition}
	Une matrice colonne est une matrice comportant une seule colonne.  Une matrice colonne a donc
	pour dimension $m\times 1$. Une matrice colonne a la forme suivante:
	\begin{equation*}
	A=\begin{bmatrix}
	a_{11}  \\
	a_{21}  \\
	\vdots  \\
	a_{n1}
	\end{bmatrix}_{n\times 1}.
	\end{equation*}
\end{definition}

\mytitle{SKILLS}{PROBLEM-SOLVING}

La matrice
$
A=
\begin{bmatrix}
-1 \\
0 \\
2 \\
4
\end{bmatrix}$ est une matrice colonne de dimension $4\times 1$.


\section{Opérations sur les matrice}

Dans cette section on va définir les opérations algébriques sur les matrices.


\subsection{Somme de deux matrices}

\begin{definition}
	Soit $A$ et $B$ deux matrices de même dimension. La somme de $A$ et $B$, écrite $A+B$, est la matrice obtenue en ajoutant les éléments correspondants des deux matrices.
	\begin{equation*}
	\begin{aligned}
	&\text{Si}\quad A=\begin{bmatrix}
	a_{11} & a_{12} & a_{13}\\
	a_{21} & a_{22}  & a_{23}
	\end{bmatrix}_{2\times 3}\;\text{et}\quad
	B=\begin{bmatrix}
	b_{11} & b_{12} & b_{13}\\
	b_{21} & b_{22}  & b_{23}
	\end{bmatrix}_{2\times 3},
	\\ \ \\
	&\text{alors}\quad A+B=\begin{bmatrix}
	a_{11}+b_{11} & a_{12}+b_{12} & a_{13}+b_{13}\\
	a_{21}+b_{21} & a_{22}+b_{22}  & a_{23}+b_{23}
	\end{bmatrix}_{2\times 3}.
	\end{aligned}
	\end{equation*}
	
\end{definition}

\mytitle{SKILLS}{PROBLEM-SOLVING}

On considère les matrices: 
\begin{equation*}
A=
\begin{bmatrix}
-2 & 0 & 1\\
1 & 3  & -1\\
\end{bmatrix},\;
B=
\begin{bmatrix}
1 & 2 \\
1 & 0 \\
0 & 4
\end{bmatrix},\;
C=
\begin{bmatrix}
-2 & 1 \\
4 & 3 \\
-1 & -1
\end{bmatrix},\;
D=
\begin{bmatrix}
1 & 1 & 0\\
2 & 0 & 4
\end{bmatrix}.
\end{equation*}
\begin{listexos}
	\item Calchler $B+C$.
	\begin{equation*}
	B+C=\begin{bmatrix}
	1 & 2 \\
	1 & 0 \\
	0 & 4
	\end{bmatrix}+
	\begin{bmatrix}
	-2 & 1 \\
	4 & 3 \\
	-1 & -1
	\end{bmatrix}=
	\begin{bmatrix}
	1+(-2) & 2+1 \\
	1+4 & 0+3 \\
	0+1 & 4+(-1)
	\end{bmatrix}=
	\begin{bmatrix}
	-1 & 3 \\
	5 & 3 \\
	1 & 3
	\end{bmatrix}.
	\end{equation*} 
	\item Calculer $A+D$.
	\begin{equation*}
	A+D=\begin{bmatrix}
	-2 & 0 & 1 \\
	1 & 3  & -1
	\end{bmatrix}+
	\begin{bmatrix}
	1 & 1 & 0\\
	2 & 0 & 4
	\end{bmatrix}=
	\begin{bmatrix}
	-2+1 & 0+1 & 1+0\\
	1+2 & 3+0 & -1+4
	\end{bmatrix}=
	\begin{bmatrix}
	-1 & 1 & 1\\
	3 & 3 & 3
	\end{bmatrix}.
	\end{equation*}
	
	\item  La somme de $A$ et $B$ n’est pas définie car $A$ et $B$ ne sont pas de même dimension.
\end{listexos}

\newpage

\section{Exercices}

\begin{multicols*}{2} 
	\sect{Calcul matriciel}
	\begin{Exercice}
		On considère les matrices
		\[ 
		A=\begin{pmatrix}
		4 & 8 \\
		1 & 2  
		\end{pmatrix},\;
		B=\begin{pmatrix}
		3 & 9 \\
		1 & 1  
		\end{pmatrix}.
		\]
		
		\begin{listexos}
			\item Calculer $A+B,\,AB,\;BA,\,A^2$ et $B^2$.
			\item A-t-on $(A+B)^2=A^2+2AB+B^2$?	
			\item Mêmes questions pour les matrices 
			\[ 
			A=\begin{pmatrix}
			1 & 0 \\
			2 & 1 
			\end{pmatrix},\;B=\begin{pmatrix}
			2 & 0 \\
			1 & 2  
			\end{pmatrix}.
			\]
		\end{listexos}
	\end{Exercice}
	
	\begin{Exercice}
		On considère les matrices
		\[ 
		A=\begin{pmatrix}
		x & 5 \\
		0 & 2x  
		\end{pmatrix},\;B=\begin{pmatrix}
		y & 7 \\
		-1 & 3y  
		\end{pmatrix}.
		\] 
		\begin{listexos}
			\item  Trouver $x,y\in\mathbb{R}$ tels que
			\[ 
			A+B=\begin{pmatrix}
			4 & 12 \\
			-1 & 17  
			\end{pmatrix}.
			\]
			\item Trouver $x,y\in\mathbb{R}$ tels que
			\[ 
			2A-4B=\begin{pmatrix}
			-5 & -18 \\
			4 & -16  
			\end{pmatrix}.
			\]
		\end{listexos}
	\end{Exercice}  
	\begin{Exercice}
		On considère la matrice
		\[ 
		A=
		\begin{pmatrix}
		x & 1 \\
		2 & 3  
		\end{pmatrix},\quad x\in\mathbb{R}.
		\]
		\begin{itemize}
			\item Trouver $x$ tel que $A^2=\begin{pmatrix}
			6 & 1 \\
			2 & 11  
			\end{pmatrix}$ 
		\end{itemize}
	\end{Exercice}  
	\begin{Exercice}
		On considère les matrices 
		\[ 
		\begin{aligned}
		&A=
		\begin{pmatrix}
		1 & 3 \\
		-4 & 2 \\
		0 & 7 
		\end{pmatrix},
		\;
		B=
		\begin{pmatrix}
		-2 & 0 \\
		-2 & 1 \\
		8 & 1 
		\end{pmatrix},
		\\
		&C=
		\begin{pmatrix}
		-4 & 6 \\
		-14 & 7 \\
		24 & 17 
		\end{pmatrix}.
		\end{aligned}
		\]
		\begin{itemize}
			\item Trouver $x, y\in\mathbb{R}$ tels que $xA+yB=C$.
		\end{itemize}
	\end{Exercice}  
	\begin{Exercice}
		Calculer si possible les produits matriciels suivants: 
		\begin{listexos}
			\item 
			\[ 
			\begin{pmatrix}
			2 & 5 \\
			3 & 6\\
			4 & 7 
			\end{pmatrix}
			\times
			\begin{pmatrix}
			2 & 5 \\
			4 & 6 
			
			\end{pmatrix}
			\]
			\item 
			\[ 
			\begin{pmatrix}
			-1 & 4 & 5
			\end{pmatrix}
			\times
			\begin{pmatrix}
			0 & 1 & 6\\
			3 & -1 & 4\\
			3 & 5 & -2
			\end{pmatrix}
			\]
			\item 
			\[ 
			\begin{pmatrix}
			2 & -3 & 4\\
			-1 & 2& 6\\
			4 & -3 & -3
			\end{pmatrix}
			\times
			\begin{pmatrix}
			1 & 2 & -3\\
			0 & -4 & 1
			\end{pmatrix}
			^T
			\]
		\end{listexos}
		
	\end{Exercice}  
	
\end{multicols*}

\end{document}