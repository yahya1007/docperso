\documentclass[a4paper,addpoints,12pt]{exam}

\usepackage{dlds}

 
\begin{document}
\devoir[prv=false,ds=false,num=1 ,niv=3 ,date=19/10/2022,Rdate=31/10/2022]

\begin{exo}
\begin{questions}
\question Développer et simplifier ce qui suit : 
\[ (x-\sqrt{3})^{2}\hspace*{0.5cm}et\hspace*{0.5cm}
	(\sqrt{2}y+2\sqrt{3})^{2}\hspace*{0.5cm}et\hspace*{0.5cm}
		(y-2\sqrt{2})(y+2\sqrt{2})\]
\[
A=(\sqrt{3}+2)^{2}-(5-2\sqrt{3})^{2}\hspace*{0.5 cm}et\hspace*{0.5 cm}
B=(7-2\sqrt{3})(7+2\sqrt{3})\hspace*{0.5 cm}et\hspace*{0.5 cm}
C=(x-2)(x+3)-5(x-1)
\]		
\question Factoriser :
\[ D=x^{2}-6\sqrt{3}x+27\hspace*{0.5cm}et\hspace*{0.5cm}
	E=4x^{2}+4\sqrt{2}x+2\hspace*{0.5cm}et\hspace*{0.5cm}
		F=\sqrt{3}x-3\hspace*{0.5cm}et\hspace*{0.5cm}
		G=5y^{2}-1\]
	\[K=(x+1)^{2}-(\sqrt{3}x-1)(x+1)\]
	\[
L=(x-3)(x-2)-5x+15\hspace*{0.5 cm}et\hspace*{0.5 cm}
M=(3x-1)^{2}-4+36x^{2}\]
\[N=9x^{2}-6\sqrt{5}x+5\hspace*{0.5 cm}
P=(2x+1)^{2}-(3x-5)^{2}
\]
\end{questions}
\end{exo}

\begin{exo}
\begin{questions}

\question Donner l'écriture scientifique :
\[0.00000000125 \]
\[300000000000000000000000 \]
\[
S=300000000 \times 0.0000000000005 \hspace*{0.5cm}et\hspace*{0.5cm}
T=24.1 \times 10^{-11}+ 7.59 \times 10^{-10}
\]
\question écrire sous la forme d'une puissance de 10
\[ A=\dfrac{0.001 \times (10^{-5})^{-2}}{(10^{-3})^{-5}\times 100000}\]
\question Simplifier :
\[ B=\dfrac{a^{2}b^{-3}(a^{-4}b^{-1})^{-2}\times a^{-5}}{(a^{-1}b^{5})^{-4}\times (a^{-3}b^{10})(b^{-2})^{-3}} \]
\[
U=\left( \dfrac{x^{-5}}{x^{-3}}\right)^{-2} \hspace*{0.5cm}et\hspace*{0.5cm}
V=\dfrac{(x \times x^{6})^{-3}}{x^{-4}\times (x^{-3})^{-2}}
\]
\end{questions}
\end{exo}

\begin{exo}
\begin{questions}
\question Calculer :
\[ \sqrt{16^{2}} \hspace*{0.5cm} et\hspace*{0.5cm}
 \sqrt{10^{-2}} \hspace*{0.5cm}et\hspace*{0.5cm}
  \sqrt{2^{4}\times 3^{2}\times 5^{6}} \hspace*{0.5cm}et \hspace*{0.5cm}
   (5\sqrt{3})^{2}  \]
   
 \question Écrire sous forme $a\sqrt{b}$:  
 \[  \sqrt{288}\hspace*{0.5cm}et\hspace*{0.5cm}
\sqrt{60}\hspace*{0.5cm}et\hspace*{0.5cm}
\sqrt{27}   \]

\question Écrire sous forme $\sqrt{a}$:
\[  3\sqrt{5}\hspace*{0.5cm}et\hspace*{0.5cm}
8\sqrt{8}\hspace*{0.5cm}et\hspace*{0.5cm}
3\sqrt{2}   \]

\question Calculer :
\[   (5+\sqrt{6})^{2}\hspace*{0.5cm}et\hspace*{0.5cm}
	(6\sqrt{6}+5)^{2}  \]
	
\question Déduire que :
\[  \sqrt{241+60\sqrt{6}}-\sqrt{31+10\sqrt{6}}=5\sqrt{6}  \]

\question Simplifier :
\[
A=\sqrt{12}-\sqrt{48}+\sqrt{3}
\hspace*{0.5cm}et\hspace*{0.5cm}
B=\sqrt{8}+\sqrt{32}-\sqrt{50}
\]

\question Résoudre les équations :
\[
4x^{2}=64
\hspace{1cm}et\hspace{1cm}
-x^{2}=144
\]	
\end{questions}

\end{exo}

\begin{exo}
\begin{questions}
\question Calculer :
\[
N=\sqrt{2^{2}+3^{2}+6^{2}}\hspace*{0.5cm}et\hspace*{0.5cm}
M=\left( \left( \dfrac{3}{2}\right)^{2}+4^{-1}\right)^{-2}  \hspace*{0.5cm}et\hspace*{0.5cm}
L=\dfrac{3^{-2}-(\dfrac{2}{3})^{2}}{1-(\dfrac{5}{3})^{-3}}
\]
\question Simplifier :
\[
P=\sqrt{21+\sqrt{13+\sqrt{9}}}\hspace*{0.5cm}et\hspace*{0.5cm}
R=\sqrt{2+\sqrt{2}}\times \sqrt{3+\sqrt{7+\sqrt{2}}}\times \sqrt{3-\sqrt{7+\sqrt{2}}}
\]

\[
H=2\sqrt{50}+\sqrt{200}-\sqrt{32}\hspace*{0.5 cm}et\hspace*{0.5 cm}
I=\sqrt{63}-5\sqrt{28}+\sqrt{175}
\]
\question Rendre le dénominateur un nombre rationnel :
\[
J=\dfrac{2\sqrt{3}}{3\sqrt{17}}\hspace*{0.5 cm}et\hspace*{0.5 cm}
K=\dfrac{1}{9-2\sqrt{3}}-\dfrac{1}{9+2\sqrt{3}}
\]
\end{questions}
\end{exo}

\end{document}