\documentclass[a4paper,addpoints,12pt]{exam}

\usepackage{dlds}
 
\begin{document}

\devoir[prv=false,num=2 ,niv=2 , date=01/12/2022,Rdate=12/12/2022 ]

\begin{exo}
\begin{questions}
\question Calculer : 
\[ 2022^{0} \quad; \quad (-1)^{101} \quad; \quad 0^{89} \quad; \quad (\dfrac{1}{5})^{-3}\]
\question Déterminer le signe des puissances suivantes :
\[ (\dfrac{-1}{13})^{-12} \quad; \quad 
	(-7)^{2007}
\] 
\question Écrire sous forme $a^{n}$ les expressions suivantes :
\[\dfrac{11^{-6}}{11^{-8}}\quad; \quad
	\dfrac{3^{5}\times (-3)^{8}}{9^{9}}\quad; \quad
	\dfrac{10^{5}\times 10^{-8}\times 10^{4}}{2^{6}\times 2^{-7}\times 2^{2}}\quad; \quad
\dfrac{a^{-9}\times a^{5}\times (-a)^{18}}{a^{-4}}
\]
\question Simplifier : 
\[
\dfrac{ab^{-4}\times (a^{2}b^{-1})^{3}\times a^{-2}b^{3}}{a^{-5}\times (ab^{-1})^{2}\times (ab)^{-3}}
\]
\question Donner l'écriture scientifique des nombres suivants :
\[
	A=0.0000000025\times 0.00000000042 \quad; \quad
	B=\dfrac{540\times 10^{-9}}{6\times 10^{-22}}\quad; \quad
\]
\end{questions}
\end{exo}

\begin{exo}
\begin{minipage}{.5\textwidth}
\begin{questions}
\question Reproduire cette figure et construire $E$ et $F$ les symétriques de $A$ et $B$ par rapport à la droite  $(\Delta)$.
\question Quel est le symétrique de $O$ par rapport à $(\Delta)$? En justifiant ta réponse .
\question Montrer que $AB=EF$.
\question Montrer que les points $O$ , $E$ et $F$ sont alignés.
\question Montrer que $O$ est le milieu du segment $[EF]$.
\end{questions}
\end{minipage}
\begin{minipage}{.5\textwidth}
\begin{tikzpicture}
\tkzDefPoints{-3/0/A,0/0/O,3/0/B}
\tkzDefPoint(30:2){D}
\tkzDrawSegment(A,B)
\tkzDrawLine[add=2 and 1](O,D)
\tkzDrawPoints(A,B,O)
\tkzLabelPoints(A,O,B)
\tkzLabelLine[pos=2,above](O,D){$(\Delta)$}
\tkzMarkSegments[mark=||](A,O O,B)

\end{tikzpicture}
\end{minipage}
\end{exo}

\begin{exo}
Soit $ABC$ un triangle tel que : $\widehat{BAC}=130^{\circ}$ et $\widehat{ACB}=30^{\circ}$
\begin{questions}
\question Construire la figure , puis le point $A'$ symétrique du point $A$ par rapport à la droite $(BC)$.
\question
	\begin{parts}
	\part Montrer que $CA=CA'$
	\part Donner la mesure de l'angle $\widehat{A'CA}$
	\part En déduire la nature du triangle $AA'C$
	\end{parts}
\question Soit $I$ le point de $[BC]$ tel que $[AI)$est  la bissectrice de l'angle $\widehat{BAC}$
		\begin{parts}
		\part Quel est le symétrique de l'angle $\widehat{BAI}$ par rapport à la droite $(BC)$ ?
		\part Montrer que $[A'I)$ est la bissectrice de l'angle $\widehat{BA'C}$.
		\end{parts}	
\end{questions}
\end{exo}

\begin{exo}
Soit $OAB$ un triangle et $(d)$ une droite passant par le point $O$ et parallèle à la droite $(AB)$
\begin{questions}
\question Construire les points $A'$ et $B'$ symétriques respectifs des points $A$ et $B$ par rapport à $(d)$.
\question Montrer que $(AB)//(A'B')$
\question Montrer que $AB'=A'B$
\question En déduire que le quadrilatère $ABB'A'$ est un rectangle.
\question Quel est le symétrique du cercle $(C)$ de centre $O$ passant par le point $A$ par rapport à la droite $(d)$?
\end{questions}
\end{exo}

\begin{exo}
\begin{questions}
\question Tracer un segment $[AB]$ puis sa médiatrice $(d)$.
\question Quel est le symétrique de $A$ par rapport à $(d)$ ?
\question Quel est le symétrique de $B$ par rapport à $(d)$ ?
\question Placer un point $K$ sur $(d)$ avec $K\notin [AB]$.
\question Quel est le symétrique de $K$ par rapport à $(d)$ ?
\question Que peut-on dire des longueurs $KA$ et $KB$ ?
\question Déduire la nature du triangle $BAK$.
\end{questions}
\end{exo}

\end{document}