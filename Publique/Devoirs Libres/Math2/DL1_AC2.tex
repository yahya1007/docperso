\documentclass[a4paper,addpoints,12pt]{exam}

\usepackage{dlds}
 
\begin{document}

\devoir[prv=false,ds=false,num=1 ,niv=2 ,date=14/10/2022,Rdate=31/10/2023]

\begin{exo}
\begin{questions}
\question Compléter par ce qui convient :
\end{questions}
\[\dfrac{-8}{14}=\dfrac{-4}{....} \hspace{1cm}
\dfrac{6}{33}=\dfrac{-2}{...}\hspace{1cm}
\dfrac{25}{....}=\dfrac{....}{2} \hspace{1cm}
\dfrac{....}{15}=\dfrac{4}{....}\hspace{1cm}
\dfrac{2\times .... +3}{7}=\dfrac{3}{4}\]

\end{exo}

\begin{exo}
\begin{questions}
\question Calculer et simplifier :
\end{questions}
$\dfrac{3}{4}+\dfrac{1.5}{2.4}=\dfrac{....}{.....}\hspace{2cm}
\dfrac{-(-(-(6)))}{-42}-\dfrac{-13}{-2.1}=\dfrac{.........}{..........}\hspace*{1cm} \dfrac{3}{4}+\dfrac{3}{2}\times \dfrac{1}{2}=\dfrac{............}{................}$
\end{exo}


\begin{exo}
\begin{questions}
\question Rendre les écritures irréductibles :
\end{questions}
\[A=\dfrac{-5\times7}{28\times(-5)} =\dfrac{.........}{...........}\hspace{1cm}
B=\dfrac{2\times(-3)}{(-3)\times22}=\dfrac{...........}{...........}\]
\[C=\dfrac{24\times3\times\times55}{11\times3\times5}=\dfrac{.........}{.............} \hspace{1cm}
D=\dfrac{-2+5}{4+5}=\dfrac{..............}{.............}\]
\end{exo}

\begin{exo}
\begin{questions}
\question Calculer :
\end{questions}
$\left( \dfrac{1}{5}+\dfrac{4}{15}\right) -\left( \dfrac{-9}{5}-\dfrac{3}{20}\right) -\left( -\dfrac{1}{3}-\dfrac{1}{6} \right) = ........... $

$ 10 -\left[ 1-\left( \dfrac{-9}{4}-\dfrac{-13}{2}\right) -\left( -\dfrac{1}{8}-\dfrac{-1}{16} \right)\right]= ............... $

\end{exo}

\begin{exo}
\begin{questions}
\question Représente sur une droite graduée les points $A$ ,$B$, $C$  et $D$ d'abscisses respectives 
 \[ -\dfrac{1}{5}\quad et \quad \dfrac{-4}{5}\quad et \quad  2 \quad et \quad \dfrac{3}{5} \]
 \end{questions}
 \begin{tikzpicture}
\tkzInit[xmin=-9, xmax=7]
\tkzDrawX
\end{tikzpicture}
\end{exo}

\begin{exo}
\begin{questions}
\question Comparer les nombres suivants : 

\[\dfrac{-8}{4}.......\dfrac{9}{-4} \hspace*{1cm}et \hspace*{1cm}\dfrac{9}{17}.......... 1 \]

\question Ranger les nombres dans l'ordre croissant :
\[ 2.5 , \dfrac{-5}{6}, \dfrac{8}{12}, \dfrac{-8}{18}, \dfrac{3}{9}, \dfrac{1}{6}, \dfrac{-5}{3}\]
\end{questions}
\notes[10pt]{5}{\linewidth}
\end{exo}



\end{document}