\documentclass[a4paper,addpoints,12pt]{exam}

\usepackage{dlds}


\setlength{\multicolsep}{6.0pt plus 2.0pt minus 1.5pt}% 50% of original values

\begin{document}

\devoir[ds=true,num=3 ,niv=3 , date=12/01/2023 ,prv=true]


\begin{exo}[8]
$ABCD$ est un rectangle tel que : $AB=6 $ et  $AD=9 $ et soit  $I$ le milieu de $[AB]$ et $J$ un point de  $[AD]$ tel que $AJ=1 $ .
\begin{questions}
\begin{multicols}{3}
\question[2] Calculer  $IJ$
\notes[10pt]{5}{\linewidth}
\columnbreak
\question[2] Calculer $IC$
\notes[10pt]{5}{\linewidth}
\columnbreak
\question[2] Calculer  $JC$ 
\notes[10pt]{5}{\linewidth}
\end{multicols}
\question[2] Est-ce que  $IJC$ est un triangle rectangle ?
\end{questions}
\notes[10pt]{5}{\linewidth}
\end{exo}

\begin{exo}[7]
$ABC$ est un triangle rectangle en $A$ tel que : $AB=6$ et $cos\widehat{B}=\dfrac{3}{4}$ .
\begin{questions}
\begin{multicols}{2}
\question[1] Calculer $sin\widehat{B}$
\notes[10pt]{4}{\linewidth}
\columnbreak
\question[1] Calculer  $tan\widehat{B}$ 
\notes[10pt]{4}{\linewidth}
\end{multicols}%
\begin{multicols}{2}
\question[1] Calculer  $BC$ 
\notes[10pt]{5}{\linewidth}
\columnbreak
\question[1] Calculer $AC$ 
\notes[10pt]{5}{\linewidth}
\end{multicols}%
\question[3] Calculer les rapports trigonométriques de l'angle.$\widehat{C}$.
\notes[10pt]{5}{\linewidth}
\end{questions}
\end{exo}

\begin{exo}[5]
\begin{questions}
\question[3] Calculer ce qui suit :

$A=2cos15^{\circ}+cos^{2}36^{\circ}-2sin75^{\circ}+cos^{2}54^{\circ}=$\anserline[2]
$B=cos^{2}28^{\circ}-sin^{2}51^{\circ}+cos^{2}62^{\circ}+cos^{2}39^{\circ}= $\anserline[2]
$C=tan73^{\circ}\times tan17^{\circ}-sin^{2}40^{\circ}-sin^{2}50^{\circ} =$\anserline[2]\vspace{-10pt}
\question[2] Simplifier :

$D=cosx(sinx+cosx)-sinx(cosx-sinx)  =$ \anserline[3]
$E= 1+tan^{2}a-\dfrac{1}{cos^{2}a}=$\anserline[3]
\end{questions}
\end{exo}

\end{document}