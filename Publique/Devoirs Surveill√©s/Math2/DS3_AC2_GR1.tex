\documentclass[a4paper,addpoints,12pt]{exam}

\usepackage{dlds}
 
\begin{document}

\devoir[ds=true,num=3 ,niv=2 ,date=12/01/2023,prv=false ]

\begin{exo}[6]
\begin{minipage}{0.6\linewidth}
On considère la figure ci-contre tel que $(EF) // (BC)$  et $BC=12$ et $AE=4$ et $AC=9$ et $AB=7$
\begin{questions}
\question[3] Calculer  $AF$.
\notes[10pt]{4}{\linewidth}
\end{questions}
\end{minipage}
\begin{minipage}{0.4\linewidth}
\begin{tikzpicture}
\tkzDefPoints{0/0/C,5/1/A,3/4/B}
\tkzDefPointOnLine[pos=0.6](A,B)\tkzGetPoint{E}
\tkzDefPointOnLine[pos=0.6](A,C)\tkzGetPoint{F}
\tkzLabelPoints(A,C,F)
\tkzLabelPoints[above, right=4pt](E,B)
\tkzDrawSegments(A,B B,C C,A)
\tkzDrawLine(E,F)

\end{tikzpicture}
\end{minipage}
\begin{questions}
\setcounter{question}{1}
\question[3] Calculer  $EF$.
\notes[10pt]{4}{\linewidth}
\end{questions}
\end{exo}

\begin{exo}[14]
$MNP$ est un triangle tel que  $\widehat{MNP}=70^{\circ}$ et$\widehat{MPN}=80^{\circ}$ et $NP=8 cm$
\begin{questions}
\question[4] Tracer la figure
\question[2] Calculer $\widehat{NMP}$
\question[2] Construire  $O$ le centre du cercle inscrit dans le triangle $MNP$
\question[2] Calculer $\widehat{PON}$
\question[2] Construire le point  $A$ milieu du  $[MP]$ et le point $B$ le milieu de$[MN]$
\question[2] Montrer que $(AB) // (PN)$ et calculer $AB$
\end{questions}
\end{exo}
\anspage{1}




\end{document} 