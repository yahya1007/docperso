\documentclass[a4paper,addpoints,12pt]{exam}

\usepackage{dlds}

 \usepackage[column=O]{cellspace}
\setlength{\cellspacebottomlimit}{5pt}
\setlength{\cellspacetoplimit}{5pt}

\begin{document}

\devoir[sem=2,prv=false,ds=true,num=6 ,niv=2 ,date=,Rdate=]


\begin{exo}[4]
On considère le tableau de proportionnalité suivant :
\begin{tabular}{|*5{Oc|}}
\hline 
4 &  & 7 &  & 5 \\ 
\hline 
12 & 45 &  & 15.5 &  \\ 
\hline 
\end{tabular} 
\begin{questions}
\question compléter le tableau 
\question calculer le coefficient de proportionnalité .\anserline[1]
\end{questions}
\end{exo}


\begin{exo}[2]
\begin{minipage}{.5\linewidth}
Un TGV roule  90 min  à la vitesse de 600 km/h.

Quelle distance parcourt-il ?
\end{minipage}
\begin{minipage}{.5\linewidth}
\anserline[3]
\end{minipage}

\end{exo}

\begin{exo}[4]
\begin{minipage}{.5\linewidth}
\begin{questions}
\fullwidth{Aprés une remise de 135DH , un costume coûte 765DH.}
\question  De quel pourcentage son prix a-t-il diminué ?
\fullwidth{ Sachant que l'on a fixé une solde de -30\% sur un article coûte 480DH.}
\question Quel est le prix à payer ?
\end{questions}
\end{minipage}
\begin{minipage}{.5\linewidth}
\anserline[8]
\end{minipage}
\end{exo}
\newpage

\begin{exo}[10]
Voici la liste des notes d'un devoir de mathématiques 
\begin{tabular}{|*8{Oc|}}
12 & 15 & 14 & 8 & 6 & 8 & 14 & 14	 \\ 
\hline 
14 & 12 & 8 & 6 & 3 & 8 & 14 & 15	 \\ 
\hline 
12 & 8 & 6 & 6 & 3 & 8 & 14 & 15	 \\ 
\hline 
14 & 15 & 12 & 8 & 8 & 3 & 3 & 15 \\ 
\hline 
20 & 20 & 6 & 6 & 3 & 3 & 8 & 9 \\ 
\hline 
8 & 6 & 8 & 8 & 8 & 14 & 12 & 2 \\ 
\end{tabular} 
\begin{questions}
\question compléter le tableau ci-dessous.
\question Quel est l'effectif total de cette série statistique?\anserline[1]
\question Calculer la moyenne de cette série statistique.\anserline[2]
\question Donner le pourcentage des élèves qui ont obtenue la moyenne.\anserline[2]
\question Représenter le tableau des effectifs par un histogramme.
\end{questions}

\begin{tabular}{|Oc|Oc|Oc|Oc|Oc|}
\hline 
Classe : note  & $2\leq n < 6$ & $6\leq n < 10$ &$10\leq n < 16$ & $16\leq n \leq 20$ \\ 
\hline 
Nombre des élèves &  &  &  &  \\ 
\hline 
Effectif cumulé &  &  &  &  \\ 
\hline 
Fréquence &  &  &  &  \\ 
\hline
Fréquence cumulé &  &  &  &  \\ 
\hline 
\end{tabular} 
\vspace{0.5cm}
\anserline[12]
\end{exo}









\end{document}