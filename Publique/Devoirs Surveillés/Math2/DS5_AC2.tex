\documentclass[a4paper,addpoints,12pt]{exam}

\usepackage{dlds}
\setlength{\columnseprule}{1pt}
\setlength{\columnsep}{3em}
\renewcommand{\columnseprulecolor}{\color{gray}}
\pointsinmargin 
\pointformat{\color{gray}\footnotesize{(\themarginpoints\,\points)}\hspace{0.5cm}}
\pointpoints{pt}{pts}

\begin{document}

\devoir[sem=2,prv=false,ds=true,num=5 ,niv=2 ,date=08/05/2023]

\begin{exo}[8]
\begin{minipage}{.65\linewidth}
On considère la figure ci-contre tel que $BC=6$ ,$AC=8$ .
\begin{questions}
\question[2] Calculer  $AB$.

\anserline[4]
\end{questions}
\end{minipage}
\begin{minipage}{.35\linewidth}
\begin{tikzpicture}
\tkzDefPoints{0/0/C,4/0/A}
\tkzDefTriangle[pythagore](A,C)
\tkzGetPoint{B}

\tkzDefPointBy[projection=onto B--A](C)
\tkzGetPoint{H}
\tkzDrawSegments(C,A A,B B,C C,H)
\tkzLabelPoints(A,C)
\tkzLabelPoint[left](B){B}
\tkzLabelPoint[above](H){H}
\tkzMarkRightAngle(B,C,A)
\tkzMarkRightAngle(A,H,C)
\end{tikzpicture}
\end{minipage}
\begin{questions}
\setcounter{question}{1}
\question[3] Calculer $\cos(\widehat{ABC})$ et $\cos(\widehat{CAB})$.
\begin{multicols}{2}
\anserline[2]
\columnbreak

\anserline[2]
\end{multicols}
\question[3] Calculer $HB$ et $HC$ et $HA$.
\begin{multicols}{3}
\anserline[3]
\columnbreak

\anserline[3]
\columnbreak

\anserline[3]
\end{multicols}
\end{questions}
\end{exo}

\begin{exo}[6]
Soient $a$ , $b$ et $c$ des nombres rationnels, tels que $-3\leq a \leq 5$ et $4\leq b \leq 7$.
\begin{questions}
\question[6] Encadrer ce qui suit :\vspace{-1cm}
\begin{multicols}{3}
\[a+b\]\\ \anserline[3]
\[3a-5\]\\ \anserline[3]
\[\frac{3a+2b}{5}\]\\ \anserline[3]
\end{multicols}
\end{questions}
\end{exo}


\begin{exo}[3]
\begin{questions}
\question[3] Simplifier :

\(\vv{EG}+\vv{GE}+\vv{AG}= \)\anserline[2]
\(\vv{AM}-\vv{MD}+\vv{CA}-\vv{CM}= \)\anserline[2]
\(2\vv{FE}-\vv{GE}-4\vv{GF}+\vv{GH}+4\vv{GF}+\vv{EH}= \)\anserline[2]
\end{questions}
\end{exo}

\begin{exo}[3]
$ABCD$ est un parallélogramme de centre $I$.
\begin{questions}
\question[2] Simplifier :
\begin{multicols}{2}
$\vv{IA}+\vv{CD}+\vv{AB}-\vv{ID}+\vv{BC}$

\anserline[5]
\columnbreak

$\vv{ID}+\vv{IC}+\vv{IB}+\vv{IA}$

\anserline[5]
\end{multicols}
\question[1] Montrer que : $2\vv{AD}+\vv{DC}-\vv{BD}-\vv{BC}-2\vv{AB}=\vv{0}$

\anserline[10]
\end{questions}
\end{exo}


\end{document}