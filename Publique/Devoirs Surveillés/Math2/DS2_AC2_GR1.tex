\documentclass[a4paper,addpoints,12pt]{exam}

\usepackage{dlds}

\begin{document}

\devoir[prv=false,ds=true,num=2 ,niv=2 , date=19/12/2022]

\begin{exo}[12]
\begin{questions}
\question[4] Calculer : 
\[
 (-1)^{-345}=\cdotsx{6} \quad; \quad (-12)^{-1}=\cdotsx{6} \quad; \quad 1^{891}=\cdotsx{6} \quad; \quad (\dfrac{3}{-4})^{-3}=\cdotsx{6}
\]
\question[2] Déterminer le signe des puissances suivantes :
\[ (\dfrac{-1}{-13})^{-24}=\cdotsx{12} \quad; \quad 
	(71)^{-3}=\cdotsx{12}
\] 
\question[4] Écrire sous forme $a^{n}$ les expressions suivantes :
\[\dfrac{7^{-6}}{7^{-25}}=\cdotsx{6}\quad; \quad
	\dfrac{5^{-5}\times (-5)^{20}}{25^{9}}=\cdotsx{6}\quad; \quad
	\dfrac{10^{15}\times 10^{-8}\times 10^{13}}{10^{-30}}=\cdotsx{6}
\]
\[
\dfrac{a^{-15}\times a^{12}\times (-a)^{-18}}{a^{-6}}=\cdotsx{6}
\]
\question[2] Donner l'écriture scientifique des nombres suivants :
\[
	A=0.0000000056\times 0.0000000004=\cdotsx{50} 
\]
\[	
	B=\dfrac{810\times 10^{-90}}{90\times 10^{-97}}=\cdotsx{70}
\]
\end{questions}
\end{exo}

\begin{exo}[8]
\begin{questions}
\question Tracer un segment $[AB]$ puis sa médiatrice $(d)$.
\question Quel est le symétrique de $A$ par rapport à $(d)$ ?
\question Quel est le symétrique de $B$ par rapport à $(d)$ ?
\question Placer un point $K$ sur $(d)$ avec $K\notin [AB]$.
\question Quel est le symétrique de $K$ par rapport à $(d)$ ?
\question Que peut-on dire des longueurs $KA$ et $KB$ ?
\question Déduire la nature du triangle $BAK$.
\end{questions}
\end{exo}
\anspage{1}
\end{document}