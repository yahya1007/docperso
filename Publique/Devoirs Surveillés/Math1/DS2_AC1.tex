\documentclass[a4paper,addpoints,12pt]{exam}

\usepackage{dlds}

\begin{document}

\devoir[prv=false,ds=true,num=2 ,niv=1 , date=16/12/2022 ]

\begin{exo}[5]
\begin{minipage}{.5\linewidth}
On considère la figure ci-contre :
\begin{questions}
\question Compléter par $\in$ ou $\notin$

$A\cdots [BC]$ ; $A\cdots [AC]$ ; $C\cdots [AB)$\\
$A\cdots [BC)$ ; $B\cdots [CA)$ ; $C\cdots [BA)$\\
$A\cdots [BE)$ ; $C\cdots [ED)$ ; $D\cdots [BE)$

\end{questions}
\end{minipage}
\begin{minipage}{.5\linewidth}
\begin{tikzpicture}[scale=.7]
\tkzDefPoints{-3/0/A,2/0/C,3/1/D,-4/-1/E}
\tkzInterLL(A,C)(E,D)
\tkzGetPoint{B}
\tkzDrawLines(A,C D,E)
\tkzDrawPoints(A,B,C,D,E)
\tkzLabelPoints(B,C)
\tkzLabelPoints[above](E,A,D)

\end{tikzpicture}
\end{minipage}
\end{exo}

\begin{exo}[4]
\begin{minipage}{.6\linewidth}
A, B ,et C sont trois points non alignés. 
\begin{questions}
\question Trace la droite $(D_{1})$ perpendiculaire à $(AB)$ passant par C.
\question Trace la droite $(D_{2})$ perpendiculaire à $(BC)$ passant par A.
\question Trace la droite $(D_{3})$ perpendiculaire à $(AC)$ passant par B.
\question Que peut-on dire des droites $(D_{1})$ et $(D_{2})$ et $(D_{3})$ ? Justifier ta réponse.

\end{questions}
\end{minipage}
\begin{minipage}{.4\linewidth}
\begin{tikzpicture}
\tkzDefPoints{0/0/B,2/3/A,4/1/C}
\tkzDrawLines(A,B B,C C,A)
\tkzLabelPoints[above=2pt](A,B,C)
\end{tikzpicture}
\end{minipage}
\end{exo}

\begin{exo}[7]
\begin{questions}
\question Calculer ce qui suit : 
\[-14+13=\cdots \hspace{2cm}
	23+(-19)=\cdots \hspace{2cm}
	-14.26+(-13.5)=\cdots
\]
\[-143-134=\cdots \hspace{2cm}
	213-(-139)=\cdots \hspace{2cm}
	-104.26-(-130.5)=\cdots
\]
\[-14-(-9)+(+48)-(68)-(-51)+(-47)-(+23)-(-13)+(+68)-48=\cdots
\]
\end{questions}
\end{exo}

\begin{exo}[4]
\begin{questions}
\question Donner les abscisses des points $A$ , $B$ , $O$ et $I$ 
\end{questions}
\begin{tikzpicture}
\tkzInit[xmin=-7.5, xmax=6]
\tkzAxeX
\tkzDefPoints{0/0/O,1/0/I,3/0/A,-4/0/B}
\tkzLabelPoints[above](O,I,A,B)
\end{tikzpicture}
\end{exo}


\end{document}