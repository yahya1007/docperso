\documentclass[a4paper,12pt]{article}

\usepackage{ProfModels}
\usepackage{fig3d}
\usepackage{diagrammes}

  
\begin{document}
\begin{Maquette}[DM]{Prive=false, Niveau=3, Numero=6, Date=19/10/2024, FinDate=31/12/2024, Semestre=2}

\begin{exercice}
\begin{enumerate}
\item Soit $f$ une fonction linéaire et $(D)$ sa représentation graphique passe par le point $A(2;4)$.
\begin{enumerate}
\item Montrer que $f(x)=2x$.
\item Calculer $f(-3)$.
\end{enumerate}
\item Soit $g$ une fonction affine telle que $g(3)-g(-1)=-2$ et $g(0)=5$ et $(\Delta)$ sa représentation graphique.

\begin{minipage}{0.5\linewidth}
\begin{enumerate}
\item Monter que $g(x)=-\dfrac{1}{2}x+5$.
\item Calculer $g(4)$.
\item Quel nombre a pour image -18 par $g$.
\end{enumerate}
\end{minipage}%
\begin{minipage}{0.5\linewidth}
\item Montrer que $(D)$ et $(\Delta)$ sont perpendiculaires.
\item Construire  $(D)$ et $(\Delta)$.
\end{minipage}
\end{enumerate}
\end{exercice}

\begin{exercice}
\begin{minipage}{.5\linewidth}
Ce graphique représente les ventes des voitures d'une société durant les jours du mois du Juin.
\begin{enumerate}
\item Recopier et compléter le tableau suivant:

\begin{tabular}{|*7{Oc|}}
\hline 
Caractère & 2 & 3 & 4 & 5 & 6 & 7 \\ 
\hline 
Effectif &  &  &  &  &  &  \\ 
\hline 
Effectif cumulé  &  &  &  &  &  &  \\ 
\hline 
\end{tabular}

\item Déterminer le mode de cette série statistique.
\item Calculer la moyenne des ventes par jours dans cette société.
\item Déterminer la médiane de cette série statistique. 
\end{enumerate}
\end{minipage}%
\begin{minipage}{.5\linewidth}
\Histogramme[Largeur=6,Hauteur=6,%
ListeCouleurs={orange,gray,blue,pink,red,black},%
DebutOx=0,FinOx=8,GradX={1,2,...,8},GradY={0,1,...,8},%
AffEffectifs=false,LabelX={},LabelY={}]%
{1.8/2.2/6 2.8/3.2/8 3.8/4.2/4 4.8/5.2/6 5.8/6.2/5 6.8/7.2/2}
\end{minipage}
\end{exercice}

\begin{exercice}
Soit $ABCDEFGH$ un parallélépipède rectangle tel que $AB=4$ ; $AE=6$ et $AD=3$.
\begin{enumerate}
\begin{minipage}{.7\linewidth}
\item Calculer $EB$.
\item Montrer que $EBC$ est un triangle rectangle.
\item Calculer $EC$.
\item Déterminer le volume du pyramide $EABCD$.
\item La pyramide $EIJKL$ est une réduction de la pyramide $EABCD$ par le rapport $\dfrac{1}{2}$ 
\begin{enumerate}
\item Calculer $IJ$
\item Calculer le volume de la pyramide $EIJKL$.
\end{enumerate}
\end{minipage}%
\begin{minipage}{.3\linewidth}

\begin{tikzpicture}
\cube[50]{3}{4}
\tkzDrawSegment[dashed](E,D)
\tkzDrawSegment(E,B)
\tkzDrawSegment[dashed](E,C)
\tkzDefPointOnLine[pos=0.5](E,A)\tkzGetPoint{I}
\tkzDefPointOnLine[pos=0.5](B,E)\tkzGetPoint{J}
\tkzDefPointOnLine[pos=0.5](C,E)\tkzGetPoint{K}
\tkzDefPointOnLine[pos=0.5](D,E)\tkzGetPoint{L}
\tkzFillPolygon[color=gray!30,opacity=0.4](I,J,K,L)
\tkzFillPolygon[color=gray!30,opacity=0.4](A,B,C,D)
\tkzDrawSegments(I,J J,K K,L L,I)
\tkzLabelPoint[left](I){I}
\tkzLabelPoint[below](J){J}
\tkzLabelPoint[above](K){K}
\tkzLabelPoint[left](L){L}
\end{tikzpicture}

\end{minipage}
\end{enumerate}
\end{exercice}

\end{Maquette}
\end{document}
