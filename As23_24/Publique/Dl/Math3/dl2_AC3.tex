\documentclass[a4paper,12pt]{article}

\usepackage{ProfModels}

 
\begin{document}

\begin{Maquette}[DM]{Prive=false, Numero=2, Niveau=3, Date=\today}

\begin{exercice}
\begin{enumerate}
\item Comparer a et b en justifiant la réponse :

\begin{minipage}{.5\linewidth}
\begin{itemize}
\item $ a=\sqrt{8}$ et $ b=3\sqrt{2}$
\item $a=5-2\sqrt{3} $ et $  b=4-\sqrt{12}$
\end{itemize}
\end{minipage}
\begin{minipage}{.5\linewidth}
\begin{itemize}
\item $ a=(\sqrt{3}+\sqrt{2})^{2}$ et $ b=2\sqrt{6}$
\item $ a=-5\sqrt{3}$ et $ b=-\sqrt{76}$
\end{itemize}
\end{minipage}
\begin{minipage}{.5\textwidth}
\item Soient $2\leq x\leq 5$ et $-3\leq y\leq -1$
\begin{itemize}
\item Encadrer $2x+1$ et $-4y+5$
\item Encadrer $x+y$ et $x-y$
\item Encadrer $x^{2}$ et $y^{2}$ et $xy$
\end{itemize}
\end{minipage}
\begin{minipage}{.5\textwidth}
\item Sachant que : $0.6\leq \dfrac{-2a+1}{5} \leq 1.4$
\begin{itemize}
\item[a)] Montrer que : $-3\leq a \leq -1$
\item[b)] Donner un encadrement de $a^{2}$

 puis $(a+4)^{2}$
\end{itemize}
\end{minipage}
\end{enumerate}
\end{exercice}

\begin{exercice}
$ABCD$ est un parallélogramme. Soient $E$ et $I$ deux points de $[AB]$ et $[AC]$ respectivement tels que : $(IE)$ et $(BC)$ sont parallèles et $AB=10$ ; $BC=8$ ; $AI=3$ et $AE=6$.
\begin{enumerate}
\item Calculer $AC$ et $IE$.
\item Soit $F$ un point de $[AD]$ tel que : $AF=4.8$
\begin{itemize}
\item[a.]Comparer les rapports : $\dfrac{AI}{AC}$ et $\dfrac{AF}{AD}$
\item[b.]En déduire que $(IF)$ et $(DC)$ sont parallèles.
\item Démontrer que $(BD)$ et $(EF)$ sont parallèles.
\end{itemize}
\end{enumerate} 
\begin{tikzpicture}
\tkzDefPoints{0/0/A,6/1/B,-2/5/C}
\tkzDefPointOnLine[pos=0.7](A,B)\tkzGetPoint{E}
\tkzDefParallelogram(A,B,C)\tkzGetPoint{D}
\tkzDefPointOnLine[pos=0.7](A,D)\tkzGetPoint{F}
\tkzDefParallelogram(F,A,E)\tkzGetPoint{I}
\tkzDrawPolygon(A,B,C,D)
\tkzDrawPoints(E,F,I)
\tkzDrawPoints(A,B,C,D)
\tkzDrawSegments(A,C E,I F,I)
\tkzDrawSegments[dashed](E,F B,D)
\tkzLabelPoints(A,B,D)
\tkzLabelPoints(E,F,I)
\tkzLabelPoints[above](C)
\end{tikzpicture}
\end{exercice}
\end{Maquette}


\end{document}