\documentclass[a4paper,12pt]{article}

\usepackage{ProfModels}

  
\begin{document}
\begin{Maquette}[DM]{Prive=false, Niveau=3, Numero=5, Date=19/10/2024, FinDate=31/12/2024, Semestre=2}

\begin{exercice}
Le plan est muni du repère orthonormé  $(O,I,J)$, soit $A(-2,3)$ , $B(1,2)$ et $C(0,3)$.
\begin{enumerate}
\item Détermine les coordonnées de $\vv{AB}$ puis calcule $AB$.
\item Détermine les coordonnées de $M$ le milieu de $[AB]$
\item Détermine les coordonnées de $N$ tel que $C$ le milieu de $[BN]$.
\item Détermine les coordonnées de $E$ tel que $\vv{BE}(-3,4)$.
\item Détermine les coordonnées de $D$ pour que $ABDJ$ être un parallélogramme.
\item Détermine les coordonnées de $R$ le centre du parallélogramme $ABDJ$.
\item Détermine les coordonnées de $W$ tel que $\vv{BW}=2\vv{AC}+3\vv{AB}$.
\item Détermine les coordonnées de $K$ l'image de $O$ par la translation qui transforme $A$ au $C$.
\item On considère $P(7,0)$ , montre que les points $A$,$B$ et $P$ sont alignés.
%\item Détermine les coordonnées de $H$ le projeté orthogonal du point $A$ sur $(BC)$.
\end{enumerate}
\end{exercice}

\begin{exercice}
Le plan est muni du repère orthonormé  $(O,I,J)$, on considère les points $M(-3,12)$,$N(2,2)$,$Q(4,-5)$ et $P(4,3)$
\begin{enumerate}
\item Montre que l'équation réduite de la droite $(MN)$ est $y=-2x+6$.
\item Détermine l'équation réduite de la droite $(PQ)$.
\item Détermine les coordonnées de $K$ le point d'intersection des deux droites $(MN)$ et $(PQ)$.
\item Détermine l'équation réduite de la droite $(\Delta)$ qui est parallèle au $(MN)$ et passe par $Q$.
\item Détermine l'équation réduite de la droite $(\Delta)'$ la médiatrice de $[MN]$.
\item Détermine l'équation réduite de la droite $(D)$ l'image de $(MN)$ par la translation qui transforme  $M$ au $I$.
%\item construit la représentation graphique de la droite $(\Delta)$ .
\end{enumerate}
\end{exercice}

\begin{exercice}
\begin{enumerate}
\item Résous les systèmes suivants :
\(
\systeme{2x+3y=5,x-4y=-3}      
\qquad
\systeme{2x-y=5,x+y=4}
\qquad
\systeme{\sqrt{2}x+\sqrt{3}y=3,x-\sqrt{2}y=8}
\)
\end{enumerate}
\end{exercice}
\begin{exercice}
Yassir a 10 pièces dans son porte-monnaie. Ce sont uniquement des pièces de 5dh et 10dh . Le montant contenu dans le porte monnaie est de 75dh . Combien a-t-il de pièces de chaque sorte ?
\end{exercice}
%\begin{exercice}
%Soit $(S)$ le système :$\systeme{5x+2y=30, x+3y=19}$
%\begin{enumerate}
%\item le couple (-4; 25) est-il solution du système (S)?
%\item Résoudre le système (S).
%\end{enumerate}
%Rachide a acheté 10 stylos et 4 crayons tandis que Meryem a acheté un stylo et 3 crayons à la même librairie.Les stylos et les crayons sont de même types.Déterminer le prix d'un stylo et le prix d'un crayon sachant que Rachide a payé 60 DH et que  Meryem a payé 19 DH.
%
%Résoudre graphiquement le système : $\systeme{2x-y-3=0,2x+3y-7=0}$.
%\end{exercice}

\end{Maquette}

\end{document}