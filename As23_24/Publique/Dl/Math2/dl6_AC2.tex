\documentclass[a4paper,12pt]{article}

\usepackage{ProfModels}

 
\begin{document}

\begin{Maquette}[DM]{Prive=false, Numero=6, Niveau=2, Date=\today}

\begin{exercice}
On considère le tableau de proportionnalité suivant :
\begin{tabular}{|c|c|c|c|c|}
\hline 
3 &  & 7 &  & 5 \\ 
\hline 
12 & 45 &  & 15.5 &  \\ 
\hline 
\end{tabular} 
\begin{enumerate}
\item calculer le coefficient de proportionnalité .
\item compléter le tableau 
\end{enumerate}
\end{exercice}


\begin{exercice}
Un TGV roule pendant 90 min  à la vitesse de 300 km/h.

Quelle distance parcourt-il ?
\end{exercice}

\begin{exercice}
\begin{enumerate}
\fullwidth{Aprés une remise de 135DH , un costume coûte 765DH.}
\item  De quel pourcentage son prix a-t-il diminué ?
\fullwidth{ Sachant que l'on a fixé une solde de -30\% sur un article coûte 480DH.}
\item Quel est le prix à payer ?
\end{enumerate}
\end{exercice}

\begin{exercice}
Le tableau suivant représente le nombre d'enfant par famille.

\begin{tabular}{|c|c|c|c|c|c|}
\hline 
Nombre d'enfant & 1 & 2 & 3 & 4 & 5 \\ 
\hline 
nombre de famille & 3 & 10 & 7 & 4 & 8 \\ 
\hline 
Effectif cumulé  &  &  &  &  &  \\ 
\hline
Fréquence  &  &  &  &  &  \\ 
\hline 
Fréquence cumulé  &  &  &  &  &  \\ 
\hline
pourcentage &  &  &  &  &  \\ 
\hline 
\end{tabular} 
\begin{enumerate}
\item compléter le tableau
\item Quel est le caractère de cette série statistique ?
\item Quel est l'effectif total de cette série statistique?
\item Calculer la moyenne de cette série statistique.
\item Tracer le diagramme en bâtons des effectifs.
\end{enumerate}
\end{exercice}
\end{Maquette}









\end{document}