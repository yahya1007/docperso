\documentclass[a4paper,12pt]{article}

\usepackage{ProfModels}
\usepackage{diagrammes}
 
\begin{document}

\begin{Maquette}[DM]{Prive=false, Numero=6, Niveau=1, Date=\today}


\begin{exercice}
Sur la droite graduée ci-dessous, donner la distance à zéro et l’abscisse de chacun des points   :

\begin{tikzpicture}
\tkzInit[xmin = -8,xmax = 5]
\tkzDrawX
\tkzLabelX
\tkzDefPoints{3/0/M,-4/0/N,-5.5/0/P,0.5/0/Q}   
\tkzDrawPoints(M,N,P,Q)
\tkzLabelPoints[above](M,N,P,Q)
\end{tikzpicture}

\begin{tabular}{|c|c|c|c|c|}
\hline 
point & M & N & P & Q \\ 
\hline 
Abscisse &  &  &  &  \\ 
\hline 
Distance à 0 &  &  &  &  \\ 
\hline 
\end{tabular} 
\begin{enumerate}
\item donner les coordonnées des points $A$,$B$,$C$ et $D$.
\item Placer les points $E(-4, 0)$, $F(-2, 2)$, $G(1, -4)$ et $H(3, 3)$.
\end{enumerate}
\begin{AffRepere}
\coordpoints{1}{2}{A}
\coordpoints{4}{0}{B}
\coordpoints{0}{-2}{C}
\coordpoints{-3}{-2}{D}
\end{AffRepere}
\end{exercice}

\begin{exercice}
Voici un tableau de proportionnalité.
\begin{tabular}{|c|c|c|c|}
\hline 
tours de pédalier & 2 & 5 & y \\ 
\hline 
distance (m) & 3.6 & x & 12.6 \\ 
\hline 
\end{tabular} 
\begin{enumerate}
\item Calculer x et y .
\item Quel est le coefficient de proportionnalité.
\end{enumerate}
\end{exercice}

\begin{exercice}
Voici un tableau représentant le nombre de téléphones portables par famille.

\begin{tabular}{|c|c|c|c|c|c|}
\hline 
Nombre de portables & 1 & 2 & 3 & 4 & 5 \\ 
\hline 
Nombre de familles & 3 & 8 & 11 & 2 & 1 \\ 
\hline 
\end{tabular} 

\begin{enumerate}
\item Calculer l'effectif total.
\item Calculer la fréquence qui correspondant à la valeur 2 du caractère.
\item Représenter les données par un diagramme en bâtons.
\end{enumerate}
\end{exercice}
\end{Maquette}













\end{document}