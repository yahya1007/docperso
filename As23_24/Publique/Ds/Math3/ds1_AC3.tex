\documentclass[a4paper,12pt]{article}

\usepackage{ProfModels}


\begin{document}
\begin{Maquette}[DS]{Numero=1, Niveau=3, Date=13/06/2024}

\begin{exercice}[BaremeDetaille]
\begin{enumerate}
\begin{multicols}{2}
\item Développer et simplifier :
 
$A=(\sqrt{3}+2)^{2}=$\anserline[4]

$B=(x-2\sqrt{3})(x+2\sqrt{3})=$\anserline[4]

$C=(x-2)(2x+3)-5(2x-1)=$\anserline[4]
\columnbreak

\item\brm{2} Factoriser :

\brm{2}$ D=4x^{2}-4x+1=$\anserline[3]

$E=49x+56=$\anserline[3]

$F=81x^{2}-5=$\anserline[3]

$G=(3x-1)(5x+6)+9x-3=$\anserline[4]
\end{multicols}
\item\brm{2} Développer $(\sqrt{5}+1)^{2}=$\anserline[1]
\item\brm{2} Déduire $\sqrt{6+2\sqrt{5}}=$\anserline[1]
\end{enumerate}
\end{exercice}

\begin{exercice}[BaremeDetaille]
\begin{enumerate}
\item\brm{2} Donner l'écriture scientifique :

$ S=70000000000 \times 0.00000000000087=$\anserline[6]
\end{enumerate}
\end{exercice}

\begin{exercice}[BaremeDetaille]
\begin{enumerate}
\begin{multicols}{2}
\item Simplifier :
$U=\left( \dfrac{x^{-7}}{x^{-4}}\right)^{-2}=$\anserline[6]
\columnbreak

\brm{2}$V=\dfrac{(x \times x^{8})^{-3}}{x^{-9}\times (x^{-6})^{-2}}=$
\anserline[6]
\end{multicols}
\item\brm{2} Simplifier :

$A=\sqrt{63}-\sqrt{28}+\sqrt{112}=$\anserline[3]
\item\brm{2} Résoudre l équation :

$x^{2}=64$\anserline[3]

\item\brm{4} Rendre le dénominateur un nombre rationnel :

$J=\dfrac{2\sqrt{3}}{3\sqrt{8}}=$\anserline[3]

$K=\dfrac{4}{3-2\sqrt{4}}-\dfrac{1}{9+2\sqrt{3}}=$\anserline[8]
\end{enumerate}
\vspace{1cm}
\end{exercice}

\end{Maquette}
\end{document}