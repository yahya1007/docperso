\documentclass[a4paper,12pt]{article}

\usepackage{ProfModels}


\setlength{\multicolsep}{6.0pt plus 2.0pt minus 1.5pt}% 50% of original values

\begin{document}
\begin{Maquette}[DS]{Numero=2, Niveau=3, Date=04/01/2025}


\begin{exercice}[BaremeDetaille]
$ABCD$ est un rectangle tel que : $AB=6 $ et  $AD=9 $ et soit  $I$ le milieu de $[AB]$ et $J$ un point de  $[AD]$ tel que $AJ=1 $ .
\begin{enumerate}
\begin{multicols}{3}
\item\brm{2} Calculer  $IJ$\newline
\anserline[5]
\columnbreak
\item\brm[0.5]{2} Calculer $IC$\newline
\anserline[5]
\columnbreak
\item\brm[1]{2} Calculer  $JC$ \newline
\anserline[5]
\end{multicols}
\item\brm{2} Est-ce que  $IJC$ est un triangle rectangle ?
\end{enumerate}
\anserline[5]
\end{exercice}

\begin{exercice}[BaremeDetaille]
$ABC$ est un triangle rectangle en $A$ tel que : $AB=6$ et $cos\widehat{B}=\dfrac{3}{4}$ .
\begin{enumerate}
\begin{multicols}{2}
\item\brm{1.5} Calculer $sin\widehat{B}$\newline
\anserline[6]
\columnbreak
\item\brm[0.5]{1.5} Calculer  $tan\widehat{B}$ \newline
\anserline[6]
\end{multicols}%
\begin{multicols}{2}
\item\brm{1.5} Calculer  $BC$ \newline
\anserline[6]
\columnbreak
\item\brm[0.5]{1.5} Calculer $AC$ \newline
\anserline[6]
\end{multicols}%
\item\brm{3} Calculer les rapports trigonométriques de l'angle.$\widehat{C}$.\newline
\anserline[8]
\end{enumerate}
\end{exercice}

\begin{exercice}[BaremeDetaille]
\begin{enumerate}
\item\brm{3} Calculer ce qui suit :

$A=2cos15^{\circ}+cos^{2}36^{\circ}-2sin75^{\circ}+cos^{2}54^{\circ}=$\anserline[10]

%$B=cos^{2}28^{\circ}-sin^{2}51^{\circ}+cos^{2}62^{\circ}+cos^{2}39^{\circ}= $\anserline[3]
%
%$C=tan73^{\circ}\times tan17^{\circ}-sin^{2}40^{\circ}-sin^{2}50^{\circ} =$\anserline[3]
%\item\brm{1} Simplifier :
%
%$D=cosx(sinx+cosx)-sinx(cosx-sinx)  =$ \anserline[5]
\end{enumerate}
\end{exercice}
\end{Maquette}
\end{document}