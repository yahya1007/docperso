\documentclass[12pt]{article}

\usepackage{ProfModels}
\usepackage{fig3d}

\pagestyle{empty}
\begin{document}

\begin{Maquette}[DS]{Niveau=3, Numero=6, Date=22/05/2025}
\begin{exercice}[BaremeDetaille=true]
\begin{minipage}{.42\linewidth}
Le tableau suivant donne la répartition des notes d'une classe à un contrôle.
\begin{enumerate}
\item\brm{2} Compléter le tableau.
 \end{enumerate}
 \end{minipage}%
 \begin{minipage}{.58\linewidth}
\begin{tabular}{|Oc|Oc|Oc|Oc|Oc|Oc|Oc|Oc|Oc|Oc|}
\hline 
Note & 8 & 10 & 12 & 14 & 15 & 16 & 18 & 19 & 20 \\ 
\hline 
Effectif & 2 &  & 4 & 5 &  & 6 & 7 & 2 & 1 \\ 
\hline 
Eff cumulé &  & 4 &  &  &  &  &  &  & 30 \\ 
\hline 
\end{tabular} 
\end{minipage}%
\begin{enumerate}[start=2]
\item\brm{1} Quel est l'effectif total ?
 \anserline[1]
\item\brm{1} Calculer la fréquence   de 16.\anserline[1]
\item\brm{1} Calculer la note moyenne de la classe.\anserline[2]
\item\brm{1} Déterminer le mode de cette série.\anserline[1]
\item\brm{1} Déterminer la médiane de cette série.\anserline[1]
\item\brm{1} Quel est le pourcentage des élèves qui ont une note supérieur à la moyenne de classe.\par\anserline[1]
\end{enumerate}
\end{exercice}

\begin{exercice}[BaremeDetaille=true]
Soit $f$ une fonction linéaire tel que $f(-3)=4$ et $g$ une fonction affine tel que $g(x)=4x-10$.
\begin{tasks}[label=\arabic*.](2)
\task\brm{2} Montrer que $f(x)=-\dfrac{4}{3}x$.
\par\anserline[4]
\task\brm[0.5]{1} Calculer $f(3)$.\par\anserline[4]
\task\brm{1.5}  Montrer que $g(3)=2$.
\par\anserline[4]
\task\brm[0.5]{1.5} Quel nombre a pour image 0 par $g$.\par\anserline[4]
\end{tasks}
\end{exercice}

\begin{exercice}[BaremeDetaille=true]
\begin{minipage}{0.65\linewidth}
$SABCD$ est un pyramide de base rectangle $ABCD$, de centre $O$, $AB=3$ et $BD=5$.La hauteur $\lrc{SO}$ mesure $6$.
\begin{enumerate}
\item Montrer que $AD=4$.\par
\anserline[4]
\end{enumerate}
\end{minipage}%
\begin{minipage}{0.35\linewidth}
\begin{tikzpicture}[scale=0.8]
\pyramide[30]{4}{65}{65}
\tkzInterLL(A,C)(B,D)
\tkzGetPoint{O}
\tkzDefPointOnLine[pos=0.4](S,A)\tkzGetPoint{A'}
\tkzDefPointOnLine[pos=0.4](S,B)\tkzGetPoint{B'}
\tkzDefPointOnLine[pos=0.4](S,C)\tkzGetPoint{C'}
\tkzDefPointOnLine[pos=0.4](S,D)\tkzGetPoint{D'}
\tkzDrawSegments(A',B' B',C')
\tkzDrawSegments[dashed](A',D' D',C')
\tkzDrawSegments[dotted](A,C B,D)
\tkzLabelPoints[left](A',D')
\tkzLabelPoints[right](B',C')
\tkzLabelPoint[below](O){O}
\end{tikzpicture}
\end{minipage}%
\begin{enumerate}[start=2]
\item\brm{2} Calculer le volume de $SABCD$.\par
\anserline[3]
\item\brm{1} La pyramide $SA'B'C'D'$ est une réduction de la pyramide $SABCD$ avec un rapport de $\dfrac{1}{2}$.
\begin{enumerate}
\item\brm{1} Monter que $V_{SA'B'C'D'}=3$.\par
\anserline[4]
\item\brm{1} Calculer $SO'$.\par
\anserline[5]
\item\brm{1} Calculer l'aire du quadrilatère $A'B'C'D'$.\par
\anserline[4]
\end{enumerate}
\end{enumerate}



\end{exercice}




\end{Maquette}
\end{document}