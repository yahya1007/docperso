\documentclass[a4paper,12pt]{article}

\usepackage{ProfModels}

\setlength{\columnseprule}{1pt}
\setlength{\columnsep}{3em}
\renewcommand{\columnseprulecolor}{\color{gray}}
\begin{document}
\begin{Maquette}[DS]{Numero=5, Niveau=3, Date=28/04/2025,Semestre=2}

\begin{exercice}[BaremeDetaille=true]
Le plan est muni d'un repère orthonormé $\oij$ et on considère les points $A(0,1)$ ; $B(3,7)$ et la droite $(D):y=\dfrac{-1}{2}x+5$.
\begin{enumerate}
\item \brm{4}Détermine les coordonnées du vecteur $\vv{AB}$ et déduire $AB$\newline
\anserline[2]
\item \brm{2}Donne les coordonnées du point $K$ le milieu du segment $[AB]$.\newline
\anserline[2]
\item \brm{2}La droite $(D)$ passe-t-elle par le point $K$?
\newline
\anserline[2]
\item \brm{2}Montrer que l'équation réduite de la droite $(AB)$ est $y=2x+1$.
\newline
\anserline[7]
\item\brm{2} Donne l'équation réduite de la droite $(D')$ qui passe par le point $C(3,2)$ et parallèle à la droite $(AB)$.
\newline\anserline[4]
\item \brm{2}Montre que la droite $(D)$ est perpendiculaire à $(AB)$.\newline\anserline[4]
\end{enumerate}
\end{exercice}

\begin{exercice}[BaremeDetaille=true]
	\begin{enumerate}
		\item \brm{4} Résoudre le système : 	$\systeme{x+y=110,2x+5y=340}$
\end{enumerate}
\begin{multicols}{2}
	\anserline[8]
	\columnbreak
	
	\anserline[8]
\end{multicols}
\begin{enumerate}[start=2]
		\item\brm{2} Problème : \newline
		 Un théâtre propose deux types de billets les uns à 100dh et les autres à 250dh. On sait que 110 spectateurs ont assisté à cette représentation théâtrale et que la recette totale s'élève à 17000dh.\newline
		Calcule le nombre de billets vendus pour chaque type.
	\end{enumerate}
	\begin{minipage}{0.48\linewidth}
			\anserline[13]
	\end{minipage}\hfill\vrule\hfill
		\begin{minipage}{0.48\linewidth}
		\anserline[13]
	\end{minipage}

\end{exercice}


\end{Maquette}
\end{document}