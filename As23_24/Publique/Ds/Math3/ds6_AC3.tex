\documentclass[12pt]{article}

\usepackage{ProfModels}
\usepackage{fig3d}

\pagestyle{empty}
\begin{document}

\begin{Maquette}[DS]{Niveau=3, Numero=6, Date=22/05/2025}


\begin{exercice}[BaremeDetaille=true]
\begin{minipage}{.42\linewidth}
Le tableau suivant donne la répartition des notes d'une classe à un contrôle.
\begin{enumerate}
\item\brm{2} Compléter le tableau.
 \end{enumerate}
 \end{minipage}%
 \begin{minipage}{.58\linewidth}
\begin{tabular}{|Oc|Oc|Oc|Oc|Oc|Oc|Oc|Oc|Oc|Oc|}
\hline 
Note & 8 & 10 & 12 & 14 & 15 & 16 & 18 & 19 & 20 \\ 
\hline 
Effectif & 2 &  & 4 & 5 &  & 6 & 7 & 2 & 1 \\ 
\hline 
Eff cumulé &  & 4 &  &  &  &  &  &  & 30 \\ 
\hline 
\end{tabular} 
\end{minipage}%
\begin{enumerate}[start=2]
\item\brm{1} Quel est l'effectif total ?
 \anserline[1]
\item\brm{1} Calculer la fréquence   de 16.\anserline[1]
\item\brm{1} Calculer la note moyenne de la classe.\anserline[2]
\item\brm{1} Déterminer le mode de cette série.\anserline[1]
\item\brm{1} Déterminer la médiane de cette série.\anserline[1]
\item\brm{1} Quel est le pourcentage des élèves qui ont une note supérieur à la moyenne de classe.\par\anserline[1]
\end{enumerate}
\end{exercice}

\begin{exercice}[BaremeDetaille=true]
Soit $f$ une fonction linéaire tel que $f(-3)=4$ et $g$ une fonction affine tel que $g(3)=2$ et $g(4)=6$.
\begin{tasks}[label=\arabic*.](2)
\task\brm{2} Montrer que $f(x)=-\dfrac{4}{3}x$.
\par\anserline[3]
\task\brm[0.5]{1} Calculer $f(3)$.\par\anserline[3]
\task\brm{1.5}  Montrer que $g(x)=4x-10$.
\par\anserline[4]
\task\brm[0.5]{1.5} Quel nombre a pour image 0 par $g$.\par\anserline[4]
\end{tasks}
\end{exercice}

\begin{exercice}[BaremeDetaille=true]
Soit $SABC$  un tétraèdre de volume $160 cm^3$ et sa hauteur $SO=8cm$.
\begin{enumerate}
\begin{minipage}{.65\linewidth}
\item\brm{1} Calculer l'aire de la base $ABC$.\\\anserline[3]
\item  Le tétraèdre $SA'B'C'$ est l'agrandissement du tétraèdre $SABC$ et son volume est égal à $540 cm^3$.
\end{minipage}%
\begin{minipage}{.35\linewidth}
\begin{tikzpicture}
\tetra[60]{3}{70}{70}
\tkzDefPointOnLine[pos=1.4](S,A)\tkzGetPoint{A'}
\tkzDefPointOnLine[pos=1.4](S,B)\tkzGetPoint{B'}
\tkzDefPointOnLine[pos=1.4](S,C)\tkzGetPoint{C'}
\tkzDefPoint(2.5,0.4){O}
\tkzDefPointOnLine[pos=1.4](S,O)\tkzGetPoint{O'}
\tkzDrawPoints[shape=cross out,size=4](O,O')
\tkzDrawSegment[dashed](S,O')
\tkzLabelPoint[left](O){O}
\tkzLabelPoint[left](O'){$O'$}
\tkzDrawSegments(A,A' B,B' C,C' A',B' B',C')
\tkzDrawSegment[dashed](A',C')
\tkzLabelPoints(A',B')
\tkzLabelPoint[right](C'){$C'$}
\end{tikzpicture}
\end{minipage}%
\begin{enumerate}
\item \brm{1} Montrer que le rapport d'agrandissement est $\dfrac{3}{2}$.\\\anserline[3]
\item\brm{1} Montrer que $\dfrac{BC}{B'C'}=\dfrac{2}{3}$.\\\anserline[3]
\end{enumerate}
\item\brm{1} Calculer $SO'$ la hauteur du tétraèdre après l'agrandissement.\\\anserline[3]
\item\brm{1} Calculer l'aire de la base $A'B'C'$ après l'agrandissement.\\\anserline[3]
\item\brm{1} Calculer le volume du solide ABCA'B'C'.\par\anserline[3]
\end{enumerate}
\end{exercice}

\end{Maquette}



\end{document}