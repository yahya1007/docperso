\documentclass[a4paper,12pt]{article}

\usepackage{ProfModels}

\begin{document}
\begin{Maquette}[DS]{Niveau=2, Numero=6, Date=19/10/2024, Semestre=1, Calculatrice=false}


\begin{exercice}
On considère le tableau de proportionnalité suivant :
\begin{tabular}{|c|c|c|c|c|}
\hline 
3 &  & 7 &  & 5 \\ 
\hline 
12 & 45 &  & 15.5 &  \\ 
\hline 
\end{tabular} 
\begin{enumerate}
\item\brm{2} calculer le coefficient de proportionnalité .\anserline[1]
\item\brm{2} compléter le tableau 
\end{enumerate}
\end{exercice}


\begin{exercice}
\begin{minipage}{.5\linewidth}
\brm{2} Un TGV roule  90 min  à la vitesse de 300 km/h.

Quelle distance parcourt-il ?
\end{minipage}
\begin{minipage}{.5\linewidth}
\anserline[3]
\end{minipage}

\end{exercice}

\begin{exercice}
\begin{minipage}{.5\linewidth}
\begin{enumerate}
\fullwidth{Aprés une remise de 135DH , un costume coûte 765DH.}
\item\brm{2}  De quel pourcentage son prix a-t-il diminué ?
\fullwidth{ Sachant que l'on a fixé une solde de -30\% sur un article coûte 480DH.}
\item\brm{2} Quel est le prix à payer ?
\end{enumerate}
\end{minipage}
\begin{minipage}{.5\linewidth}
\anserline[8]
\end{minipage}
\end{exercice}
\newpage

\begin{exercice}
Le tableau suivant représente le nombre d'enfant par famille.

\begin{tabular}{|c|c|c|c|c|c|}
\hline 
Nombre d'enfant & 1 & 2 & 3 & 4 & 5 \\ 
\hline 
nombre de famille & 3 & 10 & 7 & 4 & 8 \\ 
\hline 
Effectif cumulé  &  &  &  &  &  \\ 
\hline
Fréquence  &  &  &  &  &  \\ 
\hline 
Fréquence cumulé  &  &  &  &  &  \\ 
\hline
pourcentage &  &  &  &  &  \\ 
\hline 
\end{tabular} 
\begin{enumerate}
\item\brm{2} compléter le tableau
\item\brm{2} Quel est le caractère de cette série statistique ?\anserline[1]
\item\brm{2} Quel est l'effectif total de cette série statistique?\anserline[1]
\item\brm{2} Calculer la moyenne de cette série statistique.\anserline[2]
\item\brm{2} Tracer le diagramme en bâtons des effectifs.

\anserline[14]
\end{enumerate}
\end{exercice}

\end{Maquette}
\end{document}