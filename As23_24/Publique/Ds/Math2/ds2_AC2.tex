\documentclass[a4paper,12pt]{article}

\usepackage{ProfModels}

\begin{document}
\begin{Maquette}[DS]{Niveau=2, Numero=2, Date=19/10/2024, Semestre=1, Calculatrice=false}

\begin{exercice}
\begin{enumerate}
\item\brm{2} Calculer : 
\[
 -22^{0}=\cdotsx{6} \quad; \quad (-1)^{-11}=\cdotsx{6} \quad; \quad 1^{89}=\cdotsx{6} \quad; \quad (\dfrac{7}{5})^{-3}=\cdotsx{6}
\]
\item\brm{2} Déterminer le signe des puissances suivantes :
\[ (\dfrac{-1}{-13})^{-11}=\cdotsx{6} \quad; \quad 
	(-71)^{2022}=\cdotsx{6}
\] 
\item\brm{4} Écrire sous forme $a^{n}$ les expressions suivantes :
\[\dfrac{17^{-6}}{17^{-25}}=\cdotsx{6}\quad; \quad
	\dfrac{5^{5}\times (-5)^{20}}{25^{9}}=\cdotsx{6}\quad; \quad
	\dfrac{10^{-5}\times 10^{-8}\times 10^{3}}{2^{6}\times 2^{7}\times 2^{-3}}=\cdotsx{6}
\]
\[
\dfrac{a^{-15}\times a^{12}\times (-a)^{-18}}{a^{-6}}=\cdotsx{6}
\]
\item\brm{4} Donner l'écriture scientifique des nombres suivants :
\[
	A=0.0000005\times 0.00000004=\cdotsx{60} 
\]
\[	
	B=\dfrac{480\times 10^{-93}}{6\times 10^{-97}}=\cdotsx{70}
\]
\end{enumerate}
\end{exercice}

\begin{exercice}
$ABC$ est un triangle tel que : $AB=6$ et $\widehat{BAC}=100^{\circ}$ et $\widehat{ABC}=30^{\circ}$.

Soit $M$ le milieu du segment $[BC]$ 
\begin{enumerate}
\item\brm{2} Faire un schéma.
\item\brm{2} Construire $E$ et $F$ les symétriques respectives   de $B$ et $C$ par rapport à la droite $(AM)$.
\item\brm{2} Montrer que $AE=6$.
\item\brm{2} Quel est la mesure de l'angle $\widehat{EAF}$? Justifie ta réponse.
\end{enumerate}
\end{exercice}
\end{Maquette}
\anserpage{1}

\end{document}