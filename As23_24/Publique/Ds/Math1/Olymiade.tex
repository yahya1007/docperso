\documentclass[a4paper,12pt]{article}

\usepackage{ProfModels}

\begin{document}
\begin{Maquette}[Olym]{Date=27/12/2024}

\begin{exercice}
\begin{minipage}{.3\linewidth}
\begin{enumerate}
\item\brm{4} Calculer le nombre A

$A=1+\dfrac{2}{2-\dfrac{1}{3+\dfrac{1}{4}}} $
\end{enumerate}
\end{minipage}%
\begin{minipage}{.7\linewidth}
\anserline[10]
\end{minipage}

\end{exercice}

\begin{exercice}
\begin{enumerate}
\item\brm{4} Sachant que $\dfrac{111111}{1001}=111$.
 Calculer la somme $\dfrac{444444}{1001}+\dfrac{999999}{3003}$, en justifiant votre réponse.
 \end{enumerate}
 \anserline[11]
\end{exercice}

\begin{exercice}
\begin{minipage}{0.7\linewidth}
Le quadrilatère\brm{4} $ABCD$ est un rectangle; le point $E$ appartient au côté $\lrc{AB}$.
Le triangle $CDE$ est-il rectangle en $E$ ? Justifier votre réponse.
\end{minipage}%
\begin{minipage}{0.3\linewidth}
\begin{tikzpicture}
\tkzDefPoints{0/0/A,5/-2/C}
\tkzDefRectangle(A,C)
\tkzGetPoints{B}{D}
\tkzDefPointOnLine[pos=0.3](A,B)
\tkzGetPoint{E}
\tkzDrawPolygon(A,B,C,D)
\tkzDrawSegments(D,E E,C)
\tkzLabelPoints[above](A,B,E)
\tkzLabelPoints(D,C)
\tkzLabelAngle(E,C,D){$36^{\circ}$}
\tkzLabelAngle(A,E,D){$54^{\circ}$}
\tkzMarkAngle[arc=l , size=0.5](E,C,D)
\tkzMarkAngle[arc=l , size=0.5 , mark=|](A,E,D)
\end{tikzpicture}
\end{minipage}
\anserline[6]
\end{exercice}

\begin{exercice}
Un théâtre reçoit 1700 personnes. 500 places sont situées devant la scène. Les autres sont un peu plus loin (les gradins).
Lors d'un concert, la salle est pleine. Les billets des places devant la scène sont vendus à 240 DH. Sachant que la recette du concert est de 300000 DH.
\begin{enumerate}
\item Déterminer le montant d'un billet des places en gradins.

\anserline[5]
\item\brm{4} Lors d'un deuxième concert, les places devant la scène sont de nouveau pleines et la recette est de 262500 Dh.
\begin{enumerate}
\item\brm{4} Déterminer le nombre de spectateurs dans les gradins. 

\anserline[9]
\end{enumerate}
\end{enumerate}
\end{exercice}

\end{Maquette}
\end{document}