\documentclass[a4paper,12pt]{article}

\usepackage{ProfModels}

\begin{document}
\begin{Maquette}[DS]{Niveau=1, Numero=1, Date=19/10/2024, Semestre=1, Calculatrice=false}

\begin{exercice}
\begin{enumerate}
\item\brm{2} Calculer ce qui suit :
\begin{tasks}(2)		
\task[A=] $ 14+5\times 3-12$\newline
\anserline[3]
\task[B=] $ 124+2\times 3+155\div 5$\newline
\anserline[3]
\task[C=] $ 270-144\div 12-100$\newline
\anserline[3]
\task[D=] $ 155+(17-34\div 2)\times 85$\newline
\anserline[3]
\end{tasks}
	\end{enumerate}
\end{exercice}

\begin{exercice}
	\begin{enumerate}
\item Calculer et simplifier le plus possible :
		\end{enumerate}
\begin{tasks}(3)		
\task \(\dfrac{185}{5}+\dfrac{65}{5}=\dfrac{\cdotsx{6}}{\cdotsx{6}}\)
\task \(\dfrac{20}{81}-\dfrac{7}{81}=\dfrac{\cdotsx{6}}{\cdotsx{6}}\)
\task \(\dfrac{6}{5}+\dfrac{12}{15}=\dfrac{\cdotsx{6}}{\cdotsx{6}}\)
\task \(\dfrac{15}{7}\times \dfrac{7}{15}=\dfrac{\cdotsx{6}}{\cdotsx{6}}\)
\task \(\dfrac{44}{51}\div\dfrac{22}{7}=\dfrac{\cdotsx{6}}{\cdotsx{6}}\)
\task \(\dfrac{4}{9}\div 2=\dfrac{\cdotsx{6}}{\cdotsx{6}}\)
\end{tasks}
\end{exercice}

\begin{exercice}
\begin{enumerate}
\item Calculer :	
\begin{tasks}
\task $-1+(-3)\times (-5)-7=$\anserline[4]
\task $6+(-8)-(-8)-(-9)+(-4)=$\anserline[6]	
\end{tasks}
	\end{enumerate}
\end{exercice}




\end{Maquette}



\end{document}