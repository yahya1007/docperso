\documentclass[a4paper,12pt]{article}
\usepackage{ProfModels}
\usepackage{diagrammes}

\usepackage[column=O]{cellspace}
\setlength{\cellspacebottomlimit}{5pt}
\setlength{\cellspacetoplimit}{5pt}

\begin{document}
\begin{Maquette}[DS]{Niveau=1, Numero=6, Date=22/05/2025, Semestre=1, Calculatrice=true}


\begin{exercice}[BaremeDetaille]
\begin{minipage}{0.6\linewidth}
Sur la droite graduée ci-dessous, donner la distance à zéro et l’abscisse de chacun des points   :

\begin{tikzpicture}
\tkzInit[xmin = -6,xmax = 3]
\tkzDrawX
\tkzLabelX
\tkzDefPoints{2/0/M,-4/0/N,-5.5/0/P,0.5/0/Q}   
\tkzDrawPoints(M,N,P,Q)
\tkzLabelPoints[above](M,N,P,Q)
\end{tikzpicture}
\end{minipage}\hfill
\begin{minipage}{.36\linewidth}
\begin{tabular}{|*5{Oc|}}
\hline 
point & M & N & P & Q \\ 
\hline 
Abscisse &  &  &  &  \\ 
\hline 
Distance à 0 &  &  &  &  \\ 
\hline 
\end{tabular} 
\end{minipage}
\begin{minipage}{.4\linewidth}
\begin{enumerate}
\item\brm{4} Compléter le tableau ci-contre.
\item\brm{4} Donner les coordonnées des points $A$,$B$,$C$ et $D$.
\par
\anserline[2]
\item\brm{2} Placer les points\par
 $E(-4, 0)$, $F(-2, 2)$, $G(1, -4)$ et $H(3, 3)$.
\end{enumerate}
\end{minipage}\hfill
\begin{minipage}{0.58\linewidth}
\begin{AffRepere}
\tkzSetUpPoint[shape=cross out,size=5]
\coordpoints{1}{2}{A}
\coordpoints{4}{0}{B}
\coordpoints{0}{-2}{C}
\coordpoints{-3}{-2}{D}
\end{AffRepere}
\end{minipage}
\end{exercice}


\begin{exercice}[BaremeDetaille]
\begin{minipage}{0.6\linewidth}
Voici un tableau de proportionnalité.
\begin{enumerate}
\item\brm{2} Remplir le tableau.
\item\brm{2} Quel est le coefficient de proportionnalité ?
\end{enumerate}
\end{minipage}%
\begin{minipage}{0.4\linewidth}
\begin{tabular}{|Oc|Oc|Oc|Oc|}
\hline 
tours de pédalier & 2 & 5 &  $\cdots$ \\ 
\hline 
distance (m) & 64 & $\cdots$ & 12 \\ 
\hline 
\end{tabular} 
\end{minipage}
\end{exercice}

\begin{exercice}[BaremeDetaille]
\begin{minipage}{0.6\linewidth}
On a demandé aux élèves d'une classe la somme d'argent de poche que leurs parents donnent chaque jour.
\end{minipage}\hfill
\begin{minipage}{0.38\linewidth}
\begin{tabular}{|c|c|c|c|c|}
 \hline 
 Argent de poche & 0 & 5 & 10 & 20 \\ 
 \hline 
 Effectifs & 4 & 8 & 16 & 4 \\ 
 \hline 
 Fréquence &  &  &  &  \\ 
 \hline
 \end{tabular} 
\end{minipage}
 \begin{enumerate}
 \item\brm{2} Combien y a-t-il d'élèves dans cette classe ?\anserline
 \item\brm{2} Calculer la fréquence de chaque valeur.\anserline
 \item\brm{2} Calculer le pourcentage des élèves qui reçoivent moins de 10 Dhs de leurs parents.\par\anserline
 \end{enumerate}
\end{exercice}







\end{Maquette}

\end{document}