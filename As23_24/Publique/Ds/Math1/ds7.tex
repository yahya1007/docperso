\documentclass[a4paper,12pt]{article}

\usepackage{ProfModels}
\usepackage{diagrammes}

\newcommand{\tr}[1][1cm]{
\textcolor{gray}{\rule[-1pt]{#1}{0.5pt}}%
}

\begin{document}
\begin{Maquette}[DevS]{Numero=2,Niveau=1,Date=28/05/2024}

\begin{exercice}[BaremeTotale,BaremeDetaille]
Répondre par vrai(V) ou faux(F)
\begin{enumerate}
\item\brm{2} Un losange peut avoir des côtés de longueurs différentes.\tr[2cm]
\item\brm{2} Le carré est un losange.\tr[2cm]
\item\brm{2} Le rectangle a forcément les diagonales de même longueurs.\tr[2cm]
\item\brm{2} Un rectangle est forcément un parallélogramme.\tr[2cm]
\item\brm{2} N'importe quel quadrilatère qui a un angle droit est un rectangle.\tr[2cm]
\end{enumerate}
\end{exercice}


\begin{exercice}[BaremeTotale,BaremeDetaille]
\begin{minipage}[c]{0.4\linewidth}
\begin{enumerate}
\item\brm{4} donner les coordonnées des points $A$,$B$,$C$ et $D$.\newline
\lines[6]
\item\brm{2} Placer les points $E(-4, 0.5)$, $F(-2.5, 2)$.
\end{enumerate}
\end{minipage}%
\begin{minipage}[c]{0.6\linewidth}
\begin{AffRepere}
\coordpoints{1}{2}{A}
\coordpoints{4}{0}{B}
\coordpoints{0}{-2}{C}
\coordpoints{-3}{-2}{D}
\end{AffRepere}
\end{minipage}
\end{exercice}

\begin{exercice}[BaremeTotale,BaremeDetaille]
\begin{minipage}{0.62\linewidth}
Voici un tableau de proportionnalité.
\begin{enumerate}
\item\brm{2} Remplir le tableau.
\item\brm{2} Un pantalon d'une valeur de 160 Dhs est vendu avec une réduction de 20\%. Quel est le prix de ce pantalon ?
\end{enumerate}
\end{minipage}%
\begin{minipage}{0.38\linewidth}
\begin{tabular}{|Oc|Oc|Oc|Oc|}
\hline 
tours de pédalier & 2 & 5 &  $\cdots$ \\ 
\hline 
distance (m) & 64 & $\cdots$ & 12 \\ 
\hline 
\end{tabular} 
\end{minipage}\bigskip
\lines[2]
\end{exercice}
















\end{Maquette}

\end{document}