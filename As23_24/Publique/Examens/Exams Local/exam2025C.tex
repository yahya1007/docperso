\documentclass[a4paper,12pt]{article}

\usepackage{ProfModels}
\usepackage{adjustbox}
\usepackage{cancel}
 
%\settasks{
%% label = \theexercise.\arabic* ,
% item-indent = 0em ,
% label-width = 0em ,
% label-offset = 0pt ,
% column-sep = {10pt} 
% }

\begin{document}
\begin{Maquette}[Exam]{Prive=false, Date=2025}

\begin{exercice}[BaremeDetaille]
\begin{enumerate}
\item\brm{2} Calculer :
$$\left(\dfrac{3}{\sqrt{2}}\right)^{2}+\dfrac{1}{2}=\dfrac{9}{2}+\dfrac{1}{2}=\dfrac{10}{2}=\dfrac{5\times \cancel{2}}{\cancel{2}}=5$$

$$\sqrt{\sqrt{25}}\times \sqrt{\sqrt{16}}\times\sqrt{5}=\sqrt{5}\times\sqrt{4}\times\sqrt{5}=5\times 2=10$$
\item\brm{1} Simplifier 
$$A=2\sqrt{63}+\sqrt{28}-\sqrt{7}=6\sqrt{7}+2\sqrt{7}-\sqrt{7}=7\sqrt{7}$$
\item\brm{1} Développer et simplifier
 $$\left(\sqrt{7}-1\right)^{2}+2(\sqrt{7}-2)=7-2\sqrt{7}+1+2\sqrt{7}
 -4=4$$
 \item\brm{1} Factoriser :
$$(3-x)(x+1)+2x+2=(3-x)(x+1)+2(x+1)=(x+1)(3-x+2)=(x+1)(5-x)$$
\item\brm{1}Donner l'écriture scientifique
$$39000000\times 10^{-26}=3.9\times 10^{7}\times 10^{-26}=3.9\times 10^{-19}$$
\item\brm{1} Ecrire le dénominateur sans radicale :
$$\dfrac{\sqrt{3}+1}{\sqrt{3}-1}=\dfrac{(\sqrt{3}+1)(\sqrt{3}+1)}{(\sqrt{3}-1)(\sqrt{3}+1)}=\dfrac{(\sqrt{3}+1)^{2}}{3-1}=\dfrac{(\sqrt{3}+1)^{2}}{2}$$
\end{enumerate}	
\end{exercice}

\begin{exercice}[BaremeDetaille]
\begin{enumerate}
\item\brm{1} Comparer $3\sqrt{5}$ et $\sqrt{47}$\newline
$(3\sqrt{5})^{2}=45$ et $(\sqrt{47})^{2}=47$ donc $3\sqrt{5}< \sqrt{47}$
\item\brm{1}\brm[0.5]{1}\brm[1]{1} Soient $2\leq a\leq 5$ et $1\leq b\leq 7$. Encadrer :
\end{enumerate}
\begin{minipage}{.3\linewidth}
$$encadrons \,2a+3b$$
$$2\leq a\leq 5$$
$$4\leq 2a\leq 10$$
$$1\leq b\leq 7$$
$$3\leq 3b\leq 21$$
$$7\leq 2a+3b\leq 31$$
\end{minipage}\hfill\vrule\hfill%
\begin{minipage}{.3\linewidth}
$$encadrons\, a-b$$
$$2\leq a\leq 5$$
$$1\leq b\leq 7$$
$$-7\leq -b\leq -1$$
$$2-7\leq a-b\leq 5-1$$
$$-5\leq a-b\leq 4$$
\end{minipage}\hfill\vrule\hfill%
\begin{minipage}{.3\linewidth}
$$encadrons\, \dfrac{a}{b}$$
$$2\leq a\leq 5$$
$$1\leq b\leq 7$$
$$\dfrac{1}{7}\leq \dfrac{1}{b}\leq \dfrac{1}{1}$$
$$\dfrac{2}{7}\leq \dfrac{a}{b}\leq \dfrac{5}{1}$$
\end{minipage}
\end{exercice}

\begin{exercice}[BaremeDetaille]
On considère la figure ci-contre tel que $BH=2$, $HC=8$ et $AH=4$.
\vspace*{3mm}

\begin{minipage}{.7\linewidth}
\begin{enumerate}
\item\brm{1} Calculer $AC$.\newline
On considère le triangle rectangle $AHC$\\
d'après le théorème de Pythagore\\
$$AC^{2}=AH^{2}+HC^{2}$$
$$AC^{2}=4^{2}+8^{2}$$
$$AC^{2}=16+64$$
$$AC^{2}=80$$
$$AC=\sqrt{80}=4\sqrt{5}$$
\end{enumerate}
\end{minipage}%
\begin{minipage}{.3\linewidth}
\begin{tikzpicture}[scale=0.8]
\tkzDefPoints{0/0/A,5/1/C}
\tkzDefTriangle[school](C,A)
\tkzGetPoint{B}
\tkzDefPointBy[projection=onto B--C](A)
\tkzGetPoint{H}
\tkzDrawPolygon
(A,B,C)
\tkzDrawSegment(A,H)
\tkzLabelPoint[above](A){A}
\tkzLabelPoint[below](B){B}
\tkzLabelPoint[above](C){C}
\tkzLabelPoint[below right](H){H}
\tkzMarkRightAngle(A,H,C)
\tkzLabelSegment(B,H){2}
\tkzLabelSegment(H,C){8}
\tkzLabelSegment(A,H){4}
\end{tikzpicture}
\end{minipage}%
\begin{enumerate}[start=2]
\item\brm{1} Calculer $AB$.
On considère le triangle rectangle $AHB$\\
d'après le théorème de Pythagore\\
$$AB^{2}=AH^{2}+HB^{2}$$
$$AB^{2}=4^{2}+2^{2}$$
$$AB^{2}=16+4$$
$$AB^{2}=20$$
$$AB=\sqrt{20}=2\sqrt{5}$$
\item\brm{1} Prouver que $ABC$ est un triangle rectangle.
$$BC^{2}=10^{2}=100$$
$$AB^{2}+AC^{2}=80+20=100$$
donc $BC^{2}=AB^{2}+AC^{2}$ d'après la réciproque de Pythagore $ABC$ est un triangle rectangle.
\end{enumerate}
\end{exercice}

\begin{exercice}[BaremeDetaille]
Soit $x$ la mesure d'un angle aigu tel que : $\cos x =\dfrac{1}{3}$
\begin{enumerate}
\item\brm{0.5}\brm[0.5]{0.5} Calculer : $\sin x$ et $\tan x$\vspace{5mm}

\begin{minipage}{0.48\linewidth}
$$\sin^{2}x+\cos^{2}x=1$$
$$\sin^{2}x+\left(\dfrac{1}{3}\right)^{2}=1$$
$$\sin^{2}x+\dfrac{1}{9}=1$$
$$\sin^{2}x=1-\dfrac{1}{9}$$
$$\sin^{2}x=\dfrac{8}{9}$$
$$\sin x=\sqrt{\dfrac{8}{9}}=\dfrac{2\sqrt{2}}{3}$$
\end{minipage}\hfill\vrule\hfill%
\begin{minipage}{0.48\linewidth}
$$\tan x=\dfrac{\sin x}{\cos x}$$
$$\tan x=\dfrac{\dfrac{2\sqrt{2}}{3}}{\dfrac{1}{3}}$$
$$\tan x=\dfrac{2\sqrt{2}}{1}=2\sqrt{2}$$
\end{minipage}
\item\brm{1} Simplifier  :
$A = \sin^{2}25^{\circ} +\cos50^{\circ} +\sin^{2}65^{\circ}-\sin40^{\circ}$
$$A = \sin^{2}25^{\circ} +\sin^{2}65^{\circ}+\cos50^{\circ}-\sin40^{\circ}$$
$$A = \sin^{2}25^{\circ} +\cos^{2}25^{\circ}+\cos50^{\circ}-\cos50^{\circ}$$
$$A = 1+0=1$$
\item\brm{1} Montrer que : $\sin^{2}a = \dfrac{\tan^{2}a}{1+\tan^{2}a}$
$$\dfrac{\tan^{2}a}{1+\tan^{2}a}=\dfrac{\dfrac{\sin^{2}a}{\cos^{2}a}}{1+\dfrac{\sin^{2}a}{\cos^{2}a}}=\dfrac{\sin^{2}a}{\cos^{2}a+\sin^{2}a}=\dfrac{\sin^{2}a}{1}$$

\end{enumerate}
\end{exercice}

\begin{exercice}[BaremeDetaille]
On considère la figure ci-contre tel que $(AB)//(EF)$, $AB=24$, $OB=21$, $OE=12$, $OF=14$, $BC=7$ et $BD=8$.
\vspace*{2mm}

\begin{minipage}{.7\linewidth}
\begin{enumerate}
\item\brm{1}\brm[0.5]{1} Calculer $OA$ et $EF$.

Les droites $(BF)$ et $(AE)$ sont sécantes en $O$, et $(AB)//(EF)$ d'après le théorème de Thalès 

\begin{minipage}{.5\linewidth}
$$\dfrac{OA}{OE}=\dfrac{OB}{OF}=\dfrac{AB}{EF} $$
$$\dfrac{OA}{12}=\dfrac{21}{14}=\dfrac{24}{EF} $$
\end{minipage}%
\begin{minipage}{.5\linewidth}
$$OA=\dfrac{12\times 21}{14}=18$$
$$EF=\dfrac{14\times 24}{21}=16$$
\end{minipage}
\end{enumerate}
\end{minipage}%
\begin{minipage}{.3\linewidth}
\begin{tikzpicture}
\tkzDefPoints{0/0/A,5/1/B,2/3/O}
\tkzDefPointOnLine[pos=1.2](B,O)\tkzGetPoint{F}
\tkzDefPointOnLine[pos=1.2](A,O)\tkzGetPoint{E}
\tkzDefPointOnLine[pos=0.4](B,O)\tkzGetPoint{C}
\tkzDefPointOnLine[pos=0.4](B,A)\tkzGetPoint{D}
\tkzDrawSegments(A,B B,F F,E E,A)
\tkzDrawPoints(C,D)
\tkzDrawLine[dashed,add=0.2 and 0.3](C,D)
\tkzLabelPoints(A,D,B)
\tkzLabelPoints[right](O,C)
\tkzLabelPoints[above](F,E)
\end{tikzpicture}
\end{minipage}\vspace{5mm}

\begin{enumerate}[start=2]
\item\brm{1} Montrer que $(OA)//(DC)$.

On a $\dfrac{BC}{BO}=\dfrac{7}{21}=\dfrac{1}{3}$

et on a $\dfrac{BD}{BA}=\dfrac{8}{24}=\dfrac{1}{3}$

donc $\dfrac{BC}{BO}=\dfrac{BD}{BA}$.

Les points $B$,$C$,$O$ et $B$,$D$,$A$ sont alignés et dans le même ordre, d'après la réciproque du Thalès $(AO)//(DC)$.
\end{enumerate}
\end{exercice}

\end{Maquette}
\end{document}

