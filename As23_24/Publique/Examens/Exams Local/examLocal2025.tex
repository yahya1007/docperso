\documentclass[a4paper,12pt]{article}

\usepackage{amsmath,amssymb}

\usepackage{iftex}
\ifpdftex
\usepackage[utf8]{inputenc} 
\usepackage[T1]{fontenc}
 \usepackage{lmodern}
 \usepackage{tkz-base}
\usepackage{tkz-euclide} 
\else
\usepackage{unicode-math}
\usepackage{libertinus-otf}
\usepackage{tkz-base}
\usepackage[lua]{tkz-euclide}
\fi

 %----------------------------------------------
\usepackage[a4paper,left=15mm,right=10mm,top=10mm,
bottom=10mm,marginparwidth=1cm,marginparsep=2pt,noheadfoot]{geometry}
%-------------------------------------------------
\usepackage{xcolor}
\usepackage[most]{tcolorbox}
\usetikzlibrary{calc}
\usepackage{changepage}
\usepackage{multicol}
\usepackage[a]{esvect}
\usepackage{graphicx}
\usepackage{systeme}
\usepackage{ifthen}
\usepackage{marginnote}

%---------------------------------------------
\usepackage[inline]{enumitem}
\setlist[enumerate,1,2]{font=\bfseries,leftmargin=*}
%----------------------------------------------
\usepackage{tasks}
\settasks{label-format =\bfseries, label-offset ={5pt},label-width={1.5em},item-indent ={3em}}
%-------------------------------------------------------
\usepackage[column=O]{cellspace}
\setlength{\cellspacebottomlimit}{5pt}
\setlength{\cellspacetoplimit}{5pt}
%----------------------------------------------------
%---------------------------------------------------
% packages nécessaires
\usepackage{cancel}
\usepackage{fontawesome}
\usepackage{siunitx}
\usepackage{calc}
\usepackage{xlop}
\usepackage{varwidth}
\usepackage{xinttools}%calculer le bareme total

%--------------------------------------------------------
%définition des clés
\usepackage{simplekv}
\usepackage{adjustbox}
 
%\settasks{
%% label = \theexercise.\arabic* ,
% item-indent = 0em ,
% label-width = 0em ,
% label-offset = 0pt ,
% column-sep = {10pt} 
% }

\begin{document}
\begin{Maquette}[Exam]{Prive=false, Date=2025}

\begin{exercice}[BaremeDetaille]
\begin{enumerate}
\item\brm{2} Calculer :\newline
$\left(\dfrac{3}{\sqrt{2}}\right)^{2}=$\anserline[1]

$\sqrt{25}\times \sqrt{16}\times\sqrt{9}=$
\anserline[2]
\item\brm{1} Simplifier \newline
$A=2\sqrt{63}+\sqrt{28}-\sqrt{7}=$\anserline[2]
\item\brm{1} Développer et simplifier\newline
 $\left(\sqrt{7}-1\right)^{2}=$\anserline[2]
 \item\brm{1} Factoriser :\newline
$(3-x)(x+1)+2x+2=$\anserline[2]
\item\brm{1}Donner l'écriture scientifique\newline
$39000000\times 10^{-26}=$\anserline[1]
\item\brm{0.5}\brm[0.5]{0.5} Ecrire le dénominateur sans radicale :\newline
\begin{minipage}{0.65\linewidth}
$\dfrac{1}{\sqrt{3}-1}=$\anserline[3]
\end{minipage}\hfill\vrule\hfill%
\begin{minipage}{0.33\linewidth}
$\dfrac{2}{\sqrt{5}}=$\anserline[3]
\end{minipage}
\end{enumerate}
\end{exercice}

\begin{exercice}[BaremeDetaille]
\begin{enumerate}
\item\brm{1} Comparer $3\sqrt{5}$ et $\sqrt{47}$\newline
\anserline[2]
\item\brm{1}\brm[0.5]{1}\brm[1]{1} Soient $2\leq a\leq 5$ et $1\leq b\leq 7$. Encadrer :
\end{enumerate}
\begin{minipage}{.3\linewidth}
$$a+b$$\newline\anserline[8]
\end{minipage}\hfill\vrule\hfill%
\begin{minipage}{.3\linewidth}
$$a-b$$\newline\anserline[8]
\end{minipage}\hfill\vrule\hfill%
\begin{minipage}{.3\linewidth}
$$a\times b$$\newline\anserline[8]
\end{minipage}
\end{exercice}

\begin{exercice}[BaremeDetaille]
On considère la figure ci-contre tel que $BH=2$, $HC=8$, $AH=4$ et $AB=2\sqrt{5}$.
\vspace*{3mm}

\begin{minipage}{.7\linewidth}
\begin{enumerate}
\item\brm{0.5} Montrer que $AC=4\sqrt{5}$.\newline\anserline[6]
\end{enumerate}
\end{minipage}%
\begin{minipage}{.3\linewidth}
\begin{tikzpicture}[scale=0.8]
\tkzDefPoints{0/0/A,5/1/C}
\tkzDefTriangle[school](C,A)
\tkzGetPoint{B}
\tkzDefPointBy[projection=onto B--C](A)
\tkzGetPoint{H}
\tkzDrawPolygon
(A,B,C)
\tkzDrawSegment(A,H)
\tkzLabelPoint[above](A){A}
\tkzLabelPoint[below](B){B}
\tkzLabelPoint[above](C){C}
\tkzLabelPoint[below right](H){H}
\tkzMarkRightAngle(A,H,C)
\tkzLabelSegment[sloped](B,H){2}
\tkzLabelSegment[sloped](H,C){8}
\tkzLabelSegment[sloped, above=1mm](A,H){4}
\tkzLabelSegment[ sloped](A,B){$2\sqrt{5}$}
\end{tikzpicture}
\end{minipage}%
\begin{enumerate}[start=2]
\item\brm{1} Prouver que $ABC$ est un triangle rectangle en $A$.\newline\anserline[7]
\item\brm{1.5} Calculer $\sin\widehat{C}$, $\cos\widehat{C}$ et $\tan\widehat{C}$\vspace{2mm}

\begin{minipage}{.3\linewidth}
$\sin\widehat{C}=$\anserline[4]
\end{minipage}\hfill\vrule\hfill%
\begin{minipage}{.3\linewidth}
$\cos\widehat{C}=$\anserline[4]
\end{minipage}\hfill\vrule\hfill%
\begin{minipage}{.3\linewidth}
$\tan\widehat{C}=$\anserline[4]
\end{minipage}
\end{enumerate}
\end{exercice}

\begin{exercice}[BaremeDetaille]
Soit $x$ la mesure d'un angle aigu tel que : $\cos x =\dfrac{1}{2}$
\begin{enumerate}
\item\brm{0.5}\brm[0.5]{0.5} Calculer : $\sin x$ et $\tan x$\vspace{5mm}

\begin{minipage}{0.48\linewidth}
\anserline[6]
\end{minipage}\hfill\vrule\hfill%
\begin{minipage}{0.48\linewidth}
\anserline[6]
\end{minipage}
\item\brm{1} Simplifier  :
$A = 3\sin^{2}x+3\cos^{2}x-3$\newline\anserline[7]
\item\brm{1} Montrer que : $ \dfrac{\tan^{2}a}{1+\tan^{2}a}=\sin^{2}a$

\anserline[7]
\end{enumerate}
\end{exercice}

\begin{exercice}[BaremeDetaille]
On considère la figure ci-contre tel que $(AB)//(EF)$, $AB=24$, $OB=21$, $OE=12$, $OF=14$, $BC=7$ et $BD=8$.
\vspace*{2mm}

\begin{minipage}{.7\linewidth}
\begin{enumerate}
\item\brm{1}\brm[0.5]{1} Calculer $OA$ et $EF$.\newline\anserline[6]
\end{enumerate}
\end{minipage}%
\begin{minipage}{.3\linewidth}
\begin{tikzpicture}
\tkzDefPoints{0/0/A,5/1/B,2/3/O}
\tkzDefPointOnLine[pos=1.2](B,O)\tkzGetPoint{F}
\tkzDefPointOnLine[pos=1.2](A,O)\tkzGetPoint{E}
\tkzDefPointOnLine[pos=0.4](B,O)\tkzGetPoint{C}
\tkzDefPointOnLine[pos=0.4](B,A)\tkzGetPoint{D}
\tkzDrawSegments(A,B B,F F,E E,A)
\tkzDrawPoints(C,D)
\tkzDrawLine[dashed,add=0.2 and 0.3](C,D)
\tkzLabelPoints(A,D,B)
\tkzLabelPoints[right](O,C)
\tkzLabelPoints[above](F,E)
\end{tikzpicture}
\end{minipage}\vspace{5mm}

\begin{adjustbox}{minipage=0.97\linewidth, right}
\anserline[8]
\end{adjustbox}

\begin{enumerate}[start=2]
\item\brm{1} Montrer que $(OA)//(DC)$.\newline\anserline[12]
\end{enumerate}
\end{exercice}

\end{Maquette}
\end{document}

