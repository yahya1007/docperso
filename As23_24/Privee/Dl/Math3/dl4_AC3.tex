\documentclass[a4paper,12pt]{article}

\usepackage{ProfModels}

  
\begin{document}
\begin{Maquette}[DM]{Niveau=3, Numero=4, Date=20/02/2025, FinDate=24/02/2025, Semestre=2}

\begin{exercice}
\begin{enumerate}
\item Résoudre les équations suivantes :
\begin{tasks}(3)
\task $4x-6=7x+11$ 
\task $(3x-5)(8x+7)=0$ 
\task $4^{2}-16=0$ 
\task $7x^{2}-2\sqrt{7}x+1=0$ 
\task $-7x^{2}-6=0$
\task $2x^{2}-1=0$
\end{tasks}
\item Résoudre les inéquations suivantes et représenter les solutions sur une droite graduée :
\begin{tasks}(2)
\task $2x-8\leq 3x+4$ 
\task $5(x-1)< 5x+6$
\end{tasks}
\end{enumerate}
\end{exercice}

\begin{exercice}
\begin{enumerate}
\item En utilisant la relation de Chasles , simplifier les écritures des vecteurs suivants :
\[
\vv{CB}+\vv{DA}+\vv{BD}\quad ;;\quad
 \vv{AB}-\vv{BD}+\vv{CA}-\vv{CB}\quad ;;\quad
 4\vv{AM}-3\vv{KG}-4\vv{GM}-\vv{AG}
\]
\end{enumerate}
\end{exercice}

\begin{exercice}
Soit $ABC$ un triangle.
\begin{enumerate}
\item Construis le point $E$ tel que :$\vv{AE}=\dfrac{3}{2}\vv{AB}$
\item Construis le point $F$ tel que : $\vv{EF}=\dfrac{-3}{2}\vv{CB}$
\item Construis le point $H$ tel que : $\vv{AH}=\vv{AB}+\dfrac{1}{2}\vv{AC}$
\item Montrer que $(EF)//(BC)$
\item Montrer que : $\vv{EH}=\dfrac{1}{2}\vv{AC}-\dfrac{1}{2}\vv{AB}$
\item Déduire que $E$ , $F$ et $H$ sont des points alignés.
\end{enumerate}
\end{exercice}

%\begin{exercice}
%Soit $ABC$ un triangle et $T$ la translation qui transforme le point $B$ au point $C$.
%\begin{enumerate}
%\item Construis le point $E$ l'image du point $A$ par la translation $T$.
%\item Construis le point $D$ l'image du point $C$ par la translation $T$.
%\item Déterminer l'image du triangle $ABC$ par la translation $T$.Justifie
%\end{enumerate}
%\end{exercice}

\begin{exercice}
$ABC$ un triangle rectangle en A tel que $AC=4 $ et $E$ le milieu de $[BC]$. Soit $t$ la translation qui transforme $A$ en $E$.
\begin{enumerate}
\item Construire $F$ image de $B$ et $G$ image de $C$ par la translation $t$.
\item Déterminer la longueur $EG$.
\item Prouver que $(FG)//(BC)$.
\item Déterminer la nature du triangle $EFG$. 
\end{enumerate}
\end{exercice}

\end{Maquette}
\end{document}
