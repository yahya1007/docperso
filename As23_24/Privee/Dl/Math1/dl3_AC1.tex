\documentclass[a4paper,12pt]{article}

\usepackage{ProfModels}
\usepackage{pstyle}


 
\begin{document}

\begin{Maquette}[DM]{Niveau=1, Numero=3, Date=19/10/2024,Semestre=1, FinDate=20/12/2024}

\begin{exercice}
\begin{enumerate}
\item Calculer les expressions suivantes :
\begin{itemize}
\item \(A= \dfrac{-42\times (-63)\times 7}{49\times (-21)\times 9}\)
\item \( B=-17+(-41)-(-29)+7+(-48)\times (-3)\div (-4)-(-6)\)
\end{itemize}

\item Compléter par le nombre qui convient :
\begin{tasks}[style=itemize](2)
\task \( (-9)\times \cdots = -63 \)
\task \( \cdots \times (-4) = 48\)
\task \( (-3+ \cdots )\times (-54)= -108\)
\task \(\dfrac{\cdots}{-7}=-6\)
\end{tasks}

\end{enumerate}
\end{exercice}

\begin{exercice}
\begin{minipage}{.58\linewidth}
On considère la figure ci-contre ; tel que les points A , B et C sont alignés .
\begin{enumerate}
\item Calculer la mesure de tous les angles de la figure .
\item Déduire la nature du triangle BEC
\item les points A,B et E sont ils alignés ?
\end{enumerate}
\end{minipage}
\begin{minipage}{.4\linewidth}
\begin{tikzpicture}
\tkzDefPoints{-2/0/A,3/0/E}
\tkzDefMidPoint(A,E)
\tkzGetPoint{B}
\tkzDefPointOnCircle[through= center B angle 60 point A]
\tkzGetPoint{C}
\tkzDrawSegments(A,C C,E E,A B,C)
\tkzLabelPoints(B,A,E)
\tkzLabelPoint[above](C){C}
\tkzMarkSegments[mark=s|](A,B B,C C,E)
\tkzInterLC(A,C)(C,E)
\tkzGetPoints{F}{D}
\tkzDrawSegments(C,D E,D)
\tkzDrawPoints(A,B,C,D,E)
\tkzMarkSegment[mark=s|](C,D)
\tkzLabelPoint[right](D){D}

\tkzMarkRightAngle(A,C,E)

\tkzMarkAngle[arc=l , size=.3 ](C,B,A)
\tkzLabelAngle[pos=0.6](C,B,A){$120^{\circ}$}

\end{tikzpicture}
\end{minipage}
\end{exercice}

\begin{exercice}
ABC est un triangle tel que : BC=7 cm AC=4 cm et AB=6 cm , la médiatrice $(\Delta)$  de [BC] coupe la droite (AB) en M .
\newline
Soit N la projection orthogonal de A sur (BC) .
\begin{enumerate}
\item Faire une figure 
\item Montrer que MBC est un triangle isocèle .
\item Montrer que : $(\Delta)$ //(AN)
\end{enumerate}
\end{exercice}




\end{Maquette}

\end{document}
