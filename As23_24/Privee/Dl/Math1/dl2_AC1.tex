\documentclass[a4paper,12pt]{article}

\usepackage{ProfModels}


 
\begin{document}

\begin{Maquette}[DM]{Niveau=1, Numero=2, Date=19/10/2024,Semestre=1, FinDate=20/12/2024}

\begin{exercice}
\begin{minipage}{.66\linewidth}
On considère la figure ci-contre :
\begin{enumerate}
\item  Compléter par $\in$ ou $\notin$
\end{enumerate}
\end{minipage}%
\begin{minipage}{.33\linewidth}
\begin{tikzpicture}[scale=0.5]
\tkzDefPoints{-3/0/A,2/0/C,3/1/D,-4/-1/E}
\tkzInterLL(A,C)(E,D)
\tkzGetPoint{B}
\tkzDrawLines(A,C D,E)
\tkzDrawPoints(A,B,C,D,E)
\tkzLabelPoints(B,C)
\tkzLabelPoints[above](E,A,D)
\end{tikzpicture}
\end{minipage}

\begin{tasks}(3)
\task $A\cdotsx{} [BC]$
\task $A\cdotsx{} [AC]$%\startnewitemline
\task $C\cdotsx{} [AB)$ 
\task $A\cdotsx{} [BC)$
\task $B\cdotsx{} [CA)$
\task $C\cdotsx{} [BA)$
\task $A\cdotsx{} [BE)$
\task $C\cdotsx{} [ED)$
\task $D\cdotsx{} [BE)$
\end{tasks}
\end{exercice}

\begin{exercice}
\begin{minipage}{.7\linewidth}

\begin{enumerate}
\item Trace la droite $(D_{1})$ perpendiculaire à $(AB)$ passant par C.
\item Trace la droite $(D_{2})$ perpendiculaire à $(BC)$ passant par A.
\end{enumerate}
\end{minipage}%
\begin{minipage}{.3\linewidth}
\begin{tikzpicture}[scale=0.8]
\tkzDefPoints{0/0/B,2/3/A,4/1/C}
\tkzDrawLines(A,B B,C C,A)
\tkzLabelPoints[above=2pt](A,B,C)
\end{tikzpicture}
\end{minipage}

\begin{enumerate}[start=3]
\item Trace la droite $(D_{3})$ perpendiculaire à $(AC)$ passant par B.
\item Que peut-on dire des droites $(D_{1})$ et $(D_{2})$ et $(D_{3})$ ? Justifier ta réponse.
\end{enumerate}



\end{exercice}
\begin{exercice}
\begin{enumerate}
\item Calculer ce qui suit : 
\end{enumerate}

\begin{tasks}(3)
  \task $-14+13=\cdotsx[3]$
  \task $23+(-19)=\cdotsx[2]$
  \task $	-14.26+(-13.5)=\cdotsx[2]$
  \task $-143-134=\cdotsx[4]$
  \task $	213-(-139)=\cdotsx[4]$
  \task $	-104.26-(13.5)=\cdotsx[4]$
  \task* $-14-(-9)+(+48)-(68)-(-51)+(-47)-(+23)-(-13)+(+68)-48= \cdotsx[6]$
\end{tasks}

\end{exercice}

\begin{exercice}
\begin{enumerate}
\item Donner les abscisses des points $A$ , $B$ , $O$ et $I$ 
\end{enumerate}
\begin{tikzpicture}
\tkzInit[xmin=-7.5, xmax=6]
\tkzAxeX
\tkzDefPoints{0/0/O,1/0/I,3/0/A,-4/0/B}
\tkzLabelPoints[above](O,I,A,B)
\end{tikzpicture}


\end{exercice}



\end{Maquette}
\end{document}
