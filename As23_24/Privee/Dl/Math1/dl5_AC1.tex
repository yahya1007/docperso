\documentclass[a4paper,12pt]{article}

\usepackage{ProfModels}



 
\begin{document}

\begin{Maquette}[DM]{Niveau=1, Numero=5, Date=19/05/2024,Semestre=2, FinDate=20/12/2024}

\begin{exercice}
Soit ABC un triangle tel que : $AC=6cm$ et $\widehat{BAC}=60^{\circ}$
\begin{enumerate}
\item Sur la figure ci-contre construire les points $E$,$F$ et $G$ les symétriques respectifs de $A$,$B$ et $C$ par rapport au point $O$.
\item Construire le point $M$ le milieu du segment $[AC]$, et $N$ le symétrique de $M$ par rapport à $O$.
\item Montrer que les points $E$,$N$ et $G$ sont alignés.
\item Monter que $(AB)//(EF)$.
\item Calculer la mesure de l'angle $\widehat{EFG}$.
\item Calculer la distance $EG$.
\item Montrer que $N$ est le milieu du segment $[EG]$.
\end{enumerate}
\end{exercice}

\begin{exercice}
Répondre par vrai ou faux
\begin{enumerate}
\item Un losange peut avoir des côtés de longueurs différentes.
\item Le carré est un losange.
\item Le rectangle a forcément les diagonales de même longueurs.
\item Un rectangle est forcément un parallélogramme.
\item N'importe quel quadrilatère qui a un angle droit est un rectangle.
\end{enumerate}
\end{exercice}

\begin{exercice}
\begin{minipage}{.58\linewidth}
$ABC$ et $CDA$ sont deux triangles équilatéraux.

Quelle est la nature du quadrilatère $ABCD$? Jusfifier.
\end{minipage}
\begin{minipage}{.4\linewidth}
\begin{tikzpicture}
\tkzDefPoint(0,0){D}
\tkzDefPoint(3,1){A}
\tkzDefTriangle[equilateral](D,A)
\tkzGetPoint{C}
\tkzDefTriangle[equilateral](C,A)
\tkzGetPoint{B}
\tkzDrawSegments(D,A A,B B,C C,D C,A)
\tkzLabelPoint[below](D){D}
\tkzLabelPoint[below](A){A}
\tkzLabelPoint[left](C){C}
\tkzLabelPoint[right](B){B}

\end{tikzpicture}
\end{minipage}

\end{exercice}

\begin{exercice}
Construis les quadrilatères demandés.
\begin{enumerate}
\item Le parallélogramme $IFGH$ avec $IF=5cm$ , $FG=4cm$ et $\widehat{IFG}=32^{\circ}$
\item Le losange $PLOT$ tel que $PL=6cm$ et $\widehat{LOT}=47^{\circ}$
\item Le rectangle $KRAC$ tel que $\widehat{RKA}=36^{\circ}$ et $RA=3cm$.
\end{enumerate}
\end{exercice}

\begin{exercice}
\begin{minipage}{.58\linewidth}
On considère la figure ci-contre, tel que $(AB)//(CD)$  et $\widehat{CDA}=38^{\circ}$ et $\widehat{CBA}=49^{\circ}$
\begin{enumerate}
\item Calculer la mesure des angles $\widehat{IAB}$ et $\widehat{ICD}$.
\item Déduire la mesure de $x$.
\end{enumerate}
\end{minipage}%
\begin{minipage}{.40\linewidth}
  \begin{tikzpicture}[scale=0.8]
\tkzDefPoints{0/0/A,4/0/B,1/3/C,6/3/D}
\tkzLabelPoints[below](A,B)
\tkzLabelPoints[above](C,D)
\tkzDrawPoints(A,B,C,D)
\tkzDrawLines(A,B C,D D,A C,B)
\tkzInterLL(A,D)(B,C)
\tkzGetPoint{I}
\tkzLabelPoint[above](I){I}
\tkzMarkAngle[arc=l , size=0.5cm , mark=|](C,D,A)
\tkzMarkAngle[arc=l , size=0.5cm , mark= ||](C,B,A)
\tkzMarkAngle[arc=l , size=0.5cm , mark= |||](B,I,D)

\tkzLabelAngle(C,D,A){$38^{\circ}$}
\tkzLabelAngle(C,B,A){$49^{\circ}$}
\tkzLabelAngle(B,I,D){$x$}

\end{tikzpicture}
\end{minipage}

\end{exercice}

\begin{exercice}
Soit $[AB]$ un segment de longueur $6cm$.
\begin{enumerate}
\item Tracer les cercles $C_1(A,3)$ et $C_2(B,4)$ .
\fullwidth{Les deux cercles se coupent en $E$ et $F$.}
\item Quelle est la nature du triangle $ABE$? Justifier.
\item Quelle est la nature du quadrilatère $AEBF$? Justifier.
\end{enumerate}
\end{exercice}

\begin{exercice}
Soit $C(O,4)$ un cercle de diamètre $[MN]$.
\begin{enumerate}
\item Tracer $(D)$ la tangente à $C$ en $M$ et $(D')$ la tangente à $C$ en $N$.
\item Montrer que $(D)//(D')$.
\end{enumerate}
\end{exercice}

\end{Maquette}
\end{document}
