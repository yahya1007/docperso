\documentclass[a4paper,12pt]{article}

\usepackage{ProfModels}
 
\begin{document}

\begin{Maquette}[DM]{Niveau=2, Numero=3, Date=06/01/2025, FinDate=09/01/2025, Semestre=1}

%\begin{exercice}
%\begin{enumerate}
%\item Calculer : 
%\[ 2022^{0} \quad; \quad (-1)^{101} \quad; \quad 0^{89} \quad; \quad (\dfrac{1}{5})^{-3}\]
%\item Déterminer le signe des puissances suivantes :
%\[ (\dfrac{-1}{13})^{-12} \quad; \quad 
%	(-7)^{2007}
%\] 
%\item Écrire sous forme $a^{n}$ les expressions suivantes :
%\[\dfrac{11^{-6}}{11^{-8}}\quad; \quad
%	\dfrac{3^{5}\times (-3)^{8}}{9^{9}}\quad; \quad
%	\dfrac{10^{5}\times 10^{-8}\times 10^{4}}{2^{6}\times 2^{-7}\times 2^{2}}\quad; \quad
%\dfrac{a^{-9}\times a^{5}\times (-a)^{18}}{a^{-4}}
%\]
%\item Simplifier : 
%\[
%\dfrac{ab^{-4}\times (a^{2}b^{-1})^{3}\times a^{-2}b^{3}}{a^{-5}\times (ab^{-1})^{2}\times (ab)^{-3}}
%\]
%\item Donner l'écriture scientifique des nombres suivants :
%\[
%	A=0.0000000025\times 0.00000000042 \quad; \quad
%	B=\dfrac{540\times 10^{-9}}{6\times 10^{-22}}\quad; \quad
%\]
%\end{enumerate}
%\end{exercice}
\begin{exercice}
$ABCD$ est un parallélogramme et  $I$ le milieu de  $[AD]$ et $J$ milieu de $[DC]$. Les droites$(IJ)$ et $(BD)$ se coupent en  $M$ .
\begin{enumerate}
\item Construire une figure.
\item Démontrer que $M$ est le milieu de $[IJ]$.
\end{enumerate}
\end{exercice}

\begin{exercice}
\begin{minipage}{0.6\linewidth}

On considère la figure ci-contre tel que $AD=4$ et $BD=6$ et $DE=3$. 
\begin{enumerate}
\item Sachant que $(DE)//(BC)$ , calculer $BC$
\item Construire le centre de gravité du triangle $ABC$ et $I$ le milieu de $[BC]$.
\item Sachant que  $IG=3$ calculer $AG$.
\end{enumerate}
\end{minipage}
\begin{minipage}{0.4\linewidth}
\begin{tikzpicture}
\tkzDefPoints{1/2/A,5/0/B,2/4/C}
\tkzDefPointOnLine[pos=.3](A,B)\tkzGetPoint{D}
\tkzDefPointOnLine[pos=.3](A,C)\tkzGetPoint{E}
\tkzDrawLines(A,B A,C B,C)
\tkzDrawSegment(D,E)
\tkzLabelPoints(A,D,B)
\tkzLabelPoints[above,left ](E,C)
\end{tikzpicture}
\end{minipage}
\end{exercice}

\begin{exercice}
$a$ et $b$ deux nombres rationnels non nuls .
\begin{enumerate}
\item Simplifier ce qui suit :

\begin{tasks}[style=itemize](4)
\task $A=a^{2}\times a \times a^{-4} $ 
\task $B=\dfrac{a^{2}\times a^{3}}{a^{-4}}$
\task* $C=\dfrac{ab^{-4}\times (a^{2}b^{-1})^{3}\times a^{-2}b^{3}}{a^{-5}\times (ab^{-1})^{2}\times (ab)^{3}}$
\end{tasks}
\item Déterminer l'écriture scientifique : 

\begin{tasks}[style=itemize](3)
\task $x=0.000000000000114$
\task $y=32100000000000000$
\task $z=\dfrac{14\times 10^{14}\times 10^{-8}}{5\times 10^{-12}\times 10^{20}}$
\end{tasks}

\end{enumerate}
\end{exercice}
\begin{exercice}
$EFGH$ est un carré de centre $I$ et $O$ le milieu de $\lrc{HG}$; la droite $\lrp{FO}$ coupe $\lrp{EG}$ en $K$.
\begin{minipage}{0.55\linewidth}
\begin{enumerate}
\item Que représente le point $K$ pour le triangle $FGH$?
\item Calculer $GK$ sachant que $GI=6$.
\item Montrer que $\lrp{OI}$ est la médiatrice de $\lrc{HG}$.
\item Que représente le point $I$ pour le triangle $FGH$?
\end{enumerate}
\end{minipage}%
\begin{minipage}{0.45\linewidth}
\begin{center}
\begin{tikzpicture}[scale=0.8]
\tkzDefPoints{0/0/E,4/0/F}
\tkzDefSquare(E,F)\tkzGetPoints{G}{H}
\tkzDefPointOnLine[pos=0.5](H,G)\tkzGetPoint{O}
\tkzInterLL(E,G)(F,H)\tkzGetPoint{I}
\tkzInterLL(F,O)(E,G)\tkzGetPoint{K}
\tkzDrawPolygon(E,F,G,H)
\tkzDrawSegments(E,G F,H F,O)
\tkzLabelPoints(E,F,I)
\tkzLabelPoints[above](H,G,O)
\tkzLabelPoint[right](K){K}
\tkzMarkSegments[mark=||](H,O O,G)
\end{tikzpicture}
\end{center}
\end{minipage}
\end{exercice}

\end{Maquette}
\end{document}