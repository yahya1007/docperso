\documentclass[a4paper,12pt]{article}

\usepackage{ProfModels}



 
\begin{document}
\begin{Maquette}[DM]{Niveau=2, Numero=4, Date=20/02/2025, FinDate=24/02/2025, Semestre=2}

\begin{exercice}
\begin{enumerate}
\item Réduire les expressions suivantes :
\begin{tasks}[style=itemize]
\task $A=2x-\left(7+51a+9x-\left(6x-21a+70\right)-4\right)$
\task $B=-12a-71x+9x^{2}-2x+7x^{2}$
\end{tasks}
\item Développer et réduire :
\begin{tasks}[style=itemize](4)
\task*(2) $A=-6\left(1-4x\right)+3x\left(-6x+2\right)$
\task*(2) $B=\left(x+5\right)\left(2x-3\right)-5\left(x-3\right)$
\task $M=\left(x+7\right)^{2}$
\task $N=\left(4x-9\right)^{2}$
\task*(2) $P=\left(2x-7\right)\left(2x+7\right)$
\task*(2) $Q=\left(\dfrac{x}{8}-8\right)\left(\dfrac{x}{8}+8\right)$
\task*(2) $R=\left(7x-\dfrac{2}{7}\right)\left(7x+\dfrac{2}{7}\right)$
\end{tasks}
\end{enumerate}
\end{exercice}

\begin{exercice}
\begin{enumerate}
\item Factoriser :
\begin{tasks}[style=itemize](6)
\task*(3) $a=24\left(x-2\right)-3x\left(-2x+4\right)$
\task*(3) $b=3x^{2}+3x\left(2x-5\right)-3x$
\task* $c=12x^{3}+24x^{2}\left(x+3\right)-44x^{5}$
\task* $g=\left(\dfrac{5}{2}x-1\right)\left(x-\dfrac{3}{2}\right)+2\left(\dfrac{5}{2}x-1\right)-\dfrac{4}{3}x\left(\dfrac{5}{2}x-1\right)$
%\task*(2) $A=25x^{2}+20x+4$
%\task*(2) $B=49x^{2}-81$
%\task*(2) $C=81-36x+4x^{2}$
\end{tasks}
\end{enumerate}
\end{exercice}

\begin{exercice}
\begin{enumerate}
\item Résoudre les équations suivantes 
\end{enumerate}
\begin{tasks}[style=itemize](4)
\task $3x-\dfrac{7}{3}=3$
\task $x+\dfrac{21}{13}=\dfrac{11}{2}$
\task $\dfrac{-x}{2}+\dfrac{8}{7}=\dfrac{-1}{4}$
\task $5x-\dfrac{2}{3}=\dfrac{5x}{4}+1$
\task*(2) $\dfrac{1-x}{5}+\dfrac{1-2x}{2}=1-\dfrac{x-2}{2}$
\task*(2) $-\dfrac{4x-1}{3}=3x-\dfrac{2x+5}{8}+\dfrac{23-50x}{24}$
\end{tasks}
\begin{tasks}
\task $(2x+4)(3x-7)=0$
\task $(x+1)(3x-4)=(x+1)(5x-9)$
\task $7x-7+(3x-8)(x-1)=0$
%\task $4x^{2}-81=0$
%\task*(2) $\left(4x-1\right)^{2}-(5x+6)^{2}=0$
%\task $4x^{2}-4x+1=0$
%\task $16x^{2}-25=0$
%\task $x^{2}+2x+1=0$
%\task $x^{2}=1$
%\task $4+2x^{2}=0$
\end{tasks}
\end{exercice}

\begin{exercice}
Si tous les inscrits étaient venus, la sortie en autocar aurait coûté 250 DH par personne. Mais il y a eu 3 absents et chaque participant a du payer un supplément de 15 DH . Combien y avait-il d'inscrits ?
\end{exercice}
\end{Maquette}
\end{document}