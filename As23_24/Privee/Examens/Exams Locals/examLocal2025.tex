\documentclass[a4paper,12pt]{article}

\usepackage{ProfModels}
\usepackage{adjustbox}
 
%\settasks{
%% label = \theexercise.\arabic* ,
% item-indent = 0em ,
% label-width = 0em ,
% label-offset = 0pt ,
% column-sep = {10pt} 
% }

\begin{document}
\begin{Maquette}[Exam]{Prive=true, Date=2025}

\begin{exercice}[BaremeDetaille]
\begin{enumerate}
\item\brm{2} Calculer :\newline
$\left(\dfrac{3}{\sqrt{2}}\right)^{2}+\dfrac{1}{2}=$\anserline[1]

$\sqrt{\sqrt{25}}\times \sqrt{\sqrt{16}}\times\sqrt{5}=$
\anserline[2]
\item\brm{1} Simplifier \newline
$A=2\sqrt{63}+\sqrt{28}-\sqrt{7}=$\anserline[2]
\item\brm{1} Développer et simplifier\newline
 $\left(\sqrt{7}-1\right)^{2}+2(\sqrt{7}-2)=$\anserline[2]
 \item\brm{1} Factoriser :\newline
$(3-x)(x+1)+2x+2=$\anserline[2]
\item\brm{1}Donner l'écriture scientifique\newline
$39000000\times 10^{-26}=$\anserline[2]
\item\brm{1} Ecrire le dénominateur sans radicale :\newline
$\dfrac{\sqrt{3}+1}{\sqrt{3}-1}=$\anserline[2]
\end{enumerate}
\end{exercice}

\begin{exercice}[BaremeDetaille]
\begin{enumerate}
\item\brm{1} Comparer $3\sqrt{5}$ et $\sqrt{47}$\newline
\anserline[2]
\item\brm{1}\brm[0.5]{1}\brm[1]{1} Soient $2\leq a\leq 5$ et $1\leq b\leq 7$. Encadrer :
\end{enumerate}
\begin{minipage}{.3\linewidth}
$$2a+3b$$\newline\anserline[8]
\end{minipage}\hfill\vrule\hfill%
\begin{minipage}{.3\linewidth}
$$a-b$$\newline\anserline[8]
\end{minipage}\hfill\vrule\hfill%
\begin{minipage}{.3\linewidth}
$$\dfrac{a}{b}$$\newline\anserline[8]
\end{minipage}
\end{exercice}

\begin{exercice}[BaremeDetaille]
On considère la figure ci-contre tel que $BH=2$, $HC=8$ et $AH=4$.
\vspace*{3mm}

\begin{minipage}{.7\linewidth}
\begin{enumerate}
\item\brm{1} Calculer $AC$.\newline\anserline[5]
\end{enumerate}
\end{minipage}%
\begin{minipage}{.3\linewidth}
\begin{tikzpicture}[scale=0.8]
\tkzDefPoints{0/0/A,5/1/C}
\tkzDefTriangle[school](C,A)
\tkzGetPoint{B}
\tkzDefPointBy[projection=onto B--C](A)
\tkzGetPoint{H}
\tkzDrawPolygon
(A,B,C)
\tkzDrawSegment(A,H)
\tkzLabelPoint[above](A){A}
\tkzLabelPoint[below](B){B}
\tkzLabelPoint[above](C){C}
\tkzLabelPoint[below right](H){H}
\tkzMarkRightAngle(A,H,C)
\tkzLabelSegment(B,H){2}
\tkzLabelSegment(H,C){8}
\tkzLabelSegment(A,H){4}
\end{tikzpicture}
\end{minipage}%
\begin{enumerate}[start=2]
\item\brm{1} Calculer $AB$.\newline\anserline[5]
\item\brm{1} Prouver que $ABC$ est un triangle rectangle.\newline\anserline[6]
\end{enumerate}
\end{exercice}

\begin{exercice}[BaremeDetaille]
Soit $x$ la mesure d'un angle aigu tel que : $\cos x =\dfrac{1}{2}$
\begin{enumerate}
\item\brm{0.5}\brm[0.5]{0.5} Calculer : $\sin x$ et $\tan x$\vspace{5mm}

\begin{minipage}{0.48\linewidth}
\anserline[6]
\end{minipage}\hfill\vrule\hfill%
\begin{minipage}{0.48\linewidth}
\anserline[6]
\end{minipage}
\item\brm{1} Simplifier  :
$A = \sin^{2}25^{\circ} +\cos50^{\circ} +\sin^{2}65^{\circ}-\sin40^{\circ}$\newline\anserline[7]
\item\brm{1} Montrer que : $\sin^{2}a = \dfrac{\tan^{2}a}{1+\tan^{2}a}$

\anserline[7]
\end{enumerate}
\end{exercice}

\begin{exercice}[BaremeDetaille]
On considère la figure ci-contre tel que $(AB)//(EF)$, $AB=24$, $OB=21$, $OE=12$, $OF=14$, $BC=7$ et $BD=8$.
\vspace*{2mm}

\begin{minipage}{.7\linewidth}
\begin{enumerate}
\item\brm{1}\brm[0.5]{1} Calculer $OA$ et $EF$.\newline\anserline[6]
\end{enumerate}
\end{minipage}%
\begin{minipage}{.3\linewidth}
\begin{tikzpicture}
\tkzDefPoints{0/0/A,5/1/B,2/3/O}
\tkzDefPointOnLine[pos=1.2](B,O)\tkzGetPoint{F}
\tkzDefPointOnLine[pos=1.2](A,O)\tkzGetPoint{E}
\tkzDefPointOnLine[pos=0.4](B,O)\tkzGetPoint{C}
\tkzDefPointOnLine[pos=0.4](B,A)\tkzGetPoint{D}
\tkzDrawSegments(A,B B,F F,E E,A)
\tkzDrawPoints(C,D)
\tkzDrawLine[dashed,add=0.2 and 0.3](C,D)
\tkzLabelPoints(A,D,B)
\tkzLabelPoints[right](O,C)
\tkzLabelPoints[above](F,E)
\end{tikzpicture}
\end{minipage}\vspace{5mm}

\begin{adjustbox}{minipage=0.97\linewidth, right}
\anserline[6]
\end{adjustbox}

\begin{enumerate}[start=2]
\item\brm{1} Montrer que $(OA)//(DC)$.\newline\anserline[12]
\end{enumerate}
\end{exercice}

\end{Maquette}
\end{document}

