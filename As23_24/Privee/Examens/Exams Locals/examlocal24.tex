\documentclass[a4paper,12pt]{article}

\usepackage{dlds}
 
\begin{document}
\begin{tcolorbox}[enhanced,colback=white,colframe=black] 
{\begin{center}\Large \textbf{ Préparation d'Examen Local  }\end{center}}
\end{tcolorbox}
\begin{exo}
\begin{enumerate}
\item Calculer ce qui suit :
$$
A=\left(\left(2025-2023\right)^{3}-9\right)^{1444}  
\hspace{0.5cm};\hspace{0.5cm}
B=\dfrac{\sqrt{162}+\sqrt{32}}{\sqrt{75}-2\sqrt{108}+9\sqrt{3}}
\hspace{0.5cm};\hspace{0.5cm}
C=\sqrt{4+2\sqrt{3}}
$$
\item Rendre rationnel les dénominateurs des nombres suivants : 
$$
D=\dfrac{2}{\sqrt{11}-\dfrac{\sqrt{2}}{\sqrt{11}}}
\hspace{0.5cm};\hspace{0.5cm}
E=\dfrac{1}{\sqrt{7}+\sqrt{5}-1}
$$
\item Donner l'écriture scientifique : $F=\dfrac{48\times (10^{-2})^{3}\times 10^{19}}{3\times 10^{4}}$
\item Développer : $4\left(6x-5\right)^{2}-3x\left(x-2\right)^{2}$ 
\item Factoriser : $5x^{2}-5+\left(x+1\right)\left(3x-4\right)-x^{2}-2x-1$
\end{enumerate}
\end{exo}

\begin{exo}
\begin{enumerate}
\item Comparer les deux nombres : $-3\sqrt{5}$ et $-4\sqrt{3}$.\newline
Soient $a$ , $b$ et $c$ des nombres réels tel que  : $2\leq a \leq 7$ et $-3\leq b \leq -1$. 
\item Encadrer $\dfrac{2a+5b}{3}$ et $\dfrac{3a-2b}{4}$ et $a\times b^{2}$ et $\dfrac{5a}{6b}$

\end{enumerate}
\end{exo}

\begin{exo}
\begin{minipage}{0.6\linewidth}
 On considère la figure ci-contre tel que : $AC=5$  et $HC=4$.
\begin{enumerate}
\item Calculer : $AH$.
\item Calculer les rapports trigonométriques d'angle $\widehat{ACH}$
\item Calculer $AB$ et $BH$.
\end{enumerate}
\end{minipage}
\begin{minipage}{0.4\linewidth}
\begin{tikzpicture}
\tkzDefPoints{-1/0/B,5/-1/C}
\tkzDefTriangle[two angles=50 and 40](B,C)\tkzGetPoint{A}
\tkzDefPointBy[projection=onto C--B](A)\tkzGetPoint{H}
\tkzDrawSegments(A,B B,C C,A A,H)
\tkzLabelPoints(B,C,H)
\tkzLabelPoint[above](A){A}
\tkzMarkRightAngle(C,H,A)
\tkzMarkRightAngle(B,A,C)
\end{tikzpicture}
\end{minipage}
\end{exo}

\begin{exo}
\begin{minipage}{.6\linewidth}
On considère la figure ci-contre tel que $(AB)//(EF)$ et $AB=24$ et $OB=21$ et $OE=12$ et $OF=14$ et $BC=7$ et $BD=8$.
\begin{enumerate}
\item Calculer $OA$ et $EF$.
\item Montrer que $(OA)//(DC)$.
\item Calculer $DC$.
\end{enumerate}
\end{minipage}%
\begin{minipage}{.4\linewidth}
\begin{tikzpicture}
\tkzDefPoints{0/0/A,5/1/B,2/3/O}
\tkzDefPointOnLine[pos=1.2](B,O)\tkzGetPoint{F}
\tkzDefPointOnLine[pos=1.2](A,O)\tkzGetPoint{E}
\tkzDefPointOnLine[pos=0.4](B,O)\tkzGetPoint{C}
\tkzDefPointOnLine[pos=0.4](B,A)\tkzGetPoint{D}
\tkzDrawSegments(A,B B,F F,E E,A)
\tkzDrawPoints(C,D)
\tkzDrawLine[dashed,add=0.2 and 0.3](C,D)

\tkzLabelPoints(A,D,B)
\tkzLabelPoints[right](O,C)
\tkzLabelPoints[above](F,E)
\end{tikzpicture}
\end{minipage}
\end{exo}

\begin{exo}
\begin{enumerate}
\item Simplifier :
$$A=cos a(sin a + cos a )-sin a(cos a -sin a) $$
$$B=\dfrac{1}{1+sin a}+\dfrac{1}{1-sin a}-\dfrac{2}{cos^{2}a} $$
$$C=(cosa + sina)^{2}+(cosa-sina)^{2} $$
$$D=cos^{4}a-sin^{4}a-cos^{2}a+3sin^{2}a$$
$$E=sina\times \sqrt{1-cosa}\times \sqrt{1+cosa}+cos^{2}a$$
$$F=\sqrt{2}sin^{2}a+\sqrt{2}cos^{2}a$$
\item Montrer que :
$$\dfrac{cos^{4}a-sin^{4}a}{cos^{2}a-sin^{2}a}=1$$
$$\dfrac{1-cosx}{sinx}=\dfrac{sinx}{1+cosx}$$
\item Calculer 
$$ X= 2cos75^{\circ}+6cos^{2}86^{\circ}-2sin15^{\circ}+6cos^{2}4^{\circ}$$
$$ Y=cos^{2}48^{\circ}-sin^{2}50^{\circ}+cos^{2}42^{\circ}+cos^{2}40^{\circ}$$
$$ Z=2tan73^{\circ}\times tan17^{\circ}-sin^{2}40^{\circ}-sin^{2}50^{\circ}$$
\end{enumerate}
\end{exo}
\end{document}