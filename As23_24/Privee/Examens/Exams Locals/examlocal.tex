\documentclass[a4paper,12pt]{article}

\usepackage{dlds}
 
\begin{document}
\begin{tcolorbox}[enhanced,colback=white,colframe=black] 
{\begin{center}\Large \textbf{ Préparation d'Examen Local  }\end{center}}
\end{tcolorbox}
\begin{exo}
\begin{enumerate}
\item Calculer ce qui suit :
$$
A=\left( \dfrac{\sqrt{7}}{\sqrt{3}}\right)^{2}+\left( \dfrac{\sqrt{3}}{\sqrt{5}}\right)^{-2}  
\hspace{0.5cm};\hspace{0.5cm}
B=3\sqrt{75}+2\sqrt{300}-10\sqrt{12}
\hspace{0.5cm};\hspace{0.5cm}
C=\sqrt{10+\sqrt{36}}
$$
\item Rendre rationnel les dénominateurs des nombres suivants : 
$$
D=\dfrac{2}{\sqrt{11}}
\hspace{0.5cm};\hspace{0.5cm}
E=\dfrac{1}{\sqrt{7}+\sqrt{5}}
$$
\item Donner l'écriture scientifique : $F=0.0042 \times 10^{-5}$
\item Développer  :$G=(\sqrt{3}-a)^{2}+2(1+\sqrt{3}a)$ 
\item Factoriser : $(100-a^{2})+2(10+a)$
\end{enumerate}
\end{exo}

\begin{exo}
\begin{enumerate}
\item Comparer les deux nombres : $3\sqrt{5}$ et $3\sqrt{7}$ , puis  $5-3\sqrt{5}$ et $5-3\sqrt{7}$ .\newline
Soient $a$ , $b$ et $c$ des nombres réels tel que  : $1\leq a \leq 5$ et $3\leq b \leq 4$.
\item Encadrer $a+b$ et $a-b$ et $a\times b$ et $\dfrac{a}{b}$
\item Encadrer $5a+7b$ et $\dfrac{2a-5b}{3}$
\end{enumerate}
\end{exo}

\begin{exo}
\begin{minipage}{0.6\linewidth}
 On considère la figure ci-contre tel que : $AC=\sqrt{52}$ , $BH=9$ et $HC=4$.
\begin{enumerate}
\item Calculer : $AH$ et $AB$ .
\item Montrer que $ABC$ est un triangle rectangle .
\item Calculer les rapports trigonométriques d'angle $\widehat{ABC}$

Soit $x$ la mesure d'un angle aigu tel que : $\sin x =\dfrac{1}{2}$
\item Calculer : $\cos x$ et $\tan x$
\item Simplifier  :
$A = \sin^{2}26^{\circ} +4\cos30^{\circ} +\sin^{2}64^{\circ}-4\sin60^{\circ} $
\item Montrer que : $\dfrac{1+\tan^{2}x}{\tan^{2}x}\times (1-cos^{2}x)=1$
\end{enumerate}
\end{minipage}
\begin{minipage}{0.4\linewidth}
\begin{tikzpicture}
\tkzDefPoints{-1/0/B,5/-1/C}
\tkzDefTriangle[two angles=50 and 40](B,C)\tkzGetPoint{A}
\tkzDefPointBy[projection=onto C--B](A)\tkzGetPoint{H}
\tkzDrawSegments(A,B B,C C,A A,H)
\tkzLabelPoints(B,C,H)
\tkzLabelPoint[above](A){A}
\tkzMarkRightAngle(C,H,A)
\end{tikzpicture}
\end{minipage}
\end{exo}

\begin{exo}
\begin{minipage}{0.5\linewidth}
$ABC$ est un triangle tel que : $AC=6$ , $CB=9$ , $AJ=4$  et $(BC)//(JK)$
\begin{enumerate}
\item Calculer  $JK$ .
\item Soit $I$ un point de $[CB] $ tel que: $CI=3$\newline
Monter que : $(AB)//(IJ)$
\end{enumerate}
\end{minipage}
\begin{minipage}{0.5\linewidth}
\begin{tikzpicture}
\tkzDefPoints{0/0/C,5/0/B,-1/4/A}
\tkzDefPointOnLine[pos=.3](C,A)\tkzGetPoint{J}
\tkzDefPointOnLine[pos=.3](C,B)\tkzGetPoint{I}
\tkzDefPointOnLine[pos=.3](B,A)\tkzGetPoint{K}
\tkzDrawLines(I,J J,K)
\tkzDrawSegments(A,C A,B B,C)
\tkzLabelPoints(B,C,I)
\tkzLabelPoints[below left](A,J)
\tkzLabelPoint[above](K){K}
\end{tikzpicture}
\end{minipage}
\end{exo}

\end{document}