\documentclass[a4paper,12pt]{article}

\usepackage{ProfModels}

\setlength{\columnseprule}{1pt}
\setlength{\columnsep}{3em}
\renewcommand{\columnseprulecolor}{\color{gray}}
\begin{document}
\begin{Maquette}[DS]{Numero=5, Niveau=3, Date=28/04/2025,Semestre=2}

\begin{exercice}[BaremeDetaille=true]
Le plan est muni d'un repère orthonormé $\oij$ et on considère les points $A(1,2)$ ; $B(3,6)$ et la droite $(D):y=\dfrac{-1}{2}x+5$.
\begin{enumerate}
\item \brm{2}Détermine les coordonnées du vecteur $\vv{AB}$ et déduire $AB$\newline
\anserline[2]
\item \brm{2}Donne les coordonnées du point $K$ le milieu du segment $[AB]$.\newline
\anserline[2]
\item \brm{2}La droite $(D)$ passe-t-elle par le point $K$?
\newline
\anserline[2]
\item \brm{2}Montrer que l'équation réduite de la droite $(AB)$ est $y=2x+0$.
\newline
\anserline[4]
\item\brm{2} Donne l'équation réduite de la droite $(D')$ qui passe par le point $C(3,2)$ et parallèle à la droite $(AB)$.
\newline\anserline[4]
\item \brm{2}Montre que la droite $(D)$ est perpendiculaire à $(AB)$.\newline\anserline[4]
\item\brm{2} Déduire que la droite $(D)$ est la médiatrice du segment $[AB]$.\newline\anserline[3]
\end{enumerate}
\end{exercice}

\begin{exercice}
\begin{enumerate}
\item Résoudre les systèmes :\brm{4}
\end{enumerate}
\begin{multicols}{2}
$(S_1)\systeme{3x+y=9,-2x-y=-11}$\newline
\anserline[8]
\columnbreak

$(S_2)\systeme{3x+4y=35,2x+8y=42}$\newline
\anserline[8]
\end{multicols}
\end{exercice}

\begin{exercice}
\brm{2}Une mère a dit à son fils : j'ai dans ma poche 47 DH, et si j'achète 3 Kg de tomates et 4 Kg de pommes de terre il me reste 12 DH.Et si j'achète 2 Kg de tomates et 8 Kg de pommes de terre il me reste 5 DH.

Calculer le prix d'un kilogramme de tomates et le prix d'un kilogramme de pommes de terre.
\begin{multicols}{2}
\anserline[10]
\end{multicols}
\end{exercice}
\end{Maquette}
\end{document}