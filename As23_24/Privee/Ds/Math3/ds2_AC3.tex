\documentclass[a4paper,12pt]{article}




 
\usepackage{ProfModels}
\usetikzlibrary{arrows}
\setlength{\columnseprule}{0.4pt}
%\setlength{\columnseprule}{1pt}
%\setlength{\columnsep}{2em}
%\renewcommand{\columnseprulecolor}{\color{gray}} 
\begin{document}
\begin{Maquette}[DS]{Numero=2, Niveau=3, Date=17/12/2024}

\begin{exercice}
$a$ et $b$ sont deux nombres réels tels que $a\leq b$
\begin{enumerate}

\begin{minipage}{.5\linewidth}
\item Comparer $a$ et $\dfrac{a+b}{2}$ \brm{4}
\end{minipage}%
\begin{minipage}{.5\linewidth}
\item Comparer $3\sqrt{8}$ et $\sqrt{73}$ 
\end{minipage}
\end{enumerate}
\begin{minipage}{.5\linewidth}
\notes[10pt]{8}{.9\linewidth}
\end{minipage}%
\vline
\begin{minipage}{.5\linewidth}
\notes[10pt]{8}{.9\linewidth}
\end{minipage}
\end{exercice}

\begin{exercice}
$a$ et $b$ sont deux nombres réels tels que $3\leq a\leq 7$ et $4\leq b\leq 8$.Encadrer :\brm{8}

\begin{minipage}{.24\linewidth}
$2a+3b$
\end{minipage}
\begin{minipage}{.24\linewidth}
$a-b$
\end{minipage}
\begin{minipage}{.24\linewidth}
$ab$ 
\end{minipage}
\begin{minipage}{.24\linewidth}
$\dfrac{a}{b}$ 
\end{minipage}%

\begin{minipage}{0.24\linewidth}
\notes[10pt]{12}{.9\linewidth}
\end{minipage}
\vline
\begin{minipage}{0.24\linewidth}
\notes[10pt]{12}{.9\linewidth}
\end{minipage}
\vline
\begin{minipage}{0.24\linewidth}
\notes[10pt]{12}{.9\linewidth}
\end{minipage}
\vline
\begin{minipage}{0.24\linewidth}
\notes[10pt]{12}{.9\linewidth}
\end{minipage}
\end{exercice}

\begin{exercice}
\begin{minipage}{0.4\linewidth}
\begin{tikzpicture}
\tkzDefPoint(3,-1){B}
\tkzDefShiftPoint[B](1,4){C}
\tkzDefTriangle[two angles=40 and 60](B,C)\tkzGetPoint{A}
\tkzDefPointOnLine[pos=0.3](B,A)\tkzGetPoint{M}
\tkzDefLine[parallel=through M](B,C)\tkzGetPoint{x}
\tkzInterLL(M,x)(A,C)\tkzGetPoint{N}
\tkzDefPointOnLine[pos=1.3](B,A)\tkzGetPoint{E}
\tkzDefPointOnLine[pos=1.3](C,A)\tkzGetPoint{F}

\tkzDrawLines(B,E C,F)
\tkzDrawSegments(B,C M,N E,F)
\tkzDrawPoints(A,B,C,M,N,E,F)
\tkzLabelPoints(F)
\tkzLabelPoints[above right](E,A,N,C)
\tkzLabelPoints[below=6pt](M,B)

\end{tikzpicture}
\end{minipage}
\begin{minipage}{0.6\linewidth}
$ABC$ est un triangle tels que : $AB=16$ et $AC=8$ et soit  $M$ un point de  $[AB]$ tel que : $BM=4$\brm{8}

La droite paralléle à $(BC)$ passant de $M$ coupe  $(AC)$ en $N$ voir la figure ci-contre
\begin{enumerate}
\item Montrer que : $AN=6$ puis calculer $NC$ .
\item On considère $E$ un point de $[MA)$ et $F$ un point de $[NA)$ tel que : $AE=4$ et $AF=2$

Montrer que : $(BC)//(EF)$
\end{enumerate}
\end{minipage}%

\begin{minipage}{\linewidth}
\notes[10pt]{23}{\linewidth}
\end{minipage}

\end{exercice}

\anserpage{1}

\begin{exercice}
$ABCD$ est un trapèze isocèle ses bases $[AB]$ et $[CD]$ tels que : $AB=4$ et $CD=10$ et $BC=5$.

Soit $M$ de $[BC]$ et $N$ de $[AC]$ tels que $CM=2$ et $(MN)//(CD)$
\begin{enumerate}
\item Calculer $MN$ puis le rapport $\dfrac{CN}{CA}$
\item Soit $I$ de $[CD]$ et $J$ de $[AD]$ tels que $DI=8$ et $DJ=4$. Montrer que : $(IJ)//(AC)$
\item Monter que : $IJ - 2CN=0$.
\end{enumerate}
\begin{tikzpicture}
\tkzDefPoints{0/0/A,4/0/B,5/-4/C,-1/-4/D}
\tkzDrawPolygon(A,B,C,D)
\tkzDefPointOnLine[pos=0.3](C,B)\tkzGetPoint{N}
\tkzDefPointOnLine[pos=0.3](C,A)\tkzGetPoint{M}
\tkzDefPointOnLine[pos=0.4](C,D)\tkzGetPoint{I}
\tkzDefPointOnLine[pos=0.4](A,D)\tkzGetPoint{J}
\tkzDrawSegments(M,N I,J A,C)
\tkzDrawPoints(A,B,C,D,M,N,I,J)
\tkzLabelPoints(I,M)
\tkzLabelPoints[left](A,J,D)
\tkzLabelPoints[right](B,N,C)
\end{tikzpicture}
\end{exercice}

\begin{exercice}
\begin{minipage}{0.4\linewidth}
\begin{tikzpicture}
\tkzDefPoints{0/0/C,4/1/A,2/4/B}
\tkzDefPointOnLine[pos=0.7](A,B)\tkzGetPoint{E}
\tkzDefPointOnLine[pos=0.7](A,C)\tkzGetPoint{F}
\tkzDrawLine(E,F)
\tkzDrawPolygon(A,B,C)
\tkzDrawPoints(A,B,C,E,F)

\tkzLabelPoints(C,F,A)
\tkzLabelPoints[right](E,B)
\end{tikzpicture}
\end{minipage}
\begin{minipage}{0.6\linewidth}
On considère la figure ci-contre tels que  $(EF) // (BC)$  et  $AF=x$ et $AE=x-1$ et $AC=9$ et $AB=7$

\begin{enumerate}
\item Calculer le nombre $x$.
\item Déduire que : $BC=2EF$.
\end{enumerate}
\end{minipage}
\end{exercice}
\end{Maquette}
\end{document}