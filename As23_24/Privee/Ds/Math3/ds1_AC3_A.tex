\documentclass[a4paper,12pt]{article}

\usepackage{ProfModels}


\begin{document}
\begin{Maquette}[DS]{Numero=1, Niveau=3, Date=13/06/2024}
\begin{exercice}[BaremeDetaille]
\begin{enumerate}
\begin{multicols}{2}
\item\brm{2} Développer et simplifer :

\( A= 3(x+1) \)
\notes[10pt]{3}{\linewidth}
\( B= 5a(2a-1) \)
\notes[10pt]{3}{\linewidth}
\columnbreak
\( C= (3x+4)(4x+1) \)
\notes[10pt]{3}{\linewidth}
\( D= (2x+3)^{2} \)
\notes[10pt]{3}{\linewidth}
\end{multicols}%
\begin{multicols}{2}
\item\brm{2} Factoriser :

\( E= (3x+7)^{2}-(3x-7)^{2} \)
\notes[10pt]{3}{\linewidth}
\( X= 81x-18 \)
\notes[10pt]{3}{\linewidth}
\columnbreak
\( Y= 3(x+1)-4(2x+7)(x+1) \)
\notes[10pt]{4}{\linewidth}
\( Z= 4x^{2}+4x+1 \)
\notes[10pt]{3}{\linewidth}
\end{multicols}
\end{enumerate}
\end{exercice}

\begin{exercice}[BaremeDetaille]
\begin{enumerate}
\item\brm{2} Simplifier :

$A=\sqrt{8}-\sqrt{32}+\sqrt{50}$\notes[10pt]{3}{\linewidth}
\item\brm{2} Résoudre les équations :

$x^{2}=1$\notes[10pt]{2}{\linewidth}

\item\brm{2} Rendre le dénominateur un nombre rationnel :

$J=\dfrac{2\sqrt{3}}{\sqrt{3}}$\notes[10pt]{2}{\linewidth}
$K=\dfrac{4}{3-\sqrt{2}}$\notes[10pt]{4}{\linewidth}
\end{enumerate}
\end{exercice}

\end{Maquette}


\end{document}