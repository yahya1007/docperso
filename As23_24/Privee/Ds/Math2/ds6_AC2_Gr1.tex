\documentclass[a4paper,12pt]{article}

\usepackage{ProfModels}
\usepackage{fig3d}
\usepackage{amssymb}
\newcolumntype{C}[1]{>{\centering\arraybackslash}p{#1}}


%\reversemarginpar
\begin{document}
\begin{Maquette}[DS]{Numero=6, Niveau=2, Date=13/06/2024}
\begin{exercice}
\begin{enumerate}
\begin{minipage}{.6\linewidth}
Voici la liste des notes d'un devoir de mathématiques 
\item\brm{2} Compléter le tableau ci-dessous.
\item\brm{2} Quel est l'effectif total de cette série.\anserline[1]
\item\brm{2} Calculer la moyenne de cette série statistique.\\
\anserline[2]
\item\brm{2} Donner le pourcentage des élèves qui ont obtenue une moyenne supérieur à la moyenne de la classe.\\
\anserline[2]
\end{minipage}%
\begin{minipage}{.4\linewidth}
\begin{tabular}{Oc|Oc|Oc|Oc|Oc|Oc|Oc|Oc}
10 & 18 & 14 & 18 & 6 & 8 & 4 & 14	 \\ 
\hline 
14 & 10 & 8 & 6 & 13 & 8 & 14 & 15	 \\ 
\hline 
12 & 8 & 16 & 6 & 13 & 8 & 14 & 15	 \\ 
\hline 
14 & 15 & 12 & 8 & 18 & 3 & 13 & 15 \\ 
\hline 
20 & 20 & 6 & 16 & 3 & 13 & 8 & 9 \\ 
\hline 
8 & 6 & 4 & 4 & 8 & 18 & 19 & 0 \\ 
\hline
12 & 6 & 20 & 16 & 11 & 13 & 12 & 2
\end{tabular} 
\end{minipage}%

\begin{tabular}{|*6{Oc|}}
\hline 
Classe : note  & $0\leqslant n\textless 4$ & $4\leqslant n \textless 8$ &$8\leqslant n \textless 12$ & $12\leqslant n \textless 16$ & $16\leqslant n \leqslant 20$  \\ 
\hline 
Nombre des élèves &  &  &  & & \\ 
\hline 
Effectif cumulé &  &  &  &  &\\ 
\hline
Fréquence &  &  &  &  &\\ 
\hline 
Pourcentage &  &  &  & & \\ 
\hline 
Mesure des angles &  &  &  &  &\\ 
\hline 
\end{tabular} 
\item\brm{2} Représenter le tableau par un histogramme.\\
\anserline[8]
\end{enumerate}
\end{exercice}

\begin{exercice}
\begin{enumerate}
\item\brm{2} Compléter le tableau :
\end{enumerate}
 
\begin{tabular}{|c|C{1cm}|C{1cm}|C{1cm}|C{1cm}|C{1cm}|C{1cm}|C{1cm}|}
\hline 
Nombre & 60 &  &  &  &  &  &  \\ 
\hline 
Pourcentage & 100 & 50 & 25 & 10 & 20 & 5 & 1 \\ 
\hline 
\end{tabular} 
\begin{enumerate}
\item\brm{2} Un pantalon d'une valeur de 160 Dhs est vendu avec une réduction de 20\%. Quel est le prix de ce pantalon ?
\end{enumerate}
\anserline[4]
\end{exercice}

\begin{exercice}
\begin{enumerate}
\item\brm{2} Un cycliste parcourt 48 km en une heure et demie. Quelle est alors sa vitesse moyenne ?
\par\anserline[5]
\item\brm{2} Plus tard, il fait le même trajet à la vitesse moyenne de 38,4 km/h.\par
 Combien de temps roule-t-il ?
\par\anserline[5]
\item\brm{2} Quelle distance parcourt-il s'il roule pendant 1 h 40 min à la vitesse moyenne de 35 km/h ?
\par\anserline[5]
\end{enumerate}
\end{exercice}
\end{Maquette}
\end{document}