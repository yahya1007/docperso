\documentclass[a4paper,12pt]{article}

\usepackage{ProfModels}
\usepackage{fig3d}

 
\begin{document}

\begin{Maquette}[DS]{Niveau=2, Numero=6, Date=12/06/2025, Semestre=1, Calculatrice=false}


  \begin{exercice}[BaremeDetaille]
  Le tableau suivant représente le nombre d'enfant par famille.
\begin{enumerate}
    \begin{minipage}{.54\linewidth}
\item\brm{2} compléter le tableau
\item\brm{2} Quel est le caractère de cette série statistique ?\\
\anserline[2]
\item\brm{2} Quel est l'effectif total de cette série statistique?\\
\anserline[2]
\end{minipage}\hfill%
    \begin{minipage}{.44\linewidth}
\begin{tabular}{|Oc|Oc|Oc|Oc|Oc|Oc|}
\hline 
Nombre d'enfant & 1 & 2 & 3 & 4 & 5 \\ 
\hline 
nombre de famille & 3 & 10 & 7 & 4 & 8 \\ 
\hline 
Effectif cumulé  &  &  &  &  &  \\ 
\hline
Fréquence  &  &  &  &  &  \\ 
\hline 
Fréquence cumulé  &  &  &  &  &  \\ 
\hline
pourcentage &  &  &  &  &  \\ 
\hline 
\end{tabular} 
\end{minipage}
\end{enumerate}
  \begin{enumerate}[start=4]
      \begin{minipage}{0.54\linewidth}
\item\brm{2} Calculer la moyenne de cette série statistique.\\
\anserline[9]
    \end{minipage}\hfill%
      \begin{minipage}{0.44\linewidth}
      \item\brm[0.5]{2} Tracer le diagramme en bâtons des effs.
\anserline[9]
      \end{minipage}
\end{enumerate}
\end{exercice}

\begin{exercice}[BaremeDetaille]
  \brm{3}Un TGV roule pendant 190 min  à la vitesse de 380 km/h.

Quelle distance parcourt-il ?

\begin{minipage}{0.48\linewidth}
  \anserline[7]
\end{minipage}\hfill%
\begin{minipage}{0.48\linewidth}
\anserline[7]
\end{minipage}
\end{exercice}


\begin{exercice}[BaremeDetaille]
Un club de sports compte 260 membres dont 120 garçons.
15\% des garçons et 25\% des filles participent à des compétitions.
\begin{enumerate}
  \item\brm{1} Combien de garçons participent à des compétitions ?

  \anserline[3]
\item \brm{1} Combien de filles participent à des compétitions ?
  
    \anserline[3]
\item \brm{1} Quel pourcentage des membres de ce club participent à des compétitions ?

    \anserline[4]
\end{enumerate}
\end{exercice}


\begin{exercice}[BaremeDetaille]
On considère la fonction linéaire $f$ tel que $f(x)=-3x$.
\begin{enumerate}
\item\brm{2} Calculer $f(3)$, $f(-1)$ et $f(\dfrac{2}{3})$.\\
\anserline[6]
\item\brm{2} Quel nombre a pour image -18 par $f$.\\
\anserline[3]
\end{enumerate}
\end{exercice}

\end{Maquette}
\end{document}
