\documentclass[a4paper,12pt]{article}

\usepackage{ProfModels}

  
\begin{document}
\begin{Maquette}[DS]{Niveau=2, Numero=2, Date=19/12/2024, Semestre=1, Calculatrice=false}

\begin{exercice}[BaremeDetaille]
\begin{enumerate}
\item Calculer \brm{1}
$\dfrac{12}{-16}\times \dfrac{16}{12}=$\anserline[1]

$\dfrac{-5}{-19}\div \dfrac{25}{19}=$\anserline[1]

\item Calculer puis simplifier si possible\brm{2}

$A=\dfrac{2+\dfrac{-8}{3}}{\dfrac{5}{3}-6}=$\anserline[2]

$B=\dfrac{-4}{-(-5)}-\dfrac{9}{3}\div\dfrac{-9}{5}+1 =$\anserline[3]

\item Simplifier \brm{1}

$X=\dfrac{-4\times 81\times 125}{150\times (-27)\times 16}=$\anserline[4]

\end{enumerate}
\end{exercice}


\begin{exercice}[BaremeDetaille]
\begin{enumerate}
\item\brm{4} Calculer : 
\[
 (-1)^{-345}=\cdotsx{6} \quad; \quad (-12)^{-1}=\cdotsx{6} \quad; \quad 1^{891}=\cdotsx{6} \quad; \quad (\dfrac{3}{-4})^{-3}=\cdotsx{6}
\]
\item\brm{2} Déterminer le signe des puissances suivantes :
\[ (\dfrac{-1}{-13})^{-24}=\cdotsx{12} \quad; \quad 
	(71)^{-3}=\cdotsx{12}
\] 
\item\brm{1} Écrire sous forme $a^{n}$ les expressions suivantes :

\(	\dfrac{10^{15}\times 10^{-8}\times 10^{13}}{10^{-30}}=
\)
\anserline[4]
\item\brm{2} Donner l'écriture scientifique des nombres suivants :

$A=0.0000000056\times 0.0000000004=$\anserline[1]
$B=\dfrac{810\times 10^{-90}}{90\times 10^{-97}}=$\anserline[1]
\end{enumerate}
\end{exercice}

\begin{exercice}[BaremeDetaille]
\begin{enumerate}
\item\brm{1} Tracer un segment $[AB]$ puis sa médiatrice $(d)$.
\item\brm{1} Quel est le symétrique de $A$ par rapport à $(d)$ ?
\item\brm{1} Quel est le symétrique de $B$ par rapport à $(d)$ ?
\item\brm{1} Placer un point $K$ sur $(d)$ avec $K\notin [AB]$.
\item\brm{1} Quel est le symétrique de $K$ par rapport à $(d)$ ?
\item\brm{1} Que peut-on dire des longueurs $KA$ et $KB$ ?
\item\brm{1} Déduire la nature du triangle $BAK$.
\end{enumerate}
\anserline[14]
\end{exercice}
\end{Maquette}
\end{document}