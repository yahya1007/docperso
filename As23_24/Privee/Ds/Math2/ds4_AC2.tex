\documentclass[a4paper,12pt]{article}

\usepackage{ProfModels}

 \setlength{\columnseprule}{1pt}
\setlength{\columnsep}{2em}
\renewcommand{\columnseprulecolor}{\color{gray}}

\begin{document}
\begin{Maquette}[DS]{Niveau=2, Numero=4, Date=27/02/2025, Semestre=1, Calculatrice=true}

\begin{exercice}
\begin{enumerate}
\item\brm{6} Développer et réduire :

$
A=\left(2x-7\right)\left(2x+7\right)-x\left(2x-3\right)=$\anserline[2]

$
B=\left(x-5\right)^{2}=$
\anserline[1]

$C=\left(7x-1\right)\left(7x+1\right)=$\anserline[1]

\end{enumerate}
\end{exercice}

\begin{exercice}
\begin{enumerate}
\item\brm{7} Factoriser :

$a=4x^{2}+44x=$\anserline[1]

$b=16x^{2}-24y=$\anserline[1]

$c=5\left(x-2\right)-5x\left(-2x+7\right)=$\anserline[1]

$d=17x^{2}+17\left(2x-5\right)-51x=$\anserline[1]

$e=\left(x-1\right)\left(x-3\right)+2\left(x-1\right)-x\left(3x-3\right)=$\anserline[2]

\end{enumerate}
\end{exercice}

\begin{exercice}
\begin{enumerate}
\item\brm{5} Résoudre les équations suivantes 
\end{enumerate}
\begin{multicols}{3}
\(3x+3=3\)\newline
\anserline[8]
\columnbreak

\(\dfrac{-x}{3}+\dfrac{3}{3}=\dfrac{-3}{3}\)\newline
\anserline[8]
\columnbreak

\(5x-\dfrac{2}{5}=\dfrac{5x}{3}+1\)\newline
\anserline[8]
\end{multicols}

\begin{multicols}{2}
\((2x+1)(x+1)=3x(x+1)\)\newline
\anserline[12]
\columnbreak

\(\left(4x-1\right)^{2}\times(5x+6)^{2}=0\)\newline
\anserline[12]
\end{multicols}
\end{exercice}

\begin{exercice}
\brm{2}Karim a obtenu 11 et 16 aux deux premiers contrôles de Maths.

Quelle note doit-il avoir au troisième contrôle pour obtenir 16 de moyenne ?
\newline
\anserline[17]
\end{exercice}
\end{Maquette}
\end{document}