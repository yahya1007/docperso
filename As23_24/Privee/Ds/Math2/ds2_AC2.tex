\documentclass[a4paper,12pt]{article}

\usepackage{ProfModels}

\begin{document}
\begin{Maquette}[DS]{Niveau=2, Numero=2, Date=19/10/2024, Semestre=1, Calculatrice=false}


\begin{exercice}
\begin{enumerate}
\item Calculer 
$\dfrac{12}{-16}\times \dfrac{16}{12}=$\anserline[1]

$\dfrac{-5}{-19}\div \dfrac{25}{19}=$\anserline[1]

\item Calculer puis simplifier si possible

$A=\dfrac{2+\dfrac{-8}{3}}{\dfrac{5}{3}-6}=$\anserline[2]

$B=\dfrac{-4}{-(-5)}-\dfrac{9}{3}\div\dfrac{-9}{5}+1 =$\anserline[2]

\item Simplifier 

$X=\dfrac{-4\times 81\times 125}{150\times (-27)\times 16}=$\anserline[3]

\item Calculer

$Y=1-2+3-4+5-6+7-8+.....+99-100=$\anserline[4]
\end{enumerate}
\end{exercice}

\begin{exercice}
$ABC$ est un triangle tel que : $AB=6$ et $\widehat{BAC}=100^{\circ}$ et $\widehat{ABC}=30^{\circ}$.

Soit $M$ le milieu du segment $[BC]$ 
\begin{enumerate}
\item Faire un schéma.
\item Construire $E$ et $F$ les symétriques respectives   de $B$ et $C$ par rapport à la droite $(AM)$.
\item Montrer que $AE=6$.
\item Quel est la mesure de l'angle $\widehat{EAF}$? Justifie ta réponse.
\end{enumerate}
\end{exercice}
%\anspage{1}


\end{Maquette}

\end{document}