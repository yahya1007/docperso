\documentclass[a4paper,12pt]{article}

\usepackage{ProfModels}

  
\begin{document}
\begin{Maquette}[DS]{Niveau=1, Numero=5, Date=19/10/2024, Semestre=2, Calculatrice=false}
%\newcommand{\tf}[1][{}]{%
%\fillin[#1][1in]%
%}
\begin{exercice}
Répondre par vrai(V) ou faux(F)
\begin{enumerate}
\item\brm{2} Un losange peut avoir des côtés de longueurs différentes.
\item\brm{2} Le carré est un losange.
\item\brm{2} Le rectangle a forcément les diagonales de même longueurs.
\item\brm{2} Un rectangle est forcément un parallélogramme.
\item\brm{2} N'importe quel quadrilatère qui a un angle droit est un rectangle.
\end{enumerate}
\end{exercice}

\begin{exercice}
\begin{minipage}{.68\linewidth}
$ABC$ et $CDA$ sont deux triangles équilatéraux.
\begin{enumerate}
\item\brm{4} Quelle est la nature du quadrilatère $ABCD$?
\end{enumerate}
\anserline[5]
\end{minipage}
\begin{minipage}{.32\linewidth}
\begin{tikzpicture}
\tkzDefPoint(0,0){D}
\tkzDefPoint(3,1){A}
\tkzDefTriangle[equilateral](D,A)
\tkzGetPoint{C}
\tkzDefTriangle[equilateral](C,A)
\tkzGetPoint{B}
\tkzDrawSegments(D,A A,B B,C C,D C,A)
\tkzLabelPoint[below](D){D}
\tkzLabelPoint[below](A){A}
\tkzLabelPoint[left](C){C}
\tkzLabelPoint[right](B){B}
\end{tikzpicture}
\end{minipage}

\end{exercice}

\begin{exercice}
\begin{minipage}{.5\linewidth}
On considère la figure ci-contre, tel que $(AB)//(CD)$ et $\widehat{CDA}=38^{\circ}$ et $\widehat{CBA}=49^{\circ}$
\begin{enumerate}
\item\brm{3} Calculer la mesure des angles $\widehat{IAB}$ et $\widehat{ICD}$.\newline
\anserline[2]
\item\brm{3} Déduire la mesure de $x$.\newline
\anserline[2]
\end{enumerate}
\end{minipage}%
\begin{minipage}{.5\linewidth}
\begin{tikzpicture}
\tkzDefPoints{0/0/A,4/0/B,1/3/C,6/3/D}
\tkzLabelPoints[below](A,B)
\tkzLabelPoints[above](C,D)
\tkzDrawPoints(A,B,C,D)
\tkzDrawLines(A,B C,D D,A C,B)
\tkzInterLL(A,D)(B,C)
\tkzGetPoint{I}
\tkzLabelPoint[above](I){I}
\tkzMarkAngle[arc=l , size=0.5cm , mark=|](C,D,A)
\tkzMarkAngle[arc=l , size=0.5cm , mark= ||](C,B,A)
\tkzMarkAngle[arc=l , size=0.5cm , mark= |||](B,I,D)

\tkzLabelAngle(C,D,A){$38^{\circ}$}
\tkzLabelAngle(C,B,A){$49^{\circ}$}
\tkzLabelAngle(B,I,D){$x$}

\end{tikzpicture}
\end{minipage}

\end{exercice}
\end{Maquette}
\end{document}