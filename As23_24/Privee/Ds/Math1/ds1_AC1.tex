\documentclass[a4paper,12pt]{article}



\usepackage{ProfModels}

  
\begin{document}
\begin{Maquette}[DS]{Niveau=1, Numero=1, Date=19/10/2024, Semestre=1, Calculatrice=false}

\begin{exercice}
\begin{enumerate}
\item\brm{10} Calculer ce qui suit :
\begin{multicols}{2}		
$ A=14+5\times 3-12=$\anserline[3]

$ B=124+2\times 3+155\div 5=$\anserline[3]

$ C=270-144\div 12-100=$\anserline[3]
\columnbreak

$ D=155+(17-34\div 2)\times 85=$\anserline[4]

$ E=(36\div 3)+(3\times(12\div 4+8))=$\anserline[5]
\end{multicols}
\end{enumerate}	
\end{exercice}

\begin{exercice}
	\begin{enumerate}
\item\brm{5} Calculer et simplifier le plus possible :
		\end{enumerate}
\begin{multicols}{3}		

\[\dfrac{185}{5}+\dfrac{65}{5}=\dfrac{\cdotsx{6}}{\cdotsx{6}}\]

\[\dfrac{20}{81}+\dfrac{7}{81}=\dfrac{\cdotsx{6}}{\cdotsx{6}}\]

\[\dfrac{6}{5}+\dfrac{12}{15}=\dfrac{\cdotsx{6}}{\cdotsx{6}}\]

\[\dfrac{15}{7}\times \dfrac{7}{15}=\dfrac{\cdotsx{6}}{\cdotsx{6}}\]

\[\dfrac{44}{51}\times\dfrac{7}{22}=\dfrac{\cdotsx{6}}{\cdotsx{6}}\]

\[\dfrac{21}{71}-\dfrac{16}{71}=\dfrac{\cdotsx{6}}{\cdotsx{6}}\]

\[\dfrac{26}{5}-\dfrac{6}{10}=\dfrac{\cdotsx{6}}{\cdotsx{6}}\]

\[\dfrac{1}{5}\div\dfrac{8}{5}=\dfrac{\cdotsx{6}}{\cdotsx{6}}\]

\[\dfrac{14}{13}\div\dfrac{42}{13}=\dfrac{\cdotsx{6}}{\cdotsx{6}}\]
\end{multicols}	
\vspace{1cm}	
\end{exercice}
\newpage
\begin{exercice}
 \brm{3} Réduire le plus possible :
		
$$\dfrac{250}{2050}=$$
$$\dfrac{17\times 8\times 761}{34\times 48 \times 19}=$$
$$\dfrac{101+19}{101-1}=$$

\end{exercice}

\begin{exercice}
\begin{enumerate}
\item\brm{1} Comparer ce qui suit  :	
\[\dfrac{938}{98}\cdotsx{6} 1 \hspace{2cm}
\dfrac{36}{98}\cdotsx{6}\dfrac{98}{36}\]

\item\brm{1} Ranger dans l'ordre croissant les nombres suivants après les rendre aux même dénominateurs :

\[\dfrac{8}{9} ;; 1 ;; \dfrac{12}{18} ;; \dfrac{30}{36}  ;; \dfrac{24}{18}\]	

\end{enumerate}
\anserline[6]
\end{exercice}

\end{Maquette}
\end{document}
