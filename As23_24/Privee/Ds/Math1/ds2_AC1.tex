\documentclass[a4paper,12pt]{article}

\usepackage{ProfModels}

  
\begin{document}
\begin{Maquette}[DS]{Niveau=1, Numero=2, Date=19/10/2024, Semestre=1, Calculatrice=false}

\begin{exercice}
\begin{minipage}{.5\linewidth}
On considère la figure ci-contre :
\begin{enumerate}
\item\brm{5} Compléter par $\in$ ou $\notin$

$A\cdotsx{3} (BC)$ ; $B\cdotsx{3} [AC]$ ; $C\cdotsx{3} [AB)$

$A\cdotsx{3} (CB]$ ; $D\cdotsx{3} [EB)$ ; $E\cdotsx{3} [DB)$

$E\cdotsx{3} [BD)$ ; $C\cdotsx{3} [AB]$ ; $D\cdotsx{3} [BE)$
\end{enumerate}
\end{minipage}%
\begin{minipage}{.5\linewidth}
\begin{tikzpicture}[scale=.7]
\tkzDefPoints{-3/0/A,2/0/C,3/1/D,-4/-1/E}
\tkzInterLL(A,C)(E,D)
\tkzGetPoint{B}
\tkzDrawLines(A,C D,E)
\tkzDrawPoints(A,B,C,D,E)
\tkzLabelPoints(B,C)
\tkzLabelPoints[above](E,A,D)

\end{tikzpicture}
\end{minipage}
\end{exercice}

\begin{exercice}
\begin{minipage}{.66\linewidth}
A, B ,et C sont trois points non alignés. 
\begin{enumerate}
\item\brm{1} Trace la droite $(D_{1})$ perpendiculaire à $(AB)$ passant par C.
\item\brm{1} Trace la droite $(D_{2})$ perpendiculaire à $(BA)$ passant par C.
\item\brm{1} Trace la droite $(D_{3})$ parallèle à $(BC)$ passant par A.
\item\brm{1} Que peut-on dire des droites $(D_{1})$ et $(D_{2})$ et $(D_{3})$ ? 
\end{enumerate}
\end{minipage}
\begin{minipage}{.33\linewidth}
\begin{tikzpicture}
\tkzDefPoints{0/0/B,2/3/A,4/1/C}
\tkzDrawLines(A,B B,C C,A)
\tkzLabelPoints[above=2pt](A,B,C)
\end{tikzpicture}
\end{minipage}%

\notes[10pt]{2}{\linewidth}
\end{exercice}

\begin{exercice}
\begin{enumerate}
\item\brm{7} En utilisant les règle de calculer, effectuer ce qui suit : 
\end{enumerate}
\begin{multicols}{2}
$(-14)+13=$\anserline[2] 
$(23)+(-19)$=\anserline[2]
$-14.26+(-13.5)=$\anserline[2]
\columnbreak

$-143-134=$\anserline[2]
$213-(-139)=$\anserline[2]  
$-104.26-(-130.5)=$\anserline[2] 
\end{multicols}
$-14-(-9)+(+48)-(68)-(-51)+(-47)-(+23)-(-13)+(+68)-48=$\anserline[1]
\anserline[1]

\end{exercice}

\begin{exercice}
\begin{enumerate}
\item\brm{4} Donner les abscisses des points $A$ , $B$ , $O$ et $I$ 
\end{enumerate}
\begin{tikzpicture}
\tkzInit[xmin=-7.5, xmax=6]
\tkzAxeX
\tkzDefPoints{0/0/O,1/0/I,3/0/A,-4/0/B}
\tkzLabelPoints[above](O,I,A,B)
\end{tikzpicture}
\notes[10pt]{5}{\linewidth}
\end{exercice}


\end{Maquette}
\end{document}