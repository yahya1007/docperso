\documentclass[12pt,a4paper,x11names,usenames,dvipsnames,svgnames,oneside]{book}
\usepackage[table]{xcolor}
\usepackage[utf8]{inputenc}
\usepackage[T1]{fontenc}
\usepackage[left=0.7 cm,right=0.5 cm,top=1.2cm,bottom=1.5cm]{geometry}
\usepackage{amsmath,amssymb,amsfonts,mathrsfs}
\usepackage{mathpazo,tikz,mathtools}
\usepackage{,fontawesome,float,lmodern,multirow,bidicontour,contour}
\usepackage{fancybox,lipsum,graphics,graphicx,multicol,fancyhdr,ulem,url}
\usepackage{pgf,eso-pic,lastpage,diagbox,fourier,frcursive,pgfplots,tkz-tab}
\usepackage[most]{tcolorbox}
\usepackage[]{bclogo}
\usetikzlibrary{patterns,shapes,shapes.geometric,arrows,arrows.meta,shadings}
\usetikzlibrary{calc,scopes,backgrounds,fadings,shadows}
\usetikzlibrary{shapes.arrows,decorations, decorations.text}
\usetikzlibrary{decorations.pathmorphing,shadows.blur}
\tcbuselibrary{skins,breakable}
\tcbuselibrary{skins,theorems,breakable}
\usepackage{fontspec,color,pifont,bbding,tablists,pstricks-add,esvect,sectsty}
\usepackage{pdfpages,array,colortbl,varwidth} 
\usepackage[explicit]{ titlesec} 
\usepackage{titletoc}
\usepackage{hyperref}
\usepackage{mdframed}
\usepackage{polyglossia}

%------------------------------------------------enumerate-------------------------------------------------------------------------------------------------------------------------
\renewcommand{\labelenumi}{%
	\begin{tikzpicture}[baseline=(1.base)]
		\node[rectangle,fill=cyan,text=white,font=\bfseries,minimum size=5mm,inner sep=0mm](1){\arabic{enumi}};
	\end{tikzpicture}
}
\renewcommand{\labelenumii}{%
	\begin{tikzpicture}[baseline=(1.base)]
		\node[circle,fill=cyan,text=white,font=\bfseries,minimum size=5mm,inner sep=0mm](1){\alph{enumii}};
	\end{tikzpicture}
}
%----------------------------------------------------------------------------------------------------------------------------------------------------------------------------------
\hypersetup{                    % parametrage des hyperliens
	colorlinks=true,                % colorise les liens
	breaklinks=true,                % permet les retours à la ligne pour les liens trop longs
	urlcolor= blue,                 % couleur des hyperliens
	linkcolor= blue,                % couleur des liens internes aux documents (index, figures, tableaux, equations,...)
	citecolor= blue                % couleur des liens vers les references bibliographiques
}
%---------------------------------langue-------------------------------------------------------------------------------------------------------------------------------------------
\setdefaultlanguage{french}
\newfontfamily\frenchfont[Scale=1.2]{Amiri}%
\newfontfamily{\Ar}[Scale=2]{Arial}
\newfontfamily\frenchfontm[Scale=1.5]{Mistral}
\newfontfamily\frenchfonts[Scale=1.7]{Mistral}
\setotherlanguage{arabic}
%---------------------------------stylemaths---------------------------------------------------------------------------------------------------------------------------------------
\everymath{\displaystyle \color{violet}   \boldmath}
\mathversion{bold}
\renewcommand{\headrulewidth}{0pt}
%--------------------------colors--------------------------------------------------------------------------------------------------------------------------------------------------
\definecolor{olive}{rgb}{1,.972,.864}
\definecolor{col}{RGB}{0,100,0}  
\definecolor{col1}{RGB}{255, 102, 0}
\definecolor{col2}{RGB}{125, 125, 10}
%--------------------------stylepages----------------------------------------------------------------------------------------------------------------------------------------------
\AddToShipoutPictureBG{\ifnum\value{page}=1
	\begin{tikzpicture}[remember picture,overlay]
		\fill[col1]([yshift=1cm]current page.south west)rectangle(current page.south east) ;
		\node[cloud,draw=white,line width=2pt,minimum size=0.3cm,fill=col1]at([yshift=0.9cm]current page.south){$ {\color{white}\thepage} $};
		\node[anchor=west]at([yshift=.5cm]current page.south west){{ \textbf{\color{white} Continuité et limites}}};
		\node[anchor=east]at([yshift=.5cm]current page.south east){{ \textbf{\color{white}Prof: Mersani Imed}}};
	\end{tikzpicture}
	\else
	\begin{tikzpicture}[remember picture,overlay]
		\fill[col1]([yshift=-0.5cm]current page.north west)rectangle(current page.north east) ;
		\fill[col1]([yshift=1cm]current page.south west)rectangle(current page.south east) ;
		\node[cloud,draw=white,line width=2pt,minimum size=0.3cm,fill=col1]at([yshift=0.9cm]current page.south){${\color{white} \thepage} $};
		\node[anchor=west]at([yshift=.5cm]current page.south west){{ \textbf{\color{white} Continuité et limites}}};
		\node[anchor=east]at([yshift=.5cm]current page.south east){{ \textbf{\color{white}Prof: Mersani Imed}}};
	\end{tikzpicture}
	\fi}
\parindent=0mm
%------------------------table-des-matiéres----------------------------------------------------------------------------------------------------------------------------------------
\contentsmargin{1cm}
\definecolor{doc}{RGB}{0,60,110}
\titlecontents{chapter}[0pc]
{\addvspace{30pt}%
	\begin{tikzpicture}[remember picture, overlay]%
		\draw[fill=doc!30,draw=doc!30] (-0.5,-.2) rectangle (3,.5);%
		\pgftext[right,x=2.5cm,y=0.2cm]{\color{white}\Large\sc\bfseries Cours\
			\thecontentslabel};%
	\end{tikzpicture}\color{doc!40}\large\sc\bfseries\hspace*{3.5cm}}%
{}
{}
{\;\titlerule\;\large\sc\bfseries Page \thecontentspage
	\begin{tikzpicture}[remember picture, overlay]
		\draw[fill=doc!25,draw=doc!20] (2pt,0) rectangle (1cm,0.1pt);
	\end{tikzpicture}	\vskip4pt}%
\titlecontents{section}[2.4pc]
{\addvspace{1pt}}
{\contentslabel[\thecontentslabel]{2.4pc}}
{}
{\;\;\color{doc!40}\dotfill\;\;\small \thecontentspage}
[]
\titlecontents{subsection}[5pc]
{\addvspace{1pt}}
{\contentslabel[\thecontentslabel]{2.4pc}}
{}
{\;\;\color{doc!40}\dotfill\;\;\small \thecontentspage}
[]
\makeatletter
%-----------------------------boxes------------------------------------------------------------------------------------------------------------------------------------------------
%--------------------remarque--------------------------------------------------------------------------------------------------------------------------------------------------
\def\skn{\tcbsubskin{skn}{empty}{frame code={%\draw [black,transform canvas={xscale=1.5},decorate,decoration={zigzag,segment length=3mm,amplitude=.5mm},line width=1.3pt]([yshift=-5pt]frame.north west)--(frame.south west);
}}}
\usetikzlibrary{decorations.pathmorphing}
\newtcolorbox{boxxr}{enhanced,breakable,detach title,coltitle=black,fonttitle=\bfseries,leftrule=2mm,title={\bclampe ~~Remarque\par},skin=skn,left=0mm,top=0mm,bottom=0mm,right=0mm,
	skin first=skn,
	skin middle=skn,
	skin last=skn
}
\newenvironment{rem}{\skn\boxxr\tcbtitle}{\endboxxr}
%--------------------notation--------------------------------------------------------------------------------------------------------------------------------------------------
\def\skn{\tcbsubskin{skn}{empty}{frame code={%\draw [black,transform canvas={xscale=1.5},decorate,decoration={zigzag,segment length=3mm,amplitude=.5mm},line width=1.3pt]([yshift=-5pt]frame.north west)--(frame.south west);
}}}
\usetikzlibrary{decorations.pathmorphing}
\newtcolorbox{boxxn}{enhanced,breakable,detach title,coltitle=black,fonttitle=\bfseries,leftrule=2mm,title={\bcinfo ~~Notations\par},skin=skn,left=0mm,top=0mm,bottom=0mm,right=0mm,
	skin first=skn,
	skin middle=skn,
	skin last=skn
}
\newenvironment{nt}{\skn\boxxn\tcbtitle}{\endboxxn}
%----------------------------proposition-------------------------------------------------------------------------------------------------------------------------------------------
\newtcolorbox{pr}{breakable,enhanced,colback=MediumSpringGreen!5,colframe=MediumSpringGreen,title=Proprietés,top=2mm,boxsep=3mm,
	attach boxed title to top left=
	{yshift=-\tcboxedtitleheight/2 ,xshift=0.3cm},
	boxed title style={size=small,colback=MediumSpringGreen}}

%------------------------------------exemple---------------------------------------------------------------------------------------------------------------------------------------
\newtcolorbox{expl}{breakable,enhanced,detach title,blanker,title={Exemple~:},coltitle=black,
	boxsep=3mm,top=2.7mm,left=2mm,right=2mm
	,before skip=0.5cm,
	after skip=0.5cm,
	overlay={
		\node[anchor=west,font=\large\bfseries, scale=1.2] (1) at ([xshift=0.7cm]interior.north west){\tcbtitle};
%		\fill(1.west)circle(2pt);
%		\scoped[on background layer]{
%			\draw(1.west)--(interior.north west)--
%			(interior.south west);
%	}
},overlay first={
		\node[anchor=west,font=\large\bfseries, scale=1.2] (1) at ([xshift=0.7cm]interior.north west){\tcbtitle};
%		\fill(1.west)circle(2pt);
%		\scoped[on background layer]{
%			\draw(1.west)--(interior.north west)--(interior.south west);
	%	}
	},overlay middle={
%		\draw(interior.north west)--
%		(interior.south west);
	},overlay last={
%		\draw(interior.north west)--
%		(interior.south west);
}
}
%----------------------------------------------theoreme----------------------------------------------------------------------------------------------------------------------------
\newtcolorbox{thr}{
	enhanced,breakable,
	before skip=2mm,after skip=0.3cm,
	bottom=0.5cm,colframe=magenta,coltitle=white,
	colback=magenta!4,boxrule=0.2mm,arc=0mm,leftrule=2mm,
	attach boxed title to top left={xshift=0cm,yshift*=0mm-\tcboxedtitleheight},
	varwidth boxed title*=-3cm,
	boxed title style={frame code={
			\coordinate (A) at (frame.north east);
			\coordinate (B) at (frame.south east);
			\path[fill=magenta](frame.north east)--([xshift=-2mm]frame.north west)--([xshift=-2mm]frame.south west)--(frame.south east) --([xshift=3mm]$(A)!0.5!(B)$)--cycle;
			\draw[white,line width=2pt]([xshift=-2mm]frame.north east)--([xshift=1mm]$(A)!0.5!(B)$)--([xshift=-2mm]frame.south east);
			\path[fill=magenta!50]([xshift=-2mm]frame.south west)--+(2mm,-2mm)|-cycle;
		},interior engine=empty,
	},
	title={\large\bf\color{white} Théorème}}
%-----------------------------------definition-------------------------------------------------------------------------------------------------------------------------------------
\newtcolorbox{df}{breakable,enhanced,detach title,blanker,title={Définition},coltitle=white,
	boxsep=3mm,top=2mm,left=2mm,right=2mm
	,before skip=1cm,after skip=0.1cm,
	overlay={\scoped[on background layer]{
			\draw[fill=red!60!blue!3,draw=none](interior.north west)rectangle(interior.south east);
			\draw[red!60!blue](interior.north west)|-(interior.south east);},\node[anchor=west,font=\bfseries,
		scale=1.2,above right,fill=red!60!blue,xshift=-2mm,yshift=-2mm,drop shadow] (1) at (interior.north west){\tcbtitle};
	},overlay first={\scoped[on background layer]{
			\draw[fill=red!60!blue!3,draw=none](interior.north west)rectangle(interior.south east);
			\draw[red!60!blue](interior.north west)--(interior.south west);},\node[anchor=west,font=\bfseries,
		scale=1.2,above right,fill=red!60!blue,xshift=-2mm,yshift=-2mm,drop shadow] (1) at (interior.north west){\tcbtitle};
	},overlay middle={\draw[fill=red!60!blue!3,draw=none](interior.north west)rectangle(interior.south east);
		\draw[red!60!blue](interior.north west)--(interior.south west);},overlay last={\draw[fill=red!60!blue!3,draw=none](interior.north west)rectangle(interior.south east);
		\draw[red!60!blue](interior.north west)|-(interior.south east);}}
%-----------------------------activité---------------------------------------------------------------------------------------------------------------------------------------------
\newcounter{activ}
\newcommand{\activite}{\addtocounter{activ}{1}\textbf{Activité}~\theactiv}%
\newtcolorbox{act}{reset,breakable,blanker,
	before skip=6pt,after skip=6pt,
	borderline west={1mm}{0pt}{orange!60!blue},
	title={\activite},
	coltitle=purple,left=4mm}
%------------------Démonstration---------------------------------------------------------------------------------------------------------------------------------------------------
\newtcolorbox{dem}{reset,breakable,blanker,
	before skip=6pt,after skip=6pt,
	title={\underline{\sc{Démonstration}}},
	coltitle=gray,left=4mm}
%-------------------------solution-------------------------------------------------------------------------------------------------------------------------------------------------
\newtcolorbox{sol}{reset,breakable,blanker,
	before skip=6pt,after skip=6pt,
	title={\underline{\textbf{Solution}}},
	coltitle=purple,left=4mm}
%|-------------------Exercices-----------------------------------------------------------------------------------------------------------------------------------------------------
\newcounter{exerc}
\newcommand{\exercicenum}{\addtocounter{exerc}{1}\Large \theexerc}%
\newtcolorbox{exr}{breakable,enhanced,detach title,blanker,title={{ Exercice}},coltitle= blue,
	boxsep=2mm,top=2mm,left=2mm,right=2mm
	,before skip=0.7cm,
	after skip=0.5cm,
	overlay={
		\scoped[on background layer]{
			\draw[fill=white,draw=none](interior.north west)rectangle
			(interior.south east);}
		\node[anchor=north west,above right,font=\large\bfseries] (1) at (interior.north west){\tcbtitle};
		\draw[very thick, blue](1.south west)--([xshift=4mm]1.south east)node[above=-3mm ,circle,draw,fill=white,font=\large\bfseries , anchor=south, minimum size=0.9cm](2){\ };
		\node[font=\bf,  blue] at (2.center){\exercicenum };
		\scoped[on background layer]{
			\draw[very thick, blue](interior.south west)|-(1.south west)--([xshift=4mm]1.south east);}
	},overlay first={
		\scoped[on background layer]{
			\draw[fill=white,draw=none](interior.north west)rectangle
			(interior.south east);}
		\node[anchor=north west,above right,font=\large\bfseries] (1) at (interior.north west){\tcbtitle};
		\draw[very thick, blue](1.south west)--([xshift=4mm]1.south east)node[above=-3mm ,circle,draw,fill=white,text=black,font=\large\bfseries , anchor=south, minimum size=0.9cm](2){\ };
		\node[font=\bf , blue] at (2.center){\exercicenum };
		\scoped[on background layer]{
			\draw[very thick, blue](interior.south west)|-(1.south west)--([xshift=4mm]1.south east);}
	},
	overlay middle={
		\draw[fill=white,draw=none](interior.north west)rectangle
		(interior.south east);
		\draw[very thick, blue](interior.north west)--
		(interior.south west);
	},overlay last={
		\draw[fill=white,draw=none](interior.north west)rectangle
		(interior.south east);
		\draw[very thick, blue](interior.north west)--
		(interior.south west);
}}
%-------------------------------Applications---------------------------------------------------------------------------------------------------------------------------------------
\newtcolorbox{apll}{breakable,enhanced,detach title,blanker,title={Application},coltitle=black,
	boxsep=3mm,top=2mm,left=2mm,right=2mm
	,before skip=1cm,
	after skip=1cm,
	overlay={
		\scoped[on background layer]{
			\draw([xshift=0.3cm]interior.north west)|-
			(interior.south east)--(interior.north east);}
		\node[anchor=west,font=\large\bfseries, scale=1.2,above right] (1) at (interior.north west){\tcbtitle};
		\scoped[on background layer]{
			\draw([yshift=-0.5mm]1.north west)|-(1.south east);
			\fill(1.south east) circle(2pt);
			\fill[black](interior.north west)--+(0.3cm,-0.3cm)|-cycle;
	}},overlay first={
		\node[anchor=west,font=\large\bfseries, scale=1.2,above right] (1) at (interior.north west){\tcbtitle};
		\scoped[on background layer]{
			\draw([xshift=0.3cm]interior.north west)--
			([xshift=0.3cm]interior.south west);
			\draw(interior.north east)--
			(interior.south east);
			\draw([yshift=-0.5mm]1.north west)|-(1.south east);
			\fill[black](interior.north west)--+(0.3cm,-0.3cm)|-cycle;}
	},overlay middle={
		\draw([xshift=0.3cm]interior.north west)--
		([xshift=0.3cm]interior.south west);
		\draw(interior.north east)--
		(interior.south east);
	},overlay last={
		\draw([xshift=0.3cm]interior.north west)|-
		(interior.south east)--(interior.north east);}
}
%---------------------stylepourenumerate-------------------------------------------------------------------------------------------------------------------------------------------
\newcommand{\pc}[1]{\begin{tikzpicture}[baseline=(1.base)]
		\node[draw,fill=green!20!white,circle,inner sep=.5mm,font=\bfseries\footnotesize,minimum size=5mm](1){#1};
\end{tikzpicture}}
\newcommand{\pg}[1]{	\begin{tikzpicture}[baseline=(1.base)]
		\node[draw,,fill=orange!20!white,regular polygon,regular polygon sides=8,inner sep=.5mm,font=\bfseries](1){#1};
\end{tikzpicture}}
\newcommand{\ps}[1]{	\begin{tikzpicture}[baseline=(1.base)]
		\node[draw,fill=violet!20!white,star,star points=7,star point ratio=.2mm,font=\bfseries](1){#1};
\end{tikzpicture}}
%---------------------------------newcommand---------------------------------------------------------------------------------------------------------------------------------------
\newcommand{\Oij}{$( O, \overrightarrow{i},\overrightarrow{j})$}
\def\Oj{$\left(\text{O},~\overrightarrow{j}\right)$}
\def\Oi{$\left(\text{O},~\overrightarrow{i}\right)$}
\newcommand{\Oijk}{$\left( O, \overrightarrow{i},\overrightarrow{j},\overrightarrow{k} \right)$}
\newcommand{\Ouv}{$\left( O, \overrightarrow{u},\overrightarrow{v} \right)$}
\newcommand{\Oee}{$\left( O, \overrightarrow{e}_1,\overrightarrow{e}_2 \right)$}
\def\vv{\overrightarrow}
\def\N{\mathbb{N}}
\def\NN{\mathbb{N}}
\def\Z{\mathbb{Z}}
\def\D{\mathbb{D}}
\def\Q{\mathbb{Q}}
\def\R{\mathbb{R}}
\def\RR{\mathbb{R}}
\def\C{\mathbb{C}}
\def\K{\mathbb{K}}
\def\ds{\displaystyle}
\def\calc{$\mathscr{C}~$}
\def\calp{$\mathscr{P}~$}
\def\cald{$\mathscr{D}~$}
\def\cala{$\mathscr{A}~$}
\def\cale{$\mathscr{E}~$}
\def\calv{$\mathscr{V}~$}
\def\calq{$\mathscr{Q}~$}
\def\calr{$\mathscr{R}~$}
\newcommand{\nex}{01}
\newcommand{\CC}{\mathbb{C}}
\newcommand{\cf}{$ \mathscr{C}_{f} $}
\newcommand{\cg}{$ \mathscr{C}_{g} $}
\newcommand{\oij}{$ (O,\overrightarrow{i},\overrightarrow{j}) $}
\newcommand{\ouv}{$ (O,\overrightarrow{u},\overrightarrow{v}) $}
\newcommand{\Pf}{ +\infty }
\newcommand{\Nf}{ -\infty }
\newcommand{\qst}[2]{$ #1 $&#2\\}
\newcommand{\chaptitle}[1]{%
	\thispagestyle{empty}
	\begin{tikzpicture}[remember picture,overlay,decoration=saw]
		\fill[fill=col1,decorate,decoration={snake,amplitude=1mm,segment length=2mm,post length=1mm}] ([xshift=-10pt,yshift=.5cm]current page.north west)rectangle([xshift=10pt,yshift=-2.2cm]current page.north east);
		\node[text width=16.5cm,align=center ]at ([yshift=-1.1cm]current page.north) {
			\LARGE\color{white}\bfseries #1}; 
		\fill[violet]([xshift=2cm,yshift=-1.1cm]current page.north west)circle(1.1cm);
		\fill[white]([xshift=2cm,yshift=-1.1cm]current page.north west)circle(1cm);
		\fill[col1]([xshift=2cm,yshift=-1.1cm]current page.north west)circle(0.9cm);
		\node [scale=1.2,anchor=east](0)at ([xshift=3cm,yshift=-1.1cm]current page.north west){{\ifnum\thechapter<10 \LARGE\color{white}\textbf{{0\thechapter}}\else \LARGE\color{white}\textbf{{\thechapter}}\fi}};
	\end{tikzpicture}
}
\newcommand{\chapTitle}[1]{%
	\thispagestyle{empty}
	\begin{tikzpicture}[remember picture,overlay]
		\fill[col1] (current page.north west)rectangle([yshift=-2.2cm]current page.north east);
		\node[text width=16.5cm,align=center,scale=2]at ([xshift=-1.5cm,yshift=-1.1cm]current page.north) {
			\LARGE\color{white}\bfseries #1 }; 
	\end{tikzpicture}
}
%--------------------stylechapter--------------------------------------------------------------------------------------------------------------------------------------------------
\titleformat{\chapter}[block]
{ }{ }{10pt}{\Huge
	\centering \chaptitle{#1}\vskip-4cm}
\titleformat{name=\chapter,numberless}[block]
{ }{ }{10pt}{\Huge
	\centering \chapTitle{#1}\vskip-5cm}
\renewcommand{\thesection}{\Roman{section}}
\renewcommand{\thesubsection}{\arabic{subsection}}
\renewcommand{\thesubsubsection}{\alph{subsubsection}}
%--------------------stylesection--------------------------------------------------------------------------------------------------------------------------------------------------
\titleformat{\section}
{\bf\Large}
{
	\tikzpicture[baseline=(2.base)]
	\node [text=blue!70!red,anchor=west] (2){#1};
	\draw[blue!70!red]([xshift=-2mm]2.north east)--([xshift=-1.1cm]2.north west)node[fill=blue!70!red,anchor=north west,text=white,scale=1.1]{\thesection};
	\endtikzpicture}
{0pt}{}
\parindent=0pt
%--------------------stylesaubsection----------------------------------------------------------------------------------------------------------------------------------------------
\titleformat{\subsection}[block]{\bfseries\Large\color{blue}}{\quad\thesubsection. #1}{2mm}{}[]
%-----stylesubsubsection------------------------------------------
\titleformat{\subsubsection}[block]{\bfseries\Large\color{yellow!80!red}}{\quad\quad\thesubsubsection~-~#1}{2mm}{}[\vskip5pt]
\setcounter{secnumdepth}{3}
%--------------------------------------------------------------------------------------------------------------------------------------------------------------------------------------------------------------------------------------------------------------------------------------------------------------------
\begin{document}\pagestyle{empty}
	\tableofcontents
	\chapter{\sc{Continuité et limites}}
	\section{Rappels:}
	\subsection{Continuité:}
	\begin{act}
	\begin{enumerate}
	\item \begin{enumerate}
	\item Vérifier que pour tout $x\in\left]-\dfrac{\pi}{2},\dfrac{\pi}{2}\right[\setminus\{0\}$, $\dfrac{1-\cos x}{x^2}=\dfrac{\sin^2x}{x^2}.\dfrac{1}{1+\cos x}$.
	\item On admet que: $\ds \lim_{x\to 0} \dfrac{\sin x}{x}=1$.\\
	Déduire que: $\ds \lim_{x\to 0}\dfrac{1-\cos x}{x^2}=\dfrac{1}{2}$ et $\ds \lim_{x\to 0} \dfrac{1-\cos x}{x}=0$.
	\end{enumerate}
	\item Déterminer les limites suivantes: $\ds \lim_{x\to 0}\dfrac{\sin x}{1-\cos x}$ et $\ds \lim_{x\to 0}\dfrac{\sqrt{1+\cos x}-\sqrt{2}}{\sin^2x}$.
	\end{enumerate}
	\end{act}
	\begin{thr}
	Soit $a$ un réel non nul.
	\begin{multicols}{3}
	\begin{itemize}
	\item $\ds \lim_{x\to 0}\dfrac{\sin x}{x}=1$
	\item $\ds \lim_{x\to 0}\dfrac{\sin ax}{x}=a$
	\item $\ds \lim_{x\to 0}\dfrac{\tan x}{x}=1$
	\item $\ds \lim_{x\to 0}\dfrac{\tan ax}{x}=a$
	\item $\ds \lim_{x\to 0}\dfrac{1-\cos x}{x}=0$
	\item $\ds \lim_{x\to 0}\dfrac{1-\cos x}{x^2}=\dfrac{1}{2}$
	\end{itemize}
	\end{multicols}
	\end{thr}
\begin{thr}
Soit $f$ une fonction définie sur un intervalle ouvert $I$ et $a$ un réel de $I$.\\
La fonction $f$ est continue en $a$, si et seulement si, $\lim_{x\to a^-}f(x)=\lim_{x\to a^+}f(x)=f(a)$.\\
Autrement dit: La fonction $f$ est continue en $a$, si et seulement si, $f$ est continue à gauche et à droite en $a$.
\end{thr}
\begin{act}
Soit $f$ la fonction définie sur $\R$ par $g(x)=\left\lbrace \begin{array}{ll}
\dfrac{x^2+5x-6}{\left|x-1\right|}&\text{si}~x\neq 1\\
7&\text{si}~x=1

\end{array} \right. $.
\'{E}tudier la continuité de $f$ en $1$.
\end{act}

\begin{act}
Soit $g$ la fonction définie sur $\R$ par $g(x)=\left\lbrace \begin{array}{ll}
\dfrac{\sin 2x}{x}&\text{si}~x<0\\
x^2-1&\text{si}~ 0\leqslant x <2\\
\sqrt{x+7}&\text{si}~x \geqslant 2

\end{array} \right. $. 
\'{E}tudier la continuité de $g$ en $0$ et $2$.
\end{act}
\begin{thr}
Soient $f$ et $g$ deux fonctions définies sur un intervalle ouvert $I$ et $a$ un réel de $I$.
\begin{itemize}
\item Soit $f$ est continue en $a$ alors les fonctions $\alpha f,(\alpha\in\R)$, $\left|f\right|$ et $f^n, (n\in\N^*)$ sont continues en $a$.
\item Si $f$ est continue en $a$ et $f(a)\neq 0$ alors les fonctions $\dfrac{1}{f}$ et $\dfrac{1}{f^n}, (n\in\N^*)$ sont continues en $a$.
\item Si $f$ et $g$ sont continues en $a$ et $g(a)\neq 0$ alors la fonction $\dfrac{f}{g}$ est continue en $a$.
\item Si $f$ est continue en $a$ et $f$ est positive sur $I$ alors $\sqrt{f}$ est continue en $a$.
\end{itemize}
\end{thr}
\begin{thr}
\begin{itemize}
\item Toute fonction polynôme est continue en tout réel.
\item Toute fonction rationnelle est continue en tout réel de son ensemble de définition.
\item Les fonctions $x\longmapsto \sin x$ et $x\longmapsto \cos x$ sont continues en tout réel.
\end{itemize}
\end{thr}
\subsection{Prolongement par continuité:}
\begin{thr}
Sot $f$ une fonction définie sur un intervalle ouvert $I$, sauf en un réel $a$ de $I$.\\
Si la fonction $f$  admet une limite finie $\ell$ lorsque $x$ tend vers $a$, alors la fonction $g$ définie sur $I$ par $g(x)=\left\lbrace \begin{array}{ll}
f(x)&\text{si}~x\neq a\\
\ell &\text{si}~x=a

\end{array} \right. $ est continue en $a$.\\
La fonction $g$ est appelée le prolongement par continuité de $f$ en $a$.
\end{thr}
\begin{act}
Soit $f$ la fonction définie sur $\R^*$ par $f(x)=\dfrac{x^2+\sin x}{x}$.\\
Montrer que $f$ est prolongeable par continuité en $0$ et déterminer son prolongement.
\end{act}
\subsection{Continuité sur un intervalle:}
\begin{df}
\begin{itemize}
\item Une fonction continue sur un intervalle ouvert $I$ si elle est continue en tout réel de $I$.
\item Une fonction est continue sur un intervalle $\big[a, b\big]$ si elle est continue sur $\big]a, b\big[$, à droite en $a$ et à gauche en $b$.
\item De façon analogue, on définit la continuité d'une fonction $f$ sur les intervalles $\big[a, b\big[$, $\big]a,b\big]$, $\big[a, +\infty\big[$ et $\big]-\infty, a\big]$.

\end{itemize}
\end{df}
\begin{act}
Soit $f$ la fonction définie sur $\left[0,\pi \right]$ par $f(x)=\left\lbrace \begin{array}{ll}
\dfrac{1-\cos x}{x}&\text{si}~x\in\left]0,\pi\right]\\
0&\text{si}~x=0

\end{array} \right.$.\\
Montrer que la fonction $f$ est continue sur $\left[0,\pi\right]$.
\end{act}
\subsection{Opérations sur les limites:}
Les résultats résumés dans le tableau ci-dessous concernent les opérations sur les limites des fonctions en un réel $a$, à droite en $a$, à gauche en $b$ ou à l'infini. Soit $\ell$ et $\ell '$ deux réels.
\subsubsection{Limite d'une somme:}
\begin{center}

\begin{tabular}{|c|c|c|}
\hline
limite de $f$ & limite de $g$ & limite de $f+g$\\
\hline 
$\ell$ & $\ell '$ & $\ell+\ell'$\\
\hline
$\ell$ & $+\infty$ & $+\infty$\\
\hline
$\ell$ & $-\infty$ & $-\infty$\\
\hline
$+\infty$ & $+\infty$ & $+\infty$\\
\hline
$-\infty$ & $-\infty$ & $-\infty$\\
\hline
$-\infty$ & $+\infty$ & F.I\\
\hline
\end{tabular}
\end{center}
\subsubsection{Limite d'un produit:}
\begin{center}
\begin{tabular}{|c|c|c|}
\hline
limite de $f$ & limite de $g$ & limite de $f\times g$\\
\hline 
$\ell$ & $\ell '$ & $\ell \times \ell'$\\
\hline
$\ell\neq 0$ & $\infty$ & (R.S) $\infty$\\
\hline
$\infty$ & $\infty$ &(R.S) $\infty$\\
\hline
$0$ & $\infty$ & F.I\\
\hline
\end{tabular}
\end{center}
\subsubsection{Limite d'un quotient:}
\begin{center}
\begin{tabular}{|c|c|c|}
\hline
limite de $f$ & limite de $g$ & limite de $\dfrac{f}{g}$\\
\hline 
$\ell$ & $\ell '\neq 0$ & $\dfrac{\ell}{\ell'}$\\
\hline
$\ell$ & $+\infty$ & $+\infty$\\
\hline
$\ell$ & $\infty$ & $0$\\
\hline
$\infty$ & $\ell'$ & (R.S) $\infty$\\
\hline
$\ell \neq 0$ & $0 $ & (R.S) $\infty$\\
\hline
$0$ & $0$ & F.I\\
\hline
$\infty$ & $\infty$ & F.I\\
\hline
\end{tabular}
\end{center}
\begin{thr}
\begin{itemize}
\item La limite d'une fonction polynôme à l'infini est la même que celle de son terme de plus haut degré.
\item La limite d'une fonction rationnelle à l'infini est la même que celle du quotient des termes de haut degré.
\end{itemize}
\end{thr}
\begin{act}
Déterminer les limites ci-dessous:
\begin{enumerate}
\begin{multicols}{3}
\item $\lim_{x\to+\infty}3x^2-x+1$
\item $\lim_{x\to-\infty}\dfrac{2x^2-x+1}{x^2+2x-3}$
\item $\lim_{x\to+\infty}\dfrac{2x+3}{x^2+5x-4}$

\item $\lim_{x\to+\infty}\dfrac{\sqrt{4x^2+2x-3}}{2x+5}$
\item $\lim_{x\to-\infty}\sqrt{9x^2+3}+x$
\item $\lim_{x\to+\infty}\sqrt{4x^2+1}-2x$

\item $\lim_{x\to 0}\dfrac{\sqrt{x+4}-2}{x}$
\item $\lim_{x\to-\infty}\sqrt{x^2+x}+x$
\item $\lim_{x\to 1}\dfrac{x^2-2x+1}{x^2+x-2}$
\end{multicols}
\end{enumerate}
\end{act}
\section{Branches infinies:}
\begin{act}
Activité 1 page 9
\end{act}
\begin{df}
\begin{itemize}
\item Soit $f$ une fonction définie sur un intervalle ouvert $I$, sauf en un réel $a$ de $I$ et \cf  sa courbe représentative dans un repère orthogonl \oij.\\
Lorsque $\lim_{x\to a^-}f(x)=\pm \infty$ ou $\lim_{x\to a^+}f(x)=\pm \infty$ , on dit que la droite d'équation $x=a$ est une asymptote verticale à la courbe \cf.
\item Soit $f$ une fonction et \cf sa courbe représentative dans un repère orthonornal \oij.\\
Lorsque $\lim_{x\to+\infty}f(x)=L$ ou $\lim_{x\to-\infty}f(x)=L$ $(L\in\R)$, on dit que la droite d'équation $y=L$ est une asymptote horizontale à la courbe \cf.\\
Lorsque $\lim_{x\to+\infty}(f(x)-(ax+b))=0$ ou $\lim_{x\to-\infty}(f(x)-(ax+b))=0$ $(a\in\R^*, b\in\R)$ alors la droite d'équation $y=ax+b$ est une asymptote oblique à la courbe \cf.
\end{itemize}
\end{df}
\begin{exr}
Soit $f$ la fonction définie sur $\mathbb{R}\backslash \left\{ 1 \right\}$ par $f(x)=\dfrac{2x+1}{\left|x-1\right|}$. On désigne par \cf la courbe représentative de $f$ dans un repère orthonormé \oij.
\begin{enumerate}
\item Déterminer $\lim_{x\to 1}f(x)$. Interpréter graphiquement.
\item Calculer $\lim_{x\to+\infty}f(x)$ et $\lim_{x\to-\infty}f(x)$. Interpréter graphiquement les résultats obtenus.
\end{enumerate}
\end{exr}
\begin{exr}
Soit $g$ la fonction définie sur $\R$ par $g(x)=\dfrac{x}{\sqrt{x^2+1}}$. On désigne par \cg la courbe représentative de $g$ dans un repère orthonormé \oij.\\
Calculer $\lim_{x\to+\infty}g(x)$ et $\lim_{x\to-\infty}g(x)$. Interpréter graphiquement.
\end{exr}
\begin{exr}
Soit $h$ la fonction définie sur $\R$ par $h(x)=\sqrt{x^2+2x+2}$. On désigne par $\mathscr C_h$ la courbe représentative de $h$ dans un repère orthonormé \oij du plan.\\
Montrer que la droite $\Delta$ d'équation $y=x+1$ est une asymtote oblique à la courbe $\mathscr C_h$ au voisinage de $+\infty$.
\end{exr}
\begin{thr}
Soit $f$ une fonction et \cf sa courbe représentative dans un repère orthogonal \oij du plan.\\
Lorsque $\lim_{x\to+\infty}f(x)$ est infinie, alors la branche infinie de \cf au voisinage de $+\infty$ dépend de $\lim_{x\to+\infty}\dfrac{f(x)}{x}$.
\begin{enumerate}

\item Si $\lim_{x\to+\infty}\dfrac{f(x)}{x}$ est infinie, alors la courbe \cf admet une branche parabolique de dierction \Oj au voisinage de $+\infty$.
\item Si $\lim_{x\to+\infty}\dfrac{f(x)}{x}=0$, alors la courbe \cf admet une branche parabolique de direction \Oi au voisinage de $+\infty$.
\item Si $\lim_{x\to+\infty}\dfrac{f(x)}{x}=a, (a\in\R^*)$ alors deux cas peuvent se présenter selon $\lim_{x\to+\infty}(f(x)-ax)$.
\begin{enumerate}
\item Si $\lim_{x\to+\infty}(f(x)-ax)=b, (b\in\R)$ alors la droite d'équation $y=ax+b$ est une asymptote oblique à la courbe \cf au voisinage de $+\infty$.
\item Si $\lim_{x\to+\infty}(f(x)-ax)$ est infinie alors la droite d'équation $y=ax$ est une direction asymptotique à la courbe \cf au voisinage de $+\infty$.
\end{enumerate}
\end{enumerate}
\end{thr}
\begin{rem}
Les autres cas se déterminent d'une façon analogue.
\end{rem}
\begin{exr}
Soit $f$ la fonction définie sur $\left[ {1, + \infty } \right[$ par $f(x)=\sqrt{x-1}$. On note \cf la courbe représentative de $f$ dans un repère orthonormé \oij du plan.\\
Déterminer $\lim_{x\to+\infty}f(x)$ et $\lim_{x\to+\infty}\dfrac{f(x)}{x}$. Interpréter graphiquement.
\end{exr}
\begin{exr}
Soit $g$ la fonction définie sur $\R_+$ par $g(x)=x\sqrt{x}$. On appelle \cg la courbe représentative de $g$ dans un plan muni d'un repère orthonormé \oij.\\
Déterminer $\lim_{x\to+\infty}g(x)$ et $\lim_{x\to+\infty}\dfrac{g(x)}{x}$. Interpréter graphiquement.
\end{exr}
\begin{exr}
Soit $h$ la fonction définie sur $\R_+$ par $h(x)=x-\sqrt{x}$. On désigne par $\mathscr C_h$ la courbe représentative de $h$ dans le plan muni d'un repère orthonormé \oij.\\
Déterminer $\lim_{x\to+\infty}h(x)$, $\lim_{x\to+\infty}\dfrac{h(x)}{x}$ et $\lim_{x\to+\infty}(h(x)-x)$. Interpréter graphiquement.
\end{exr}
\section{Continuité et limite d'une fonction composée:}
\subsection{Composée de deux fonctions:}
\begin{df}
Soit $u$ une fonction définie sur un ensemble $I$ et $v$ une fonction définie sur un ensemble $J$ telle que $u(I) \subset J$. La fonction notée $v \circ u$, définie sur $I$ par $v\circ u(x)=v(u(x))$, est appelée fonction composée de $u$ et $v$.
\end{df}
\begin{act}
Activité 2 page 12
\end{act}
\begin{thr}
Soit $u$ une fonction définie sur un intervalle ouvert $I$ contenant un réel $a$.\\
Soit $v$ une fonction définie sur un intervalle ouvert $J$ contenant le réel $u(a)$.\\
Si $u$ est continue en $a$ et $v$ est continue en $u(a)$ alors la fonction $v\circ u$ est continue en $a$.
\end{thr}
\begin{dem}

\end{dem}
\begin{thr}
La composée de deux fonctions continues est une fonction continue.\\
\textbf{Plus précisément:} Si $u$ est continue sur un intervalle $I$ et $v$ est continue sur un intervalle $J$ telle que $u(I) \subset J$  alors la fonction $v\circ u$ est continue sur l'intervalle $I$.
\end{thr}
\begin{rem}
Si $J=\R$ alors la condition $u(I)\subset J$ devient unitile.
\end{rem}
\begin{act}
Activité page 12
\end{act}
\subsection{Limite d'une fonction composée:}
\begin{thr}
Soient $u$ et $v$ deux fonctions et $a$, $b$ et $c$ finis ou infinis.\\
Si $\lim_{x\to a}u(x)=b$ et $\lim_{x\to b}v(x)=c$ alors $\lim_{x\to a}v\circ u(x)=c$.
\end{thr}
\begin{act}
Les activités 1, 2, 3, 4 et 5 page 12
\end{act}
\section{Limites et ordre:}
\begin{thr}
Soient $f$, $u$ et $v$ des fonctions définies sur un intervalle ouvert $I$ suaf peut-être en un réel $a$ de $I$.\\
Soient $\ell$ et $\ell'$ deux réels.
\begin{enumerate}
\item Si $u(x) \leqslant v(x)$, pour tout $x\in I\backslash \left\{ a \right\}$ et si $\lim_{x\to a} u(x)=\ell$ et $\lim_{x\to a}v(x)=\ell'$ alors $\ell \leqslant \ell'$.
\item Si $u(x) \leqslant f(x) \leqslant v(x)$, pour tout $x\in I\backslash \left\{ a \right\}$ et si $\lim_{x\to a}u(x)=\lim_{x\to a}v(x)=\ell$ alors $\lim_{x\to a}f(x)=\ell$.
\item Si $u(x)  \leqslant f(x)$, pour tout $x\in I\backslash \left\{ a \right\}$ et si $\lim_{x\to a}u(x)=+\infty$ alors $\lim_{x\to a}f(x)=+\infty$.
\item Si $f(x) \leqslant u(x)$, pour tout $x\in I\backslash \left\{ a \right\}$ et si $\lim_{x\to a}u(x)=-\infty$ alors $\lim_{x\to a}f(x)=-\infty$.
\end{enumerate}
\end{thr}
\begin{rem}
Ces résultats restent valables lorsque l'on considère des limites à gauche en $a$, à droite en $a$ ou à l'infini.
\end{rem}
\begin{dem}
\end{dem}
\begin{exr}
Soient $f$ et $g$ les fonctions définies sur $\R$ par $f(x)=\dfrac{x+\cos x}{x^2+1}$ et $g(x)=2x-\sin x$.
\begin{enumerate}
\item Vérifier que: $\forall x\in\R$, $\dfrac{x-1}{x^2+1} \leqslant f(x) \leqslant \dfrac{x+1}{x^2+1}$, en déduire $\lim_{x\to+\infty}f(x)$ et $\lim_{x\to-\infty}f(x)$.
\item Vérifier que: $\forall x\in\R$, $2x-1 \leqslant g(x) \leqslant 2x+1$, en déduire $\lim_{x\to+\infty}g(x)$ et $\lim_{x\to-\infty}f(x)$.
\end{enumerate}
\end{exr}
\begin{exr}
Soient les fonctions $f$ et $g$ définies sur $\R$ par $f(x)=xE\left(\dfrac{1}{x}\right)$ et $g(x)=x^2E\left(\dfrac{1}{x}\right)$.
\begin{enumerate}
\item Montrer que: $\forall x\in\R^*$, $x-x^2 \leqslant g(x) \leqslant x$, en déduire $\lim_{x\to 0}g(x)$.
\item Déterminer alors $\lim_{x\to 0}f(x)$.
\end{enumerate}
\end{exr}
\section{Image d'un intervalle par une fonction continue:}
\begin{thr}
L'image d'un intervalle par une fonction continue est un intervalle
\end{thr}
\begin{thr}
Soit $f$ une fonction continue sur un intervalle $I$.\\
Soit $a$ et $b$ deux réels de $I$ tels que $a<b$.\\
Pour tout réel $k$ compris entre $f(a)$ et $f(b)$, l'équation $f(x)=k$ admet au moins une solution $\alpha$ dans l'intervalle $\left[ {a,b} \right]$.\\
En particulier si $f(a) \times f(b) <0$ alors l'équation $f(x)=0$ admet au moins une solution dans l'intervalle $\left] {a,b} \right[$.
\begin{center}
\newrgbcolor{qqqqcc}{0 0 0.8}
\newrgbcolor{zzqqzz}{0.6 0 0.6}
\psset{xunit=1.2cm,yunit=1.2cm,algebraic=true,dotstyle=o,dotsize=3pt 0,linewidth=0.8pt,arrowsize=3pt 2,arrowinset=0.25}
\begin{pspicture*}(-2.25,-0.48)(3.26,4.67)
\psaxes[labelFontSize=\scriptstyle,xAxis=true,yAxis=true,Dx=1,Dy=1,ticksize=-2pt 0,subticks=2]{->}(0,0)(-2.25,-0.48)(3.26,4.67)
\psplot[linewidth=1.2pt,linecolor=qqqqcc,plotpoints=200]{-1.2}{1.8}{x^4-3*x^2+0.3*x+3}
\psline[linestyle=dashed,dash=1pt 1pt,linecolor=red](-1.2,0.39)(-1.2,0)
\psline[linestyle=dashed,dash=1pt 1pt,linecolor=red](-1.2,0.39)(0,0.39)
\psline[linestyle=dashed,dash=1pt 1pt,linecolor=red](1.8,4.32)(1.8,0)
\psline[linestyle=dashed,dash=1pt 1pt,linecolor=red](1.8,4.32)(0,4.32)
\psplot{-2.25}{3.26}{(--2.3-0*x)/1}
\psline[linestyle=dashed,dash=1pt 1pt,linecolor=red](-0.45,2.3)(-0.45,0)
\psline[linestyle=dashed,dash=1pt 1pt,linecolor=red](0.57,2.3)(0.57,0)
\psline[linestyle=dashed,dash=1pt 1pt,linecolor=red](1.59,2.3)(1.59,0)
\psline[linewidth=1.2pt,linecolor=red]{->}(0,0)(1,0)
\psline[linewidth=1.2pt,linecolor=red]{->}(0,0)(0,1)
\rput[tl](-0.56,-0.2){$\zzqqzz{x_1}$}
\rput[tl](0.44,-0.14){$\zzqqzz{x_2}$}
\rput[tl](1.43,-0.1){$\zzqqzz{x_3}$}
\rput[tl](-1.43,-0.04){$\zzqqzz{a}$}
\rput[tl](1.81,-0.05){$\zzqqzz{b}$}
\rput[tl](1.97,2.01){$\zzqqzz{\Delta:y=k}$}
\rput[tl](0.05,0.74){$\zzqqzz{f(a)}$}
\rput[tl](-0.84,3.99){$\zzqqzz{f(b)}$}
\rput[tl](-1.17,1.9){$\zzqqzz{\mathscr{C}_f}$}
\rput[tl](0.12,2.82){$\zzqqzz{k}$}
\psdots[dotstyle=*,linecolor=red](-1.2,0.39)
\psdots[dotstyle=*,linecolor=red](1.8,4.32)
\psdots[dotstyle=*,linecolor=red](-0.45,2.3)
\psdots[dotstyle=*,linecolor=red](0.57,2.3)
\psdots[dotstyle=*,linecolor=red](1.59,2.3)
\psdots[dotstyle=*,linecolor=red](0,0)
\rput[bl](-0.21,-0.35){\red{$O$}}
\end{pspicture*}
\end{center} 
\end{thr}
\begin{thr}
Soit $f$ une fonction continue et strictement monotone sur un intervalle $I$.\\
Soit $a$ et $b$ deux réels de $I$ tels que $a<b$.\\
Pour tout réel $k$ compris entre $f(a)$ et $f(b)$, l'équation $f(x)=k$ admet une unique solution dans l'intervalle $\left[ {a,b} \right]$.
\end{thr}
\begin{act}
Activité 5 page 17
\end{act}
\begin{act}
Soit $f$ une fonction continue sur $\left[ {0,1} \right]$ et telle que ppur tout $x\in\left[ {0,1} \right]$, $f(x)\in\left[ {0,1} \right]$.\\
Montrer qu'il existe au moins un réel $\beta$ de $\left[ {0,1} \right]$ tel que $f(\beta)=\beta$.
\end{act}
\begin{thr}
Soit $f$ une fonction continue sur un intervalle $I$.\\
Si $f$ ne s'annule en aucun réel de $I$ alors $f$ gardre un signe constant sur $I$.
\end{thr}
\begin{dem}

\end{dem}
\begin{thr}
L'image d'un intervalle fermé borné $\left[ {a,b} \right]$ par une fonction continue est un intervalle fermé borné $\left[ {m,M} \right]$.\\
$m$ est minimum de $f$ sur $\left[ {a,b} \right]$. Il existe un réel $\alpha \in \left[ {a,b} \right]$ tel que $m=f(\alpha)$.\\
$M$ est le maximum de $f$ sur $\left[ {a,b} \right]$. Il existe un réel $\beta\in \left[ {a,b} \right]$ tel que $M=f(\beta)$.\\
On dit que $f$ atteint des bornes en $\alpha$ et $\beta$.
\end{thr}
\begin{exr}
Soit $f$ une fonction définie sur $\R$ par $f(x)=\sin^2x$\\
Montrer que $f\left( {\left[ { - \frac{\pi }
{6},\frac{\pi }
{6}} \right]} \right) = \left[ {0,\frac{1}
{4}} \right]$.
\end{exr}
\section{Image d'un intervalle par une fonction strictement monotone:}
\begin{thr}
Soit $f$ une fonction définie sur un intervalle de type $\left[ {a,b} \right[$ ($b$ fini ou infini)
\begin{enumerate}
\item Si la fonction $f$ est croissante et majorée alors $f$ possède une limite finie en $b$.
\item Si la fonction $f$ est croissante et non majorée alors $f$ tend vers $+\infty$ en $b$.
\item Si la fonction $f$ est décroissante et minorée alors $f$ possède une limite finie en $b$.
\item Si la fonction $f$ est décroissante et non minorée alors $f$ tend vers $-\infty$ en $b$.
\end{enumerate}
\end{thr}
\begin{act}
Activité 1 page 19
\end{act}
\begin{thr}
L'image d'un intervalle par une fonction continue et strictement monotone est un intervalle de même nature.
\end{thr}
\begin{expl}
Soient $a$ et $b$ deux réels.
\begin{center}
\begin{tabular}{|c|c|c|}
\hline 
Intervalle $I$&Si $f$ est continue et strictement & Si $f$ est continue et strictement \\
 & croissante sur $I$& décroissante sur $I$\\
\hline
$I=\big[a,b\big]$&$f(I)=\big[f(a),f(b)\big]$&$f(I)=\big[f(b),f(a)\big]$\\
\hline
$I=\big[a,b\big[$&$f(I)=\big[f(a),\lim_{b^-}f\big[$&$f(I)=\big]\lim_{b^-}f,f(a)\big]$\\
\hline
$I=\big[a,+\infty\big[$&$f(I)=\big[f(a),\lim_{+\infty}f\big[$&$f(I)=\big]\lim_{+\infty}f,f(a)\big]$\\
\hline
$I=\big]a,b\big[$&$f(I)=\big]\lim_{a^+}f,\lim_{b^-}f\big[$&$f(I)=\big]\lim_{b^-}f,\lim_{a^+}f\big[$\\
\hline
\end{tabular}
\end{center}
\end{expl}
\begin{exr}
Soit $f$ la fonction définie sur $\big]2,+\infty\big[$ par $f(x)=\dfrac{x+1}{x-2}$.\\
Déterminer $f\left( {\left] {2, + \infty } \right[} \right)$.
\end{exr}
\begin{exr}
On donne ci-dessous le tableau de variation d'une fonction $f$ définie et continue sur chacun des intervalles $]-\infty,3[$ et $]3,+\infty[$.
\begin{center}
\begin{tikzpicture}
\tkzTabInit[espcl=2]{$x$/1, $f$/2}{$-\infty$, $-5$, $2$, $3$, $+\infty$}
\tkzTabVar{+/$+\infty$,-/$-3$,+/$8$,-D-/$-\infty$ /$-\infty$,+/$7$}
\end{tikzpicture}
\end{center}
Déterminer l'image par $f$ de chacun des intervalles suivants:
\begin{multicols}{3}
\begin{enumerate}
\item $]-\infty,-5]$
\item $]-\infty,3[$
\item $]3,+\infty[$
\end{enumerate}
\end{multicols}
\end{exr}
\end{document}