\usepackage{xkeyval}[2005/11/25]
\usepackage{eso-pic}
\usepackage{fp}
\usepackage{tikz}
\usetikzlibrary{calc}
\usetikzlibrary{decorations.pathmorphing}
\usetikzlibrary{decorations.pathreplacing} 
\usetikzlibrary{decorations.shapes} 
\usetikzlibrary{shadows}
\usetikzlibrary{arrows}
\usetikzlibrary{shapes,snakes,shapes.geometric,shapes.misc}
%----------------------------------------------------------\cadre[x=... ,y=.....,  ]------------------------------------------------------%

% \cadre[			shadow (bool�en),
%					lw = , (�paisseur du trait, en pt)
%					style = , (dashed ou dotted)
%					x = abscisse du sommet en bas � gauche,
%					y = ordonn�e du sommet en bas � gauche,
%					xshadow = d�calage selon x de l'ombre,
%					yshadow = d�calage selon y de l'ombre,
%					color = couleur du cadre,
%					colorshadow = couleur de l'ombre,
%					decoration = nom de la d�coration,
%					doubleline (bool�en) pour pstricks
%				  ]
\makeatletter
\define@cmdkey [PAS] {cadre} {x}{}
\define@cmdkey [PAS] {cadre} {y}{}
\define@cmdkey [PAS] {cadre} {decoration}{}
\define@cmdkey [PAS] {cadre} {shape}{}
\define@cmdkey [PAS] {cadre} {lw}{}
\define@cmdkey [PAS] {cadre} {xshadow}{}
\define@cmdkey [PAS] {cadre} {yshadow}{}
\define@cmdkey [PAS] {cadre} {bordercolor}{}
\define@cmdkey [PAS] {cadre} {incolor}{}
\define@cmdkey [PAS] {cadre} {shadowcolor}{}
\define@cmdkey [PAS] {cadre} {style}{}
\define@boolkey[PAS] {cadre} {shadow}[true]{}
\define@boolkey[PAS] {cadre} {doubleline}[true]{}

\presetkeys    [PAS] {cadre} {	shadow = false,
								lw = 2,
								x = 0.2,
								y = 0.2,
								xshadow =- 0.35,
								yshadow =0.2,
								bordercolor = black,
								incolor = white,
								shadowcolor = gray,
								style = ,
								doubleline = false,
								decoration = , 
								shape = }{}

\newcommand*{\cadre}[1][]{\pasCadre[#1]}

\def\pasCadre[#1]{
	\setkeys[PAS]{cadre}{#1}
	
	\AddToShipoutPicture
	{
		\put(\LenToUnit{\cmdPAS@cadre@x cm},\LenToUnit{\cmdPAS@cadre@y cm})
		{
				\ifPAS@cadre@doubleline
					\edef\dl{double}
				\else
					\edef\dl{}
				\fi
				\begin{tikzpicture}[decoration={\cmdPAS@cadre@decoration,shape=\cmdPAS@cadre@shape}]
					\pgfsetlinewidth{\cmdPAS@cadre@lw pt}
					\ifPAS@cadre@shadow
						\filldraw[
						decorate,
						fill=\cmdPAS@cadre@incolor,
						draw=\cmdPAS@cadre@bordercolor,
						style=\cmdPAS@cadre@style,
						drop shadow={fill=\cmdPAS@cadre@shadowcolor,shadow xshift=\cmdPAS@cadre@xshadow cm,shadow yshift=-\cmdPAS@cadre@yshadow cm},
						\dl] (0,0) -- (0,\paperheight-2*\cmdPAS@cadre@y cm) -- (\paperwidth-2*\cmdPAS@cadre@x cm,\paperheight-2*\cmdPAS@cadre@y cm) -- (\paperwidth-2*\cmdPAS@cadre@x cm,0) -- cycle;
					\else
						\filldraw[
						decorate,
						fill=\cmdPAS@cadre@incolor,
						draw=\cmdPAS@cadre@bordercolor,
						style=\cmdPAS@cadre@style,
						\dl] (0,0) -- (0,\paperheight-2*\cmdPAS@cadre@y cm) -- (\paperwidth-2*\cmdPAS@cadre@x cm,\paperheight-2*\cmdPAS@cadre@y cm) -- (\paperwidth-2*\cmdPAS@cadre@x cm,0) -- cycle;
					\fi
				\end{tikzpicture}
		}
	}
}

\def\nocadre{\ClearShipoutPicture}
\makeatother