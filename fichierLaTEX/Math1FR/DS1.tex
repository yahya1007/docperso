\documentclass[addpoints]{exam}
\usepackage[most]{tcolorbox}
\usepackage{tikz}
\usetikzlibrary{babel}

\usepackage{lmodern}
\usepackage[main=french]{babel}
\usepackage[utf8]{inputenc}
\usepackage[T1]{fontenc}

\usepackage{xspace}
\usepackage{amsmath}
\usepackage{listings}

\tcbset{coltitle=black ,colbacktitle= white}

\begin{document}

\begin{tcolorbox}
\begin{minipage}{0.2\linewidth}
niveau : 1 APIC
\end{minipage}\hfill
\begin{minipage}{0.5\linewidth}
Devoir surveillez de Mathématiques 1
\end{minipage}\hfill
\begin{minipage}{0.3\linewidth}
Date : 28/10/2019
\end{minipage}
\end{tcolorbox}
	
	
\begin{tcolorbox}[title=Exercice 1]
	\begin{questions}
		\question calculer ce qui suit :

$ B=(12+2)\times( 3+15\div 5)= ..............................................$\vspace{0.3cm}

$ C=250-210\div 30-10= ...............................................$\vspace{0.3cm}

$ D=115+(7-14\div 2)\times 3=.........................................$\vspace{0.3cm}

$ E=(18\div 3)+(3\times(12+7-8))=....................................$

$ F=\dfrac{7}{4}+\dfrac{8}{4}\times 5 =......................................... $
		
	\end{questions}
\end{tcolorbox}

	
\begin{tcolorbox}[title=Exercice 2]
	\begin{questions}
		\question calculer et réduire si possible :
		
\begin{minipage}{0.4\linewidth}
\[\dfrac{108}{18}+\dfrac{60}{18}=\dfrac{.......}{.....}\]
\[\dfrac{15}{81}+\dfrac{6}{81}=\dfrac{.......}{.....}\]
\[\dfrac{6}{5}+\dfrac{16}{15}=\dfrac{.......}{.....}\]
\[\dfrac{1}{5}\times \dfrac{6}{51}=\dfrac{.......}{.....}\]
\[\dfrac{4}{51}\times\dfrac{7}{2}=\dfrac{.......}{.....}\]

\end{minipage}
\begin{minipage}{0.4\linewidth}
\[\dfrac{201}{17}-\dfrac{16}{17}=\dfrac{.......}{.....}\]
\[\dfrac{8}{5}-\dfrac{6}{10}=\dfrac{.......}{.....}\]
\[\dfrac{1}{5}\div\dfrac{6}{5}=\dfrac{.......}{.....}\]
\[\dfrac{14}{3}\div\dfrac{16}{3}=\dfrac{.......}{.....}\]
\end{minipage}		
	\end{questions}
\end{tcolorbox}

\begin{tcolorbox}[title=Exercice 3]
	\begin{questions}
		\question réduire le plus possible :
		
		\[\dfrac{250}{50}=\dfrac{.......}{.....}\]
\[\dfrac{8\times 571}{8 \times 107}=\dfrac{.......}{.....}\]
\[\dfrac{10+18}{10+12}=\dfrac{.......}{.....}\]
		
	\end{questions}
\end{tcolorbox}		

\begin{tcolorbox}[title=Exercice 4]
	\begin{questions}
		\question comparer ce qui suit  :	
$$\dfrac{973}{987}......... 1 \hspace{1cm}   \hspace{1cm}
\dfrac{93}{98}.........\dfrac{98}{93}$$

\question ranger dans l'ordre croissant:

$$\dfrac{8}{6} ;; 1 ;; \dfrac{14}{12} ;; \dfrac{25}{30}  ;; \dfrac{7}{42}$$	

\end{questions}
\end{tcolorbox}	
	
	
\end{document}