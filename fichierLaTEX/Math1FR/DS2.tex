\documentclass[addpoints]{exam}
\usepackage[most]{tcolorbox}
\usepackage{tikz}
\usetikzlibrary{babel}

\usepackage{lmodern}
\usepackage[main=french]{babel}
\usepackage[utf8]{inputenc}
\usepackage[T1]{fontenc}

\usepackage{xspace}
\usepackage{amsmath}
\usepackage{listings}

\tcbset{coltitle=black ,colbacktitle= white}

\begin{document}
Nom :.................................................
Classe :...........................
N : ....................
\vspace{0.5cm}
\begin{tcolorbox}
\begin{minipage}{0.2\linewidth}
niveau : 1 APIC
\end{minipage}\hfill
\begin{minipage}{0.5\linewidth}
Devoir surveillez de Mathématiques 2
\end{minipage}\hfill
\begin{minipage}{0.3\linewidth}
Date : 08/12/2019
\end{minipage}
\end{tcolorbox}

\begin{tcolorbox}[title=Exercice 1]
1) ranger les nombres dans l'ordre croissant
\[-17 ** 5.36 ** -6.35 ** - 14 ** -7 ** 26 ** 10.36 **-12.36   \]

\vspace{1cm}
2) tracer une droite gradue et poser les points d'abscisses $x_{A}=-3.5 $ et $x_{B}=-5 $ et $x_{C}=2.5 $ et $x_{D}=4 $

\vspace{1cm}
\end{tcolorbox}

\begin{tcolorbox}[title=Exercice 2]
calculer ce qui suit :

 $-14 + (-20)= . . . . \hspace{1cm} 17+(-18)=. . . . \hspace{1cm} -15+14=. . . .  $
 
\vspace{1cm}
 $15-30= . . . . \hspace{1cm} -12-20= . . . . \hspace{1cm} 14-(-20)= . . . .  $
 
\vspace{1cm}
 $-5\times (-7)= . . . . \hspace{1cm}  3\times (-7) = . . . . \hspace{1cm} -5\times 6 = . . . . $ 
 
\vspace{1cm}
$-20\div 4 = . . . . \hspace{1cm}  -16\div (-4)= . . . .  $ 
\[A=-13+(-15)-(-19)-(-25)-(+15)-(+36)=. .. .. .. .. ..\] 
\[B=-20-15-35-17+18-6+58=. .. . .... .... ......\] 
\end{tcolorbox}

\begin{tcolorbox}[title=Exercice 2]
calculer le perimetre et la surface de la figure : 

\definecolor{cqcqcq}{rgb}{0.75,0.75,0.75}
\begin{tikzpicture}[line cap=round,line join=round,x=0.5012671434015871cm,y=0.47197113350550923cm]
\draw [color=cqcqcq,dash pattern=on 1pt off 1pt, xstep=0.5012671434015871cm,ystep=0.47197113350550923cm] (3.61,-5.89) grid (15.58,4.7);
\clip(3.61,-5.89) rectangle (15.58,4.7);
\draw (14,-5)-- (14,-1);
\draw (14,-1)-- (5,-1);
\draw (5,-1)-- (5,-5);
\draw [shift={(11,-1)}] plot[domain=0:3.14,variable=\t]({1*2*cos(\t r)+0*2*sin(\t r)},{0*2*cos(\t r)+1*2*sin(\t r)});
\draw (8,-1)-- (8,3);
\draw (8,3)-- (7,3);
\draw (7,3)-- (7,4);
\draw (7,4)-- (6,4);
\draw (6,4)-- (6,3);
\draw (6,3)-- (5,3);
\draw (5,3)-- (5,-1);
\draw (14,-5)-- (12,-5);
\draw (12,-5)-- (12,-4);
\draw (12,-4)-- (11,-4);
\draw (11,-4)-- (11,-5);
\draw (11,-5)-- (5,-5);
\end{tikzpicture}
\end{tcolorbox}
\end{document}