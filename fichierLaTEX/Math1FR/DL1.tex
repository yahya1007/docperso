\documentclass[addpoints]{exam}
\usepackage[most]{tcolorbox}
\usepackage{tikz}
\usetikzlibrary{babel}

\usepackage{lmodern}
\usepackage[main=french]{babel}
\usepackage[utf8]{inputenc}
\usepackage[T1]{fontenc}

\usepackage{xspace}
\usepackage{amsmath}
\usepackage{listings}

\tcbset{coltitle=black ,colbacktitle= white}

\begin{document}

\begin{tcolorbox}
\begin{minipage}{0.2\linewidth}
niveau : 1 APIC
\end{minipage}\hfill
\begin{minipage}{0.5\linewidth}
Devoir libre de Mathématiques 1
\end{minipage}\hfill
\begin{minipage}{0.3\linewidth}
Date : 05/11/2020
\end{minipage}
\end{tcolorbox}
	
	
\begin{tcolorbox}[title=Exercice 1]
	\begin{questions}
		\question calculer ce qui suit :
		
		$ A=17+5\times 3-12=.................................................$\vspace{0.3cm}

$ B=12+2\times 3+15\div 5= ..............................................$\vspace{0.3cm}

$ C=270-210\div 30-100= ...............................................$\vspace{0.3cm}

$ D=115+(7-14\div 2)\times 3=.........................................$\vspace{0.3cm}

$ E=(18\div 3)+(3\times(12+7-8))=....................................$
		
	\end{questions}
\end{tcolorbox}

	
\begin{tcolorbox}[title=Exercice 2]
	\begin{questions}
		\question calculer :
		
\begin{minipage}{0.4\linewidth}
\[\dfrac{18}{5}+\dfrac{61}{5}=\dfrac{.......}{.....}\]
\[\dfrac{15}{81}+\dfrac{6}{81}=\dfrac{.......}{.....}\]
\[\dfrac{6}{5}+\dfrac{16}{15}=\dfrac{.......}{.....}\]
\[\dfrac{1}{5}\times \dfrac{6}{51}=\dfrac{.......}{.....}\]
\[\dfrac{4}{51}\times\dfrac{7}{2}=\dfrac{.......}{.....}\]

\end{minipage}
\begin{minipage}{0.4\linewidth}
\[\dfrac{201}{17}-\dfrac{16}{17}=\dfrac{.......}{.....}\]
\[\dfrac{8}{5}-\dfrac{6}{10}=\dfrac{.......}{.....}\]
\[\dfrac{1}{5}\div\dfrac{6}{5}=\dfrac{.......}{.....}\]
\[\dfrac{14}{3}\div\dfrac{16}{3}=\dfrac{.......}{.....}\]
\end{minipage}		
	\end{questions}
\end{tcolorbox}

\begin{tcolorbox}[title=Exercice 3]
	\begin{questions}
		\question réduire le plus possible :
		
		\[\dfrac{25}{250}=\dfrac{.......}{.....}\]
\[\dfrac{18\times 51}{18 \times 17}=\dfrac{.......}{.....}\]
\[\dfrac{10+8}{10+2}=\dfrac{.......}{.....}\]
		
	\end{questions}
\end{tcolorbox}		

\begin{tcolorbox}[title=Exercice 4]
	\begin{questions}
		\question comparer ce qui suit  :	
$$\dfrac{93}{98}......... 1 \hspace{1cm}   \hspace{1cm}
\dfrac{93}{98}.........\dfrac{98}{93}$$

\question ranger dans l'ordre croissant:

$$\dfrac{8}{7} ;; 1 ;; \dfrac{12}{14} ;; \dfrac{30}{21}  ;; \dfrac{24}{28}$$	

\end{questions}
\end{tcolorbox}	
	
	
\end{document}