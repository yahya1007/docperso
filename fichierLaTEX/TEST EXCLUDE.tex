\documentclass[12pt,a4paper]{article}
\usepackage[T1]{fontenc}          
\usepackage[utf8]{inputenc}
\usepackage[french]{babel}
\usepackage{fancybox}		
\usepackage{amssymb}
\usepackage{amsmath}
\usepackage{amsthm}
\usepackage{ mathrsfs }
\usepackage{stmaryrd}
\usepackage{tkz-tab}
\usepackage{comment}
\usepackage{lipsum}
\usepackage{makeidx}
\usepackage{multicol}
\usepackage{tikz}    
\usetikzlibrary{decorations.pathreplacing,shapes, backgrounds, patterns, arrows,positioning,automata,shadows,fit,shapes,trees} 
\usepackage{tikz-qtree,tikz-qtree-compat}
\usepackage{graphicx} 
\usepackage{amsfonts}
\usepackage{wrapfig}
\usepackage{enumitem}
\usepackage{pict2e}
\usepackage{graphicx}
\usepackage{systeme}
\usepackage{tikz}
\usetikzlibrary{shapes}
\usepackage[left=2cm, right =2cm, top =2cm, bottom =2cm]{geometry}
\usepackage{fullpage}
\usepackage{eso-pic}
\usepackage{ hyperref}
\usepackage{biblatex}
\usepackage{xcolor} 
\frenchbsetup{StandardLists=true}
\usepackage{mdframed}
\usepackage{lipsum}
\usepackage[most]{tcolorbox}
\usepackage{comment}

\geometry{left=2cm, right =2cm, top =0.8cm, bottom =2cm}


\theoremstyle{definition}

\newtheorem{exo}{Exercice}

\newtcbtheorem[no counter]{rep}{Réponse}{%
                lower separated=false,
                colback=white,
colframe=black, fonttitle=\bfseries,
colbacktitle=white,
coltitle=black,
enhanced,
boxed title style={colframe=black},
attach boxed title to top left={xshift=0.5cm,yshift=-2mm},
}{theo}


%%------------------------------------------------------------------
% Rédaction du document
%%----------------------------------------------------------------

\excludecomment{exo}

\begin{document}
\begin{center}

\Large
\textbf{{TD11 : Probabilités (\textbf{Correction})}}\\


\end{center}

\begin{exo}
On tire une carte au hasard dans un jeu de 52 cartes. Soit les événements : \\
$T$ : "La carte tirée est un trèfle." \\
$F$ : "La carte tirée est une figure." \\
$S$ : "La carte tirée est un chiffre inférieur ou égal à 7." \\
$C$ : "La carte tirée est un carreau". 


\begin{enumerate}
   
    \item Décrire avec des phrases les événements suivants.
    \begin{multicols}{3}
    

    \begin{enumerate}
        \item $T \cap S$.
        \item $C \cup F$.
        \item $\overline{F}$.
        \item $\overline{S} \cap C$.
        \item $F \cap S$.
        \item $F \cup \overline{T}$.
    \end{enumerate}
        \end{multicols}

\begin{rep}{}{}

\begin{enumerate}
        \item $T \cap S$ : "La carte tirée est un trèfle inférieur ou égal à 7"
        \item $C \cup F$ : "La carte tirée est un carreau ou une figure"
        \item $\overline{F}$ : "La carte tirée n'est pas une figure"
        \item $\overline{S} \cap C$ : "La carte tirée est un carreau strictement plus grand que 7"
        \item $F \cap S$ : "La carte tirée est une figure inférieure ou égale à 7"
        \item $F \cup \overline{T}$ : "La carte tirée est une figure qui n'est pas un trèfle"
    \end{enumerate}

\end{rep}
        
        \item Écrire les événements suivants à l’aide des événements $T$, $F$, $A$ et $C$.
        
    \begin{enumerate}
        \item "La carte tirée est un trèfle ou une figure."
        \item "La carte tirée est un carreau inférieur ou égal à 7."
        \item "La carte tirée n'est ni une figure, ni un carreau."
        \item "La carte tirée est une figure plus grande que 7"
        \item "La carte tirée n'est ni un trèfle, ni une figure, ni un carreau."
    \end{enumerate}

    \begin{rep}{}{}

         
    \begin{enumerate}
        \item "La carte tirée est un trèfle ou une figure." : $T \cup F$
        \item "La carte tirée est un carreau inférieur ou égal à 7." : $C \cap S$
        \item "La carte tirée n'est ni une figure, ni un carreau." : $\overline{C} \cap \overline{F}$
        \item "La carte tirée est une figure plus grande que 7" : $F \cap \overline{S}$
        \item "La carte tirée n'est ni un trèfle, ni une figure, ni un carreau." : $\overline{T} \cap \overline{F} \cap \overline{C}$
    \end{enumerate}
    

    \end{rep}
    
\end{enumerate}

\end{exo}
\end{document}