\documentclass[12pt,a4paper,fleqn]{article}
\usepackage[utf8]{inputenc}
\usepackage{amssymb,mathtools,amsthm,amsmath,amsfonts,array}
\usepackage[left=1cm,right=1cm,top=1cm,bottom=1cm]{geometry}
\everymath{\displaystyle}
\usepackage[many]{tcolorbox}
\usetikzlibrary{calc}
\usepackage{wrapfig} %%% images et figures
\usepackage{fourier}
\usepackage{tikz}
\usepackage{lipsum}
\usepackage{tabularx}
\usepackage{graphicx}
\pagestyle{empty} %%%% aucune en-tete et aucun pied de page

\usepackage[T1]{fontenc}
\usepackage[frenchb]{babel}

\usepackage{enumitem} %%% modifie numérotation item
\usepackage{hyperref}
%\usepackage[french]{babel}



\usepackage{xcolor}
\usepackage{color} %% colorer texte

%%%%%%%%%%%%%%%%%%%%%  EXERCICE  %%%%%%%%%%%%%%%%%%%%%%%%%%%%%%%
\newcommand\dunderline[3][-1pt]{{%
\setbox0=\hbox{#3}
\ooalign{\copy0\cr\rule[\dimexpr#1-#2\relax]{\wd0}{#2}}}}
\newtcolorbox[auto counter]{Exercice}[1][]{
enhanced,left=80pt,
interior code app={\fill[blue!25,line width=1cm] ([xshift=-0cm,yshift=-0.3cm]frame.north west)rectangle([xshift=2.7cm]frame.south west);} ,
colframe=white,colback=white,
attach boxed title to top left={xshift=-0cm,yshift=-\tcboxedtitleheight/3},
boxed title style={colback=red!44,left=-2pt},
coltitle=black,
title=\dunderline{2.5pt}{\sffamily \textbf{Exercice n$^{\circ}$\thetcbcounter :}}\quad$\big($\textbf{#1}$ \big)$ }
\newcommand*\circled[1]{\tikz[baseline=(char.base)]{%
\node[shape=circle,fill=black,draw,inner sep=2pt] (char) {#1};}}
\tikzstyle{every picture}+=[remember picture]
\newcommand{\pts}[2]{\tikz[overlay,anchor=base, baseline,xshift=-#1cm] {\node[](4){\textbf{#2}};}}


%%%%%%%%%%%%%%%%%%%%%% l'entête %%%%%%%%%%%%%%%%%%%%%%%%%%%%%%%%%%%%
\newtcolorbox[]{entete}[1][]{
breakable,
enhanced,
arc=2pt,
top=3mm,
left=3mm,
colframe=black,
colback=blue!15
}

\begin{document}
\begin{entete}
\begin{tabular}{m{5cm}m{7.2cm}m{5cm}}
Niveau: 2APIC  \newline Durée:\quad 1h \newline
Prof : \emph{\href{https://mathelatex.com}{Mathe\LaTeX}} & \centering \begin{large}
\textbf{DEVOIR SURVEILLE N\textsuperscript{o} 1}
\end{large}\vspace{0.05cm}\newline
\centering  Semestre 2 $\qquad\quad$ & \centering \href{https://mathelatex.com}{Mathe\LaTeX} \quad \href{https://mathelatex.com}{Mathe\LaTeX} \\ \href{https://mathelatex.com}{Mathe\LaTeX} \newline
Année scolaire :2022-2023
\end{tabular}
\end{entete}

\large
%%%%%%%%%%%%%%%%%%%%%%%%%%%%%%%%%%%%%%%%%%%%%%%%%%%%%%%%%%%%%%%%%%%%%%%%%%%%%%
%%%%%%%%%%%%%%%%%%%%%%%%%%%%%%%%%%%%%%%%%%%%%%%%%%%%%%%%%%%%%%%%%%%%%%%%%%%%%%
%%%%%%%%%%%%%%%%%%%%%%%%%%%%%% EXERCIC 1 %%%%%%%%%%%%%%%%%%%%%%%%%%%%%%%%%%%%%
%%%%%%%%%%%%%%%%%%%%%%%%%%%%%%%%%%%%%%%%%%%%%%%%%%%%%%%%%%%%%%%%%%%%%%%%%%%%%%
%%%%%%%%%%%%%%%%%%%%%%%%%%%%%%%%%%%%%%%%%%%%%%%%%%%%%%%%%%%%%%%%%%%%%%%%%%%%%%

\begin{Exercice}[4 points]
Complète les phrases suivantes:
\begin{enumerate}
      \item[•] \pts{2.6}{2 pts} La bissectrice d'un angle est  \dotfill
      \item[•] \pts{2.6}{1 pt} Le centre de gravité d'un triangle est le point de concours des \dotfill
      \item[•] \pts{2.6}{1 pt} .\dotfill est le point de concours des bissectrices.
      \item[•] \pts{2.6}{1 pt} .\dotfill est le point de concours des hauteurs.
\end{enumerate}
\end{Exercice}

%%%%%%%%%%%%%%%%%%%%%%%%%%%%%%%%%%%%%%%%%%%%%%%%%%%%%%%%%%%%%%%%%%%%%%%%%%%%%%
%%%%%%%%%%%%%%%%%%%%%%%%%%%%%%%%%%%%%%%%%%%%%%%%%%%%%%%%%%%%%%%%%%%%%%%%%%%%%%
%%%%%%%%%%%%%%%%%%%%%%%%%%%%%% EXERCIC 2 %%%%%%%%%%%%%%%%%%%%%%%%%%%%%%%%%%%%%
%%%%%%%%%%%%%%%%%%%%%%%%%%%%%%%%%%%%%%%%%%%%%%%%%%%%%%%%%%%%%%%%%%%%%%%%%%%%%%
%%%%%%%%%%%%%%%%%%%%%%%%%%%%%%%%%%%%%%%%%%%%%%%%%%%%%%%%%%%%%%%%%%%%%%%%%%%%%%

\begin{Exercice}[7 points]
\begin{enumerate}
\item Développer puis réduire les expressions suivantes:
     \begin{enumerate}
     \item \pts{3.6}{1 pt} $A=2x(1-x)$
     \item \pts{3.6}{1 pt} $B=(2x-1)(x+3)$
     \item \pts{3.6}{1 pt} $C=(3x-2)^2$
     \item \pts{3.6}{1 pt} $D=(x-3)(x+3)$
     \end{enumerate}
\item Factoriser les expressions suivantes:
     \begin{enumerate}
     \item \pts{3.6}{1 pt} $E=36x^2+12$
     \item \pts{3.6}{1 pt} $F=x^2-8x+16$
     \item \pts{3.6}{1 pt} $G=9x^2-100$
     \end{enumerate}
\end{enumerate}
\end{Exercice}

%%%%%%%%%%%%%%%%%%%%%%%%%%%%%%%%%%%%%%%%%%%%%%%%%%%%%%%%%%%%%%%%%%%%%%%%%%%%%%
%%%%%%%%%%%%%%%%%%%%%%%%%%%%%%%%%%%%%%%%%%%%%%%%%%%%%%%%%%%%%%%%%%%%%%%%%%%%%%
%%%%%%%%%%%%%%%%%%%%%%%%%%%%%% EXERCIC 3 %%%%%%%%%%%%%%%%%%%%%%%%%%%%%%%%%%%%%
%%%%%%%%%%%%%%%%%%%%%%%%%%%%%%%%%%%%%%%%%%%%%%%%%%%%%%%%%%%%%%%%%%%%%%%%%%%%%%
%%%%%%%%%%%%%%%%%%%%%%%%%%%%%%%%%%%%%%%%%%%%%%%%%%%%%%%%%%%%%%%%%%%%%%%%%%%%%%

\begin{Exercice}[5 points]
\begin{enumerate}
\item Considérons l'expression suivante: $H=16-(2x+3)^2$
     \begin{enumerate}
     \item \pts{3.6}{1 pt} Développer et réduire $H$.
     \item \pts{3.6}{1 pt} Factoriser $H$.
     \item \pts{3.6}{1 pt} Calculer la valeur de $H$ pour $x=\dfrac{1}{2}$
     \end{enumerate}
\item \pts{2.6}{2 pts} Soit $x$ et $y$ deux nombres rationnels.\\
 Montrer que: $(x+y)^2+(x-y)^2=2(x^2+y^2)$
\end{enumerate}
\end{Exercice}

%%%%%%%%%%%%%%%%%%%%%%%%%%%%%%%%%%%%%%%%%%%%%%%%%%%%%%%%%%%%%%%%%%%%%%%%%%%%%%
%%%%%%%%%%%%%%%%%%%%%%%%%%%%%%%%%%%%%%%%%%%%%%%%%%%%%%%%%%%%%%%%%%%%%%%%%%%%%%
%%%%%%%%%%%%%%%%%%%%%%%%%%%%%% EXERCIC 4 %%%%%%%%%%%%%%%%%%%%%%%%%%%%%%%%%%%%%
%%%%%%%%%%%%%%%%%%%%%%%%%%%%%%%%%%%%%%%%%%%%%%%%%%%%%%%%%%%%%%%%%%%%%%%%%%%%%%
%%%%%%%%%%%%%%%%%%%%%%%%%%%%%%%%%%%%%%%%%%%%%%%%%%%%%%%%%%%%%%%%%%%%%%%%%%%%%%

\begin{Exercice}[4 points]
Résoudre les équations suivantes:
\begin{enumerate}
\item \pts{2.6}{1 pt} $x+5=16$
\item \pts{2.6}{1.5 pts} $7x-6=2x+2$
\item \pts{2.6}{1.5 pts} $2(5x-1)-(6x-4)=0$
\end{enumerate}
\end{Exercice}



\hfill\textbf{\huge\textcolor{red}{ $\mathcal{BON}$ \ $\mathcal{COURAGE}$}}


\end{document}