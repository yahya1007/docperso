\documentclass[a4paper, 12pt]{report}
\usepackage{etex}
\usepackage[top=0.5cm, bottom=1.5cm, left=2cm, right=1cm]{geometry}
\usepackage{tikz}
\usepackage{etoolbox}
\usepackage{lastpage} 

\usepackage{pstricks-add}
\pagestyle{empty}

\usepackage{amsmath,amssymb}
\usepackage{float}
\usepackage[multidot]{grffile}
\usepackage{graphicx}
\usepackage{soul}
\usepackage{ulem}
\usepackage{amsthm}
\usepackage{mathptmx}
\usepackage{fancyhdr}
\usepackage{color,colortbl}
\usepackage{array}
\usepackage{tabularx}
\usepackage{fancybox}
\usepackage{pstricks,pst-plot,pst-text,pst-tree,pst-eps,pst-fill,pst-node,pst-math}
\usepackage{pifont}
\usepackage{xargs}
\usepackage{framed}
\usepackage[tikz]{bclogo}
\usepackage{multirow}
\usepackage[most]{tcolorbox}
\renewcommand{\headrulewidth}{0pt}
\usepackage{pgf,tikz,pgfplots}
\pgfplotsset{compat=1.15}
\usepackage{mathrsfs}
\usetikzlibrary{arrows}
\usepackage{multicol}
\pagestyle{fancy}
\usepackage{tcolorbox}
\usepackage{amsmath,amsfonts,amssymb}
\usetikzlibrary{calc}
\usepackage[object=vectorian]{pgfornament}
\tcbuselibrary{skins,xparse,hooks,vignette}
\usepackage{tikzrput}
\usepackage[tikz]{bclogo}
\usepackage{varwidth}
%\usepackage[]{anttor}
%\usepackage[]{kurier}
%\usepackage{arev}
%\usepackage{avant}
%\usepackage{bera}
%\usepackage{bookman}
%\usepackage{chancery}

%\usepackage{charter}
%\usepackage{cmbright}
%\usepackage{concrete}
%\usepackage{courier}
%\usepackage{euler}
%\usepackage{fourier}
%\usepackage{fouriernc}
%\usepackage{helvet}
%\usepackage{inconsolata}
%\usepackage{kpfonts}
%\usepackage{lato}
%\usepackage{mathpazo}
%\usepackage{lmodern}
%\usepackage{tgpagella}
%\usepackage{tgtermes}
%\usepackage{yfonts}
 \newcommand{\R}{\mathbb{R}}
\newcommand{\N}{\mathbb{N}}
%<<<<<<<WARNING>>>>>>>
% PGF/Tikz doesn't support hatch filling very well
% Use PStricks for a perfect hatching export

\usetikzlibrary[patterns]

\makeatother

   \usepackage{enumitem}
\usetikzlibrary{shapes.geometric} 
\usetikzlibrary{shapes.symbols}
\newcommand{\circled}[1]{\tikz[baseline=-6pt] 
 \node[circle,draw=red,inner sep=1.5pt]{#1};} 
 \newcommand{\diamonde}[1]{\tikz[baseline=-6pt] \node[shape=diamond,draw=black,shade,ball color=black,inner sep=2pt]{#1};} \newcommand{\polygone}[1]{\tikz[baseline=-6pt]
 \node[shape=regular polygon,regular polygon sides=8,draw=blue,inner sep=2pt,shade,ball color=red]{#1};} \newcommand{\signale}[1]{\tikz[baseline=-6pt] \node[signal,signal to =right,draw=blue,inner sep=2pt]{#1};} 
\newcommand{\signalle}[1]{\tikz[baseline=-6pt] \node[signal,signal from =west,signal to =east,draw=blue,inner sep=2pt]{#1};}
 \definecolor{uclagold}{rgb}{1.0, 0.7, 0.0}
\definecolor{gold}{rgb}{1.0, 0.84, 0.0}
\parindent=0pt
\newcommand{\be}{\begin{enumerate}}
	\newcommand{\ee}{\end{enumerate}}
\newcommand{\vect}{\overrightarrow}
\renewcommand{\footrulewidth}{1pt}
\renewcommand{\headrulewidth}{1pt}
\newcommand{\textreflect}[3]
{%
 \parbox[c]{5ex}
{
\begin{tikzpicture}[text=#2]
 \node[at={(0,0)},yshift=1ex,yslant=#3]{%
 \LARGE \bfseries #1};
 \node[above,at={(0,0)},yslant=#3,yscale=-1,
 scope fading=south,
 opacity=0.5]{\LARGE \bfseries #1};
 \end{tikzpicture}
 }%
 }
  \setlist[enumerate,1]
  {
    itemsep=1ex, 
    leftmargin=5ex, 
    label=\fcolorbox{black}{blue!20}{\textcolor{black}{\large\bfseries$\arabic*$}}
  }
%%%
\usetikzlibrary{shapes.symbols}
\usetikzlibrary{decorations.pathmorphing}
\tcbuselibrary{skins}

%%%%%%%%%%%

\usepackage{eso-pic}


\newtcolorbox{myboxe2}[2][]{enhanced,breakable,
	before skip=2mm,after skip=2mm,
	colback=gray!20,colframe=black!50,boxrule=0.2mm,
	attach boxed title to top left={xshift=1cm,yshift*=1mm-\tcboxedtitleheight},
	varwidth boxed title*=-3cm,
	boxed title style={frame code={
			\path[fill=tcbcolback!30!black]
			([yshift=-1mm,xshift=-1mm]frame.north west)
			arc[start angle=0,end angle=180,radius=1mm]
			([yshift=-1mm,xshift=1mm]frame.north east)
			arc[start angle=180,end angle=0,radius=1mm];
			\path[left color=tcbcolback!60!black,right color=tcbcolback!60!black,
			middle color=tcbcolback!80!black]
			([xshift=-2mm]frame.north west) -- ([xshift=2mm]frame.north east)
			[rounded corners=1mm]-- ([xshift=1mm,yshift=-1mm]frame.north east)
			-- (frame.south east) -- (frame.south west)
			-- ([xshift=-1mm,yshift=-1mm]frame.north west)
			[sharp corners]-- cycle;
		},interior engine=empty,
	},
	fonttitle=\bfseries,
	title={#2},#1}

\cfoot{\thepage/\pageref{LastPage}}
\lfoot{{\LARGE\ding{45}}
\textcolor{black}{Pr:S.BASSY}}
\rfoot{{\LARGE\ding{45}}
\textcolor{black}{
{\bf 2020~ \copyright~ Easy maths}}}
\parindent=0pt
\def\dg#1{%
	\begin{tikzpicture}[remember picture,overlay]
	\node [anchor=#1,rectangle,fill=gray!30,minimum width=1.5cm,minimum height=\paperheight,align=center]at(current page.#1){\rotatebox{90}{%
		Pr: S.BASSY  \hspace*{2cm} 	lycée technique IBN ALHAITAM \hspace*{2cm} 2BAS SC EXP  \hspace*{2cm} Année scolaire: 2019-2020
	}};
	\end{tikzpicture}
	
}
\AddToShipoutPicture{%
	\ifodd \thepage
	\dg{west}
	\else
	\dg{west}\fi
}

\begin{document}

\begin{myboxe2}[colbacktitle=gray]{Exercice $1$}
		
			\begin{enumerate}
		
		\item Montrer que 
		\begin{enumerate}[label=\protect\circled{\alph*}]
			
			
			\begin{multicols}{2}
				
				\item text
				\item text
				\item text
				
				\item text
				
				
			\end{multicols}
			
		\end{enumerate}
		\item text
		\begin{enumerate}
			[label=\protect\circled{\alph*}]
			
			\begin{multicols}{3}
				
					\item text
				\item text
				\item text
				
				\item text
					\item text
				\item text
				
			\end{multicols}
			
		\end{enumerate}
	\end{enumerate}

	\end{myboxe2}%%%%%%%%%%%%%%%%%%%%%%%%%%%%%%%%%%%%%%%%%%%%%%%%%%%%%%%%ù	
				\vspace*{0.1cm}
			\begin{myboxe2}[colbacktitle=gray]{Exercice $2$}	
				
			On 
		\begin{enumerate}
		
			\item \begin{enumerate}		[label=\protect\circled{\alph*}]
				\item 	 Qst1
				\item  Qst$2$	
			
		\end{enumerate}   
		\item \begin{enumerate}		[label=\protect\circled{\alph*}]
		\item 	 Qst1
		\item  Qst$2$	
		
	\end{enumerate}   
			
		\end{enumerate}
		
		
	\end{myboxe2}	
	
%%%%%%%%%%%%%%%
\vspace*{0.1cm}
\begin{myboxe2}[colbacktitle=red]{Exercice $3$}	
	
	On 
	\begin{enumerate}
		
		\item \begin{enumerate}		[label=\protect\circled{\alph*}]
			\item 	 Qst1
			\item  Qst$2$	
			
		\end{enumerate}   
		\item \begin{enumerate}		[label=\protect\circled{\alph*}]
			\item 	 Qst1
			\item  Qst$2$	
			
		\end{enumerate}   
		
	\end{enumerate}
	
	
\end{myboxe2}
%%%%%%%%%%%%%%%%%%%%%%%%%%%%%%%%%%%%%%%%%%%%%%%%%%%%%%%
\vspace*{0.1cm}
\begin{myboxe2}[colbacktitle=blue]{Exercice $4$}	
	
	On 
	\begin{enumerate}
		
		\item \begin{enumerate}		[label=\protect\circled{\alph*}]
			\item 	 Qst1
			\item  Qst$2$	
			
		\end{enumerate}   
		\item \begin{enumerate}		[label=\protect\circled{\alph*}]
			\item 	 Qst1
			\item  Qst$2$	
			
		\end{enumerate}   
		
	\end{enumerate}
	
	
\end{myboxe2}
%%%%%%%%%%%%%%%%%%%%%%%%%%%%%%%%%%%%%%%%%%%%%%ù
\vspace*{0.1cm}
\begin{myboxe2}[colbacktitle=black]{Exercice $5$}	
	
	On 
	\begin{enumerate}
		
		\item \begin{enumerate}		[label=\protect\circled{\alph*}]
			\item 	 Qst1
			\item  Qst$2$	
			
		\end{enumerate}   
		\item \begin{enumerate}		[label=\protect\circled{\alph*}]
			\item 	 Qst1
			\item  Qst$2$	
			
		\end{enumerate}   
		
	\end{enumerate}
	
	
\end{myboxe2}

\end{document}
