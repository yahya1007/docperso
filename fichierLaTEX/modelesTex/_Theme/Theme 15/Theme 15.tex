\documentclass[12pt,a4paper,fleqn]{article}

\usepackage[utf8]{inputenc}
\usepackage{amsmath,amssymb,amsfonts,array}
\usepackage[left=1cm,right=1cm,top=.5cm,bottom=1cm]{geometry}
\everymath{\displaystyle}
\usepackage[many]{tcolorbox}


\usetikzlibrary{calc}




\newtcolorbox[auto counter ]{Exercice}[1][]{
breakable,
enhanced,
arc=2pt,
top=3mm,
left=3mm,
colframe=black,
colback=gray!20,
attach boxed title to top left={xshift=2mm,yshift=-3mm},
boxed title style={
colback=cyan!25!white,
outer arc=2pt,
arc=2pt,
top=1pt,
bottom=1pt,
left=1mm,
right=2mm
},
coltitle=black,
fonttitle=\sffamily,
title=Exercice~\thetcbcounter
}
\newtcolorbox[]{entete}[1][]{
breakable,
enhanced,
arc=2pt,
top=3mm,
left=3mm,
colframe=black,
colback=blue!20
}
\begin{document}

\begin{entete}
 \begin{tabular}{m{4.5cm}m{7.2cm}m{5cm}}
  1$^\text{eme}$ année Bac\newline
  Prof : Mohamed .. & Série  N:1 \newline
  \centering Semestre 1 & Collège   \newline
  Année scolaire :2019-2020
 \end{tabular}
\end{entete}

\begin{Exercice}
 Soit $x$ un nombre reel 
\end{Exercice}

\begin{Exercice}
 contenu...
\end{Exercice}

\begin{Exercice}
 contenu...
\end{Exercice}

\begin{Exercice}
 contenu...
\end{Exercice}

\begin{Exercice}
 contenu...
\end{Exercice}

\end{document}