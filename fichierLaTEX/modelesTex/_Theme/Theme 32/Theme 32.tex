\documentclass[a4paper,10pt]{article}
\usepackage{array}
\usepackage{polyglossia}
\usepackage{geometry}
\usepackage{enumerate}
\usepackage{enumitem}
\usepackage{amsmath,amssymb}
\usepackage{pifont}
\usepackage{fancybox}
\usepackage{fancyhdr}
\usepackage{multicol}
\usepackage{mathrsfs} % pour \mathscr
\usepackage{tcolorbox,varwidth}
\usepackage{fontspec}
\tcbuselibrary{skins,breakable}
\usepackage{pgf,tikz,pgfplots}
\setmainfont[Scale=1.1]{Times New Roman}
\usetikzlibrary{arrows}
\pgfplotsset{compat=1.15}
\pagestyle{empty}
\usepackage{pgfornament}
\usepackage{tasks}
\usepackage{psvectorian}
\usepackage{lastpage}
\usepackage{eso-pic,graphicx}
\newcommand{\N}{\mathbb{N}}
\newcommand{\Z}{\mathbb{Z}}
\newcommand{\R}{\mathbb{R}}
\newcommand{\Cc}{\mathbb{C}}
\newcounter{numexo}
\setcounter{numexo}{1}
\newcommand{\numexo}{Exercice \thenumexo \addtocounter{numexo}{1} }
\geometry{
paperwidth=21cm,
left=1cm,
right=1cm,
paperheight=29.7cm,
top=.5cm,
height=24.5cm,
headheight=12.0pt,
headsep=1cm,
footskip=.5cm,
bottom=1cm
}
\rfoot{\textit{}} %Bonne chance
\cfoot{\textit{\underline{Page ~ \thepage ~ sur ~ \pageref{LastPage}}}}
\renewcommand{\headrulewidth}{0.1mm}
\renewcommand{\footrulewidth}{0.1mm}
\renewcommand{\baselinestretch}{1}
\pagestyle{fancy}
%================= New box ======================
\usepackage{varwidth}
\newtcolorbox{mynewbox}[2][]{skin=enhancedlast jigsaw,breakable,interior hidden,
boxsep=4pt,top=0pt,colframe=violet,coltitle=violet!50!black,
fonttitle=\bfseries\sffamily,
attach boxed title to top center,
boxed title style={empty,boxrule=0.5mm},
varwidth boxed title=0.5\linewidth,
underlay boxed title={
\draw[white,line width=0.5mm]
([xshift=0.3mm-\tcboxedtitleheight*2,yshift=0.3mm]title.north west)
--([xshift=-0.3mm+\tcboxedtitleheight*2,yshift=0.3mm]title.north east);
\path[draw=violet,top color=white,bottom color=violet!50!white,line width=0.5mm]
([xshift=0.25mm-\tcboxedtitleheight*2,yshift=0.25mm]title.north west)
cos +(\tcboxedtitleheight,-\tcboxedtitleheight/2)
sin +(\tcboxedtitleheight,-\tcboxedtitleheight/2)
-- ([xshift=0.25mm,yshift=0.25mm]title.south west)
-- ([yshift=0.25mm]title.south east)
cos +(\tcboxedtitleheight,\tcboxedtitleheight/2)
sin +(\tcboxedtitleheight,\tcboxedtitleheight/2); },
title={#2},#1}
%================================================
\begin{document}
\begin{center}
{\Large\textit{\textcolor{violet}{1er BAC Sciences Expérimentales - Fr}}} \vspace*{0.5cm} \\
{\Large \bf Série : Limite d'une fonction numérique } \\
\end{center}
\vspace*{-0.7cm}
\begin{center}
\rule{0.75\linewidth}{1pt}
\end{center}
\begin{multicols}{2}
\begin{mynewbox}{\numexo}
Calculer les limites suivantes:
\begin{tasks}(2)
\task $\lim\limits_{x \rightarrow 0} x^{2}-4 x+1$
\task $\lim\limits_{x \rightarrow-1} 3 x^{4}-x^{2}+3$
\task $\lim\limits_{x \rightarrow+\infty} 3 x^{2}-x+3$
\task $\lim\limits_{x \rightarrow-\infty} x^{2}-x+3$
\end{tasks}
\end{mynewbox}
\begin{mynewbox}{\numexo}
Calculer les limites suivantes:
\begin{tasks}(2)
\task $\lim\limits_{x \rightarrow+\infty} \cfrac{x^{2}-x}{x^{2}-4 x+1}$
\task $\lim\limits_{x \rightarrow-\infty} \cfrac{x^{2}-4 x}{1+x}$
\task $\lim\limits_{x \rightarrow+\infty} \cfrac{-x^{2}+3}{3 x^{4}+x+2}$
\task $\lim\limits_{x \rightarrow+\infty} 3 x^{4}-\cfrac{x^{2}+3}{1+x}$
\end{tasks}
\end{mynewbox}
\begin{mynewbox}{\numexo}
Calculer les limites suivantes:
\begin{tasks}(2)
\task $\lim\limits_{x \rightarrow 2} \cfrac{x^{2}-x}{-2 x+4}$
\task $\lim\limits_{x \rightarrow 0} \cfrac{x^{2}+5 x-1}{x^{2}}$
\task $\lim\limits_{x \rightarrow-2} \cfrac{x^{3}+5 x}{x^{2}+4 x+4}$
\task $\lim\limits_{x \rightarrow 2^{-}} \cfrac{-x^{2}+x}{x-2}$
\task $\lim\limits_{x \rightarrow 1^{+}} \cfrac{x^{2}+x}{x-1}$
\task $\lim\limits_{x \rightarrow 1^{+}} \cfrac{1+x^{2}}{x^{2}-3 x+2}$
\end{tasks}
\end{mynewbox}
\begin{mynewbox}{\numexo}
Calculer les limites suivantes:
\begin{tasks}(2)
\task $\lim\limits_{x \rightarrow 1} \cfrac{x^{2}-x}{x^{2}-1}$
\task $\lim\limits_{x \rightarrow 1} \cfrac{1-x^{2}}{x^{2}-5 x+4}$
\task $\lim\limits_{x \rightarrow-2^{+}} \cfrac{4-x^{2}}{x^{2}+4 x+4}$
\task $\lim\limits_{x \rightarrow 1} \cfrac{x^{2}+x-2}{x-1}$
\end{tasks}
\end{mynewbox}
\begin{mynewbox}{\numexo}
Calculer les limites suivantes:
\begin{tasks}(2)
\task $\lim\limits_{x \rightarrow-\infty} \sqrt{x^{2}+2 x-5}$
\task $\lim\limits_{x \rightarrow-\infty} \sqrt{4 x^{2}-3}-2 x$
\task $\lim\limits_{x \rightarrow+\infty} \sqrt{x^{2}+3}-x$
\task $\lim\limits_{x \rightarrow+\infty} \cfrac{\sqrt{x^{2}-4}}{x-2}$
\task $\lim\limits_{x \rightarrow-\infty} \cfrac{x^{2}-x \sqrt{x}}{x^{2}-1}$
\task $\lim\limits_{x \rightarrow+\infty} \cfrac{x}{\sqrt{x^{2}+1}-1}$
\end{tasks}
\end{mynewbox}
\begin{mynewbox}{\numexo}
Calculer les limites suivantes:
\begin{tasks}(2)
\task $\lim\limits_{x \rightarrow 1}\left(\frac{\sqrt{2 x+1}-3}{x-4}\right)$
\task $\lim\limits_{x \rightarrow 1}\left(\frac{x-1}{\sqrt{x}-1}\right)$
\task $\lim\limits_{x \rightarrow 0}\left(\frac{\sqrt{4-x}-\sqrt{4+x}}{x}\right)$
\task $\lim\limits_{x \rightarrow 1^{+}} \frac{\sqrt{2+x}}{x^{2}-x-2}$
\end{tasks}
\end{mynewbox}
\begin{mynewbox}{\numexo}
Calculer les limites suivantes:
\begin{tasks}(2)
\task $ \lim\limits_{x \rightarrow 0} \cfrac{\sin 3 x}{x}$
\task $ \lim\limits_{x \rightarrow 0} \cfrac{1-\cos x}{x}$
\task $ \lim\limits_{x \rightarrow 0} \cfrac{\tan x}{\sin 2 x}$
\task $ \lim\limits_{x \rightarrow 0} \cfrac{\sin x}{2 x}$
\task $ \lim\limits_{x \rightarrow 0} \cfrac{\sin (x)}{\sqrt{x+1}-1}$
\task $ \lim\limits_{x \rightarrow 0} \cfrac{\sin (2 x)}{\sin (4 x)}$
\end{tasks}
\end{mynewbox}
\begin{mynewbox}{\numexo}Calculer les limites suivantes:
\begin{tasks}(2)
\task $\lim\limits_{x \rightarrow+\infty} \cfrac{|x|-5}{|x+5|}$
\task $\lim\limits_{x \rightarrow 1^{-}} \cfrac{x-1}{\left|x^{2}-1\right|}$
\task $\lim\limits_{x \rightarrow+\infty}\left|3 x^{2}-4 x+1\right|$
\task $\lim\limits_{x \rightarrow+\infty}|2 x-1|-3 x$
\end{tasks}
\end{mynewbox}
\begin{mynewbox}{\numexo}
\begin{enumerate}
\item Soit $f$ la fonction définie sur $\mathbb{R}^{*}$ par: $$ f(x)=1+\cfrac{\sin x}{x} $$
Déterminer la limite de la fonction $f$ en $+\infty$.
\item Montrer que pour tout réel strictement positif: $$ \cfrac{x-1}{x+1} \leq \cfrac{x-\sin x}{x+1} \leq 1 $$
En déduire la limite de la fonction $u$ définie sur $\mathbb{R}_{+}^{*}$ par: $\cfrac{x-\sin x}{x+1}$
\end{enumerate}
\end{mynewbox}
\begin{mynewbox}{\numexo}
Soit $f$ la fonction définie sur $I =] 0 ; 3\left[\right.$ par $$ f(x)=\cfrac{3 x-1}{x^{2}-3 x} $$
\begin{enumerate}
\item Étudier le signe de $f$ sur l'intervalle $\boldsymbol{I}$.
\item Calculer les limites de $f$ aux bornes de $\boldsymbol{I}$.
\end{enumerate}
\end{mynewbox}
\begin{mynewbox}{\numexo}
Soit $\boldsymbol{u}$ une fonction définie sur $ ] 0 ;+\infty[$ telle que pour tout $x \in[1 ;+\infty[, \quad 0 \leq u(x) \leq x .$\\
Et soit $f$ la fonction définie sur $] 0 ;+\infty[$ par :
$$ f(x)=1+\cfrac{u(x)}{x^{2}} $$
Montrer que si $x \geq 1, |f(x)-1| \leq \cfrac{1}{x}$.\\
En déduire $\lim\limits_{x \rightarrow+\infty} f(x)$
\end{mynewbox}
\end{multicols}
\end{document}
