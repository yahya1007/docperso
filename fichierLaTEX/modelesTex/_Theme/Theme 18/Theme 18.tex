\documentclass{book}
\usepackage[utf8]{inputenc}
\usepackage[left=1cm,right=1cm,top=.5cm,bottom=1cm]{geometry}
\usepackage[french]{babel}
\usepackage{xcolor}
\usepackage[explicit]{titlesec}
\titleformat{\chapter}[display]
{\bfseries\Large}
{\filright\MakeUppercase{\chaptertitlename} \Huge\thechapter}
{1ex}
{\titlerule\vspace{1ex}\filcenter {\Large #1}}
[\vspace{1ex}\titlerule]
 \titlespacing{\chapter}{0pt}{-5ex}{10ex}
\titleformat{\section}{\normalfont\large\bfseries}{\color{red!55!yellow}{\thesection}}{0.6em}{#1:}
\titleformat{\subsection}{\normalfont\large\bfseries}{\color{red!55!yellow}{\thesubsection}}{0.6em}{#1:}
%%%%%%%%%%%%%%%%%%%%%%%%%%%%%%%%%%%%%%%%%%%%%%%%%%
\usepackage[most]{tcolorbox}
\tcbuselibrary{listings,theorems}
\newtcbtheorem[number within=section]{theoreme}{Théorème}
{enhanced,
 lower separated=false,
 sharp corners,
 before upper={\parindent28mm},
 attach boxed title to top left={
 xshift=1mm,
 yshift=-5mm,
 yshifttext=5mm
 },
 top=1.5em,
 coltitle=red!60!yellow,
 colback=gray!10,
 colframe=red!60!yellow, 
 rightrule=0pt,
 titlerule=0pt, 
 toprule=0pt,
 colbacktitle=gray!10,
 fonttitle=\bfseries,
 boxed title style={
 sharp corners=all,
 leftrule=0pt,
 rightrule=0pt,
 bottomrule=0pt, 
 toprule=0pt,
 pad after break=0pt
} }{th}
\begin{document}
\chapter{Distributions}
\section{Introduction}
Il y a plusieurs raisons pour introduire la notion de distribution. 
Certaines sont d'ordre purement physique (exp expérimental même ) 
alors que d’autres sont des raisons plus mathématiques.
\subsection{Unité pour la convolution} 
\begin{theoreme}{}{}Comme nous l'avons remarqué lors de l'étude de la convolution des fonctions intégrables, il n'y a pas de fonction intégrable 
$\Delta$ qui puisse servir d'unité pour le produit de convolution de ces fonctions. Dans l'espace des distributions, il y a effectivement une telle distribution unité pour le produit de convolution : c'est la distribution de Dirac, qui satisfait à
$$\delta\times T=T\times\delta=T$$
\end{theoreme}
\end{document}