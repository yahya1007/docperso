\documentclass[12pt,a4paper]{article}
\usepackage[utf8]{inputenc}
\usepackage{amsmath,amssymb,amsfonts,tikz}
\usepackage[left=2cm,right=1cm,top=1.5cm,bottom=2.5cm]{geometry}
\usepackage{varwidth}
\definecolor{col1}{RGB}{72, 114, 198}
\definecolor{col2}{RGB}{32, 58, 95 }
\definecolor{col3}{RGB}{45, 85, 155}
\everymath{\displaystyle}
\pagestyle{empty}
\usepackage{pifont,fourier}
\usepackage[most,breakable]{tcolorbox}
\usetikzlibrary{shapes.arrows,calc}
\newtcolorbox{mybox}[1][]{%
 enhanced,interior code app={\fill[line width=1pt,col1] ( [yshift=-1cm]frame.north east)to[out=140,in=60]( [xshift=-2cm]frame.south east)--++(2cm,0)--cycle;},width=15.7cm,arc=0pt,outer arc=0pt ,colback=white,colframe=blue,
 frame code app={\fill[line width=0.2pt,col1]( frame.north west)rectangle (frame.south east);}
}
\linespread{1.3} 
%%%%%%%%%%%%%%%%%%%%%%%%%%
\usepackage{eso-pic}
\AddToShipoutPictureBG{\ifnum\value{page}=1\begin{tikzpicture}[remember picture,overlay]
\fill[col1]([yshift=1cm]current page.south west)rectangle(current page.south east) ;
\node[circle,draw=white,line width=2pt,minimum size=0.3cm,fill=col1]at([yshift=0.9cm]current page.south){{\thepage}};
\node[anchor=north west,rectangle,fill=col1!85,rotate=90,inner sep=0.3cm,](A)at([yshift=1.5cm]current page.south west){{\large \textbf{\color{white}Mathématique}}};
\node[anchor=west]at([yshift=.5cm]current page.south west){{ \textbf{\color{white}Lycée secondaire }}};
\node[anchor=east]at([yshift=.5cm]current page.south east){{ \textbf{\color{white}Cette fiche contient 5 exercices}}};
\end{tikzpicture}\else
\begin{tikzpicture}[remember picture,overlay]
\fill[col1]([yshift=-0.5cm]current page.north west)rectangle(current page.north east) ;
\fill[col1]([yshift=1cm]current page.south west)rectangle(current page.south east) ;
\node[circle,draw=white,line width=2pt,minimum size=0.3cm,fill=col1]at([yshift=0.9cm]current page.south){{\thepage}};
\node[anchor=north west,rectangle,fill=col1!85,rotate=90,inner sep=0.3cm,](A)at([yshift=1.5cm]current page.south west){{\large \textbf{\color{white}Mathématique}}};
\node[anchor=north east,rectangle,fill=col1!85,rotate=90,inner sep=0.3cm,](A)at([yshift=-0.5cm]current page.north west){{\large \textbf{\color{white}Les nombres complexes}}};
\node[anchor=west]at([yshift=.5cm]current page.south west){{ \textbf{\color{white}Lycée secondaire }}};
\node[anchor=east]at([yshift=.5cm]current page.south east){{ \textbf{\color{white}Cette fiche contient 5 exercices}}};
\end{tikzpicture}
\fi }
%%%%%%%%%%%%%%%%%%%%%%%%%%%%%%%%%
\newtcolorbox[auto counter ]{Exercice}[2][N.supplémentaire]{
 enhanced,top=0.5cm,bottom=0.5cm,
 overlay app={\draw[col1,line width=0.5cm] (frame.south west)--(frame.north west);
 \node[rectangle,fill=violet,inner xsep=0.3cm, anchor=south east] at([xshift=-0.3cm]frame.north east){\textbf{\color{white}#1}};},
   colback=white,colbacktitle=white,breakable,after=\vskip1cm,
 fonttitle=\bfseries,coltitle=col1,colframe=col1,
 attach boxed title to top left={xshift=0.5cm,yshift=-0.25mm-\tcboxedtitleheight/2,yshifttext=2mm-\tcboxedtitleheight/2},
 title={\textbf{Exercice n$^{\circ}$\thetcbcounter :}}}
%%%%%%%%%%%%%%%%%%%%%%%%%%%%%%%%%%
\newcommand*\cir[1]{\tikz[baseline=(char.base)]{%
 \node[shape=circle,right color=col1, left color=red,draw=col1,minimum size=0.6cm,inner sep=2pt] (char) {\color{white}#1};}}
\usepackage{enumitem}
\begin{document} 
 \begin{tikzpicture}[remember picture,overlay]
\fill[col1]([xshift=2cm,yshift=-0.5cm]current page.north west)rectangle([xshift=-1cm,yshift=-3cm]current page.north east);
\fill[col2]([xshift=-1cm,yshift=-3cm]current page.north east)--++(-0.5,0)--++(0,-0.5)--cycle;
\fill[col2]([xshift=2cm,yshift=-3cm]current page.north west)--++(0.5,-0.5)coordinate(a)--++(-1.5,-0.5)coordinate(b)--++(2,-1.5)--++(0,2.5)--cycle;
\fill[col3](b)--(a)--++(-0.5,0.5)--++(0,2)--++(-1,-0.5)--cycle;
\node[white]at([yshift=-1.75cm]current page.north){{\huge \textbf{Série d'exercice n$^\circ $ 1}}};
\node[white]at([yshift=-1.75cm]current page.north){{\huge \textbf{Série d'exercice n$^\circ $ 1}}};
\node[anchor=north west,align=flush left,rectangle,fill=col1,inner sep=0.3cm,minimum height=1cm,text width=10.95cm ](B)at([xshift=-3cm,yshift=-6.66cm]current page.north){{\Large \textbf{\color{white} Les nombres complexes}}};
\node[anchor=north west](A)at([xshift=3.2cm,yshift=-3.2cm]current page.north west){
 \begin{mybox}
 \vskip1mm
\begin{minipage}{8cm}
\begin{itemize}
\renewcommand\labelitemi{ 
 \begin{tikzpicture}
 \node[minimum width=0.3cm,minimum height=0.3cm](a) {\qquad };
 \fill[red]([yshift=-0.2cm]a.north)--([yshift=-0.2cm]a.west)--([yshift=-0.2cm]a.south)--([yshift=-0.2cm]a.east)--cycle;
 \fill[col1]([xshift=0.1cm,yshift=-0.2cm]a.north)--([xshift=0.1cm,yshift=-0.2cm]a.west)--([xshift=0.1cm,yshift=-0.2cm]a.south)--([xshift=0.1cm,yshift=-0.2cm]a.east)--cycle;
 \end{tikzpicture}
}
\item  \textbf{{\large \color{violet}Prof : ...}}
\item  {\large \textbf{\color{violet}Classe : ...}}
\item \textbf{{\large \color{violet} Année scolaire : ...}}
\end{itemize}
\end{minipage}
\end{mybox}
};
 \end{tikzpicture}
%%%%%%%%%%%%%%%%%%%%
\vskip7cm
\begin{Exercice}{}
 math math math math math math math math math math math math math math math math math math math math math 
 \begin{enumerate}[label=\protect\cir{\arabic*}]
 \item math math math math math math math math math 
 \item math math math math math math math math math 
 \begin{enumerate}[label=\protect\cir{\alph*}]
 \item math math math math math math math math math  
 \item math math math math math math math math math 
 \end{enumerate}
 \item math math math math math math math math math 
 \item math math math math math math math math math 
 \item math math math math math math math math math 
 \item math math math math math math math math math  
 \end{enumerate}
\end{Exercice}
\begin{Exercice}
math math math math math math math math math math math math math math math math math math math math math 
\begin{enumerate}[label=\protect\cir{\arabic*}]
 \item math math math math math math math math math 
 \item math math math math math math math math math 
 \item math math math math math math math math math 
 \item math math math math math math math math math 
 \item math math math math math math math math math 
 \item math math math math math math math math math 
 \item math math math math math math math math math 
 \item math math math math math math math math math 
 \item math math math math math math math math math 
\end{enumerate}
\end{Exercice}
\end{document}