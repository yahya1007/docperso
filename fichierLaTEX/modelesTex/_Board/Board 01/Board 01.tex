\documentclass[12pt,a4paper,fleqn]{article}
\usepackage[utf8]{inputenc}
\usepackage[most]{tcolorbox}
\usepackage{lipsum,tikz,amssymb,amsmath,xcolor}
\everymath{\displaystyle}
\usetikzlibrary{through,intersections,calc}  
\definecolor{col}{RGB}{183, 35, 35}
\newtcolorbox{mybox}[1][]{enhanced, 
  colframe=red,colback=white,colbacktitle=blue!5!yellow!50!white,
  colback=white,colframe=col,
frame code app={\fill[rounded corners,col] ([yshift=-1cm]frame.north west)rectangle([yshift=1cm]frame.north east);
\foreach \i in {1,4.75,8.5,12.25}
\fill[rounded corners,col!55,draw=red!44] ([xshift=\i cm,yshift=0.6cm]frame.north west)rectangle++(0.25,1.2);
} }
\begin{document}
\begin{mybox}
\textcolor{blue}{\textbf{\underline{Théorème de Stolz :}}}\vskip3mm
Soit $(x_n)$ et $(y_n)$ deux suites vérifient les propriétés suivantes :
\begin{enumerate}
 \item $(y_n)$ est strictement croissante et $ \lim_{n\to +\infty}y_n=+\infty $ . 
 \item $ \lim_{n\to +\infty} \dfrac{x_{n+1}-x_n}{y_{n+1}-y_n}=\ell$ avec $\ell \in \mathbb R$ .
\end{enumerate}
Alors  {$$ \lim_{n\to +\infty} \dfrac{ x_n}{ y_n}=\ell$$ .}
\end{mybox}
\end{document}