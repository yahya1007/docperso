\documentclass[a4paper,12pt]{article}

\usepackage{dlds}

\begin{document}

\devoir[prv=true,ds=true,num=2 ,niv=2 , date=19/12/2022]

\begin{exo}[12]
\begin{enumerate}
\item Calculer : 
\[
 -22^{0}=\cdotsx{6} \quad; \quad (-1)^{-11}=\cdotsx{6} \quad; \quad 1^{89}=\cdotsx{6} \quad; \quad (\dfrac{7}{5})^{-3}=\cdotsx{6}
\]
\item Déterminer le signe des puissances suivantes :
\[ (\dfrac{-1}{-13})^{-11}=\cdotsx{20} \quad; \quad 
	(-71)^{2022}=\cdotsx{20}
\] 
\item Écrire sous forme $a^{n}$ les expressions suivantes :

$\dfrac{17^{-6}}{17^{-25}}=\cdotsx{6}\quad; \quad
	\dfrac{5^{5}\times (-5)^{20}}{25^{9}}=$\anserline[1]
$\dfrac{10^{-5}\times 10^{-8}\times 10^{3}}{2^{6}\times 2^{7}\times 2^{-3}}=$\anserline[1]
$\dfrac{a^{-15}\times a^{12}\times (-a)^{-18}}{a^{-6}}=$\anserline[1]

\item Donner l'écriture scientifique des nombres suivants :

$A=0.0000005\times 0.00000004=$\anserline[1]	
$B=\dfrac{480\times 10^{-93}}{6\times 10^{-97}}=$\anserline[1]
\end{enumerate}
\end{exo}

\begin{exo}[8]
$ABC$ est un triangle tel que : $AB=6$ et $\widehat{BAC}=100^{\circ}$ et $\widehat{ABC}=30^{\circ}$.

Soit $M$ le milieu du segment $[BC]$ 
\begin{enumerate}
\item Faire un schéma.
\item Construire $E$ et $F$ les symétriques respectives   de $B$ et $C$ par rapport à la droite $(AM)$.
\item Montrer que $AE=6$.
\item Quel est la mesure de l'angle $\widehat{EAF}$? Justifie ta réponse.
\end{enumerate}
\end{exo}
\anspage{1}


\end{document}



\end{document}