\documentclass[a4paper,12pt]{article}

\usepackage{dlds}
 
\begin{document}

\devoir[ds=true,num=3 ,niv=2 , date=12/01/2023,prv=true ]

\begin{exo}[6]
\begin{tikzpicture}
\tkzDefPoints{0/0/A,9/0/B,3/6/C}
\tkzDrawSegments(A,B A,C C,B)
\tkzDefTriangleCenter[circum](A,B,C)\tkzGetPoint{O}
\tkzDefTriangleCenter[in](A,B,C)\tkzGetPoint{I}
\tkzDefTriangleCenter[ortho](A,B,C)\tkzGetPoint{H}
\tkzDefTriangleCenter[centroid](A,B,C)\tkzGetPoint{G}
\tkzDrawPoints(O,I,G,H)
\tkzLabelPoints(O,I,G,H,A,B)
\tkzLabelPoints[above](C)
\tkzInterLL(A,B)(C,H)\tkzGetPoint{H_1}
\tkzInterLL(A,C)(B,H)\tkzGetPoint{H_2}
\tkzDrawLines(C,H_1 B,H_2)
\tkzMarkRightAngles(B,H_1,C B,H_2,C)

\tkzInterLL(A,G)(C,B)\tkzGetPoint{G_1}
\tkzInterLL(C,G)(A,B)\tkzGetPoint{G_2}
\tkzDrawLines[dashed](A,G_1 C,G_2)

\tkzDefPointBy[projection=onto B--C](O)\tkzGetPoint{O_1}
\tkzDrawLine[add=0.5 and 1](O,O_1)
\tkzMarkRightAngle(C,O_1,O)
\tkzLabelPoints[above=6pt, right=6pt](O_1)
\tkzDrawSegment(A,I)
\tkzMarkAngle[arc=ll , size=1 , mark=|](I,A,C) 
\tkzMarkAngle[arc=ll , size=1.2 , mark=|](B,A,I) 

\end{tikzpicture}

Observer bien la figure et répondre aux questions suivantes .
\begin{enumerate}
\item Citer un médiatrice du triangle $ABC$
\anserline[1]
\item Citer un hauteur du triangle $ABC$
\anserline[1]
\item Citer un bissectrice du triangle $ABC$
\anserline[1]
\item Quel est le centre du cercle inscrit dans le triangle $ABC$?
\anserline[1]
\item Quel est le centre du cercle circonscrit au triangle $ABC$ ?
\anserline[1]
\item Quel est l'orthocentre du triangle $ABC$ ?
\anserline[1]
\end{enumerate}
\end{exo}

\begin{exo}[8]
On considère la figure ci-contre tel que $AM=3$\newline et $BM=9$ et $MN=4$ et $AN=2$ et $(MN)//(BC)$ . 
\begin{enumerate}
\begin{minipage}{0.6\linewidth}
\item Calculer $BC$ et $NC$
\anserline[4]
\item Construire le centre de gravité du triangle $ABC$ et $I$ le milieu de $[BC]$.
\end{minipage}
\begin{minipage}{0.4\linewidth}
\begin{tikzpicture}
\tkzDefPoints{0/0/A,4/1/B,1/4/C}
\tkzDrawSegments(A,B B,C C,A)
\tkzDefPointOnLine[pos=.3](A,B)\tkzGetPoint{M}
\tkzDefPointOnLine[pos=.3](A,C)\tkzGetPoint{N}
\tkzDrawSegment(M,N)
\tkzLabelPoints[left](A,N,C)
\tkzLabelPoints[below](B,M)
\end{tikzpicture}
\end{minipage}
\item Sachant que  $AI=6$ calculer $AG$.
\end{enumerate}
\anserline[2]
\end{exo}

\begin{exo}[6]
$ABCD$ est un trapèze de bases $[AB]$ et $[CD]$ . $M$ et $N$ sont respectivement les milieux des cotés $[AD]$ et $[AC]$ 
\begin{enumerate}
\item Montrer que $(MN)//(CD)$\newline
\anserline[3]
\item La droite $(MN)$ coupe la droite $(BC)$ en $K$ ,monter que $K$ est le milieu de $[BC]$\newline
\anserline[3]
\item Calculer la distance $MN$ sachant que $CD=100$\newline
\anserline[3]
\end{enumerate}
\end{exo}
\end{document} 