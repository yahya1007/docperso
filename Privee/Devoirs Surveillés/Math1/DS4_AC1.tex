\documentclass[a4paper,12pt]{article}




\usepackage{dlds}

\setlength{\columnseprule}{1pt}
\setlength{\columnsep}{3em}
\renewcommand{\columnseprulecolor}{\color{gray}}
%\pointsinmargin 
%\pointformat{\color{gray}\footnotesize{(\themarginpoints\,\points)}\hspace{0.5cm}}
%\pointpoints{pt}{pts}

%\usedecimalfrac

\begin{document}
\devoir[sem=2,prv=true,ds=true,num=4 ,niv=1 ,date=02/03/2023]

\begin{exo}[0.75]

\begin{enumerate}


\item Compléter les expressions par qui convient :
\[
 8x = 2x + \cdots\quad ;; \quad
 14xy = 2x \times \cdots\quad ;; \quad
 146 ab = 200 ab - \cdots\quad ;; \quad
 52 x = 12 x + \cdots
\]

\item Réduire ce qui suit :
\begin{multicols}{2}
\(
3x+2+6x-7=\anserline[3]
\)
\columnbreak

\(
5ab-1+3ab-8-14ab=\anserline[3]
\)
\end{multicols}
\vspace{-1cm}
\item Développer et réduire les produits suivants:
\begin{multicols}{2}
\(
-7x(2x+1)=\anserline[2]
\)
\columnbreak

\(
(6x-7)(5x-2)=\anserline[2]
\)
\end{multicols}
\vspace{-1cm}
\item Factoriser :
\begin{multicols}{3}
\(
8x-8y=\anserline[2]
\)
\columnbreak

\(
25ab+5=\anserline[2]
\)
\columnbreak

\(
49x^{2}+56y^{3}=\anserline[2]
\)
\end{multicols}
\end{enumerate}

\end{exo}

\begin{exo}
\begin{enumerate}

\item Résoudre les équations suivantes :
\begin{multicols}{3}
 \[-4-x=1\]
\anserline[4]
\columnbreak

 \[3x-5=6\]
\anserline[4]
\columnbreak

 \[4(3x-1)=2x-3\]
\anserline[4]
\end{multicols}

\end{enumerate}
\end{exo}




\end{document}