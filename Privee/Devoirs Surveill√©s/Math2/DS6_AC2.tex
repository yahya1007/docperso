\documentclass[a4paper,12pt]{article}

\usepackage{dlds}
\usepackage{fig3d}
%
 
\begin{document}

\devoir[sem=2,prv=true,ds=true,num=6 ,niv=2 ,date=29/05/2023]


\begin{exo}
\begin{enumerate}
\begin{minipage}{.55\linewidth}
Le tableau suivant représente le nombre d'enfant par famille.
\item compléter le tableau
\item Quel est le caractère de cette série statistique ?\\
\anserline[2]
\item Quel est l'effectif total de cette série statistique?\\
\anserline[2]
\item Calculer la moyenne de cette série statistique.\\
\anserline[6]
\item Tracer le diagramme en bâtons des effectifs.
\end{minipage}
\begin{minipage}{.45\linewidth}
\begin{tabular}{|Oc|Oc|Oc|Oc|Oc|Oc|}
\hline 
Nombre d'enfant & 1 & 2 & 3 & 4 & 5 \\ 
\hline 
nombre de famille & 3 & 10 & 7 & 4 & 8 \\ 
\hline 
Effectif cumulé  &  &  &  &  &  \\ 
\hline
Fréquence  &  &  &  &  &  \\ 
\hline 
Fréquence cumulé  &  &  &  &  &  \\ 
\hline
pourcentage &  &  &  &  &  \\ 
\hline 
\end{tabular} 
\anserline[10]
\end{minipage}
\end{enumerate}
\end{exo}

\begin{exo}
\begin{minipage}{.5\linewidth}
Un TGV roule pendant 90 min  à la vitesse de 300 km/h.

Quelle distance parcourt-il ?
\end{minipage}%
\begin{minipage}{.5\linewidth}
\anserline[8]
\end{minipage}%
\end{exo}

\begin{exo}
\begin{minipage}{.5\linewidth}
\begin{enumerate}
\fullwidth{Aprés une remise de 135DH , un costume coûte 765DH.}
\item  De quel pourcentage son prix a-t-il diminué?
\fullwidth{ Sachant que l'on a fixé une solde de -30\% sur un article coûte 480DH.}
\item Quel est le prix à payer?
\end{enumerate}
\end{minipage}%
\begin{minipage}{.5\linewidth}
\anserline[6]
\end{minipage}%
\end{exo}

\begin{exo}
On considère la fonction linéaire $f$ tel que $f(x)=-3x$.
\begin{enumerate}
\item Calculer $f(3)$, $f(-1)$ et $f(\dfrac{2}{3})$.\\
\anserline[4]
\item Quel nombre a pour image -18 par $f$.\\
\anserline[2]
\item Représenter graphiquement la fonction $f$.\\
\anserline[14]
\end{enumerate}
\end{exo}








\end{document}