\documentclass[a4paper,12pt]{article}

\usepackage{dlds}


\begin{document}

\devoir[ds=true,prv=true,num=1 ,niv=2 , date=03/11/2022 ]

\begin{exo}[5]
\begin{enumerate}
\item Compléter par ce qui convient :
\end{enumerate}
\[\dfrac{-8}{16}=\dfrac{6}{....} \hspace{1cm}
\dfrac{-1}{33}=\dfrac{-11}{...}\hspace{1cm}
\dfrac{2}{....}=\dfrac{....}{9} \hspace{1cm}
\dfrac{....}{-8}=\dfrac{4}{....}\hspace{1cm}
\dfrac{3\times .... -6}{4}=\dfrac{-3}{4}\]

\end{exo}

\begin{exo}[4]
\begin{enumerate}
\item Calculer et simplifier :
\end{enumerate}
$\dfrac{-3}{6}+\dfrac{1.5}{-2.4}=\dfrac{....}{.....}\hspace{2cm}
\dfrac{-(-(-(7)))}{-42}-\dfrac{-13}{-2.1}=\dfrac{.........}{..........}\hspace*{1cm} \dfrac{3}{6}+\dfrac{3}{2}\times \dfrac{1}{6}=\dfrac{............}{................}$
\end{exo}


\begin{exo}[2]
\begin{enumerate}
\item Rendre les écritures irréductibles :
\end{enumerate}
\[A=\dfrac{-55\times7}{28\times(-555)} =\dfrac{.........}{...........}\hspace{1cm}
B=\dfrac{-12\times(-3)}{(-36)\times22}=\dfrac{...........}{...........}\]
\[C=\dfrac{24\times36\times125}{50\times3\times5}=\dfrac{.........}{.............} \hspace{1cm}
D=\dfrac{-12+5}{24+4}=\dfrac{..............}{.............}\]
\end{exo}

\begin{exo}[3]
\begin{enumerate}
\item Calculer :
\end{enumerate}
\begin{multicols}{2}
$$A=\left( \dfrac{1}{-5}+\dfrac{-4}{-20}\right) -\left(\dfrac{-8}{15}-\dfrac{3}{30}\right) -\left( -\dfrac{1}{3}-\dfrac{1}{-6} \right)$$
\notes[10pt]{8}{\linewidth}
\columnbreak
$$B= \left[-41-\left( \dfrac{-9}{-4}-\dfrac{-13}{-2}\right) -\left( -\dfrac{1}{8}-\dfrac{-1}{-16} \right)\right]$$
\notes[10pt]{8}{\linewidth}
\end{multicols}
\begin{multicols}{2}
\notes[12pt]{4}{\linewidth}
\columnbreak
\notes[12pt]{4}{\linewidth}
\end{multicols}
\end{exo}

\begin{exo}[3]
\begin{enumerate}
\item Représente sur une droite graduée les points $A$ ,$B$, $C$  et $D$ d'abscisses respectives 
 \[ -\dfrac{4}{5}\quad et \quad \dfrac{-7}{5}\quad et \quad  \dfrac{6}{5} \quad et \quad \dfrac{3}{5} \]
 \end{enumerate}
\begin{tikzpicture}
\tkzInit[xmin=-9, xmax=7]
\tkzDrawX
\end{tikzpicture}
\end{exo}

\begin{exo}[3]
\begin{enumerate}
\item Comparer les nombres suivants : 

\[\dfrac{-8}{41}.......\dfrac{9}{-41} \hspace*{1cm}et \hspace*{1cm}\dfrac{91}{17}.......... 1 \]

\item Ranger les nombres dans l'ordre croissant en justifiant votre réponse :
\[ \dfrac{-4}{3} , \dfrac{-5}{4}, \dfrac{-8}{12}, \dfrac{8}{3}, \dfrac{3}{9}, \dfrac{1}{6}, \dfrac{-5}{3}\]
\end{enumerate}
\notes[12pt]{8}{\linewidth}
\end{exo}



\end{document}