\documentclass[a4paper,12pt]{article}

\usepackage{dlds}

 
\begin{document}

\devoir[sem=2,prv=true,ds=true,num=6 ,niv=1 ,date=22/05/2023]


\begin{exo}
\begin{minipage}{.6\linewidth}
Voici la liste des notes d'un devoir de mathématiques 
\begin{enumerate}
\item compléter le tableau ci-dessous.
\item Représenter le tableau par un diagramme circulaire .
\end{enumerate}
\end{minipage}
\begin{tabular}{c|c|c|c|c|c|c|c}
12 & 15 & 14 & 8 & 6 & 8 & 14 & 14	 \\ 
\hline 
14 & 12 & 8 & 6 & 3 & 8 & 14 & 15	 \\ 
\hline 
12 & 8 & 6 & 6 & 3 & 8 & 14 & 15	 \\ 
\hline 
14 & 15 & 12 & 8 & 8 & 3 & 3 & 15 \\ 
\hline 
20 & 20 & 6 & 6 & 3 & 3 & 8 & 9 \\ 
\hline 
8 & 6 & 8 & 8 & 8 & 14 & 12 & 2 \\ 
\end{tabular} 

\begin{tabular}{|c|c|c|c|c|}
\hline 
Classe : note  & $2\leq n \leq 7$ & $8\leq n \leq 11$ &$12\leq n \leq 16$ & $16\leq n \leq 20$ \\ 
\hline 
Nombre des élèves &  &  &  &  \\ 
\hline 
Fréquence&  &  &  &  \\ 
\hline 
Pourcentage &  &  &  &  \\ 
\hline 
Mesure des angles &  &  &  &  \\ 
\hline 
\end{tabular} 
\end{exo}

\begin{exo}
Le tableau suivant représente le nombre d'enfant par famille

\begin{tabular}{|c|c|c|c|c|c|}
\hline 
Nombre d'enfant & 1 & 2 & 3 & 4 & 5 \\ 
\hline 
nombre de famille & 3 & 10 & 7 & 4 & 8 \\ 
\hline 
Fréquence  &  &  &  &  &  \\ 
\hline 
pourcentage &  &  &  &  &  \\ 
\hline 
\end{tabular} 
\begin{enumerate}
\item compléter le tableau
\item Quel est le caractère de cette série statistique ?
\item Quel est l'effectif total de cette série statistique?
\item Tracer le diagramme en bâtons des effectifs.
\end{enumerate}
\end{exo}


\begin{exo}
Sur la droite graduée ci-dessous, donner la distance à zéro et l’abscisse de chacun des points   :

\begin{tikzpicture}
\tkzInit[xmin = -8,xmax = 5]
\tkzDrawX
\tkzLabelX
\tkzDefPoints{3/0/M,-4/0/N,-5.5/0/P,0.5/0/Q}   
\tkzDrawPoints(M,N,P,Q)
\tkzLabelPoints[above](M,N,P,Q)
\end{tikzpicture}

\begin{tabular}{|c|c|c|c|c|}
\hline 
point & M & N & P & Q \\ 
\hline 
Abscisse &  &  &  &  \\ 
\hline 
Distance à 0 &  &  &  &  \\ 
\hline 
\end{tabular} 
\begin{enumerate}
\item donner les coordonnées des points $A$,$B$,$C$ et $D$.
\item Placer les points $E(-4, 0)$, $F(-2, 2)$, $G(1, -4)$ et $H(3, 3)$.
\end{enumerate}
\begin{reper}
\coordpoints{1}{2}{A}
\coordpoints{4}{0}{B}
\coordpoints{0}{-2}{C}
\coordpoints{-3}{-2}{D}
\end{reper}
\end{exo}

\begin{exo}
On considère le tableau de proportionnalité suivant :
\begin{tabular}{|c|c|c|c|c|}
\hline 
3 &  & 7 &  & 5 \\ 
\hline 
12 & 45 &  & 15.5 &  \\ 
\hline 
\end{tabular} 
\begin{enumerate}
\item calculer le coefficient de proportionnalité .
\item compléter le tableau 
\end{enumerate}
\end{exo}


\begin{exo}
Un TGV roule pendant 90 min  à la vitesse de 300 km/h.

Quelle distance parcourt-il ?
\end{exo}

\begin{exo}
Aprés une remise de 135DH , un costume coûte 765DH.
\begin{enumerate}
\item  De quel pourcentage son prix a-t-il diminué ?
\end{enumerate}
 Sachant que l'on a fixé une solde de -30\% sur un article coûte 480DH.
\begin{enumerate}
\item Quel est le prix à payer ?
\end{enumerate}
\end{exo}


\end{document}