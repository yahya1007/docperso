\documentclass[a4paper,12pt]{article}

\usepackage{dlds}

\setlength{\columnseprule}{1pt}
\setlength{\columnsep}{2em}
\renewcommand{\columnseprulecolor}{\color{gray}} 
\begin{document}

\devoir[prv=true,ds=true,num=4 ,niv=3,sem=2 ,date=02/03/2023]

\begin{exo}[6]
\begin{enumerate}
\item Résoudre les équations suivantes :
\begin{multicols}{4}
$4x-5=4x+11$ \newline
\anserline[4]
\columnbreak

$(4x-5)(3x+2)=0$\newline
\anserline[4]
\columnbreak

$4x^{2}-25=0$\newline
\anserline[4]
\columnbreak

$x^{2}-x+\dfrac{1}{4}=0$\newline
\anserline[4]
\end{multicols}
\item Résoudre les inéquations suivantes et représenter les solutions sur une droite graduée :
\begin{multicols}{2}
$4x-8\leq x+1$ \newline
\anserline[4]
\columnbreak

$5x-6 > 5x+6$\newline
\anserline[4]
\end{multicols}
\end{enumerate}
\end{exo}

\begin{exo}[6]
\begin{enumerate}
\item En utilisant la relation de Chasles , simplifier les écritures des vecteurs suivants :
\begin{multicols}{3}
\(
\vv{AB}+\vv{DA}+\vv{BD}\)\newline
\anserline[4]
\columnbreak

\( \vv{AM}-\vv{MD}+\vv{KA}-\vv{KM}\)\newline
\anserline[4]
\columnbreak

\( 4\vv{AM}-4\vv{GM}-\vv{AG}\)\newline
\anserline[4]
\end{multicols}

\end{enumerate}
\end{exo}

\begin{exo}[5]
\begin{minipage}{.6\linewidth}
Soit $ABCD$ un parallélogramme.
\begin{enumerate}
\item Construis le point $M$ tel que :$\vv{DM}=\dfrac{1}{3}\vv{DC}$
\item Construis le point $N$ tel que : $\vv{BN}=-3\vv{CB}$
\item Montrer que $\vv{AM}=\vv{BC}+\vv{DM}$
\item Montrer que : $\vv{AN}=3\vv{BC}+3\vv{DM}$
\item Déduire que $A$ , $M$ et $N$ sont des points alignés.
\end{enumerate}
\end{minipage}%
\begin{minipage}[t]{.4\linewidth}
\begin{tikzpicture}
\tkzDefPoints{0/0/A,4/0/B,5/-2/C,1/-2/D}
\tkzDrawSegments(A,B B,C C,D D,A)
\tkzLabelPoints[above](A,B)
\tkzLabelPoints(C,D)
\end{tikzpicture}
\end{minipage}
\begin{multicols}{2}
\anserline[10]
\columnbreak

\anserline[10]
\end{multicols}
\end{exo}

\begin{exo}[3]
Soit $ABC$ un triangle et $T$ la translation qui transforme le point $B$ au point $C$.
\begin{enumerate}
\item Construis le point $E$ l'image du point $A$ par la translation $T$.
\item Construis le point $D$ l'image du point $C$ par la translation $T$.
\item Déterminer l'image du triangle $ABC$ par la translation $T$.
\end{enumerate}
\end{exo}

\begin{tikzpicture}
\tkzDefPoints{0/0/A,3/1/B,6/-1/C}
\tkzDrawSegments(A,B B,C C,A)
\tkzLabelPoints(A,C)
\tkzLabelPoint[above](B){B}

\end{tikzpicture}
\end{document}