\documentclass[12pt]{article}

\usepackage{dlds}
\usepackage{fig3d}


\pagestyle{empty}
\begin{document}

\devoir[sem=2,prv=true,ds=true,num=6 ,niv=3 ,date=05/05/2023,grp=5]

\begin{exo}[7]
\begin{enumerate}
\begin{EnvFullwidth}
Le tableau suivant donne la répartition des notes d'une classe à un contrôle.
\end{EnvFullwidth}
\begin{minipage}{.42\linewidth}
\item\bareme{(1)} Compléter le tableau.
\item\bareme{(1)} Quel est l'effectif total?
 \anserline[2]
\item\bareme{(1)} Calculer la fréquence   de 16.\anserline[2]
\end{minipage}%
\begin{minipage}{.58\linewidth}
\begin{tabular}{|Oc|Oc|Oc|Oc|Oc|Oc|Oc|Oc|Oc|Oc|}
\hline 
Note & 8 & 10 & 12 & 14 & 15 & 16 & 18 & 19 & 20 \\ 
\hline 
Effectif & 2 &  & 4 & 5 &  & 6 & 7 & 2 & 1 \\ 
\hline 
Eff cumulé &  & 4 &  &  &  &  &  &  & 30 \\ 
\hline 
\end{tabular} 
\end{minipage}%
\item\bareme{(1)} Calculer la note moyenne de la classe.\anserline[2]
\item\bareme{(1)} Déterminer le mode de cette série.\anserline[1]
\item\bareme{(1)} Déterminer la médiane de cette série.\anserline[1]
\item\bareme{(1)} Quel est le pourcentage des élèves qui ont une note supérieur à la moyenne de classe.\par\anserline[2]
\end{enumerate}
\end{exo}

\begin{exo}[7]
Soit $f$ une fonction linéaire tel que $f(-3)=4$ et $g$ une fonction affine tel que $g(3)=2$ et $g(4)=6$.
\begin{multicols}{2}
\begin{enumerate}
\item \bareme{(1)\\+\\(1)} Montrer que $f(x)=-\dfrac{4}{3}x$.
\par\anserline[4]
\item Calculer $f(3)$.\par\anserline[3]
\item  Montrer que $g(x)=4x-10$.
\\\anserline[4]
\item \bareme{(2)} Quel nombre a pour image 0 par $g$.\\\anserline[3]
\end{enumerate}
\end{multicols}
\end{exo}

\begin{exo}[6]
\begin{enumerate}
\begin{minipage}{.65\linewidth}
\begin{EnvFullwidth}
Soit $SABC$  un tétraèdre de volume $160 cm^3$ et sa hauteur $SO=8cm$.
\end{EnvFullwidth}
\item \bareme{(3)} Calculer l'aire de la base $ABC$.\\\anserline[3]
\item \bareme{(2)} Le tétraèdre $SA'B'C'$ est l'agrandissement du tétraèdre $SABC$ et son volume est égal à $540 cm^3$.
\end{minipage}%
\begin{minipage}{.35\linewidth}
\begin{tikzpicture}
\tetra[60]{3}{70}{70}
\tkzDefPointOnLine[pos=1.4](S,A)\tkzGetPoint{A'}
\tkzDefPointOnLine[pos=1.4](S,B)\tkzGetPoint{B'}
\tkzDefPointOnLine[pos=1.4](S,C)\tkzGetPoint{C'}
\tkzDefPoint(2.5,0.4){O}
\tkzDefPointOnLine[pos=1.4](S,O)\tkzGetPoint{O'}
\tkzDrawPoints[shape=cross out,size=4](O,O')
\tkzDrawSegment[dashed](S,O')
\tkzLabelPoint[left](O){O}
\tkzLabelPoint[left](O'){$O'$}
\tkzDrawSegments(A,A' B,B' C,C' A',B' B',C')
\tkzDrawSegment[dashed](A',C')
\tkzLabelPoints(A',B')
\tkzLabelPoint[right](C'){$C'$}
\end{tikzpicture}
\end{minipage}%
\begin{enumerate}
\item \bareme{(2)} Montrer que le rapport d'agrandissement est $\dfrac{3}{2}$.\\\anserline[3]
\item Montrer que $\dfrac{BC}{B'C'}=\dfrac{2}{3}$.\\\anserline[4]
\end{enumerate}
\item\bareme{(2)} Calculer $SO'$ la hauteur du tétraèdre après l'agrandissement.\\\anserline[3]
\item\bareme{(2)} Calculer l'aire de la base $A'B'C'$ après l'agrandissement.\\\anserline[4]
\item \bareme{(2)}Calculer le volume du solide ABCA'B'C'.\par\anserline[4]
\end{enumerate}
\end{exo}
\end{document}