\documentclass[a4paper,12pt]{article}

\usepackage{dlds}
\setlength{\columnseprule}{1pt}
\setlength{\columnsep}{3em}
\renewcommand{\columnseprulecolor}{\color{gray}}
%\pointsinmargin 
%\pointformat{\color{gray}\footnotesize{(\themarginpoints\,\points)}\hspace{0.5cm}}
%\pointpoints{pt}{pts}
 
\begin{document}

\devoir[sem=2,prv=true,ds=true,num=2 ,niv=3 ,date=27/04/2023]

\begin{exo}[14]
Le plan est muni d'un repère orthonormé $\oij$ et on considère les points $A(1,-1)$ ; $B(3,-5)$ et la droite $(L):y=\dfrac{x}{2}-4$.
\begin{enumerate}
\item Détermine les coordonnées du vecteur $\vv{AB}$ et déduire $AB$\newline
\anserline[2]
\item Donne les coordonnées du point $K$ le milieu du segment $[AB]$.\newline
\anserline[2]
\item La droite $(L)$ passe-t-elle par le point $K$?
\newline
\anserline[2]
\item Montrer que l'équation réduite de la droite $(AB)$ est $y=-2x+1$.
\newline
\anserline[4]
\item Donne l'équation réduite de la droite $(D')$ qui passe par le point $C(3,2)$ et parallèle à la droite $(AB)$.
\newline\anserline[4]
\item Montre que la droite $(L)$ est perpendiculaire à $(AB)$.\newline\anserline[4]
\item Déduire que la droite $(L)$ est la médiatrice du segment $[AB]$.\newline\anserline[3]
\end{enumerate}
\end{exo}

\begin{exo}[6]
\begin{enumerate}
\item Résoudre les systèmes :
\end{enumerate}
\begin{multicols}{2}
$(S_1)\systeme{3x+y=9,-2x-y=-11}$\newline
\anserline[8]
\columnbreak

$(S_2)\systeme{3x+4y=35,2x+8y=42}$\newline
\anserline[8]
\end{multicols}
\end{exo}

\begin{exo}[2]
Une mère a dit à son fils : j'ai dans ma poche 47 DH, et si j'achète 3 Kg de tomates et 4 Kg de pommes de terre il me reste 12 DH.Et si j'achète 2 Kg de tomates et 8 Kg de pommes de terre il me reste 5 DH.

Calculer le prix d'un kilogramme de tomates et le prix d'un kilogramme de pommes de terre.
\end{exo}

\end{document}