\documentclass[a4paper,addpoints,12pt]{exam}

\usepackage{dlds}
\pointsdroppedatright
\marginpointname{ \points}
\setlength{\rightpointsmargin}{25mm}
\pointformat{\bfseries\boldmath[\themarginpoints]}

\begin{document}

\examen[prv=false,date= 2023]

\begin{exo}[7]
\begin{questions}
\question[2]Calculer :\droppoints
\begin{multicols}{2}
$\sqrt{2}^{2}=$\anserline[1]
$\left(\dfrac{2\sqrt{3}}{3\sqrt{5}}\right)^{2}=$\anserline[2]\columnbreak

$\sqrt{5}\times \sqrt{\sqrt{16}}\times\sqrt{20}=$\anserline[3]
\end{multicols}
\question[1]Simplifier \droppoints
$A=\sqrt{63}+\sqrt{28}-\sqrt{7}=$\anserline[3]
\question[1]Développer et simplifier\droppoints
\begin{multicols}{2}
$\left(\sqrt{7}-3\right)^{2}=$\anserline[2]
$\left(3-\sqrt{2}\right)\left(3+\sqrt{2}\right)=$\anserline[2]
\end{multicols}
\question[1]Factoriser :\droppoints
$x^{2}+2\sqrt{7}x+7=$\anserline[2]
\question[1]Donner l'écriture scientifique\droppoints
$\dfrac{560\times 10^{26}}{0.000008\times 10^{-14}}=$\anserline[3]
\question[1]Ecrire le dénominateur sans radicale :\droppoints
$\dfrac{2}{2+\sqrt{3}}=$\anserline[1]
\end{questions}
\end{exo}

\begin{exo}[4]
\begin{questions}
\question[1]Comparer $2\sqrt{3}$ et $\sqrt{13}$\droppoints
\anserline[2]
\question[1]Déduire la comparaison des nombres : 
$\dfrac{3}{1+2\sqrt{3}}$ et $\dfrac{3}{1+\sqrt{13}}$\droppoints
\anserline[4]
\question[2]Soient $4\leq a\leq 7$ et $1\leq b\leq 3$ . Encadrer :\droppoints
\begin{multicols}{4}
$2a+b$\notes[10pt]{6}{\linewidth}\columnbreak

$a-b$\notes[10pt]{6}{\linewidth}\columnbreak
 
$ab$\notes[10pt]{6}{\linewidth}\columnbreak
  
$\dfrac{a}{b}$ \notes[10pt]{6}{\linewidth}
\end{multicols}
\end{questions}
\end{exo}

\begin{exo}[7]
On considère la figure ci-contre tel que : $AC=5$ , $BA=10$ et $BC=5\sqrt{5}$ et $H$ le projeté orthogonal de $A$ sur $(BC)$.
\begin{questions}
\question[1] Montrer que $ABC$ est un triangle rectangle .\droppoints
\begin{minipage}{0.6\linewidth}
\anserline[5]
\end{minipage}%
\begin{minipage}{0.4\linewidth}
\begin{tikzpicture}
\tkzDefPoints{-1/0/B,5/-1/C}
\tkzDefTriangle[two angles=50 and 40](B,C)\tkzGetPoint{A}
\tkzDefPointBy[projection=onto C--B](A)\tkzGetPoint{H}
\tkzDrawSegments(A,B B,C C,A A,H)
\tkzLabelPoints(B,C,H)
\tkzLabelPoint[above](A){A}
\tkzMarkRightAngle(C,H,A)
\end{tikzpicture}
\end{minipage}
\question[2] Calculer les rapports trigonométriques d'angle $\widehat{ACB}$\droppoints
\anserline[3]
\question[1] Calculer $AH$\droppoints
\anserline[3]
\end{questions}
Soit $x$ la mesure d'un angle aigu tel que : $\cos x =\dfrac{2}{3}$
\begin{questions}
\setcounter{question}{3}
\question[1] Calculer : $\sin x$ et $\tan x$\droppoints
\anserline[4]
\question[1] Simplifier  :\droppoints
$A = \sin^{2}40^{\circ} +\cos55^{\circ} +\sin^{2}50^{\circ}-\sin35^{\circ} =$\anserline[3]
\question[1] Montrer que : $1+\tan^{2}x = \dfrac{1}{\cos^{2}x}$\droppoints
\anserline[4]
\end{questions}
\end{exo}

\begin{exo}[2]
$ABC$ est un triangle tel que : $AC=9$ , $CB=12$ , $AM=6$  et $(BC)//(MN)$
\begin{questions}
\question[1] Calculer  $MN$ .\droppoints
\begin{minipage}{0.6\linewidth}
\anserline[6]
\end{minipage}
\begin{minipage}{0.4\linewidth}
\begin{tikzpicture}
\tkzDefPoints{0/0/C,5/0/B,-1/4/A}
\tkzDefPointOnLine[pos=.3](C,A)\tkzGetPoint{M}
\tkzDefPointOnLine[pos=.3](C,B)\tkzGetPoint{P}
\tkzDefPointOnLine[pos=.3](B,A)\tkzGetPoint{N}
\tkzDrawLines(P,M M,N)
\tkzDrawSegments(A,C A,B B,C)
\tkzLabelPoints(B,C,P)
\tkzLabelPoints[below left](A,M)
\tkzLabelPoint[above](N){N}
\end{tikzpicture}
\end{minipage}
\anserline[3]
\question[1] Soit $P$ un point de $[CB] $ tel que: $PC=4$.
Monter que : $(AB)//(MP)$\droppoints
\anserline[8]
\end{questions}
\end{exo}


\end{document}