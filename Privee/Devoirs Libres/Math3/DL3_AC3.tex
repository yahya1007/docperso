\documentclass[a4paper,12pt]{article}

\usepackage{dlds}


 
\begin{document}

\devoir[ds=false,num=3 ,niv=3 , date=05/01/2023,Rdate=09/01/2023 ,prv=true]

\begin{exo}
\begin{minipage}{0.6\linewidth}
On considère la figure ci-contre tel que $ABCD$ est un  carré  et $I$ le milieu de $[DC]$ et $AB=12$ et$JC=3$
\begin{enumerate}
\item Montrer que $AIJ$ est un triangle rectangle .
\end{enumerate}
\end{minipage}
\begin{minipage}{0.4\linewidth}
\begin{tikzpicture}
\tkzDefPoints{-1/4/A,4/4/B}
\tkzDefSquare(A,B)\tkzGetPoints{C}{D}
\tkzDrawPolygon(A,B,C,D)
\tkzDefMidPoint(D,C)\tkzGetPoint{I}
\tkzDefPointOnLine[pos=0.25](C,B)\tkzGetPoint{J}
\tkzDrawPoints(A,B,C,D,I,J)
\tkzDrawPolygon(A,I,J)
\tkzLabelPoints(A,B)
\tkzLabelPoints[above](D,I,C)
\tkzLabelPoints[right](J)
\tkzMarkRightAngles(C,B,A A,D,C D,C,B)
\end{tikzpicture}
\end{minipage}
\end{exo}

\begin{exo}

Soit  $ABC$ un triangle rectangle en $A$ tel que  $AC=7$ et $\cos \widehat{C}=\dfrac{3}{5}$
\begin{enumerate}
\item Calculer  $\sin \widehat{C}$ et $\tan \widehat{C}$ .
\item Calculer les rapports trigonométriques d'angle $\widehat{B}$.
\item Déduire $AB$ et $BC$ .
\end{enumerate}
\end{exo}

\begin{exo}
\begin{enumerate}
\item Calculer ce qui suit :
$$A= cos^{2}14^{\circ}+cos^{2}28^{\circ}+cos^{2}76^{\circ}+cos^{2}62^{\circ}$$
$$B=sin^{2}40^{\circ}-4cos^{2}30^{\circ}+sin^{2}50^{\circ}+tan^{2}45^{\circ}$$
$$C=tan70^{\circ}-\dfrac{1}{tan20^{\circ}}+\dfrac{2}{tan60^{\circ}}$$
\item Simplifier :
$$ X=(sinx+cosx)^{2}-(sinx-cosx)^{2} $$
$$ Y=1+tan^{2}x+\dfrac{1}{cosx}$$
\end{enumerate}

\end{exo}

\end{document}
